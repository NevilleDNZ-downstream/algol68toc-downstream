 $Log: chapter13.tex,v $
 Revision 1.11  2012/01/04 17:20:06  sian
 *** empty log message ***

 Revision 1.10  2005-01-28 20:26:18  sian
 Minor changes.

 Revision 1.9  2005/01/28 17:11:04  sian
 Altered the description of `establish' for `client socket channel' and
 `server socket channel' so that the `c' parameter is used as the MTU.
%
 Revision 1.8  2004/11/07 16:00:10  sian
 Removed all references to PLUS, MINUS and TIMES
%
 Revision 1.7  2003/05/31 13:02:41  sian
 Removal of compiled libraries and binaries
%
 Revision 1.6  2003/04/23 08:57:31  sian
 Debian release 1.5
%
 Revision 1.5  2002/06/20 11:49:28  sian
 mm removed, padding removed, ca68 added
%
\Chapter{Standard Prelude}{chapxiii}\catcode`\%=12
The function of this chapter is to describe all the facilities in the
standard prelude supplied with the Linux port of the
\hx{a68toc}{a68toc!standard prelude}
compiler. The standard prelude contains both implicit declarations
(facilities provided by the compiler) and explicit declarations
(those defined in, and made available by, the QAD standard
prelude\footnote{QAD stands for ``quick-and-dirty'' and was supposed
to represent the provided standard prelude. While it is not entirely
standard (as far as Algol~68 is concerned), it was certainly not
implemented quickly!}). They are classified and dealt with as follows:-
\begin{enumerate}
\item \textbf{Standard modes} \newline
These are the modes defined by the Algol 68 Revised Report, which
defines the language, plus modes required by the transput.
\item \textbf{Environment enquiries} \newline
Some of these are defined in the Revised Report.
\item \textbf{Standard operators} \newline
There are a large number of these, all defined in the Revised Report
and classified by the modes of their operands. They are preceded by a
subsection giving their priorities.
\item\textbf{Other operators} \newline
Some operators are provided which are not in the Revised Report. They
are described in this section. However, operators peculiar to the
a68toc implementation are described in the section on a68toc
extensions.
\item \textbf{Standard procedures} \newline
Only those procedures not used in transput and process control are
defined here. They all appear in the Revised Report.
\item \textbf{Other procedures} \newline
Procedures which appear neither in the Revised Report nor in any other
section appear here.
\item \textbf{\texttt{ALIEN} declarations}
This section includes all the \ixtt{ALIEN} declarations made
available by the standard prelude.
\item \textbf{a68toc extensions}\newline
All the extensions to the language are described in this section
including modes, constructs, operators and procedures.
\item \textbf{Process control} \newline
These declarations provide control over the working of the
floating-point unit, integer overflow and signal handling. They include
declarations for controlling the Algol~68 garbage collector.
\item \textbf{Transput} \newline
This very large section provides specifications for all the transput
declarations available in the Standard Prelude, but omits those
operators and procedures which are intended for internal use only.
\end{enumerate}
See the bibliography for details of the \ix{Revised Report}.

\Section{Standard modes}{stan-modes}
Many of the modes available in the standard prelude are built from the
standard modes of the language which are all defined in the Revised
Report.
\begin{enumerate}
\item \ixtt{VOID} \newline
This mode has one value: \verb|EMPTY|. It is mainly used as the yield
of routines and in unions.
\item \ixtt{BOOL} \newline
 This mode has
two values, namely \ixtt{TRUE} and
\ixtt{FALSE}.
\item \verb|INT| \newline
This is the basic arithmetic mode. Various precisions are available:-
\begin{enumerate}
\item \ixtt{LONG INT} 64-bit integer
\item \ixtt{INT} 32-bit integer
\item \ixtt{SHORT INT} 16-bit
integer
\item \ixtt{SHORT SHORT INT} 8-bit integer
\end{enumerate}
\item \verb|REAL|\newline
This mode is used mainly for approximate calculations although exact
values can be manipulated provided that the number of significant
digits does not exceed the precision of the mantissa (see section
\hyref{stan-arithenq}). The following precisions are available:-
\begin{enumerate}
\item \ixtt{REAL} 64-bit real
\item \ixtt{SHORT REAL} 32-bit real
\end{enumerate}
\item \verb|COMPL| \newline
Strictly speaking, this is not a fundamental mode because it is
regarded as a structure with two fields:-
\begin{verbatim}
   MODE COMPL = STRUCT(REAL re,im)
\end{verbatim}
\noindent
However, the widening coercion will convert a \verb|REAL| value into a
\verb|COMPL| value and \verb|COMPL| values are not straightened (see
section~\hyref{stan-straight}). Like \verb|REAL|s, the following
precisions are available:-
\begin{enumerate}
\item \ixtt{COMPL} 128-bit
\item \ixtt{SHORT COMPL} 64-bit
\end{enumerate}
\item \ixtt{CHAR} \newline
This mode is used for most character operations. See section
\hyref{stan-enqchar} for further details.
\item \verb|BITS| \newline
This mode is equivalent to a computer word regarded as a group of bits
(binary digits) numbered 1 to \verb|bits width| (see
section \hyref{stan-intsz}). Various precisions are available:-
\begin{enumerate}
\item \ixtt{LONG BITS} 64-bit
\item \ixtt{BITS} 32-bit
\item \ixtt{SHORT BITS} 16-bit
\item \ixtt{SHORT SHORT BITS} 8-bit
\end{enumerate}
\item \ixtt{BYTES} \newline
The Revised Report describes the mode, but the a68toc compiler does not
implement it.
\item \ixtt{STRING} \newline
This mode is defined as
\begin{verbatim}
   MODE STRING = FLEX[1:0]CHAR
\end{verbatim}
\noindent
and is provided with a shorthand construct for denotations of such
values (see section~\hyref{rep-mult}).
\end{enumerate}

\Section{Environment enquiries}{stan-enq}
Algol 68 was the first programming language to contain declarations
which enable a programmer to determine the characteristics of the
implementation. The enquiries are divided into a number of different
groups. The actual values of the Linux port of the a68toc compiler
are placed in square brackets. Those defined in the Revised Report
are marked with (RR).

\Subsection{Arithmetic enquiries}{stan-arithenq}
These enquiries are so numerous that they are further subdivided.

\subsubsection*{Enquiries about precisions}\hylabel{stan-prec}
Any number of \verb|LONG| or \verb|SHORT| can be given in the mode
specification of numbers, but only a few such modes are distinguishable
in any implementation. The following environment enquiries tell which
modes are distinguishable. Note particularly that there are more
distinguishable precisions available for \verb|INT| and \verb|BITS|
than there are for \verb|REAL| and \verb|COMPL| in the
\hx{a68toc}{a68toc!precisions} implementation.
\begin{enumerate}
\item \hxtt{INT int lengths}{int lengths} (RR) $[2]$ \newline
$1+$ the number of extra lengths of integers.
\item \hxtt{INT int shorths}{int shorths} (RR) $[3]$ \newline
$1+$ the number of short lengths of integers.
\item \hxtt{INT real lengths}{real lengths} (RR) $[1]$ \newline
$1+$ the number of extra lengths of real numbers.
\item \hxtt{INT real shorths}{real shorths} (RR) $[2]$ \newline
$1+$ the number of short lengths of real numbers.
\item \hxtt{INT bits lengths}{bits lengths} (RR) $[2]$ \newline
$1+$ the number of extra lengths of \verb|BITS|.
\item \hxtt{INT bits shorths}{bits shorths} (RR) $[3]$ \newline
$1+$ the number of short lengths of \verb|BITS|.
\item \hxtt{INT bytes lengths}{bytes lengths} (RR) $[0]$ \newline
Bytes are not implemented by the
\hx{a68toc}{a68toc!BYTES@\texttt{BYTES}} compiler.
\item \hxtt{INT bytes shorths}{bytes shorths} (RR) $[0]$ \newline
Bytes are not implemented by the a68toc compiler.
\end{enumerate}

\subsubsection*{Enquiries about ranges}\hylabel{stan-range}
\begin{enumerate}
\item \verb|SHORT SHORT INT| \ixtt{short short max int} (RR) $[127]$\newline
The maximum value of mode \verb|SHORT SHORT INT|.
\item \hxtt{SHORT INT short max int}{short max int} (RR) $[32\,767]$ \newline
The maximum value of mode \verb|SHORT INT|.
\item \hxtt{INT max int}{max int} (RR) $[2\,147\,483\,647]$ \newline
The maximum value of mode \verb|INT|.
\item \hxtt{LONG INT long max int}{long max int} (RR)\newline
$[9\,223\,372\,036\,854\,775\,807]$ \newline
The maximum value of mode \verb|LONG INT|.
\item \hxtt{SHORT REAL short min real}{short min real} $[0.117\,55e-37]$ \newline
The smallest representable short real. It should not be confused with
\verb|short small real|.
\item \hxtt{SHORT REAL short max real}{short max real} (RR) $[0.340\,28e+39]$ \newline
The largest short real value storable.
\item \hxtt{SHORT REAL short small real}{short small real} (RR)\newline
$[1.192\,09e-7]$\newline
The smallest short real which, when added to $1.0$ makes a discernible
difference.
\item \hxtt{REAL min real}{min real}
 $[0.197\,626\,258\,336\,50e-322]$\hfil \newline
The smallest representable real. It should not be confused with
\verb|small real|.
\item \hxtt{REAL max real}{max real}
 (RR) $[0.179\,769\,313\,486\,23e+309]$\hfil\newline
The largest real value storable.
\item \hxtt{REAL small real}{small real}
 (RR) $[0.222\,044\,604\,925\,031e-15]$\hfil\newline
The smallest real which, when added to $1.0$, makes a discernible
difference.
\item \hxtt{INT bytes per bits}{bytes per bits} (RR) $0$
\end{enumerate}

\subsubsection*{Internal sizes of modes}\hylabel{stan-intsz}
\begin{enumerate}
\item \hxtt{INT short short int width}{short short int width} $[3]$
\newline
The maximum number of decimal digits expressible by a value of mode
\verb|SHORT SHORT INT|.
\item \hxtt{INT short int width}{short int width} $[5]$ \newline
The maximum number of decimal digits expressible by a value of mode
\verb|SHORT INT|.

\item \hxtt{INT int width}{int width} $[10]$
\newline
The maximum number of decimal digits expressible by a value of mode
\verb|INT|.
\item \hxtt{INT long int width}{long int width} $[19]$ \newline
The maximum number of decimal digits expressible by a value of mode
\verb|LONG INT|.
\item \hxtt{INT short short bits width}{short short bits width}
(RR) $[8]$ \newline
The number of bits required to hold a value of mode
\verb|SHORT SHORT BITS|.
\item \hxtt{INT short bits width}{short bits width} (RR) $[16]$ \newline
The number of bits required to hold a value of mode \verb|SHORT BITS|.
\item \hxtt{INT bits width}{bits width} (RR) $[32]$ \newline
The number of bits required to hold a value of mode \verb|BITS|.
\item \hxtt{INT long bits width}{long bits width} (RR) $[64]$ \newline
The number of bits required to hold a value of mode \verb|LONG BITS|.
\item \hxtt{INT bytes width}{bytes width}
(RR) $[0]$ \newline
The mode \verb|BYTES| is not implemented.
\item \hxtt{INT short real precision}{short real precision} $[24]$
\newline
The number of bits used for the mantissa of a short real.
\item \hxtt{INT short real width}{short real width} $[6]$ \newline
The maximum number of significant decimal digits in a small real.
\item \hxtt{INT short min exp}{short min exp} $[-125]$ \newline
The minimum exponent of a short real.
\item \hxtt{INT short max exp}{short max exp} $[128]$ \newline
The maximum exponent of a short real.
\item \hxtt{INT short exp width}{short exp width} $[2]$\newline
The maximum number of decimal digits in the exponent of a short real.
This can be less than the number of digits occupied by
\verb|short max exp| because any decimal digit can be represented. For
example, 99 but not 999.
\item \hxtt{INT real precision}{real precision} $[53]$ \newline
The number of bits used for the mantissa of a real.
\item \hxtt{INT real width}{real width}
$[15]$ \newline
The maximum number of significant decimal digits in a real.
\item \hxtt{INT min exp}{min exp} $[-1021]$\newline
The minimum exponent of a real.
\item \hxtt{INT max exp}{max exp} $[1024]$\newline
The maximum exponent of a real.
\item \hxtt{INT exp width}{exp width} $[3]$\newline
The maximum number of decimal digits in the exponent of a real. See
also \verb|short exp width|.
\end{enumerate}

\subsubsection*{Sizes used in binary transput}\hylabel{stan-binary}
These values give the sizes of each mode when transput using
\ixtt{put bin} or \ixtt{get bin}.
\begin{enumerate}
\item \hxtt{INT long bits bin bytes}{long bits bin bytes} $[8]$
\item \hxtt{INT bits bin bytes}{bits bin bytes} $[4]$
\item \hxtt{INT short bits bin bytes}{short bits bin bytes} $[2]$
\item \hxtt{INT short short bits bin bytes}{short short bits bin bytes} $[1]$
\item \hxtt{INT bool bin bytes}{bool bin bytes} $[4]$
\item \hxtt{INT long int bin bytes}{long int bin bytes} $[8]$
\item \hxtt{INT int bin bytes}{int bin bytes} $[4]$
\item \hxtt{INT short int bin bytes}{short int bin bytes} $[2]$
\item \hxtt{INT short short int bin bytes}{short short int bin bytes}
$[1]$
\item \hxtt{INT real bin bytes}{real bin bytes} $[8]$
\item \hxtt{INT short real bin bytes}{real bin bytes} $[4]$
\item \hxtt{INT compl bin bytes}{compl bin bytes} $[16]$
\item \hxtt{INT short compl bin bytes}{short compl bin bytes} $[8]$
\item \hxtt{INT char bin bytes}{char bin bytes} $[1]$
\end{enumerate}

\subsubsection*{Particular arithmetic values}\hylabel{stan-arithval}
\begin{enumerate}
\item \hxtt{REAL infinity}{infinity}\newline
Defined by the C mathematics library as \verb|HUGE_VAL|.
\item \hxtt{SHORT REAL short pi}{short pi} $[3.14159]$
\item \hxtt{REAL pi}{pi} $[3.141\,592\,653\,589\,79]$
\item \hxtt{REAL log2}{log2} $[0.301\,029\,995\,663\,981]$\newline
This is the value of $\log_{10}2$.
\end{enumerate}

\Subsection{Character set enquiries}{stan-enqchar}
The \hx{a68toc}{a68toc!charset} implementation of Algol 68 is
bedevilled by the peculiar limitations of the C~programming language
in which a character is actually an integer and indistinguishable
from such. Furthermore, a~C `character' is a signed integer,
equivalent to a value of mode \verb|SHORT SHORT INT|. Thus~C
`characters' range from $-128$ to $+127$.  Algol~68, on the other
hand, has the mode \verb|CHAR| which, at a high level, is distinct
from values of both mode \verb|INT| and mode \verb|SHORT SHORT INT|.
The absolute value of Algol~68 characters range from $0$ to the value
of \verb|max abs char|. Furthermore, the operator \verb|REPR| will
convert any \verb|INT| upto \verb|max abs char| to a character. Be
warned that the~C value of \verb|REPR 225|, for example, is $-31$!
What character is represented by \verb|REPR 225| will depend on the
character set used by the displaying device. An ISO 8859-1 character
set, for example, will display `\'a'. The environment enquiries in
this section are limited to a range enquiry and the values of
commonly required characters.
\begin{enumerate}
\item \hxtt{INT max abs char}{max abs char}
(RR) $[255]$ \newline
The largest positive integer which can be represented as a character.
\item \hxtt{CHAR null character}{null character}
(RR) $[\mbox{\texttt{REPR 0}}]$
\item \hxtt{CHAR nul ch}{nul ch}
$[\mbox{\texttt{REPR 0}}]$ \newline
This is a synonym for \verb|null character|.
\item \hxtt{CHAR blank}{blank}
(RR) $[\mbox{\texttt{REPR 32}}]$ \newline
This is a space character.
\item \hxtt{CHAR error char}{error char}
(RR) $[\ast]$ \newline
This character is used by the conversion routines for invalid values.
\item \hxtt{CHAR flip}{flip} (RR) $[\mbox{T}]$
\newline
This character is used to represent \verb|TRUE| as an external value.
\item \hxtt{CHAR flop}{flop} (RR) $[\mbox{F}]$
\newline
This character is used to represent \verb|FALSE| as an external value.
\item \hxtt{CHAR cr}{cr}
$[\mbox{\texttt{REPR 13}}]$ \newline
This character is sometimes used as a line terminator, usually in
association with \verb|lf|.
\item \hxtt{CHAR lf}{lf}
$[\mbox{\texttt{REPR 10}}]$  \newline
This character terminates lines on Linux.
\item \hxtt{CHAR ff}{ff}
$[\mbox{\texttt{REPR 12}}]$ \newline
This character is the ``form-feed'' character often used for
continuous stationery.
\item \hxtt{CHAR tab ch}{tab ch}
$[\mbox{\texttt{REPR 9}}]$ \newline
This character is used to provide crude formatting of text files,
particularly those which mimic documents produced by typewriters.

\item \hxtt{CHAR esc}{esc}
$[\mbox{\texttt{REPR 27}}]$ \newline
This character is mainly used to introduce ``escape sequences'' which
control the format and colour of output on Linux virtual terminals
(VTs) and \textit{xterm} windows.\footnote{See the file
\texttt{/usr/share/doc/xtermcontrol/ctlseqs.txt.gz} for the latter.}
\item \hxtt{CHAR eof char}{eof char}
$[\mbox{\texttt{REPR 26}}]$ \newline
This character was used to denote the end of a plain text file in the
MS-DOS operating system.
\end{enumerate}

\Section{Standard operators}{stan-ops}
The number of distinct operators is vastly increased by the
availability of \verb|SHORT| and \verb|LONG| modes. Thus it is
imperative that some kind of shorthand be used to describe the
operators. Following the subsection on the method of description are
sections devoted to operators with classes of operands. The end of this
section contains tables of all the operators.

\Subsection{Method of description}{stan-opsmeth}
Where an operator has operands and yield which may include \verb|LONG|
or \verb|SHORT|, the mode is written using \verb|L|. For example,
\begin{verbatim}
   OP + = (L INT, L INT)L INT:
\end{verbatim}
\noindent
is a shorthand for the following operators:-
\begin{verbatim}
   OP + = (LONG INT,LONG INT)LONG INT:
   OP + = (INT,INT)INT:
   OP + = (SHORT INT,SHORT INT)SHORT INT:
   OP + = (SHORT SHORT INT,SHORT SHORT INT)
                SHORT SHORT INT:
\end{verbatim}
\noindent
Ensure that wherever \verb|L| is replaced by \verb|SHORT|s or
\verb|LONG|s, it should be replaced by the same number of \verb|SHORT|s
or \verb|LONG|s throughout the definition of that operator. This is
known as ``consistent substitution''. Note that any number of
\verb|SHORT|s or \verb|LONG|s can be given in the mode of any value
whose mode accepts such constructs (\verb|INT|, \verb|REAL|,
\verb|COMPL| and \verb|BITS|), but the only modes which can be
distinguished are those specified by the environment enquiries in
section~\hyref{stan-prec}. However, you should note that even though
values of modes \verb|LONG REAL| and \verb|LONG LONG REAL| cannot be
distinguished internally, the Algol~68 compiler still regards them as
having unique modes and you will need to use the \verb|LENG| operator
to convert a value of mode \verb|LONG REAL| to a value of mode
\verb|LONG LONG REAL|.

The priority of an operator is independent of the mode of the
operator and so is given in a separate subsection. Each operator is
accompanied by a short description of its function.

\Subsection{Standard priorities}{stan-prio}
The priority of declarations of the standard operators can be changed
in subsidiary ranges using the \ixtt{PRIO}
declaration (see section~\hyref{rout-dyad}). Each of the following
enumerated nine sections contains a list of those operators which
have that priority.  Operators in parentheses are not defined in the
Revised Report. See section~\hyref{stan-cont} for their details.
\goodbreak
\begin{enumerate}
\item \verb|+:=|, \verb|-:=|, \verb|*:=|, \verb|/:=|,
\verb|%:=|, \verb|%*:=|, \verb|+=:|\newline
\ixtt{PLUSAB}, \ixtt{MINUSAB}, \ixtt{TIMESAB}, \ixtt{DIVAB},
\ixtt{OVERAB}, \ixtt{MODAB}, \ixtt{PLUSTO}
\item \ixtt{OR}
\item \verb|&|, \ixtt{AND}
\item \ixtt{=}, \ixtt{/=}, \verb|EQ|, \verb|NE|
\item \ixtt{<}, \ixtt{<=}, \ixtt{>=}, \ixtt{>} \newline
\verb|LT|, \verb|LE|, \verb|GE|, \verb|GT|
\item \ixtt{-}, \ixtt{+},
\item \ixtt{*}, \ixtt{/}, \verb|%|, \verb|%*|,\newline
\ixtt{OVER}, \ixtt{MOD}, \ixtt{ELEM},
\item \ixtt{**}, \verb|UP|, \verb|DOWN|, \ixtt{SHL},
\ixtt{SHR}, \ixtt{LWB}, \ixtt{UPB}
\item \verb|+*|, \ixtt{I}, (\ixtt{MIN}, \ixtt{MAX})
\end{enumerate}

\Subsection{Operators with row operands}{stan-optrow}
Both monadic and dyadic forms are available. We shall use the mode
\verb|ROW| to denote the mode of any multiple.
\begin{enumerate}
\item Monadic.\newline
\hx{\texttt{OP LWB = (ROW)INT:}}{LWB@\texttt{LWB}!monadic}\newline
\hx{\texttt{OP UPB = (ROW)INT:}}{UPB@\texttt{UPB}!monadic}\newline
Yields the lower or upper bound of the first or only dimension of its
operand.
\item Dyadic.\newline
\hx{\texttt{OP LWB = (INT n,ROW r)INT:}}{LWB@\texttt{LWB}!dyadic}\newline
\hx{\texttt{OP UPB = (INT n,ROW r)INT:}}{UPB@\texttt{UPB}!dyadic}\newline
Yields the lower or upper bound of the $n$-th dimension of the
multiple \verb|r|.
\end{enumerate}

\Subsection{Operators with \texttt{BOOL} operands}{stan-opbool}
\begin{enumerate}
\item \hxtt{OP OR = (BOOL a,b)BOOL:}{OR}\newline
Logical \verb|OR|.
\item \verb|OP & = (BOOL a,b)BOOL:|\newline
Logical \verb|AND| (synonym \ixtt{AND}).
\item \hxtt{OP NOT = (BOOL a)BOOL:}{NOT}\newline
Logical \verb|NOT|: \verb|TRUE| if \verb|a| is \verb|FALSE| and vice
versa.
\item \hxtt{OP = =(BOOL a,b)BOOL:}{=}\newline
\verb|TRUE| if \verb|a| equals \verb|b| (synonym is \verb|EQ|).
\item \hxtt{OP /= =(BOOL a,b)BOOL:}{/=}\newline
\verb|TRUE| if \verb|a| not equal to \verb|b| (synonym is \verb|NE|).
\item \hxtt{OP ABS = (BOOL a)INT:}{ABS}\newline
\verb|ABS TRUE| is $1$ and \verb|ABS FALSE| is $0$.
\end{enumerate}

\Subsection{Operators with \texttt{INT} operands}{stan-opint}
Most of these operators take values of any precision. The \verb|L|
shorthand is used for those that can.

\subsubsection*{Monadic operators}\hylabel{stan-opintmon}
Consistent substitution applies to all those operators in this section
which use the \verb|L| shorthand: apart from \verb|LENG| and
\verb|SHORTEN|, the precision of the yield is the same as the precicion
of the operand.
\begin{enumerate}
\item \hx{\texttt{OP + = (L INT a)L INT:}}{+@\texttt{+}!monadic}\newline
The identity operator. Does nothing.
\item \hx{\texttt{OP - = (L INT a)L INT:}}{-@\texttt{-}!monadic}\newline
The negation operator.
\item \verb|OP ABS = (L INT a)L INT:|\newline
The absolute value. \verb|ABS -3 = +3|
\item \hxtt{OP SIGN = (L INT a)INT:}{SIGN}\newline
Yields $-1$ for a negative operand, $+1$ for a positive operand and $0$
for a zero operand.
\item \hxtt{OP ODD = (L INT a)BOOL:}{ODD} \newline
Yields \verb|TRUE| if the operand is odd.
\item \hxtt{OP LENG = (L INT a)LONG L INT:}{LENG} \newline
\verb|OP LENG = (SHORT L INT a)L INT:| \newline
Converts its operand to the next longer precision. Note that you cannot
use both \verb|SHORT| and \verb|LONG| in the same mode.
\item \hxtt{OP SHORTEN = (L INT a)SHORT L INT:}{SHORTEN}
\newline
\verb|OP SHORTEN = (LONG L INT a)L INT:| \newline
Converts its operand to the next shorter precision. If the value
exceeds \verb|l max int| for the next shorter precision, the value will
be truncated. This can lead to erroneous results. See also \verb|LENG|.
\end{enumerate}

\subsubsection*{Dyadic operators}\hylabel{stan-opintdy}
In this section, consistent substitution is used wherever the \verb|L|
shorthand is used. For operators with mixed operands, see
section~\hyref{stan-mixed}.
\begin{enumerate}
\item \hx{\texttt{OP + = (L INT a,L INT b)L INT:}}{+@\texttt{+}!dyadic}\newline
Arithmetic addition: $a+b$. No check is made for integer overflow.
\item \hx{\texttt{OP - = (L INT a,L INT b)L INT:}}{-@\texttt{-}!dyadic}\newline
Arithmetic subtraction: $a-b$. No check is made for integer
overflow.
\item \hxtt{OP * = (L INT a,L INT b)L INT:}{*}\newline
Arithmetic multiplication: $a\times b$. No check is made for
integer overflow.
\item \hxtt{OP / = (L INT a,L INT b)L REAL:}{/}\newline
Arithmetic fractional division. Even if the result is a whole number
(for example, $6/3$), the yield always has mode \verb|L REAL|. Where a
result of mode \verb|L REAL| needs to be output, but cannot be output
due to the limitations built into the definition of the mode
\verb|SIMPLOUT|, the operators \verb|LENG| or \verb|SHORTEN| should be
used. Floating-point overflow can be checked (see
section \hyref{stan-fpov}).
\item \verb|OP % = (L INT a,L INT b)L INT:|\newline
Arithmetic integer division. Division by zero in the
\hx{a68toc}{a68toc!division by zero}
implementation produces a floating-point exception paradoxically
(synonym \ixtt{OVER}).
\item \hxtt{OP ** = (L INT a,INT b)L INT:}{**}\newline
Computes $a^b$ for $b\geq0$.
\item \verb|OP %* = (L INT a,L INT b)L INT:|\newline
Arithmetic modulo (synonym \ixtt{MOD}). For example
\begin{verbatim}
   5 MOD 3 = 2
\end{verbatim}
\item \hxtt{OP +* = (L INT a,L INT b)L COMPL:}{+*}\newline
Converts two integers into a complex number of the same precision
(synonym \ixtt{I}).
\item \hxtt{OP < = (L INT a,L INT b)BOOL:}{<}\newline
Arithmetic ``less than'': $a<b$ (synonym \verb|LT|).
\item \hxtt{OP <= = (L INT a,L INT b)BOOL:}{<=}
\newline
Arithmetic ``less than or equals'': $a\leq b$ (synonym \verb|LE|).
\item \hxtt{OP >= = (L INT a,L INT b)BOOL:}{>=}\newline
Arithmetic ``greater than or equals'': $a\geq b$ (synonym \verb|GE|).
\item \hxtt{OP > = (L INT a,L INT b)BOOL:}{>}\newline
Arithmetic ``greater than'': $a>b$ (synonym \verb|GT|).
\item \hxtt{OP = = (L INT a,L INT b)BOOL:}{=}\newline
Arithmetic equality: $a=b$ (synonym \verb|EQ|).
\item \hxtt{OP /= = (L INT a,L INT b)BOOL:}{/=}\newline
Arithmetic inequality: $a\neq b$ (synonym \verb|NE|).
\end{enumerate}

\Subsection{Operators with \texttt{REAL} operands}{stan-opreal}
Most of these operators can have operands of any precision. The
\verb|L| shorthand is used for them.

\subsubsection*{Monadic operators}\hylabel{stan-oprealmon}
\begin{enumerate}
\item \hx{\texttt{OP + = (L REAL a)L REAL:}}{+@\texttt{+}!monadic}\newline
Arithmetic identity. Does nothing.
\item \hx{\texttt{OP - = (L REAL a)L REAL:}}{-@\texttt{-}!monadic}\newline
Arithmetic negation: $-a$.
\item \hxtt{OP ABS = (L REAL a)L REAL:}{ABS}\newline
The absolute value. \verb|ABS -3.0 = +3.0|
\item \hxtt{OP SIGN = (L REAL a)INT:}{SIGN} \newline
Yields $-1$ for negative operands, $+1$ for positive op\-erands and $0$ for
a zero operand ($0.0$).
\item \hxtt{OP ROUND = (REAL a)INT:}{ROUND} \newline
Rounds its operand to the nearest integer. If the value ends with $.5$,
it is rounded to the nearest even number. This is contrary to normal
Linux C~library practice, but is an internationally accepted standard
which ensures that rounding errors do not accumulate. The operator
checks for integer overflow (see section~\hyref{stan-fpov} for details).
\item \verb|OP ROUND = (L REAL a)L INT:|
(for any precision except \verb|REAL|)\newline Rounds its operand to
the nearest integer. Does not check integer overflow. If its operand
exceeds \verb|l max int|, an erroneous result will ensue.
\verb|ROUND| should be used for a \verb|REAL| operand if you want to
check for integer overflow (see section~\hyref{stan-fpov} for details
of floating-point overflow checking).
\item \hxtt{OP ENTIER = (REAL a)INT:}{ENTIER}\newline
Truncates its operand to the next lowest integer. The operator checks
for integer overflow (see section~\hyref{stan-fpov} for details).
\item \verb|OP ENTIER = (L REAL a)L INT:|
(for any precision except \verb|REAL|) \newline
Truncates its operand to the next lowest integer. The operator does not
check integer overflow. If its operand exceeds \verb|l max int|, an
erroneous result will ensue. Use \verb|ENTIER| for a \verb|REAL|
operand if you want to check for integer overflow (see
section~\hyref{stan-fpov} for details of floating-point overflow
checking).
\item \hxtt{OP LENG = (L REAL a)LONG L REAL:}{LENG}
\newline
\verb|OP LENG = (SHORT L REAL a)L REAL:| \newline
Converts its operand to the next longer precision. Note that you cannot
use both \verb|SHORT| and \verb|LONG| in the same mode.
\item \hxtt{OP SHORTEN = (L REAL a)SHORT L REAL:}{SHORTEN}
\newline
\verb|OP SHORTEN = (LONG L REAL a)L REAL:| \newline
Converts its operand to the next shorter precision. If a value exceeds
\verb|l max real| for the next shorter precision, the value will be
truncated leading to an erroneous result. The mantissa will always be
truncated.
\end{enumerate}

\subsubsection*{Dyadic operators}\hylabel{stan-oprealdy}
In this section, consistent substitution is used wherever the \verb|L|
shorthand appears. For operators with mixed operands, see
section~\hyref{stan-mixed}.
\begin{enumerate}
\item \hx{\texttt{OP + = (L REAL a,L REAL b)L REAL:}}{+@\texttt{+}!dyadic} \newline
Floating-point addition: $a+b$. Floating-point overflow will cause a
trappable signal (see section~\hyref{stan-fpov}).
\item \hx{\texttt{OP - = (L REAL a,L REAL b)L REAL:}}{-@\texttt{-}!dyadic} \newline
Floating-point subtraction: $a-b$. Floating-point overflow will cause a
signal which can be trapped (see section \hyref{stan-fpov}).
\item \hxtt{OP * = (L REAL a,L REAl b)L REAL:}{*} \newline
Floating-point multiplication: $a\times b$. Floating-point overflow
will cause a signal which can be trapped (see section \hyref{stan-fpov}).
\item \hxtt{OP / = (L REAL a,L REAL b)L REAL:}{/} \newline
Floating-point divison: $a/b$. Floating-point overflow and
divide-by-zero will cause a trappable signal (see
section \hyref{stan-fpov}). Where a result of mode \verb|L REAL| needs to
be output, but it cannot be output due to the limitations built into
the definition of the mode \verb|SIMPLOUT|, the operators \verb|LENG|
or \verb|SHORTEN| should be used.
\item \hxtt{OP +* = (L REAL a,L REAL b)L COMPL:}{+*}\newline
Converts two reals into a complex number of the same precision
(synonym \ixtt{I}).
\item \hxtt{OP < = (L REAL a,L REAL b)BOOL:}{<}\newline
Floating-point ``less than'': $a<b$ (synonym \verb|LT|).
\item \hxtt{OP <= = (L REAL a,L REAL b)BOOL:}{<=}\newline
Floating-point ``less than or equals'': $a\leq b$ (synonym \verb|LE|).
\item \hxtt{OP >= = (L REAL a,L REAL b)BOOL:}{>=}
\newline(synonym \verb|GE|)\newline
Floating-point ``greater than or equals'': $a\geq b$.
\item \hxtt{|OP > = (L REAL a,L REAL b)BOOL:}{>}\newline
Floating-point ``greater than'': $a>b$ (synonym \verb|GT|).
\item \hxtt{OP = = (L REAL a,L REAL b)BOOL:}{=}\newline
Floating-point equality: $a=b$ (synonym \verb|EQ|).
\item \hxtt{OP /= = (L REAL a,L REAL b)BOOL:}{/=}\newline
Floating-point inequality: $a\neq b$ (synonym \verb|NE|).
\end{enumerate}

\Subsection{Operators with \texttt{COMPL} operands}{stan-opcompl}
Algol 68 is one of the few programming languages which have a
built-in mode for complex numbers. It is complemented by a rich set
of operators, some of which are only available for values of mode
\verb|COMPL|. Again, consistent substitution is applicable to all
operators using the \verb|L| shorthand.

\subsubsection*{Monadic operators}\hylabel{stan-opcomplmon}
\begin{enumerate}
\item \hxtt{OP RE = (L COMPL a)L REAL:}{RE}\newline
Yields the real component: \verb|re OF a|.
\item \hxtt{OP IM = (L COMPL a)L REAL:}{IM} \newline
Yields the imaginary component: \verb|im OF a|.
\item \hxtt{OP ABS = (L COMPL a)L REAL:}{ABS} \newline
Yields $\sqrt{\mbox{\texttt{RE}}\,a^2+\,\mbox{\texttt{IM}}\,a^2}$.
\item \hxtt{OP ARG = (L COMPL a)L REAL:}{ARG} \newline
Yields the argument of the complex number.
\item \hxtt{OP CONJ = (L COMPL a)L COMPL:}{CONJ} \newline
Yields the conjugate complex number.
\item \hx{\texttt{OP + = (L COMPL a)L COMPL:}}{+@\texttt{+}!monadic} \newline
Complex identity. Does nothing.
\item \hx{\texttt{OP - = (L COMPL a)L COMPL:}}{-@\texttt{-}!monadic} \newline
Complex negation.
\item \hxtt{OP LENG = (L COMPL a)LONG L COMPL:}{LENG} \newline
\verb|OP LENG = (SHORT L COMPL a)L COMPL:| \newline
Converts its operand to the next longer precision. Note that you
cannot use both \verb|SHORT| and \verb|LONG| in the same mode.
Unfortunately, although \hx{a68toc}{a68toc!LENG@\texttt{LENG}} will
translate a program containing this operator apparently without
errors, the resulting C~file will not compile. The error produced
will be ``conversion to non-scalar type requested''. You should use
the following code instead:-
\begin{verbatim}
   (LENG RE z,LENG IM z)
\end{verbatim}
\item \hxtt{OP SHORTEN = (L COMPL a)SHORT L COMPL:}{SHORTEN} \newline
\verb|OP SHORTEN = (LONG L COMPL a)L COMPL:| \newline
Converts its operand to the next shorter precision. Note that you
cannot use both \verb|SHORT| and \verb|LONG| in the same mode.
Unfortunately, the \hx{a68toc}{a68toc!SHORTEN@\texttt{SHORTEN}}
translator will generate incorrect code (see the note for the
operator \verb|LENG|). Use the following code instead:-
\begin{verbatim}
   (SHORTEN RE z,SHORTEN IM z)
\end{verbatim}
\noindent
If either of the components of the complex number exceeds
\verb|l max real| for the next shorter precision, an erroneous result
will ensue, but no error will be generated.
\end{enumerate}

\subsubsection*{Dyadic operators}\hylabel{stan-opcompldy}
The remarks in section~\hyref{stan-oprealdy} concerning
floating-point overflow apply doubly here.
\begin{enumerate}
\item \hx{\texttt{OP + = (L COMPL a,L COMPL b)L COMPL:}}{+@\texttt{+}!dyadic}
\newline
Floating-point addition for both components.
\item \hx{\texttt{OP - = (L COMPL a,L COMPL b)L COMPL:}}{-@\texttt{-}!dyadic}
\newline
Floating-point subtraction for both components.
\item \hxtt{OP * = (L COMPL a,L COMPL b)L COMPL:}{*} \newline
Standard complex multiplication with floating-point arithmetic.
\item \hxtt{OP / = (L COMPL a,L COMPL b)L COMPL:}{/} \newline
Standard complex division with floating-point arithmetic.
\item \hxtt{OP = = (L COMPL a,L COMPL b)BOOL:}{=}\newline
Complex equality with floating-point arithmetic (synonym \verb|EQ|).
\item \hxtt{OP /= = (L COMPL a,L COMPL b)BOOL:}{/=}\newline
Complex inequality with floating-point arithmetic (synonym \verb|NE|).
\end{enumerate}

\Subsection{Operators with mixed operands}{stan-mixed}
Consistent substitution is applicable to all operators using the
\verb|L| shorthand. Additional shorthands are used as follows:-
\begin{itemize}
\item The shorthand \verb|P| stands for
\hx{+}{+@\texttt{+}!dyadic}, \verb|-|, \verb|*| or \verb|/|.
\item The shorthand \verb|R| stands for
\ixtt{<}, \ixtt{<=}, \verb|=|, \verb|/=|, \ixtt{>=}, \ixtt{>},\newline
or \verb|LT|, \verb|LE|, \verb|EQ|, \verb|NE|, \verb|GE|, \verb|GT|.
\item The shorthand \verb|E| stands for \verb|=| \verb|/=|,\newline
or \verb|EQ| or \verb|NE|.
\end{itemize}
\medskip
\begin{enumerate}
\item \verb|OP P = (L INT a,L REAL b)L REAL:|
\item \verb|OP P = (L INT a,L COMPL b)L COMPL:|
\item \verb|OP P = (L REAL a,L COMPL b)L COMPL:|
\item \verb|OP P = (L REAL a,L INT b)L REAL:|
\item \verb|OP P = (L COMPL a,L INT b)L COMPL:|
\item \verb|OP P = (L COMPL a,L REAL b)L COMPL:|
\item \verb|OP R = (L INT a,L REAL b)BOOL:|
\item \verb|OP R = (L REAL a,L INT b)BOOL:|
\item \verb|OP E = (L INT a,L COMPL b)BOOL:|
\item \verb|OP E = (L REAL a,L COMPL b)BOOL:|
\item \verb|OP E = (L COMPL a,L INT b)BOOL:|
\item \verb|OP E = (L COMPL a,L REAL b)BOOL:|
\item \hxtt{OP ** = (L REAL a,INT b)L REAL:}{**}\newline
\verb|OP ** = (L COMPL a,INT b)L COMPL:|\newline
Exponentiation: $a^b$ (synonym \texttt{UP}).
\item \hxtt{OP +* = (L INT a,L REAL b)L COMPL:}{+*}
\newline
\verb|OP +* = (L REAL a,L INT b)L COMPL:|\newline
(synonym \texttt{I})
\end{enumerate}

\Subsection{Operators with \texttt{BITS} operands}{stan-opbits}
Consistent substitution applies to all operators using the \verb|L|
shorthand.

\subsubsection*{Monadic operators}\hylabel{stan-opbitsmon}
\begin{enumerate}
\item \hxtt{OP BIN = (L INT a)L BITS:}{BIN} \newline
Mode conversion which does not change the internal value.
\item \hxtt{OP ABS = (L BITS a)L INT:}{ABS} \newline
Mode conversion which does not change the internal value.
\item \hxtt{OP NOT = (L BITS a)L BITS:}{NOT} \newline
Yields the bits obtained by inverting each bit in the operand. That is,
$0$ goes to $1$, $1$ goes to $0$.
\item \hxtt{OP LENG = (L BITS a)LONG L BITS:}{LENG} \newline
\verb|OP LENG = (SHORT L BITS a)L BITS:| \newline
Converts a bits value to the next longer precision by adding zero bits
to the more significant end. Note that you cannot use both \verb|SHORT|
and \verb|LONG| in the same mode.
\item \hxtt{OP SHORTEN = (L BITS a)SHORT L BITS:}{SHORTEN} \newline
\verb|OP SHORTEN = (LONG L BITS a)L BITS:|\newline
Truncates a bits value to a value of the next shorter precision. The
more significant bits are simply ignored.
\end{enumerate}

\subsubsection*{Dyadic operators}\hylabel{stan-opbitsdy}
\begin{enumerate}
\item \verb|OP & = (L BITS a,L BITS b)L BITS:|\newline
(synonym \ixtt{AND})\newline
The logical ``and'' of each pair of binary digits in $a$ and $b$.
\item \hxtt{OP OR = (L BITS a,L BITS b)L BITS:}{OR} \newline
The logical ``or'' of each pair of binary digits in $a$ and $b$.
\item \hxtt{OP SHL = (L BITS a,INT b)L BITS:}{SHL}\newline
The left operand shifted left by the number of bits specified by the
right operand. New bits shifted in are all zero. If the right operand is
negative, shifting is to the right (synonym \verb|UP|).
\item \hxtt{OP SHR = (L BITS a,INT b)L BITS:}{SHR}\newline
(synonym \verb|DOWN|)
\item \hxtt{OP ELEM = (INT a,L BITS b)BOOL:}{ELEM} \newline
Yields \verb|TRUE| if bit $a$ is $1$.
\item \hxtt{OP = = (L BITS a,L BITS b)BOOL:}{=} \newline
Logical equality (synonym \verb|EQ|).
\item \hxtt{OP /= = (L BITS a,L BITS b)BOOL:}{/=} \newline
Logical inequality (synonym \verb|NE|).
\item \hxtt{OP <= = (L BITS a,L BITS b)BOOL:}{<=}\newline
Yields \verb|TRUE| if each bit in the left operand implies the
corresponding bit in the right operand (synonym \verb|LE|).
\item \hxtt{OP >= = (L BITS a,L BITS b)BOOL:}{>=}\newline
Yields \verb|TRUE| if each bit in the right operand implies the
corresponding bit in the left operand (synonym \verb|GE|).
\end{enumerate}

\Subsection{Operators with \texttt{CHAR} operands}{stan-opchar}
The shorthands in section~\hyref{stan-mixed} apply here.
\begin{enumerate}
\item \hxtii{OP ABS = (CHAR a)INT:}{ABS}{CHAR} \newline
The integer equivalent of a character.
\item \hxtt{OP REPR = (INT a)CHAR:}{REPR} \newline
The reverse of \verb|ABS|. The operand should be in the range
\verb|[0:max abs char]|.
\item \hx{\texttt{OP + = (CHAR a,CHAR b)STRING:}}{+@\texttt{+}!dyadic!CHAR@\texttt{CHAR}}\newline
The character $b$ is appended to the character $a$ (concatenation).
\item \verb|OP E = (CHAR a,CHAR b)BOOL:| \newline
\hxrtii{Comparison}{=}{CHAR} \hxrtii{of}{/=}{CHAR}
\hxrtii{the}{<}{CHAR} \hxrtii{arithmetic}{<=}{CHAR}
\hxrtii{equivalents}{>}{CHAR} \hxrtii{of}{>=}{CHAR}
$a$ and $b$.
\end{enumerate}

\Subsection{Operators with \texttt{STRING} operands}{stan-opstr}
\begin{enumerate}
\item \hxtii{OP + = (STRING a,STRING b)STRING:}{+}{STRING}
\newline
String $b$ is appended to string $a$ (concatenation).
\item \verb|OP + = (CHAR a,STRING b)STRING:| \newline
String $b$ is appended to character $a$.
\item \verb|OP + = (STRING a,CHAR b)STRING:| \newline
Character $b$ is appended to string $a$.
\item \hxtii{OP * = (INT a,STRING b)STRING:}{*}{STRING}\newline
Yields $a$ lots of string $b$ concatenated.
\item \verb|OP * = (STRING a,INT b)STRING:| \newline
Yields $b$ lots of string $a$ concatenated.
\item \verb|OP * = (INT a,CHAR b)STRING:| \newline
Yields $a$ lots of character $b$ concatenated.
\item \verb|OP * = (CHAR a,INT b)STRING:| \newline
Yields $b$ lots of character $a$ concatenated.
\item \hxtii{OP < = (STRING a,STRING b)BOOL:}{<}{STRING}\newline
The absolute value of each character of $a$ is compared with the
absolute value of the corresponding character in $b$ (for the purpose
of the comparison, the lower bounds of both strings are regarded as
equal to $1$). If the strings are equal upto the end of the shorter of
the strings, then the longer string is the greater
(synonym \verb|LT|).
\item \hxtii{OP <= = (STRING a,STRING b)BOOL:}{<=}{STRING}\newline
(synonym \verb|LE|)\newline
The text for the operator \verb|<| in this section applies.
\item \hxtii{OP >= = (STRING a,STRING b)BOOL:}{>=}{STRING}\newline
(synonym \verb|GE|)\newline
The text for the operator \verb|<| in this section applies.
\item \hxtii{OP > = (STRING a,STRING b)BOOL:}{>}{STRING}
\newline (synonym \verb|GT|)\newline
The text for the operator \verb|<| in this section applies.
\item \hxtii{OP = = (STRING a,STRING b)BOOL:}{=}{STRING}\newline
If the strings differ in length, they are unequal, else they are
compared as for the operator \verb|<| in this section
(synonym \verb|EQ|).
\item \hxtii{OP /= = (STRING a,STRING b)BOOL}{/=}{STRING}\newline
(synonym \verb|NE|)\newline
If the strings differ in length, they are unequal, else they are
compared as for the operator \verb|<| in this section.
\item \verb|OP E = (STRING a,CHAR b)BOOL:| \newline
\verb|OP E = (CHAR a,STRING b)BOOL:| \newline
\hxrtii{The}{<}{STRING} \hxrtii{shorthand}{<=}{STRING}
\hxtii{E}{>}{STRING} \hxrtii{as}{>=}{STRING} described in
section~\hyref{stan-mixed} applies for these cases.
\end{enumerate}

\Subsection{Assigning operators}{stan-opass}
Consistent substitution applies to all operators containing the
\verb|L| shorthand.
\begin{enumerate}
\item \verb|+:=| (synonym \ixtt{PLUSAB}) \newline
The operator is a shorthand for \verb|a:=a+b|.\newline
\begin{tabular}{|l|l||l|}\hline
  Left operand    &Right operand &Yield\\ \hline
  \texttt{REF L INT}   &\texttt{L INT}     &\texttt{REF L INT}\\
  \texttt{REF L REAL}  &\texttt{L INT}     &\texttt{REF L REAL}\\
  \texttt{REF L COMPL} &\texttt{L INT}     &\texttt{REF L COMPL}\\
  \texttt{REF L REAL}  &\texttt{L REAL}    &\texttt{REF L REAL}\\
  \texttt{REF L COMPL} &\texttt{L REAL}    &\texttt{REF L COMPL}\\
  \texttt{REF L COMPL} &\texttt{L COMPL}   &\texttt{REF L COMPL}\\
  \texttt{REF STRING}  &\texttt{CHAR}      &\texttt{REF STRING}\\
  \texttt{REF STRING}  &\texttt{STRING}    &\texttt{REF STRING}\\ \hline
\end{tabular}
\item \verb|+=:| (synonym \ixtt{PLUSTO}) \newline
The operator is a shorthand for \verb|b:=a+b|.\newline
\begin{tabular}{|l|l||l|}\hline
  Left operand    &Right operand &Yield\\ \hline
  \texttt{STRING}      &\texttt{REF STRING}&\texttt{REF STRING}\\
  \texttt{CHAR}        &\texttt{REF STRING}&\texttt{REF STRING}\\ \hline
\end{tabular}
\newpage
\item \verb|-:=| (synonym \ixtt{MINUSAB})\newline
The operator is a shorthand for \verb|a:=a-b|.\newline
\begin{tabular}{|l|l||l|}\hline
  Left operand    &Right operand &Yield\\ \hline
  \texttt{REF L INT}   &\texttt{L INT}     &\texttt{REF L INT}\\
  \texttt{REF L REAL}  &\texttt{L INT}     &\texttt{REF L REAL}\\
  \texttt{REF L COMPL} &\texttt{L INT}     &\texttt{REF L COMPL}\\
  \texttt{REF L REAL}  &\texttt{L REAL}    &\texttt{REF L REAL}\\
  \texttt{REF L COMPL} &\texttt{L REAL}    &\texttt{REF L COMPL}\\
  \texttt{REF L COMPL} &\texttt{L COMPL}   &\texttt{REF L COMPL}\\ \hline
\end{tabular} \newline
\item \verb|*:=| (synonym \ixtt{TIMESAB})\newline
The operator is a shorthand for \verb|a:=a*b|.\newline
\begin{tabular}{|l|l||l|}\hline
  Left operand    &Right operand &Yield\\ \hline
  \texttt{REF L INT}   &\texttt{L INT}     &\texttt{REF L INT}\\
  \texttt{REF L REAL}  &\texttt{L INT}     &\texttt{REF L REAL}\\
  \texttt{REF L COMPL} &\texttt{L INT}     &\texttt{REF L COMPL}\\
  \texttt{REF L REAL}  &\texttt{L REAL}    &\texttt{REF L REAL}\\
  \texttt{REF L COMPL} &\texttt{L REAL}    &\texttt{REF L COMPL}\\
  \texttt{REF L COMPL} &\texttt{L COMPL}   &\texttt{REF L COMPL}\\
  \texttt{REF STRING}  &\texttt{INT}       &\texttt{REF L COMPL}\\ \hline
\end{tabular} \newline
\item \verb|/:=| (synonym \ixtt{DIVAB})\newline
The operator is a shorthand for \verb|a:=a/b|.\newline
\begin{tabular}{|l|l||l|}\hline
  Left operand    &Right operand &Yield\\ \hline
  \texttt{REF L REAL}  &\texttt{L INT}     &\texttt{REF L REAL}\\
  \texttt{REF L REAL}  &\texttt{L REAL}    &\texttt{REF L REAL}\\
  \texttt{REF L COMPL} &\texttt{L INT}     &\texttt{REF L COMPL}\\
  \texttt{REF L COMPL} &\texttt{L REAL}    &\texttt{REF L COMPL}\\
  \texttt{REF L COMPL} &\texttt{L COMPL}   &\texttt{REF L COMPL}\\ \hline
\end{tabular} \newline
\item \verb|OP %:= = (REF L INT a,L INT b)REF L INT:|\newline
(synonym \ixtt{OVERAB})\newline
The operator is a shorthand for \verb|a:=a%b|.
\item \verb|OP %*:= = (REF L INT a,L INT b)REF L INT:|\newline
(synonym \ixtt{MODAB})\newline
The operator is a shorthand for \verb|a:=a%*b|.
\end{enumerate}

\Subsection{Other operators}{stan-othop}
This section contains those operators which appear neither in the
Revised Report nor in the section concerning a68toc extensions
(section~\hyref{stan-a68toc}).
\begin{enumerate}
\item \hx{\texttt{OP \&* = (REAL r,INT e)REAL:}}{&*@\texttt{\&*}}\newline
Multiply \verb|r| by $2^e$. The routine does not use multiplication,
but simply increments the exponent of \verb|r| accordingly.
\item \hxtii{OP ELEM = (INT a)BITS:}{ELEM}{BITS}\newline
The operator yields a value with all bits zero except the bit specified
by the operand.
\item \hxtt{OP MIN = (L INT a,L INT b)L INT:}{MIN}
\newline
\verb|OP MIN = (L REAL a,L REAL b)L REAL:| \newline
\verb|OP MIN = (L INT a,L REAL b)L REAL:| \newline
\verb|OP MIN = (L REAL a,L INT b)L REAL:| \newline
The lesser of the two operands.
\item \hxtt{OP MAX = (L INT a,L INT b)L INT:}{MAX}
\newline
\verb|OP MAX = (L REAL a,L REAL b)L REAL:| \newline
\verb|OP MAX = (L INT a,L REAL b)L REAL:| \newline
\verb|OP MAX = (L REAL a,L INT b)L REAL:| \newline
The greater of the two operands.
\item \hxtt{OP VALID = (REAL r)BOOL:}{VALID}\newline
Whether the real value \verb|r| is a valid real in terms of the IEEE
standard.
\end{enumerate}

\Section{Standard procedures}{stan-proc}
These mainly consist of mathematical procedures. All the procedures
associated with interfacing with alien procedures appear in the a68toc
section and all the transput procedures appear in the transput section.
Procedures associated with floating-point, process and
garbage-collector control appear in section~\hyref{stan-cont}.

\Subsection{Mathematical procedures}{stan-math}
Strictly speaking, there are as many precisions of each of the
mathematical functions as there are for real numbers. However, in the
standard prelude provided with the a68toc compiler, the only extra
precision implemented is that for \verb|short real|. The \verb|L|
shorthand is used to simplify the list of procedures. All these
procedures depend on the corresponding C~library functions, so consult
the manual pages for details.
\begin{enumerate}
\item \hxtt{PROC l sqrt = (L REAL x)L REAL:}{sqrt} \newline
\hxrt{Yields}{short sqrt} the square root of $x$ provided that $x\geq0$.
\item \hxtt{PROC l exp = (L REAL x)L REAL:}{exp} \newline
\hxrt{Yields}{short exp} $e^x$ if such a value exists.
\item \hxtt{PROC l ln = (L REAL x)L REAL:}{ln} \newline
\hxrt{Yields}{short ln} the natural (or Napierian) logarithm of $x$
provided that $x>0$, otherwise the procedure fails and \verb|errno|
is set (see section~\hyref{stan-cont} for details).
\item \hxtt{PROC l log = (L REAL x)L REAL:}{log} \newline
\hxrt{Yields}{short log} the logarithm of \verb|x| to base 10.
\item \hxtt{PROC l cos = (L REAL x)L REAL:}{cos} \newline
\hxrt{Yields}{short cos} the cosine of $x$, where $x$ is in radians.
\item \hxtt{PROC l arccos = (L REAL x)L REAL:}{arccos}
\newline
\hxrt{Yields}{short arccos} the inverse cosine of $x$ as a value
between \verb|L 0| and \verb|2 l pi| inclusive. If \verb|ABS x > 1|
then the procedure yields an erroneous result, but \verb|errno| is
set (see section~\hyref{stan-cont} for details).
\item \hxtt{PROC l sin = (L REAL x)L REAL:}{sin}
\newline
\hxrt{Yields}{short sin} the sine of $x$, where $x$ is in radians.
\item \hxtt{PROC l arcsin = (L REAL x)L REAL:}{arcsin}
\newline
\hxrt{Yields}{short arcsin} the inverse sine of $x$ as a value
between \verb|L 0| and \verb|2 l pi| inclusive. If \verb|ABS x > 1|
then the procedure yields an erroneous result, but \verb|errno| is
set (see section~\hyref{stan-cont} for details).
\item \hxtt{PROC l tan = (L REAL x)L REAL:}{tan}\newline
\hxrt{Yields}{short tan} the tangent of $x$, where $x$ is in radians.
\item \hxtt{PROC l arctan = (L REAL x)L REAL:}{arctan}
\newline
\hxrt{Yields}{short arctan} the inverse tangent of $x$ as a value
between \verb|L 0| and \verb|2 l pi| inclusive.
\item \hxtt{PROC next random = (REF INT a)REAL:}{next random} \newline
The next \verb|INT| value after $a$ in a pseudo-random sequence
uniformly distributed in the range \verb|L 0| to \verb|max int| is
assigned to $a$. The yield $x$ is in the range $0\leq x<1$ obtained by
a uniform mapping of $a$.
\item \hxtt{INT last random}{last random} \newline
\hxtt{LONG INT long last random}{long last random} \newline
Names initialised to fixed values and used by other random procedures
as a seed.
\item \hxtt{PROC random = REAL:}{random} \newline
A call of \verb|next random(last random)|.
\item \hxtt{PROC short random = SHORT REAL:}{short random}\newline
As for \verb|random| with the yield shortened.
\item \hxtt{PROC l random int = (L INT n)L INT:}{random int}\newline
\hxrt{Yields}{long random int} \hxrt{a}{short random int}
\hxrt{pseudo-random}{short short random int} sequence of
integers in the range $1\leq x\leq n$.
\end{enumerate}

\Subsection{Other procedures}{stan-procoth}
The procedures \verb|whole|, \verb|fixed| and \verb|float| are
dealt with in the transput section (\hyref{stan-trans}).
\begin{enumerate}
\item \hxtt{PROC l bits pack = ([]BOOL a)L BITS:}{bits pack}\newline
\hxrt{Packs}{long bits pack} \verb|l bits width|
\hxrt{booleans}{bits pack} \hxrt{into}{short bits pack}
\hxrt{a}{short short bits pack} value of mode \verb|L BITS|.
\item \hxtt{PROC char in string =}{char in string}\newline
      \verb|       (CHAR c,REF INT i,STRING s)BOOL:|\newline
If the character $c$ occurs in the string $s$, the procedure yields
\verb|TRUE| and assigns the position of $c$ in $s$ to $i$, otherwise it
yields \verb|FALSE| when no assignment to $i$ takes place.
\end{enumerate}

\Subsection{\texttt{ALIEN} declarations}{stan-decalien}
This section contains all those values which are declared as
\verb|ALIEN| values, but which are not mentioned in the Revised
Report.
\begin{enumerate}
\item \ixtt{erange err} is one Linux system error used in the QAD
standard prelude.
\item The following integer values are used in the transput.
\begin{enumerate}
\item \hxtt{INT posix seek cur}{posix seek cur}\newline
Used in \verb|posix lseek| to specify the current position.
\item \hxtt{INT posix seek end}{posix seek end}\newline
Used in \verb|posix lseek| to specify the end of the file.
\item \hxtt{INT posix seek set}{posix seek set}\newline
Used in \verb|posix lseek| to specify a direct offset.
\end{enumerate}
\item These values are used in the manipulation of the \ixtt{kbd channel}.
\begin{enumerate}
\item \hxtt{INT termios vtime}{termios vtime}\newline
The offset in the \verb|termios| structure.
\item \hxtt{INT termios vmin}{termios vmin}\newline
The offset in the \verb|termios| structure.
\item \hxtt{INT tcsanow}{tcsanow}\newline
Used in the call to \ixtt{linux tc set attr}.
\item \hxtt{INT isig}{isig}\newline
Used in the call to \ixtt{linux tc set attr}.
\item \hxtt{INT icanon}{icanon}\newline
Used in the call to \ixtt{linux tc set attr}.
\item \hxtt{INT echo}{echo}\newline
Used in the call to \ixtt{linux tc set attr}.
\end{enumerate}
\item \ixtt{nil func ptr} has mode \verb|CPTR| and is used to provide
a \verb|NIL| pointer to C functions.
\item \ixtt{null c charptr} is equivalent to the C value \verb|NULL|.
\item \ixtt{prelude} is a dummy declaration used to access the
parameters provided to a program when called. It should not be used.
\item The signal names in Linux are given in the following table:
\begin{center}
\begin{tabular}{|l|l|l|l|}
\hline
sighup  & sigstkflt & sigint  & sigchld \\ \hline
sigquit & sigcont   & sigill  & sigstop \\ \hline
sigtrap & sigtstp   & sigabrt & sigttin \\ \hline
sigbus  & sigttou   & sigfpe  & sigurg  \\ \hline
sigkill & sigxcpu   & sigusr1 & sigxfsz \\ \hline
sigsegv & sigvtalrm & sigusr2 & sigprof \\ \hline
sigpipe & sigwinch  & sigalrm & sigio   \\ \hline
sigterm & sigpwr    &         &         \\ \hline
\end{tabular}
\end{center}
See \verb|signal(7)| for details of each signal.
\item \ixtt{errno} identifies a value of mode \verb|REF INT| and
contains the error number of the latest Linux system error.
\end{enumerate}

\Subsection{\texttt{ALIEN} routines}{stan-procalien}
All the \verb|ALIEN| routines made available by the QAD standard
prelude can be found in this section. No attempt has been made to
give details of their function: consult the ``man'' pages on your
Linux system for that (the appropriate page is given after the header
of each routine with the section of the Linux programming manual
given in parentheses). The \verb|ALIEN| declarations have been
classified by the first word of their identifiers which gives the
standard which specifies them. For example, the routine \ixtt{bsd
mkstemp} is specified by the BSD4.4 standard.

Many of the routines are used in the definition of other,
higher-level routines. They are made available so that implementers
of other preludes will not have to redefine them.

\subsubsection*{Routines conforming to Ansi C}\hylabel{stan-ansi}
\begin{enumerate}
\item \hxtt{PROC(INT)INT ansi raise}{ansi raise}\newline
See \verb|raise(3)|.
\item \hxtt{PROC(INT,CPTR)CPTR ansi signal}{ansi signal}\newline
See \verb|signal(2)|.
\item \hxtt{PROC(VECTOR[]CHARJ)REAL ansi strtod}{ansi strtod}\newline
See \verb|strtod(3)|.
\end{enumerate}

\subsubsection*{Routines conforming to BSD4.4}\hylabel{stan-bsd}
\begin{enumerate}
\item \hxtt{PROC(INT,VECTOR[]CHAR,REF INT)INT bsd accept}{bsd accept}\newline
See \verb|accept(2)|.
\item \hxtt{PROC(INT,VECTOR[]CHAR,INT)INT bsd bind}{bsd bind}\newline
See \verb|bind(2)|.
\item \hxtt{PROC(VECTOR[]CHAR,INT)INT bsd chmod}{bsd chmod}\newline
See \verb|chmod(2)|.
\item \hxtt{PROC(INT,VECTOR[]INT,INT)INT bsd connect}{bsd connect}\newline
See \verb|connect(2)|.
\item \hxtt{PROC(INT,INT)INT fchmod}{fchmod}\newline
See \verb|fchmod(2)|.
\item \hxtt{PROC(VECTOR[]CHAR)CCHARPTRPTR bsd gethostbyname}{bsd gethostbyname}\newline
See \verb|gethostbyname(3)|.
\item \hxtt{PROC(VECTOR[]CHAR,REF BITS)INT bsd inet aton}{bsd inet aton}\newline
See \verb|inet_aton(3)|.
\item \hxtt{PROC(INT)INT bsd is a tty}{bsd is a tty}\newline
See \verb|isatty(3)|.
\item \hxtt{PROC(INT,INT)INT bsd listen}{bsd listen}\newline
See \verb|listen(2)|.
\item \hxtt{PROC(VECTOR[]CHAR)INT bsd mkstemp}{mkstemp}\newline
See \verb|mkstemp(3)|.
\item \hxtt{PROC(VECTOR[]CHAR,VECTOR[]CHAR,REAL,INT,INT)INT bsd real snprintf}{bsd real snprintf}\newline
Although the underlying routine can be used for the transput of any
plain value, it is used here for the transput of a \verb|REAL| only.
See \verb|snprintf(3)|.
\item \hxtt{PROC(INT,INT)INT bsd shutdown}{bsd shutdown}\newline
See \verb|shutdown(2)|.
\item \hxtt{PROC(INT,INT,INT)INT bsd socket}{bsd socket}\newline
See \verb|socket(2)|.
\end{enumerate}

\subsubsection*{Routines conforming to an ISO standard}\hylabel{stan-iso}
\begin{enumerate}
\item \hxtt{PROC(CPTR)INT iso at exit}{iso at exit}\newline
See \verb|atexit(3)|.
\end{enumerate}

\subsubsection*{Routines peculiar to Linux}\hylabel{stan-linux}
\begin{enumerate}
\item \hxtt{PROC(CPTR,CSTR)INT linux on exit}{linux on exit}\newline
See \verb|on_exit(3)|.
\item \hxtt{PROC(INT,CCHARPTR)INT linux tc get attr}{linux tc get attr}\newline
See \verb|tcgetattr(3)|.
\item \hxtt{PROC(INT,BITS,CCHARPTR)INT linux tc set attr}{linux tc set attr}\newline
See \verb|tcsetattr(3)|.
\end{enumerate}

\subsubsection*{Routines conforming to POSIX}\hylabel{stan-posix}
\begin{enumerate}
\item \hxtt{PROC(INT)INT posix close}{posix close}\newline
See \verb|close(2)|.
\item \hxtt{PROC(VECTOR[]CHAR,INT)INT posix creat}{posix creat}\newline
See \verb|creat(2)|.
\item \hxtt{PROC(INT)VOID posix exit}{posix exit}\newline
See \verb|exit(2)|.
\item \hxtt{PROC(VECTOR[]CHAR)CSTR posix get env}{posix getenv}\newline
See \verb|getenv(3)|.
\item \hxtt{PROC INT posix getpid}{posix getpid}\newline
See \verb|getpid(2)|.
\item \hxtt{PROC(INT,INT,INT)INT posix lseek}{posix lseek}\newline
See \verb|lseek(2)|.
\item \hxtt{PROC(VECTOR[]CHAR,INT,INT)INT posix open}{posix open}\newline
See \verb|open(2)|.
\item \hxtt{PROC(INT,VECTOR[]CHAR,INT)INT posix read}{posix read}\newline
See \verb|read(2)|.
\item \hxtt{PROC(VECTOR[]CHAR,VECTOR[]CHAR)INT posix rename}{posix rename}\newline
See \verb|rename(2)|.
\item \hxtt{PROC(INT)CSTR posix strerror}{posx strerror}\newline
See \verb|strerror(3)|.
\item \hxtt{PROC(CSTR)INT posix strlen}{posix strlen}\newline
See \verb|strlen(3)|.
\item \hxtt{PROC(REF INT)INT posix time}{posix time}\newline
See \verb|time(2)|.
\item \hxtt{PROC(VECTOR[]CHAR)INT posix unlink}{posix unlink}\newline
See \verb|unlink(2)|.
\item \hxtt{PROC(INT,VECTOR[]CHAR,INT)INT posix write}{posix write}\newline
See \verb|write(2)|.
\end{enumerate}

\subsubsection*{Local routines}\hylabel{stan-local}
\begin{enumerate}
\item \hxtt{PROC(REF SHORT BITS)VOID get fpu cw}{get fpu cw}\newline
Gets the control word of the floating point unit.
\item \hxtt{PROC(SHORT BITS)VOID set fpu cw}{set fpu cw}\newline
Sets the control word of the floating point unit.
\item \hxtt{PROC(REAL,REF INT)VOID ph round}{ph round}\newline
Rounds a \verb|REAL| to an \verb|INT|.
\end{enumerate}

\Section{a68toc extensions}{stan-a68toc}
The a68toc manual describes the language restrictions of the
translator. Chapter~3 contains details of the \verb|FORALL|
construct.  This section is intended to document those extensions
used in the QAD standard prelude.

\Subsection{Modes peculiar to a68toc}{stan-ctmodes}
The principal extensions to Algol 68 modes are the introduction of
multiple modes whose housekeeping overhead is less than the standard
row modes.
\begin{enumerate}
\item \verb|STRUCT n MODE|\newline
This mode is called an \hx{``indexable structure''}{indexable
structure}.  The $n$, a non-negative integer, is built into the mode
and must be an integer denotation. The base mode can be any mode. It
is equivalent to a C~language ``array''.  Here is a list of modes
used in the QAD standard prelude which are either indexable
structures or references to such:-
\begin{enumerate}
\item \hxtt{CSTR=REF STRUCT 16000000 CHAR}{CSTR}\newline
This is a reference mode and is equivalent to the C~type \verb|char *|.
It is used in the \verb|ALIEN| (see section \hyref{dev-ALIEN})
definitions of \verb|linux getenv|, for example, to reference
data.
\item \hxtt{CCHARPTRPTR=REF STRUCT 16000000 CSTR}{CCHARPTRPTR}\newline
Again, this is a reference mode and is equivalent to the C~type
\verb|char **|. It is used to access the program's arguments.
\end{enumerate}
A considerable number of operators use indexable structures for
converting values of one mode to another using memory mapping (see
section \hyref{stan-ctbiop}.
\item \verb|VECTOR[n]MODE| \newline
The \ixtt{VECTOR} mode has less overhead than a row mode because its
lower bound is always one. It is commonly used to provide strings to
C procedures. The following modes are defined using \verb|VECTOR|:-
\begin{enumerate}
\item \hxtt{STR=VECTOR[0]CHAR}{STR}\newline
Due to the way in which C~multiples are defined (without bounds), the
mode \verb|STR| can be used for any length \verb|VECTOR|.
\item \hxtt{RVC=REF STR}{RVC}\hylabel{stan-RVC}\newline
This mode is used in a number of operators, such as
\begin{verbatim}
   OP MAKERVC = ([]CHAR s)RVC:
\end{verbatim}
\noindent
It is also used to construct other modes such as \verb|BOOK| (see
section \hyref{stan-trans}).
\end{enumerate}
\item Coercions provided by a68toc\newline
A value of mode \verb|STRUCT n MODE| can be coerced directly to a value
of mode \verb|VECTOR[]MODE|. Likewise, a value of the latter mode can
be coerced to a value of mode \verb|[]MODE|. Therefore, preferably use
the mode \verb|[]MODE| for a parameter to a procedure.
\item Other modes used by a68toc\newline
Some modes are provided to make interfacing with C~library procedures
easier. Here are the ones provided by the QAD standard prelude:-
\begin{enumerate}
\item \hxtt{CPTR=REF BITS}{CPTR}\newline
This mode is equivalent to the C~type \verb|void *|.
\item \hxtt{CINTPTR=REF INT}{CINTPTR}\newline
Equivalent to the C~type \verb|int *|.
\item \hxtt{CCHARPTR=REF CHAR}{CCHARPTR}\newline
Equivalent to the C~type \verb|char *|.
\item \hxtt{GCPARAM=STRUCT(STR name,INT value)}{GCPARAM}\newline
Used to access parameters of the garbage-coll\-ector (see section
\hyref{stan-gc} below).
\item \hxtt{PDESC=STRUCT(CPTR cp,CSTR env)}{PDESC}
\newline
This represents the structure created by a68toc for every Algol~68
procedure. The field \verb|cp| contains the actual memory address of
the procedure and the field \verb|env| contains data used by the
procedure.
\item \verb|VDESC=STRUCT(CCHARPTR data,BITS gc,|\newline
\hxtt{             INT upb)}{VDESC}\newline
This mode represents the housekeeping overhead of a \verb|VECTOR|.
The \verb|data| field is the actual memory address of the start of
the data and the \verb|upb| field is the upper bound of the vector.
The \verb|gc| field is used by the garbage-collector (the heap
manager).
\end{enumerate}
\end{enumerate}

\Subsection{a68toc constructs}{stan-ctcons}
This section describes those constructs which are either peculiar to
a68toc or which are in some way different from standard Algol~68.
\begin{enumerate}
\item \verb|FORALL|\newline
\verb|FORALL| is described in section \hyref{rep-FORALL}.

\item \ixtt{ALIEN} and \ixtt{CODE}\newline
Both \verb|ALIEN| and \verb|CODE| are described in the a68toc manual.
\verb|ALIEN| is also described in section \hyref{dev-ALIEN}. All
\verb|ALIEN| declarations used in the QAD standard prelude appear in
the file \verb|spaliens.a68| which you should consult for further
details.  You should note that the \verb|ALIEN| declarations were
established by trying various modes until a definition was found
which a68toc translated to a compilable C~source file. The
declarations for \verb|get fpu cw|, \verb|set fpu cw| and the like,
use the \verb|__asm__| construct of the GNU C~compiler: this provides
a means of incorporating short sequences of assembler instructions
into a C~program. Consult the node \verb|Extended| \verb|Asm| in info
file \verb|gcc.info| for details.

As described in the a68toc manual, source files may contain either a
\ixtt{PROGRAM} module or a \ixtt{DECS} module. The latter may contain
declarations and \verb|CODE| clauses only. See the file
\verb|transput.a68|, lines 1185--92, for an example of how to execute
code when a \verb|DECS| module is being elaborated.

\item \ixtt{USE} lists\newline
The \verb|USE| list of a \verb|DECS| or a \verb|PROGRAM| module
generates calls to the relevant initialiser \verb|PROC|s (see the
generated C~file for \verb|standard.a68| for an example) in the reverse
order of the given modules. Therefore, if the order matters, ensure
that the \verb|USE| clause mentions each module in the proper order.

\item The \ixtt{BY} construct in a \verb|FOR| loop\newline
Unfortunately, the \verb|BY| construct in a \verb|FOR| loop (see
section \hyref{rep-rep}) generates an internal compiler error if it is
followed by an integer denotation or a formula consisting of \verb|+|
followed by an integer denotation. This
is a definite bug. The remedy is to use a simple procedure or operator
or the denotation preceded by two minus symbols. For example, instead of writing
\begin{verbatim}
   FOR i BY 5 TO ...
\end{verbatim}
\noindent
use
\begin{verbatim}
   FOR i BY --5 TO ...
\end{verbatim}
\item The \hx{default case}{CASE default@\texttt{CASE} default} in a
\verb|CASE| clause\newline

If in a \verb|CASE| clause, whether a simple \verb|CASE| or a
conformity \verb|CASE| clause (one which determines the mode of the
value in its enquiry clause), the default clause can occur, then you
must include at least \verb|OUT SKIP|, otherwise you will get a
run-time fault.
\item \ixtt{BIOP 99}\newline
In Algol~68, a \verb|UNION| (see section \hyref{union-mode}) is a
well-defined mode composed of constituent modes. A value of one of the
constituent modes may be assigned to a name of the united mode and only
that value (with its original mode) can be extracted. In the
C~language, however, a ``free union'' or just ``union'' is a piece of
memory which can have different interpretations. The \verb|BIOP 99|
construct enables the operand of an operator using it to be
re-interpreted as a value of the mode given in the yield. for example,
the operator \verb|FLAT| declared as
\begin{verbatim}
   OP(REAL)STRUCT 8 CHAR FLAT = BIOP 99;
\end{verbatim}
\noindent
accepts a \verb|REAL| parameter which, as the yield, is regarded as an
indexable structure of 8 characters each of which can be accessed
separately. See section \hyref{stan-ctbiop} for operators using this
construct.
\end{enumerate}

\Subsection{Operators}{stan-ctops}
These are largely operators using the \verb|BIOP 99| construct, but
there are a number of other operators which ease the task of
interfacing with C~library procedures.
\newpage

\subsubsection*{Operators using \texttt{BIOP 99}}\hylabel{stan-ctbiop}
Most of the operators used in the QAD standard prelude which are
defined using the \verb|BIOP 99| construct are for internal use only.
In the following list, the full declaration
\begin{verbatim}
   OP(CPTR)CSTR TOCSTR = BIOP 99;
\end{verbatim}
\noindent
is abbreviated to
\begin{verbatim}
   OP TOCSTR=(CPTR)CSTR:
\end{verbatim}

Here is a list of operators using the \verb|BIOP 99| construct which
are made available by the QAD standard prelude:-
\begin{enumerate}
\item \hxtt{OP CCHARPTRTOCSTR=(CCHARPTR)CSTR:}{CCHARPTRTOCSTR}\newline
This operator is used to define the \verb|on exit| procedure.
\item \hxtt{OP CSTRTOCCHARPTR=(CSTR)CCHARPTR:}{CSTRTOCCHARPTR}\newline
This operator converts in the opposite direction.
\item \hxtt{OP TOCPTR=(INT)CPTR:}{TOCPTR}
\item \hxtt{OP TOCSTR=(CPTR)CSTR:}{TOCSTR}
\item \hxtt{TOPDESC=(PROC VOID)PDESC:}{TOPDESC}\newline
This operator provides a means of getting the address of a procedure
and is used to provide the identifier of an Algol~68 procedure which
must be elaborated by an \verb|ALIEN| procedure (such as a C~library
routine).\newline
\verb|OP TOPDESC=(PROC(INT,CPTR)VOID)PDESC:|\newline
\verb|OP TOPDESC=(PROC(INT)VOID)PDESC:|\newline
\verb|OP TOPDESC=(PROC(INT,RVC)PDESC:|\newline
You can define as many \verb|TOPDESC| operators as you wish with
operands of procedures you will need. You will certainly need more
definitions of \verb|TOPDESC| if you write wrapper procedures for X
Window System procedures which have procedural parameters. See the file
\verb|transput.a68| for details of how this operator is used.
\item \hxtt{OP TORPDESC=(REF PROC VOID)RPDESC:}{RPDESC}\newline
This operator converts a reference to a \verb|PROC VOID| to a
reference to a value of mode \verb|PDESC|.
\item \hxtt{OP TOVDESC=(STR)VDESC:}{TOVDESC}\newline
This operator provides a means of getting the address of the
\verb|STR| in a form suitable as a parameter to C~library routines.
\item \hxtt{OP TOVBDESC=(VECTOR[]BITS)VBDESC}{TOVBDESC}\newline
This operator provides a means of getting the address of the
\verb|VECTOR[]BITS|.
\item \hxtt{OP TOVIDESC=(VECTOR[]INT)VIDESC}{TOVIDESC}\newline
Exactly as for the previous routine, but for a \verb|VECTOR[]INT|.
\end{enumerate}

\subsubsection*{Other operators}\hylabel{stan-ctothops}
Here is a list of operators not using the \verb|BIOP 99|
construct:-
\begin{enumerate}
\item \hxtt{OP CPTRTORVC=(CPTR)RVC:}{CPTRTORVC}\newline
Used to cast the C~type \verb|void *| to the type \verb|char *|.
\item \hxtt{OP CSTRTORVC=(CSTR)RVC:}{CSTRTORVC}\newline
Converts a C string to an \verb|RVC| using the standard
RS Algol 68 coercion
\begin{verbatim}
   REF STRUCT n CHAR => REF VECTOR[]CHAR
\end{verbatim}
\noindent
It is mainly used to access C strings yielded by C library routines.
The parameter must be terminated by a \verb|null character|.
\item \hxtt{OP FLATRVB=(RVC)BITS:}{FLATRVB}\newline
Converts a \verb|VECTOR[4]CHAR| into a \verb|BITS|.
\item \hxtt{OP FLATRVLB=(RVC)LONG BITS:}{FLATRVLB}\newline
Converts a \verb|VECTOR[8]CHAR| into a \verb|LONG BITS|.
\item \hxtt{OP FLATRVSB=(RVC)SHORT BITS:}{FLATRVSB}\newline
Converts a \verb|VECTOR[2]CHAR| into a \verb|SHORT BITS|.
\item \hxtt{OP FLATRVSSB=(RVC)SHORT SHORT BITS:}{FLATRVSSB}\newline
Converts a \verb|BECTOR[1]CHAR| into a \verb|SHORT SHORT BITS|.
\item \hxtt{OP FLATRVR=(RVC)REAL:}{FLATRVR}\newline
Converts a \verb|VECTOR[8]CHAR| into a \verb|REAL|.
\item \hxtt{OP FLATRVSR=(RVC)SHORT REAL:}{FLATRVSR}\newline
Converts a \verb|VECTOR[4]CHAR| into a \verb|SHORT REAL|.
\item \hxtt{OP MAKERVC=(CHAR)RVC:}{MAKERVC}\newline
\verb|OP MAKERVC=(STR)RVC:|\newline
\verb|OP MAKERVC=([]CHAR)RVC:|
\item \hxtt{OP VBTOCPTR=(VECTOR[]BITS)CPTR:}{VBTOCPTR}\newline
This operator gets the address of the \verb|VECTOR[]BITS| in a form
suitable as a parameter to a C~library routine.
\item \hxtt{OP VCTOCHARPTR=(STR)CCHARPTR:}{VCTOCHARPTR}\newline
Yields the C pointer from a a68toc descriptor. If a C string is
expected, a \verb|null character| must be appended to the data before
the routine is called. This need not be done for string denotations.
This routine may be used to yield a C~pointer from an \verb|RVC|, as
the C~representation is the same.
\item \hxtt{OP VITOINTPTR=(VECTOR[]INT)CINTPTR:}{VITOINTPTR}\newline
Yields the address of the \verb|VECTOR[]INT| in a form suitable for
use as a parameter of a C~library routine.
\item \hxtt{OP STRTOCSTR=(STR)CSTR:}{STRTOCSTR}\newline
The operator combines the action of the operators
\texttt{CCHAR\-PTR\-TO\-CSTR} and \verb|VCTOCHARPTR|.
\item \hxtt{OP Z=(STR str)STR:}{Z}\newline
Yields a null-terminated \verb|STR| from a \verb|STR| for use with the
C~library.
\end{enumerate}

\Section{Control routines}{stan-cont}
Three groups of procedures and operators are provided to control
various aspects of the run-time environment. These are floating-point
control, process termination control and garbage-collector control.

\Subsection{Floating-point unit control}{stan-fpov}
The Intel Pentium microprocessors all have a floating-point unit (FPU)
as an integral part of the microprocessor. The action of the FPU is
determined by the contents of a 16-bit register called the ``control
word register''. Details of the register can be found in the file
\begin{verbatim}
   /usr/include/fpu_control.h
\end{verbatim}
\noindent
Details of the working of the FPU, as controlled by the control word
register can be found in the three volumes of the ``Intel Architecture
Software Developer's Manual''. The control word contains bits
which control rounding, precision and whether floating-point errors
should cause an exception. The QAD standard prelude provides two
procedures which enable you to get and set the control word register:-
\begin{enumerate}
\item \hxtt{PROC get fpu cw = (REF SHORT BITS cw)VOID:}{get fpu cw}\newline
After calling \verb|get fpu cw|, the current value of the FPU control
word will be assigned to the parameter.
\item \hxtt{PROC set fpu cw = (SHORT BITS cw)VOID:}{set fpu cw}\newline
After calling \verb|set fpu cw|, the current value of the FPU control
word will have been set to the value of the parameter.
\end{enumerate}
The QAD standard prelude provides three values of mode \verb|SHORT|
\verb|BITS| which enable you to control how rounding is performed.
They are:-
\begin{enumerate}
\item \ixtt{fpu cw ieee}\newline
This value enables you to reset the FPU control word to the standard
value for the C~library.
\item \ixtt{fpu cw algol68 round}\newline
This value ensures that the FPU will perform rounding to the nearest
number. A \verb|REAL| value ending in $0\cdot5$ will be rounded to the
nearest even number. This ensures that rounding errors in random values
will not accumulate.
\item \ixtt{fpu cw algol68 entier}\newline
This value ensures that the FPU will truncate \verb|REAL| numbers
towards $-\infty$ when converting to an integer of the equivalent
precision.
\end{enumerate}
These values are used as masks. Here, for example, is the source code
for the operator \verb|ROUND|:-
\begin{verbatim}
   OP ROUND = (REAL r)INT:
   (
      INT n;
      SHORT BITS ocw;  get fpu cw(ocw);
      set fpu cw(ocw & fpu cw algol68 round);
      ph round(r,n);
      set fpu cw(ocw);
      n
   )
\end{verbatim}
\noindent
Notice how the FPU control word is reset to its original value before
the end of the operator.

The FPU control word is also used to control whether overflow should
be detectable. The standard mode of operation is to ignore integer
overflow. The control word masks mentioned above ensure that integer
overflow can be detected using a \bfix{signal}. The procedure
\hxtt{on signal}{on signal} is declared as follows:-
\begin{verbatim}
   PROC on signal=(INT sig,
                   PROC(INT)VOID handler)VOID:
\end{verbatim}
\noindent
The example program \verb|testov.a68| shows how \verb|on signal| can be
used. The Algol~68 identifiers for all the Linux signals are the same
as the Linux signal identifiers, but in lower case. For example, the
signal used in FPU control is \verb|sigfpu|. The signal generated by
keying \verb|Ctrl-C| (sometimes depicted as \verb|^C|) on program input
is \verb|sigint|. Here is a short program which illustrates signal
trapping:-
\begin{verbatim}
   PROGRAM sig CONTEXT VOID
   USE standard
   BEGIN
      on signal(sigint,
                (INT sig)VOID:
                 (write(("sigint trapped",
                         newline));
                  exit(1)));
      write("Please key ^C: "); read(LOC CHAR);
      write(("No signal trapped",newline))
   END
   FINISH
\end{verbatim}
\noindent
Usually, when you trap a signal such as \ixtt{sigint}, your program
could close down processing in an orderly manner: files could be closed
properly, a message to the user could be issued, and so on. You can do
anything you want in the procedure provided as a parameter to
\verb|on signal|. You can also predeclare the procedure and simply
provide its identifier in the \verb|on signal| call.

Integer overflow is ignored by the microprocessor. So the
formula \verb|max int + 3| simply yields an incorrect value.

The procedure \ixtt{ansi raise} will cause any specified signal to
occur. Here is the mode of \ixtt{ansi raise}:
\begin{verbatim}
   PROC ansi raise = (INT sig)INT:
\end{verbatim}

\Subsection{Terminating a process}{stan-prcont}
As well as raising and trapping signals, it is sometimes useful to
specify procedures to be elaborated when your program has finished, for
whatever reason. Four procedures are provided for process
termination:-
\begin{enumerate}
\item \hxtt{PROC iso at exit=(PROC VOID p)INT:}{iso at exit}\newline
The procedure \verb|p| is registered to be elaborated when the program
terminates normally or when the procedure \verb|exit| (see procedure
\hyref{stan-pcexit}) is called. Registered procedures are elaborated in
the reverse order of being registered, so that the procedure registered
last is elaborated first. The procedure \verb|at exit| yields $0$ for
success, $-1$ for an error.
\item
\hxtt{PROC on exit=(PROC(INT,CPTR)VOID p,}{on exit}\newline
\verb|              []CHAR arg)INT:|\hylabel{stan-pconexit}\newline
Unlike the procedure \ixtt{at exit} (see above),
\verb|on exit| allows you to register a procedure which takes two
parameters. The first is the integer parameter given to the
\ixtt{exit}
procedure (or $0$ for normal termination) and the second is a
\verb|[]CHAR| which the procedure \verb|p| can use. \ixtt{on exit}
yields $0$ for success and $-1$ for an error.
\item \hxtt{PROC exit = (INT status)VOID:}{exit}\newline
\hylabel{stan-pcexit}
This procedure terminates the program immediately. Any procedures
registered using \ixtt{at exit} or \ixtt{on exit} will be elaborated in
the reverse order of registration. Any open files will be closed, but
Algol~68 buffers will not be flushed. The value of \verb|status| will
be passed to the parent process of the program.
\item \hxtt{PROC stop = VOID:}{stop}\newline\hylabel{stan-stop}
This is a synonym for \verb|exit(0)|.
\end{enumerate}
The example program \verb|testexit.a68| shows one way in which
\verb|at exit| and \verb|on exit| may be used.

\Subsection{Garbage-collector control}{stan-gc}
The garbage-collector manages the run-time heap. The term ``garbage''
is used to designate memory on the heap which is no longer
referenced. Although the garbage-collector is usually called whenever
space on the heap is required, a number of routines are provided for
explicit garbage collection or for fine control of the
garbage-collector.
\begin{enumerate}
\item \hxtt{PROC garbage\_collect = VOID:}{garbage\_collect}\newline
The garbage-collector can be called explicitly by an Algol~68 program
using this procedure.
\item \hxtt{PROC disable\_garbage\_collector = VOID:}{disable\_garbage\_collector}\newline
Disables the garbage-collector.
\item \hxtt{PROC enable\_garbage\_collector = VOID:}{enable\_garbage\_collector}\newline
Enables the garbage-collector.
\item \hxtt{PROC gc\_param = (VECTOR[]CHAR cmd,INT v)INT:}{gc\_param}\newline
This routine is used to set or get the values of a number of
parameters which control the garbage-collector. The \verb|cmd| should
consist of \verb*|GET | or \verb*|SET | followed by the string
identifying the required parameter followed by a \verb|nul ch|.
The available strings are
\begin{enumerate}
\item \ixtt{COLLECTION THRESHOLD} The number of bytes in use before
the next garbage collection is allowed.
\item \ixtt{HEAP INCREMENT} The number of bytes by which the heap
should be increased in size whenever the heap is grown.
\item \ixtt{MAX HEAP SIZE} The maximum size of the heap in bytes.
\item \ixtt{MIN HEAP SIZE} The minimum size of the heap in bytes.
\item \ixtt{MAX SEGMENT SIZE} The maximum size of a memory segment
acquired for the heap.
\item \ixtt{MIN SEGMENT SIZE} The minimum size of a memory segment
acquired for the heap.
\item \ixtt{POLICY} The heap policy. Three values are provided for
setting the heap policy:-
\begin{enumerate}
\item \hxtt{INT always collect}{always collect}\newline
The garbage-collector will always be called if space is required.
\item \hxtt{INT always grow heap}{always grow heap}\newline
The garbage-collector will never be called even if space is required.
\item \hxtt{INT default policy}{default policy}\newline
The garbage-collector will be called if there is insufficient space
in the heap for the memory required. Extra space will be acquired if
garbage-collection does not yield the spaced needed.
\end{enumerate}
Whether the heap is grown or whether garbage-collection takes place
depends on the current policy which is usually specified by the
environment string \ixtt{A68\_GC\_POLICY}.
\end{enumerate}
\item \hxtt{PROC get\_gc\_param = (VECTOR[]CHAR param)INT:}{get\_gc\_param}\newline
Gets the current value of the garbage-collector parameter (any one of
the strings given in this section).
\item \hxtt{PROC set\_gc\_params = (VECTOR[]GCPARAM gcpar)VOID:}{set\_gc\_params}\newline
Sets the value of the garbage-collector parameter (any one of the
strings given in this section).
\end{enumerate}

For further details about the garbage-collector, consult the code in
the \verb|library| directory in the a68toc source tree.

\Section{Transput}{stan-trans}
If you are unclear about the working of Algol~68 transput, consult
chapter~9. The function of this section is to document all the transput
declarations so that you can use it as a reference manual.

The declarations will be covered in the following order:-
\begin{enumerate}
\item Modes
\item Standard channels
\item Standard files
\item Opening files
\item Closing files
\item Transput routines
\item Interrogating files
\item File properties
\item Event routines
\item Conversion routines
\item Layout routines
\end{enumerate}
In the sequel, transput errors are mentioned using identifiers whose
values appear in the following table:-
$$\vbox{\begin{tabular}{|l|l|}\hline
  \ixtt{physical file end not mended}&255\\
  \ixtt{logical file end not mended}&254\\
  \ixtt{stand in redirected}&253\\
  \ixtt{environment string unset}&252\\
  \ixtt{environment string estab err}&251\\
  \ixtt{estab invalid parameters}&250\\
  \ixtt{open invalid parameters}&249\\
  \ixtt{no program args}&248\\
  \ixtt{value error not mended}&247\\ \hline
  \end{tabular}}
$$
\begin{center}
Identifiers for transput errors
\end{center}

\Subsection{Transput modes}{stan-trmodes}
Only five modes are available:-
\begin{description}
\item[\ixtt{FILE}] A structure containing details of a a book
accessed by the program.
\item[\ixtt{CHANNEL}] A structure containing procedures for
accessing books.
\item[\ixtt{SIMPLIN}] A union of names of all plain modes, rows of
plain modes, structures of plain modes and their combinations.
\item[\ixtt{SIMPLOUT}] A union of all plain modes, rows of plain
modes, structures of plain modes and their combinations.
\item[\ixtt{BUFFER}] A synonym for \ixtt{RVC}. It is used as the
yield of the procedure \ixtt{file buffer} (see section
\hyref{stan-trinterr}).
\end{description}
The mode \verb|NUMBER| is used as a parameter of the procedures
\verb|whole|, \verb|fixed| and \verb|float|, but because it is the
union of all number modes, it is unnecessary to specify it and so has
not been made available for general use.

\Subsection{Standard channels}{stan-trchan}
For each channel in this section, the general properties are first
given, followed by a table giving the properties of books opened on the
channel and then a list of specific properties for the following
procedures where applicable:-
$$\begin{tabular}{|l|}\hline
  establish\\
  open\\
  create\\
  close\\
  lock\\
  scratch\\
  set\\
  logical end\\
  reidf\\ \hline
  \end{tabular}
$$
\begin{enumerate}
\item \hxtt{CHANNEL stand in channel}{stand in channel}\newline
\hxtt{CHANNEL stand out channel}{stand out channel}\newline
\hxtt{CHANNEL stand back channel}{stand back channel}\newline
These three channels have similar properties because they use the
same access procedures. The standard buffered input channel is
\verb|stand in channel|. Books on this channel have the following
properties:-
$$\begin{tabular}{|l|c|}\hline
  \multicolumn{2}{|c|}{stand in channel}\\ \hline
  bin possible & \texttt{TRUE}\\
  put possible & \texttt{FALSE}\\
  get possible & \texttt{TRUE}\\
  set possible & \texttt{TRUE}\\
  reidf possible & \texttt{FALSE}\\ \hline
  \end{tabular}
$$
and are available as the \ixtt{stand in book}.

The \verb|stand out channel| is the standard buffered output channel.
Books on this channel have the following properties:-
$$\begin{tabular}{|l|c|}\hline
  \multicolumn{2}{|c|}{stand out channel}\\ \hline
  bin possible & \texttt{TRUE}\\
  put possible & \texttt{TRUE}\\
  get possible & \texttt{FALSE}\\
  set possible & \texttt{TRUE}\\
  reidf possible & \texttt{FALSE}\\ \hline
  \end{tabular}
$$
and are available as the \ixtt{stand out book}.

The \verb|stand back channel| is the standard buffered input/output
channel. Books on this channel have the following properties:-
$$\begin{tabular}{|l|c|}\hline
  \multicolumn{2}{|c|}{stand back channel}\\ \hline
  bin possible & \texttt{TRUE}\\
  put possible & \texttt{TRUE}\\
  get possible & \texttt{TRUE}\\
  set possible & \texttt{TRUE}\\
  reidf possible & \texttt{TRUE}\\ \hline
  \end{tabular}
$$
and are available as the \ixtt{stand back book}.

The channels have the following properties:-
\begin{description}
\item[\ixtt{establish}] You must have write access to the file. If it already
exists, it will be truncated to zero length. The default mode is
\verb|8r644|. If the file cannot be established, the routine will return
the value \verb|errno| (the system error name) refers to.
\item[\ixtt{open}] the file will be opened with a default mode of \verb|8r444|.
If the file cannot be opened, the routine will return the value
\verb|errno| refers to.
\item[\ixtt{create}] A zero length file with a unique identification will be
created using the default mode of \verb|8r644|.
\item[\ixtt{close}] The file will be closed. For the
\ixtt{stand out channel} and the \ixtt{stand back channel}, the
buffer will be flushed.
\item[\ixtt{lock}] The file will be closed and then all permissions will be
removed from the file provided you have write access to the directory
containing the file.
\item[\ixtt{scratch}] The file will be closed and then unlinked.
\item[\ixtt{set}] The current position will be set to any legal position in
the book (non-negative positions only). If the position is set beyond
the current logical end, a sparse file will be created.
\item[\ixtt{logical end}] The position will be set to just beyond the last
byte in the file.
\item[\ixtt{reidf}] For the \ixtt{stand back channel} only. When
the file is closed, it will be renamed. If the rename fails (an
already existing file with that name, for example), an error message
will be output on the \ixtt{stand error} file giving a description of
the error and identifying the file.
\end{description}

\item \hxtt{CHANNEL arg channel}{arg channel}\newline
The \ixtt{arg channel} gives access to the program arguments including the actual
call of the program which precedes the program arguments. Arguments are
separated by a single space. A name of mode \verb|REF FILE| opened with
this channel has \ixtt{blank} as the string terminator. The arguments,
as a book, have the following properties:-
$$\begin{tabular}{|l|l|}\hline
\multicolumn{2}{|c|}{\texttt{arg channel}}\\ \hline
  bin possible & \texttt{FALSE} \\
  put possible & \texttt{FALSE} \\
  get possible & \texttt{TRUE} \\
  set possible & \texttt{TRUE} \\
  reidf possible & \texttt{FALSE}\\ \hline
  \end{tabular}
$$
The channel has the following properties:-
\begin{description}
\item[\ixtt{establish}] Same as \verb|open|.
\item[\ixtt{open}] The program arguments will be made available. If the
arguments are unavailable, the routine will return
\verb|no program args|.
\item[\ixtt{create}] Same as \verb|open|.
\item[\ixtt{close}] No action.
\item[\ixtt{lock}] No action.
\item[\ixtt{scratch}] No action.
\item[\ixtt{set}] Provided that the required position lies between the
beginning and the end of the arguments, the position will be set
accordingly.
\item[\ixtt{logical end}] The position will be set to just beyond the last
character of the last argument.
\item[\ixtt{reidf}] Inapplicable.
\end{description}
The procedure \verb|make term| can be used to set the string delimiter
to any required value to facilitate searching for quote-delimited or
otherwise delimited arguments.

\item \hxtt{CHANNEL env channel}{env channel}\newline
The \ixtt{env channel} gives read-only access to environment strings (referred to
in Linux documentation as ``environment variables''). The environment
string, as a book, has the following properties:-
$$\begin{tabular}{|l|l|}\hline
  \multicolumn{2}{|c|}{\texttt{env channel}}\\ \hline
  bin possible & \texttt{FALSE} \\
  put possible & \texttt{FALSE} \\
  get possible & \texttt{TRUE} \\
  set possible & \texttt{TRUE} \\
  reidf possible & \texttt{FALSE}\\ \hline
  \end{tabular}
$$
The channel has the following properties:-
\begin{description}
\item[\ixtt{establish}] Yields an error of value\newline
\ixtt{environment string estab err}.
\item[\ixtt{open}] If the environment string is the null string or
is unset, \verb|open| yields an error of value\newline
\ixtt{environment string unset}.  Otherwise, the string is available
as a book.
\item[\ixtt{create}] Yields an error of value\newline
\ixtt{environment string estab err}.
\item[\ixtt{close}] No action.
\item[\ixtt{lock}] The routine will attempt to remove all
permissions from a file of the same identification as the environment
string identification.
\item[\ixtt{scratch}] The routine will attempt to unlink a file of
the given identification.
\item[\ixtt{set}] Provided that the required position lies between the
beginning and the end of the string, the position will be set
accordingly.
\item[\ixtt{logical end}] The position will be set to just beyond
the last character.
\item[\ixtt{reidf}] Inapplicable.
\end{description}
The default string terminator is \ixtt{nul ch}. You should set the string
terminator using \ixtt{make term}.

\item \hxtt{CHANNEL kbd channel}{kbd channel}\newline
The \ixtt{kbd channel} provides access to unechoed keystrokes (also referred
to as ``non-canonical input''). Be warned that if a file opened with
this channel is not closed and the program ends prematurely, none of
your keystrokes will be echoed! You can reset to canonical input using
the command
\begin{verbatim}
   stty sane
\end{verbatim}
\noindent
The keyboard is made available as a book having the following
properties:-
$$\begin{tabular}{|l|l|}\hline
  \multicolumn{2}{|c|}{\texttt{kbd channel}}\\ \hline
  bin possible & \texttt{TRUE} \\
  put possible & \texttt{FALSE} \\
  get possible & \texttt{TRUE} \\
  set possible & \texttt{FALSE} \\
  reidf possible & \texttt{FALSE}\\ \hline
  \end{tabular}
$$
The channel is usually used to access the characters input by control
and function keys as well as normal keystrokes, so it is advisable to
use \verb|get bin| rather than \verb|get|. The channel has the
following properties:-
\begin{description}
\item[\ixtt{establish}] Same as \verb|open|.
\item[\ixtt{open}] You should use the null string \verb|""| for the
identification. The routine checks to see whether stand in has been
redirected and yields the error \ixtt{stand in redirected} if so.
Otherwise, the characteristics of \verb|stand in| are changed to wait
for a single character with no minimum waiting time and with no echo
of the input.
\item[\ixtt{create}] Same as \verb|open|.
\item[\ixtt{close}] The routine resets \ixtt{stand in} to the
condition it was in previously.
\item[\ixtt{lock}] Same as \verb|close|.
\item[\ixtt{scratch}] Same as \verb|close|.
\item[\ixtt{set}] Inapplicable.
\item[\ixtt{logical end}] Inapplicable.
\item[\ixtt{reidf}] Inapplicable.
\end{description}

\item \hxtt{CHANNEL mem channel}{mem channel}\hylabel{stan-memch}\newline
The \ixtt{mem channel} provides a memory buffer with access to all transput
facilities. It is similar to the standard Algol~68 \verb|associate|
except that binary transput is also allowed. The buffer behaves as a
book with the following properties:-
$$\begin{tabular}{|l|l|}\hline
  \multicolumn{2}{|c|}{\texttt{mem channel}}\\ \hline
  bin possible & \texttt{TRUE} \\
  put possible & \texttt{TRUE} \\
  get possible & \texttt{TRUE} \\
  set possible & \texttt{TRUE} \\
  reidf possible & \texttt{FALSE}\\ \hline
  \end{tabular}
$$
The channel has the following properties:-
\begin{description}
\item[\ixtt{establish}] If the values of \verb|p| and \verb|l| are both
\verb|1| and the value of \verb|c| is a positive integer then \verb|c|
is taken to be the size of the buffer. Otherwise, the routine yields
\ixtt{estab invalid parameters} as error value. The identification
should be \verb|""|.
\item[\ixtt{open}] The routine should be called with an identification of mode
\verb|RVC| (see section \hyref{stan-RVC}). The identification will be
used as the memory buffer.
\item[\ixtt{create}] The value \ixtt{estab invalid parameters} will be
returned.
\item[\ixtt{close}] No action.
\item[\ixtt{lock}] Inapplicable.
\item[\ixtt{scratch}] Inapplicable.
\item[\ixtt{set}] Provided the position lies in or just beyond the end of the
buffer, the position will be set.
\item[\ixtt{logical end}] The position will be set to just beyond the end of
the buffer.
\item[\ixtt{reidf}] Inapplicable.
\end{description}
The channel can be used to access individual characters of integers and
reals. \ixtt{make term} can also be used.

\item \hxtt{CHANNEL client socket channel}{client socket channel}\newline
\hylabel{stan-sockch}\hxtt{CHANNEL server socket channel}{server socket channel}\newline
These channels provide UNIX- or Internet-type sockets in the form
of standard Algol~68 files.
Sockets behave as books with the following properties:-
$$\begin{tabular}{|l|l|}\hline
  \multicolumn{2}{|c|}{\texttt{client/server socket channel}}\\ \hline
  bin possible & \texttt{TRUE}\\
  put possible & \texttt{TRUE}\\
  get possible & \texttt{TRUE}\\
  set possible & \texttt{FALSE}\\
  reidf possible & \texttt{FALSE}\\ \hline
  \end{tabular}
$$
The channels have the following properties:-
\begin{description}
\item[\ixtt{establish}] The \verb|p| should be the family of socket (either
\ixtt{af unix} or \ixtt{af inet}). If the latter, the \verb|l| should
be the port. The \verb|c| should be the MTU (maximum transport unit).
This governs the size of the buffer associated with the socket. No
checks are performed on its value. If \verb|p| is neither \verb|af unix|
nor \verb|af inet|, the routine returns \ixtt{estab invalid
parameters} as error value. The server socket should be established
before the client socket.
\item[\ixtt{open}] Yields an error of \verb|open invalid parameters| for
both channels.
\item[\ixtt{create}] Inapplicable.
\item[\ixtt{close}] The buffer is flushed and the socket closed.
\item[\ixtt{lock}]  The buffer is flushed, the socket closed and then all
access permissions removed (provided that write access is available to
the directory containing the socket).
\item[\ixtt{scratch}] The buffer is flushed, the socket closed and then
unlinked.
\item[\ixtt{set}] Inapplicable.
\item[\ixtt{logical end}] Inapplicable.
\item[\ixtt{reidf}] Inapplicable.
\end{description}
An extra procedure \ixtt{accept}, which mirrors the C~library procedure,
has the following header
\begin{verbatim}
   PROC accept = (REF FILE socket)REF FILE:
\end{verbatim}
\noindent
and is used by the server to accept a \ixtt{client socket}, thereby
yielding a \verb|REF FILE| name which can be used to communicate with
the client.

The example programs \verb|client1|, \verb|server1|, \verb|client2|
and \verb|server2| (whose source can be found in the \verb|examples|
directory) demonstrate simple use of sockets.
\end{enumerate}

\Subsection{Standard files}{stan-trfiles}
Four standard files are provided:-
\begin{enumerate}
\item \hxtt{REF FILE stand in}{stand in}\newline
This file corresponds to the C~\verb|stdin|. Books connected via
\verb|stand in| differ from those connected via the channel
\ixtt{stand in channel}: \ixtt{set possible} returns \verb|FALSE|. Thus
this file must be regarded as a simple stream of bytes. When the
\ixtt{kbd channel} is being used, \ixtt{stand in} is unavailable.
\item \hxtt{REF FILE stand out}{stand out}\newline
This file corresponds to the C~\verb|stdout|. Books connected via
\verb|stand out| differ from those connected via the channel
\ixtt{stand out channel}: \ixtt{set possible} returns \verb|FALSE|.
Thus this file must be regarded as a simple stream of bytes.
\item \hxtt{REF FILE stand error}{stand error}\newline
This file corresponds to the C~\verb|stderr| and behaves like the file
\verb|stand out|.
\item \hxtt{REF FILE stand back}{stand back}\newline
This file accesses a workbook which is deleted on termination of the
program. All kinds of transput are allowed on this file.
\end{enumerate}

\Subsection{Opening files}{stan-tropen}
Three procedures are available for opening files:-
\begin{enumerate}
\item \hxtt{PROC establish=}{establish}\newline
      \verb|   (REF FILE f,STRING idf,|\newline
      \verb|    CHANNEL chann,INT p,l,c)INT:|\newline
In standard Algol~68, this procedure creates a new file with \verb|p|
pages, each page containing \verb|l| lines, each line containing
\verb|c| characters. In the QAD standard prelude, the
\ixtt{mem channel} (see section \hyref{stan-memch}) takes notice of
\verb|p|, \verb|l| and \verb|c| and both \verb|p| and \verb|l| must
be \verb|1|. The socket channels (see section
\hyref{stan-sockch}) use \verb|p| for the socket family, \verb|l| for
the port if the family is \verb|af inet| and \verb|c| for the size of
the MTU. For other channels, the values of \verb|p|, \verb|l| and
\verb|c| are ignored. The procedure yields zero on success, otherwise
an integer denoting an error (see section \hyref{stan-trchan}).
\item \hxtt{PROC open=(REF FILE f,}{open}\newline
      \verb|           UNION(CHAR,STRING,RVC) idf,|\newline
      \verb|           CHANNEL chann)INT|\newline
In standard Algol~68, the second parameter of this procedure has mode
\verb|STRING|. The above union ensures that an \ixtt{RVC} parameter can
be used to open an existing memory buffer with the memory channel. This
is particularly useful for performing transput on buffers obtained from
C~library routines. The procedure yields zero on success, otherwise an
integer denoting an error (see section \hyref{stan-trchan}).
\item \verb|PROC create=(REF FILE f,CHANNEL chann)INT:|\newline
Creates a work file with a unique identification in the directory
\verb|/tmp| using the given channel.
\end{enumerate}

\pagebreak
\Subsection{Closing files}{stan-trclose}
Three procedures are provided:-
\begin{enumerate}
\item \verb|PROC close=(REF FILE f)VOID:|\newline
This is the standard procedure for closing a file. It is standard
practice to close every opened file. The procedure checks to see
whether the file is open. If the \verb|reidf| procedure has been
called, after closing the file, the procedure renames the file to the
identification given in the \verb|reidf| field.
\item \verb|PROC lock=(REF FILE f)VOID:|\newline
The Algol~68 Revised Report requires that the file be closed in such a
manner that some system action is required before it can be reopened.
In this case, the file is closed and then all access permissions
removed. Before the file can be reopened, the user will have to use
\verb|chmod|.
\item \verb|PROC scratch=(REF FILE f)VOID:|\newline
The file is closed and then unlinked.
\end{enumerate}

\Subsection{Transput routines}{stan-trtran}
The procedures in this section are responsible for the trans\-put of
actual values. Firstly, formatless transput is covered and then binary
transput. The \ix{a68toc} compiler does not support formatted
transput. In each section, the shorthand \verb|L| is used for the
various precisions of numbers and bits values.

\subsubsection*{Straightening}\hylabel{stan-straight}
The term \bfix{straightening} is used in Algol~68 to mean the process
whereby a complex mode is separated into its constituent modes. For
example, the mode
\begin{verbatim}
   MODE X=STRUCT(INT a,
                 CHAR b,
                 UNION(REAL,VOID) u)
\end{verbatim}
\noindent
would be straightened into values of the following modes:-
\begin{itemize}
\item \verb|INT|
\item \verb|CHAR|
\item \verb|UNION(REAL,VOID)|
\end{itemize}
The mode \verb|REF[]X| would be straightened into a number of values
each having the mode \verb|REF X|, and then each such value would be
further straightened into values having the modes
\begin{itemize}
\item \verb|REF INT|
\item \verb|REF CHAR|
\item \verb|REF UNION(REAL,VOID)|
\end{itemize}
However, a value of mode \verb|COMPL| is not straightened into two
values both of mode \verb|REAL|. Instead, the real part is transput,
then an \verb|"I"| read or written followed by the imaginary part.

\subsubsection*{Formatless transput}\hylabel{stan-fmtless}
Formatless transput converts internal values into strings of
characters or strings of characters into internal values.
\begin{enumerate}
\item \hxtt{PROC write=([]UNION(SIMPLOUT,}{write}\newline
\verb|            PROC(REF FILE)VOID) x)VOID:|\newline
This is equivalent to \verb|put(stand out,x)| (synonym \ixtt{print}).
\item \hxtt{PROC put=(REF FILE f,}{put}\newline
\verb|          []UNION(SIMPLOUT,|\newline
\verb|          PROC(REF FILE)VOID) x)VOID:|\newline
The parameter \verb|x| is straightened and the resulting values are
output. Each plain mode is output as follows:-
\begin{description}
\item[\texttt{CHAR}] Output a character to the next logical position in the
file. For \verb|[]CHAR|, output all the characters on the current
line.
\item[\texttt{BOOL}] Output \verb|flip| or \verb|flop| for \verb|TRUE| or
\verb|FALSE| respectively. For \verb|[]BOOL|, output \verb|flip| or
\verb|flop| for each \verb|BOOL|.
\item[\texttt{L BITS}] Output \verb|flip| for each bit equal to one and
\verb|flop| for each bit equal to zero. \verb|l bits width|
characters are output in all. No newline or newpage is output. For
\verb|[]L BITS|, each \verb|BITS| value is output as above with no
intervening spaces.
\item[\texttt{L INT}] Output a space character if the logical position is not
at the start of a line. Then output the integer using the call
\begin{verbatim}
   whole(i,1+l int width)
\end{verbatim}
\noindent
which will right-justify the integer in
\begin{verbatim}
   1+l int width
\end{verbatim}
\noindent
positions with a preceding sign. For \verb|[]L INT|, each integer is
output as described above, preceded by a space if it is not at the
beginning of the line. No newlines or newpages are output.
\item[\texttt{L REAL}] A space is output if the logical position is not at the
start of a line and then the number is output space-filled
right-justified in
\begin{verbatim}
   l real width+l exp width+3
\end{verbatim}
\noindent
positions in floating-point format and preceded by a sign. For a
value of mode \verb|[]L REAL|, each \verb|REAL| value is output as
described above.
\item[\texttt{L COMPL}] The complex value is output as two real numbers in
floating-point format separated by \verb*| i |. For \verb|[]L COMPL|,
each complex value is output as described above.
\item[\texttt{PROC(REF FILE)VOID:}]
An \verb|lf| character is output if the routine is \texttt{newline}
and an \verb|ff| character if the routine is \texttt{newpage}.
User-defined routines with this mode can be used.
\end{description}
\item \hxtt{PROC read=([]UNION(SIMPLIN,}{read}\newline
\verb|           PROC(REF FILE)VOID) x)VOID:|\newline
This is equivalent to \verb|get(stand in,|$\ldots$\verb|)|.
\item \hxtt{PROC get=(REF FILE f,}{get}\newline
\verb|          []UNION(SIMPLIN,|\newline
\verb|          PROC(REF FILE)VOID x)VOID:|\newline
This procedure converts strings of characters into internal values.
Inputting data is covered for each plain mode and \verb|REF STRING|. In
each case, if the end of the file is detected while reading characters,
the logical file end procedure is called:-
\begin{description}
\item[\texttt{REF CHAR}]\ \newline
Any characters \verb|c| where \verb|c < blank| are
skipped and the next character is assigned to the name.

If a \verb|REF[]CHAR| is given, then the above action occurs for each
of the required characters of the multiple.
\item[\texttt{REF STRING}]\ \newline
All characters, including any control characters, are
assigned to the name until any character in the character set specified
by the \verb|string term| field of \verb|f| is read. The file is then
backspaced so that the string terminator will be available for the next
\verb|get|.
\item[\texttt{REF BOOL}]\ \newline
The next non-space character is read and, if it is
neither \verb|flip| nor \verb|flop|, the char error procedure is
called with \verb|flop| as the suggestion. For \verb|REF[]BOOL|, each
\verb|\texttt{REF BOOL}| name is assigned a value as described above.
\item[\texttt{REF L BITS}]\ \newline
The action for \verb|REF BOOL| is repeated for each bit in the name.
For \verb|REF[]L BITS|, each \texttt{REF L BITS} name is assigned a
value as described above.
\item[\texttt{REF L INT}]\ \newline
If the next non-control character (\textit{ie}, a
character which is neither a space, a tab character, a newline or
newpage character or other control character) is not a decimal
digit, then the char error procedure is called with \verb|"0"| as the
suggestion. Reading of decimal digits continues until a character
which is not a decimal digit is encountered when the file is
back\-spaced. If during the reading of decimal digits, the value of
\verb|l max int| would be exceeded, reading continues, but the input
value is not increased. For \verb|REF[]L INT|, each integer is read
as described above.
\item[\texttt{REF L REAL}]\ \newline
A real number consists of 3 parts:-
\begin{itemize}
\item an optional sign possibly followed by spaces
\item an optional integral part
\item a \verb|"."| followed by any number of control characters (such
as newline or tab characters) and the fractional part
\item an optional exponent preceded by a character in the set
\verb|"Ee\"|. The exponent may have a sign. Absence of a sign is
taken to mean a positive exponent
\end{itemize}
The number may be preceded by any number of control characters or
spaces. For \verb|REF[]L REAL|, each \verb|REAL| value is read as
described above.
\item[\texttt{REF L COMPL}]\ \newline
Two real numbers separated by \verb|"i"| are read
from the file. Newlines or newpages or other control characters can
precede each real. The first number is regarded as the real part and
the second the imaginary part. For \verb|REF[]L COMPL|, each
\verb|REF L COMPL| is read as described above.
\end{description}
\end{enumerate}

\subsubsection*{Binary transput}\hylabel{stan-bin}
Binary transput performs no conversion on internal values, thus
providing a means of storing internal values in a compact form in
books or reading such values into a program.
\begin{enumerate}
\item \hxtt{PROC write bin=([]SIMPLOUT x)VOID:}{write bin}\newline
This is equivalent to \verb|put bin(stand back,x)|.
\item \hxtt{PROC put bin=(REF FILE f,}{put bin}\newline
\verb|              []SIMPLOUT x)VOID:|\newline
This procedure outputs internal values in a compact form. Then
external size is the same as the internal size.
\item \hxtt{PROC read bin=([]SIMPLIN x)VOID:}{read bin}\newline
This procedure is equivalent to
\begin{verbatim}
      get bin(stand back,x)
\end{verbatim}
\item \hxtt{PROC get bin=(REF FILE f,[]SIMPLIN x)VOID:}{get bin}\newline
This procedure transfers external values in a compact form directly
into internal values.
\end{enumerate}
In all the above procedures, the transput is direct with no code
conversion. It should also be noted that the procedure
\ixtt{make term}, although usually used with formatless transput, can
also be used with binary transput in the QAD standard prelude for
inputting a \verb|STRING| terminated by any of a number of
characters. You should note that if \hx{set possible} for the channel
concerned, then the terminator (which will \textit{always} include
the \hx{lf} character) will not have been read when \hx{get bin}
returns. However, if not \ixtt{set possible} for the channel (and neither
\ixtt{stand in} nor \ixtt{stand out} can be set), then no backspace is
possible for binary transput and so the terminating character will
have been read.

\subsubsection*{Other procedures}\hylabel{stan-otherproc}
A number of miscellaneous procedures fall into this category.
\begin{enumerate}
\item \hxtt{PROC file buffer  = (REF FILE f)BUFFER:}{file buffer}\newline
Yields the buffer of a \verb|REF FILE| value.
\item \hxtt{PROC flush buffer = (REF FILE f) VOID:}{flush buffer}\newline
This procedures empties the buffer if it has been changed by a
\verb|put| or a \verb|put bin|.
\item \hxtt{PROC no file end=(REF FILE f)BOOL:}{no file end}\newline
One of the default procedures in \ixtt{default io procs}.
\item \hxtt{PROC ignore value error = (REF FILE f)BOOL:}{ignore value error}\newline
One of the default procedures in \ixtt{default io procs}.
\item \hxtt{PROC ignore char error = (REF FILE f, REF CHAR ch)BOOL:}{ignore char error}\newline
One of the default procedures in \ixtt{default io procs}.
\end{enumerate}

\Subsection{Interrogating files}{stan-trinterr}
A number of procedures are available for interrogating the properties
of files:-
\begin{enumerate}
\item Properties of the book:-
\begin{enumerate}
\item \hxtt{PROC bin possible=(REF FILE f)BOOL:}{bin possible}\newline
Yields \verb|TRUE| if binary transput is possible.
\item \hxtt{PROC put possible=(REF FILE f)BOOL:}{put possible}\newline
Yields \verb|TRUE| if data can be sent to the book.
\item \hxtt{PROC get possible=(REF FILE f)BOOL:}{get possible}\newline
Yields \verb|TRUE| if data can be got from the book.
\item \hxtt{PROC set possible=(REF FILE f)BOOL:}{set possible}\newline
Yields \verb|TRUE| if the book can be browsed: that is, if the position
in the book for further transput can be set.
\item \hxtt{PROC reidf possible=(REF FILE f)BOOL:}{reidf possible}\newline
Yields \verb|TRUE| if the identification of the book can be changed.
\end{enumerate}
\item \hxtt{PROC current pos=(REF FILE f)INT:}{current pos}\newline
The standard Algol~68 procedure yields a triple giving the page, line
and character number. However, the QAD standard prelude does not use
pages, lines and characters, so this procedure yields the current
character position within the book for the next trans\-put operation.
\item \hxtt{PROC file buffer=(REF FILE f)BUFFER:}{file buffer}\newline
Yields the memory \hx{buffer}{\texttt{BUFFER}} associated with the
file \verb|f|.
\item \hxtt{PROC idf=(REF FILE f)RVC:}{idf}\newline
Yields the \hx{current}{\texttt{RVC}} identification of the book.
\item \hxtt{PROC logical end=(REF FILE f)INT:}{logical end}\newline
The current length of the book connected to \verb|f|.
\end{enumerate}

\Subsection{File properties}{stan-trprop}
Three procedures are provided for altering the properties of files:-
\begin{enumerate}
\item \hxtt{PROC make term=(REF FILE f,}{make term}\newline
\verb|                STRING term)VOID:|\newline
Makes \verb|term| the current string terminator.
\item \hxtt{PROC reidf=(REF FILE f,STRING new idf)VOID:}{reidf}\newline
Changes the \verb|reidf| field of \verb|f| to the given value so that
when the file is closed, the book will be renamed.
\item \hxtt{PROC set flush after put=(REF FILE f)VOID:}{set flush after put}
\newline
Ensures that the buffer of a file is flushed whenever data is written
to the file.
\end{enumerate}

\Subsection{Event routines}{stan-trevent}
Four event routines are provided. For each routine, the default
behaviour will be described. In each case, if the user routine yields
\verb|FALSE|, the default action will be elaborated. If it yields
\verb|TRUE|, the action depends on the event.
\begin{enumerate}
\item \hxtt{PROC on char error=(REF FILE f,}{on char error}\newline
\verb|                    PROC(REF FILE,|\newline
\verb|                    REF CHAR)BOOL p)VOID:|\newline
This procedure assigns the procedure \verb|p|, which may be an
identifier or an anonymous procedure, to the \texttt{char error mended}
field of \verb|f|. The actions on character error are:-
\begin{description}
\item[Default action] Use the default character for the particular
situation.
\item[User action] A character may be assigned to the \texttt{REF
CHAR} parameter and will be used if it is in the particular character
set involved.
\end{description}
The relevant situations are:-
\begin{enumerate}
\item When reading an integer of any precision, first character,
possibly following an optional sign with following spaces, is not a
digit. Any decimal digit can be substituted. The default is \verb|"0"|.
\item When reading a real of any precision, the first non-space
character, optionally preceded by a decimal point \verb|"."|, is not a
digit. Any decimal digit can be substituted. The default is \verb|"0"|.
\item When reading a real of any precision, an exponent is present
(the character \verb|"e"| or \verb|"E"| or \verb|"\"| has been read),
and the first non-space character is not a digit. Any decimal digit
can be substituted. The default is \verb|"0"|.
\item When reading a complex number, the first non-space character
following the first real is not in the set \verb|iI|. The default is
\verb|"i"|.
\end{enumerate}
\item \hxtt{PROC on logical file end=}{on logical file end}\newline
\verb| (REF FILE f,PROC(REF FILE)BOOL p)VOID:|\newline
This procedure assigns the procedure \verb|p|, which may be an
identifier or an anonymous procedure, to the \texttt{logical file
mended} field of \verb|f|. The actions on logical file end are:-
\begin{description}
\item[Default action]
On any channel, if the end of the file has been reached or, in
unformatted character transput, an \verb|eof char| is read then the
error message \texttt{logical file end not mended} will be issued and
the program terminated with the exit value \texttt{logical file end
not mended}.
\item[User action]
Any action as specified. Care should be taken if transput is performed
on the file in question as an infinite loop could occur.
\end{description}
\item \hxtt{PROC on physical file end=}{on physical file end}\newline
\verb|   (REF FILE f,PROC(REF FILE)BOOL p)VOID:|\newline
This procedure assigns the procedure \verb|p|, whether an identifier or
an anonymous procedure, to the
\begin{verbatim}
   physical file mended
\end{verbatim}
\noindent
field of \verb|f|. The actions on physical file end are:-
\begin{description}
\item[Default action]
On any channel, if there is no more room on the physical medium, the
program issues the error message
\begin{verbatim}
   physical file end not mended
\end{verbatim}
\noindent
and then terminates the program with the exit value \texttt{physical
file end not mended}.
\item[User action]
Any action as specified. Care should be taken if transput is performed
on the file in question as an infinite loop could occur.
\end{description}
\item \hxtt{PROC on value error=}{on value error}\newline
\verb|   (REF FILE f,PROC(REF FILE)BOOL p)VOID:|\newline
This procedure assigns procedure \verb|p| to the \texttt{value error
mended} field of \verb|f|. The actions taken on a value error are:-
\begin{description}
\item[Default action]
The program issues the error message \verb|value error not mended| and
then terminates with the same exit value.
\item[User action]
Transput continues.
\end{description}
The error occurs in the following situations:-
\begin{enumerate}
\item When an integer on input exceeds \verb|max int| for the precision
concerned.
\item The size of the exponent of a real number exceeds \texttt{max
int}.
\item An input real number is $\pm\infty$ or greater than \texttt{max
real} or is less than \verb|min real| for the precision concerned.
\end{enumerate}
\end{enumerate}

\Subsection{Conversion routines}{stan-trconv}
The conversion routines consist of three procedures conversion of
numbers to strings of characters, one operator and the procedure
\texttt{char in string}. All the procedures \verb|whole|,
\verb|fixed| and \verb|float| will return a string of
\verb|error char| if the number to be converted is too large for the
given width, or, if the number is a real, if it is infinite or
otherwise invalid.
\begin{enumerate}
\item \hxtt{PROC char in string=}{char in string}\newline
      \verb|        (CHAR c,REF INT p,STRING s)BOOL:|\newline
If the character \verb|c| occurs in the string \verb|s|, its index is
assigned to \verb|p| and the procedure yields \verb|TRUE|, otherwise no
value is assigned to \verb|p| and the procedure yields \verb|FALSE|.
\item \hxtt{PROC whole=(NUMBER v,INT width)STRING:}{whole}\newline
The procedure converts integer values. Leading zeros are replaced by
spaces and a sign is included if \verb|width>0|. If \verb|width| is
zero, the shortest possible string is yielded. If a real number is
supplied for the parameter \verb|v|, then the call
\verb|fixed(v, width, 0)| is elaborated.
\item \verb|PROC fixed=(NUMBER v,|\newline
\hxtt{            INT width,after)STRING:}{fixed}\newline
The procedure converts real numbers to fixed point form, that is,
without an exponent. The total number of characters in the converted
value is given by the parameter \verb|width| whose sign controls the
presence of a sign in the converted value as for \verb|whole|. The
parameter \verb|after| specifies the number of required digits after
the decimal point. From the values of \verb|width| and \verb|after|,
the number of digits in front of the decimal point can be calculated.
If the space left in front of the decimal point is insufficient to
contain the integral part of the value being converted, digits after
the decimal point are sacrificed.
\item \hxtt{PROC float=(NUMBER v,}{float}\newline
\verb|            INT width,after,exp)STRING:|\newline
The procedure converts reals to floating-point form
(``scientific notation''). The total number of characters in the
converted value is given by the parameter \verb|width| whose sign
controls the presence of a sign in the converted value as for
\verb|whole|. Likewise, the sign of \verb|exp| controls the presence of
a preceding sign for the exponent. If \verb|exp| is zero, then the
exponent is expressed in a string of minimum length. In this case, the
value of \verb|width| must not be zero. Note that \verb|float| always
leaves a position for the sign. If there is no sign, a blank is
produced instead. The values of \verb|width|, \verb|after| and
\verb|exp| determine how many digits are available before the decimal
point and, therefore, the value of the exponent. The latter value has
to fit into the width specified by \verb|exp| and so, if it cannot fit,
decimal places are sacrificed one by one until either it fits or there
are no more decimal places (and no decimal point). If it still doesn't
fit, digits before the decimal place are also sacrificed. If no space
for digits remains, the whole string is filled with \verb|error char|.
\item \verb|OP HEX=(L BITS v)[]CHAR:|\newline
The operator \ixtt{HEX} converts a value of mode \verb|L BITS| into a
row of hexadecimal digits. The total number of digits equals
\verb|l bits width OVER 4|. For example, \verb|HEX 4r3| yields
\verb|00000003|.
\end{enumerate}
\newpage

\Subsection{Layout routines}{stan-trlay}
These routines provide formatting capability on both input and output.
\begin{enumerate}
\item \hxtt{PROC space=(REF FILE f)VOID:}{space}\newline
The procedure advances the position in file \verb|f| by 1 byte. It does
\emph{not} read or write a blank.
\item \hxtt{PROC backspace=(REF FILE f)VOID:}{backspace}\newline
The procedure advances the position in file \verb|f| by -1 bytes. It
does not read or write a backspace. Note that not every channel
permits backspacing more than once consecutively.
\item \hxtt{PROC newline=(REF FILE f)VOID:}{newline}\newline
On input, skips any remaining characters on the current line and
positions the file at the beginning of the next line. This means that
all characters on input are skipped until a linefeed character
\verb|lf| is read. On output, emits a linefeed character. This is
non-standard behaviour.
\item \hxtt{PROC newpage=(REF FILE f)VOID:}{newpage}\newline
On input, skips any remaining characters on the current page and
positions the file at the beginning of the next page. This means that
all characters on input are skipped until a formfeed character
\verb|ff| is read. Note that if the character following a number is a
formfeed character, then that character will have been read during the
read of the number. Hence, the skip to formfeed character will skip the
whole of the \emph{following} page. On output, a formfeed character is
emitted immediately.
\item \hxtt{PROC skip terminators=(REF FILE f)VOID:}{skip terminators}
\newline
Any \verb|STRING| terminators are skipped on input
and the file positioned at the next non-terminating character. The
procedure is usually called after a \verb|STRING| has been read.
\end{enumerate}

\Section{Summary}{stan-summ}
The whole of the standard prelude has been described in the above
sections. Apart from the built-in operators implemented by the a68toc
compiler, the whole of the standard prelude is implemented by
Algol~68 source code.
\catcode`\%=14
\vfill\eject
