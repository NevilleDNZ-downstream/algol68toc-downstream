% $Log: chapter3.tex,v $
% Revision 1.4  2002-06-20 11:49:28  sian
% mm removed, padding removed, ca68 added
%
\Chapter{Repetition}{chapiii}
Up to now, we have dealt with \hx{plain values}{value!plain}:
that is, values with modes \verb|INT|, \verb|REAL| or \verb|CHAR|.
In practice, plain values are of limited use when dealing with a lot
of data.  For example, commercial programs are continually dealing
with strings of characters and engineers use vectors and matrices.
In this chapter, we start the process of building more complicated
modes.  Firstly, we consider repetition of values.
\newpage

\Section{Multiples}{rep-mult}
A \bfix{multiple} consists of a number of \bfix{elements}, each of
which have the same mode (sometimes known as the
\hx{\textbf{base mode}}{mode!base}).  The \ix{mode} of a multiple
consists of the \hx{mode indicant}{mode!indicant} for each
element preceded by brackets, and is said ``row of mode''.  For
example, here is an identity declaration of a row of \verb|CHAR|
\hx{multiple}{declaration!identity![]CHAR@\texttt{[]CHAR}}:
\begin{verbatim}
   []CHAR a = "abcd"
\end{verbatim}
\noindent
The phrase on the left-hand side of the equals symbol is read ``row
of car a''.  The phrase on the right-hand side of the equals symbol
is the \hx{denotation}{denotation![]CHAR@\texttt{[]CHAR}} of a
value whose mode is \verb|[]CHAR|.  Spaces can, of course, appear
before, between or after the brackets.

Multiples of mode \verb|[]CHAR| are so common that this denotation
was devised as a kind of shorthand. The maximum number of elements in
a multiple is equal to the maximum positive integer (\ixtt{max int}),
although in practice, your program will be limited by the available
memory.  The denotation of a \verb|[]CHAR| may extend over more than
one line. There are two ways of doing this. You can simply write the
denotation on more than one line in which case every character
``between'' the starting and ending quote characters is included
except the newline characters, or you can split the denotation with
quote characters at the end of one line and at the start of the
continuation of the denotation on the next line. Here are two
declarations which exemplify these rules:
\begin{verbatim}
   []CHAR long1 = "The first stage in the develo
   pment of a new program consists of analysing 
   the problem that the program must solve.";
   []CHAR long2 = "The first stage in the "
              "development of a new "
              "program consists of "
              "analysing the problem "
              "that the program must "
              "solve."
\end{verbatim}
\noindent
Notice that the second method is neater because you can indent the
subsequent parts of the denotation. Everything ``between'' the second
and third quote characters and ``between'' the fourth and fifth quote
characters is ignored, although you should not put anything other
than spaces or tabs and newlines there.  If you want to place a quote
character (\hx{"}{""@\texttt{""}}) in the denotation, you must double it, just as
in the character denotation.  Here are two \verb|[]CHAR| denotations,
each containing two quote characters:
\begin{verbatim}
   []CHAR rca = """Will you come today?""",
          rcb = "The minority report stated "
                "that ""in their opinion""";
\end{verbatim}
\noindent
The repeated quote characters are different from the quote characters
which chain the two parts of the denotation of \verb|rcb|.

\Subsection{Row-displays}{rep-rowdisp}
Multiples of other modes cannot be denoted as shown above, but use a
construct called a \hx{\textbf{row-display}}{row!display}.  A
row-display consists of none or two or more \hx{units}{unit}
separated by commas and enclosed by \ix{parentheses} (or \ixtt{BEGIN}
and \ixtt{END}).  Here is the identity
\hx{declaration}{declaration!identity![]CHAR@\texttt{[]CHAR}} for
\verb|a| written using a row-display:
\begin{verbatim}
   []CHAR a = ("a","b","c","d")
\end{verbatim}
\noindent
It is important to note that the units in the row-display could be
quite complicated. For example, here is another declaration for a
multiple with mode \verb|[]CHAR|:
\begin{verbatim}
   []CHAR b = ("a","P",REPR 36,"""")
\end{verbatim}
\noindent
In each of these two declarations, the number of elements is \verb|4|.

Here are identity \hx{declarations}{declaration!identity} for a
multiple of mode \verb|[]INT| and a multiple of mode \verb|[]REAL|:
\begin{verbatim}
   []INT  c = (1, 2+3, -2**4, 7, -11, 2, 1);
   []REAL d = (1.0, -2.9, 3e4, -2e-2, -5)
\end{verbatim}
\noindent
Note that the last unit of the row-display for \verb|c| has the same
value as the first unit. In a multiple of mode \verb|[]INT|, the
individual elements can have any value of mode \verb|INT|: that is to
say, any integer or formula yielding an integer.  In \verb|d|, the
unit yielding the last element is written as a formula yielding a
value of mode \verb|INT|.  Since the context of the row-display is
\hx{strong}{context!strong} (because it occurs on the right-hand
side of an identity declaration), this context is passed on to its
constituent \hx{units}{unit!constituent}.  Thus, the context of
the formula is also strong, and so the value yielded by the formula
is widened to yield \hx{\texttt{-5.0}}{coercion!widening}.

An empty row-display can be used to yield a
\hx{\textbf{flat} multiple}{multiple!flat} (one with no
elements). For example, here is an identity declaration using an
empty \hx{row-display}{row!display!empty}:
\begin{verbatim}
   []REAL empty = ()
\end{verbatim}
\noindent
The denotation for a flat \verb|[]CHAR| is used in the identity
declaration
\begin{verbatim}
   []CHAR none = ""
\end{verbatim}
\noindent
A multiple can also have a single element. However, a row-display
cannot have a single \ix{unit} (because it would be an enclosed
\hx{clause}{clause!enclosed}, which is a different construct). In
this case, we use a unit (or a \ix{formula}, which is another kind of
unit) for the only element, and the value of that unit is coerced to
a multiple with a single element using the \textbf{rowing}
\hx{coercion}{coercion!rowing}. For example,
\begin{verbatim}
   []INT ri = 4
\end{verbatim}
\noindent
yields a multiple with one element. An enclosed clause can be used
instead:
\begin{verbatim}
   []INT ri1 = (4)
\end{verbatim}
\noindent
since an enclosed clause is also a unit (see section \hyref{gram-enc}).

Rowing can only occur in strong \hx{contexts}{context!strong}
(and the right-hand side of an identity declaration is a strong
context).  Here is another example:
\begin{verbatim}
   []CHAR rc = "p"
\end{verbatim}
\noindent
A row-display can only be used in a strong context. Because the
context of an operand is \hx{firm}{context!firm}, a row-display
cannot appear in a formula (but there is a way round this, see
section \hyref{gram-prim}). The shorthand denotation for a
\verb|[]CHAR| is not a row-display and so does not suffer from this
limitation.

\Subsection{Dimensions}{rep-dim}
One of the properties of a multiple is its number of
\bfix{dimensions}. All the multiples declared so far have one
dimension. The number of dimensions affects the mode.  A
two-dimensional multiple of integers has the mode
\begin{verbatim}
   [,]INT
\end{verbatim}
\noindent
(said ``row-row-of-int''), while a 3-dimensional multiple of reals
(real numbers) has the mode
\begin{verbatim}
   [,,]REAL
\end{verbatim}
\noindent
Note that the number of \hx{commas}{comma!dimensions} is always one
less than the number of dimensions. In Algol~68, multiples of any
number of dimensions can be declared.\footnote{However, the
\protect\hx{a68toc}{a68toc!dimensions} Algol~68 compiler limits the
number of dimensions to three. If you try to declare rows having more
than three dimensions, the translation proceeds without error
messages, but the resulting C code will fail to compile.}

To cater for more than one dimension, each of the units of a
\hx{row-display}{row!display} can also be a row-display. For example,
the row-display for a multiple with mode \verb|[,]INT| could be
\begin{verbatim}
   ((1,2,3),(4,5,6))
\end{verbatim}
\noindent
The fact that this is the row-display for a 2-dimensional multiple
would be clearer if it were written
\begin{verbatim}
   ((1,2,3),
    (4,5,6))
\end{verbatim}
\noindent
For two dimensions, it is convenient to talk of ``\ix{row}s'' and
``col\-umns''. \hx{Here}{columns} is an identity declaration using
the previous row-dis\-play:
\begin{verbatim}
   [,]INT e = ((1,2,3),
               (4,5,6))
\end{verbatim}
\noindent
The first ``row'' of \verb|e| is yielded by the row-display
\verb|(1,2,3)| and the second ``row'' is yielded by \verb|(4,5,6)|.
The first ``column'' of \verb|e| is yielded by the row-display
\verb|(1,4)|, the second ``column'' by \verb|(2,5)| and the third
``column'' by \verb|(3,6)|.  Note that the number of elements in each
``row'' is the same, and the number of elements in each ``column'' is
also the same, but that the number of ``rows'' and ``columns''
differ.  We say that \verb|e| is a
\hx{\textbf{rectangular}}{multiple!rectangular} multiple. If
the number of ``rows'' and ``columns'' are the same, the multiple is
said to be \hx{\textbf{square}}{multiple!square}.  Here is an
\hx{identity}{declaration!identity} declaration for a square
multiple:
\begin{verbatim}
   [,]CHAR f = (("a","b","c"),
                ("A","B","C"),
                ("1","2","3"))
\end{verbatim}
\noindent
All square multiples are also rectangular, but the converse is not
true. Note that in the row-display for a multi-dimensional multiple
of characters, it is not possible to use the special denotation for a
\verb|[]CHAR|.

The base \hx{mode}{mode!base} of a multiple can be any mode,
including another row mode. For example:
\newpage
\begin{verbatim}
   [][]CHAR days =
      ("Monday","Tuesday","Wednesday",
       "Thursday","Friday","Saturday",
       "Sunday")
\end{verbatim}
\noindent
The mode is said ``row of row of \verb|CHAR|''. Note that \verb|days|
is one-dimensional, each element consisting of a one-dim\-en\-sional
\verb|[]CHAR|. The shorthand denotation for a \verb|[]CHAR| can be
used in this case. Because the base mode is \verb|[]CHAR|, the
individual \verb|[]CHAR|s can have different lengths. Here is another
example using integers:
\begin{verbatim}
   [][]INT trapezium = ((1,2),(1,2,3),(1,2,3,4))
\end{verbatim}

\Subsection{Subscripts and bounds}{rep-bound}
Each element of a multiple has one integer associated with it for
each dimension. These integers increase by \verb|1| from the first to the
last element in each dimension. For example, in the declaration
\begin{verbatim}
   []INT r1 = (90,95,98)
\end{verbatim}
\noindent
the integers associated with the elements are \verb|[1]|, \verb|[2]|
and \verb|[3]| (see the next section for an explanation of why the
integers are written like this). Remember that the first element in a
row-display always has an associated integer of \verb|[1]|. These
integers are known as \hx{subscripts}{subscript}\footnote{From the
practice of mathematicians who write $x_1, x_2, \ldots$} or indexers.
Thus the subscript of \verb|98| in \verb|r1| is \verb|[3]|. In the
two-dimensional multiple
\begin{verbatim}
   [,]INT r2 = ((-40, -30, -20),
                (100, 130, 160))
\end{verbatim}
\noindent
the subscripts for \verb|-40| are \verb|[1,1]| and the subscripts for
\verb|160| are \verb|[2,3]|.

We say that the \hx{\textbf{lower bound}}{bound!lower} of
\verb|r1| is \verb|1|, and its \hx{\textbf{upper bound}}{bound!upper}
is~\verb|3|.  The multiple \verb|r2| has a lower bound of \verb|1| for both the
first and second dimensions, an upper bound of \verb|2| for the first
dimension (2 ``rows'') and an upper bound of \verb|3| for the second
dimension (3 ``columns''). We shall write the bounds of \verb|r1| and
\verb|r2| as \verb|[1:3]| and \verb|[1:2,1:3]| respectively.  The
bounds of a flat multiple, unless specified otherwise (see the
section on trimming), are \verb|[1:0]|.

The bounds of a multiple can be
\hx{interrogated}{bounds!interrogation} using the operators
\ixtt{LWB} for the lower bound, and \ixtt{UPB} for the upper bound.
The bounds of the first, or only, dimension can be interrogated using
the \hx{monadic}{operator!monadic} form of these operators.  For
example, using \verb|days| defined above, \verb|LWB days| yields
\verb|1|, and \verb|UPB days| yields \verb|7|. Where the multiple is
multi-dimensional, the bounds are interrogated using the
\hx{dyadic}{operator!dyadic} form of \verb|LWB| and \verb|UPB|: the
left operand is the dimension while the right operand is the
identifier of the multiple.  For example, \verb|1 UPB r2| yields
\verb|2| and \verb|2 UPB r2| yields \verb|3|.  The priority of the
dyadic operators is~\verb|8|.

\begin{exercise}
\item What is wrong with the following identity declarations?
\begin{subex}
\item \verb|()CHAR c1 = "Today"| \subans Strictly speaking, the
definition of Algol~68 allows parentheses wherever brackets (\verb|[|
and \verb|]|) are allowed. Fortunately or unfortunately, the
\ix{a68toc} compiler flags this as an error.
'
\item \verb|[]CHAR c2 = 'Yesterday'| \subans The apostrophes should
be replaced by quote symbols.
'
\item \verb|[]INT i1 = (1, 2.0, 3)| \subans The value \verb|2.0| in
the row-display cannot be coerced to a value of mode \verb|INT| in a
strong context (or any context, for that matter).
'
\end{subex}
\item Using the identifier \verb|first 4 odd numbers|, write an
appropriate identity declaration.
\ans \verb|[]INT first 4 odd numbers = (1,3,5,7)|
'
\item Given the identity declarations
\begin{verbatim}
   []CHAR s = "abcdefgh";
   []REAL r = (1.4e2, 3.5e-1, -4.0);
   [,]INT t = ((2,3,5),
               (7,11,13),
               (17,19,23))
\end{verbatim}
\noindent
what is the value of the following:
\begin{subex}
\item \verb|UPB s| \subans \verb|8|
'
\item \verb|LWB r| \subans \verb|1|
'
\item \verb|2 UPB t - 1 LWB t + 1| \subans \verb|3|
'
\end{subex}
\item Write the formul{\ae} which give the upper and lower bounds of
each of the following multiples:
\begin{subex}
\item \ %
\begin{verbatim}
   [,,]INT a = (((1,2,3),
                 (4,5,6)),
                ((7,8,9),
                 (10,11,12)))
\end{verbatim}
\subans \verb|1 LWB a|, \verb|1 UPB a|, \verb|2 LWB a|,
\verb|2 UPB a|, \verb|3 LWB a|,\newline
\texttt{3 UPB a}
'
\item \verb|[]REAL b = ()| \subans \verb|LWB b|, \verb|UPB b|
'
\end{subex}
\end{exercise}

\Section{Slicing}{rep-slice}
In the previous section, it was mentioned that a subscript is
associated with every element in a multiple.  The lower-bound of the
multiple for a dimension determines the minimum subscript for that
dimension and the upper-bound for that dimension determines the
maximum subscript.  Thus there is a set of subscripts for each
dimension.  The individual elements can be accessed by quoting all
the subscripts for that element. For example, the elements of the
multiple
\begin{verbatim}
   []INT odds = (1,3,5)
\end{verbatim}
\noindent
can be accessed as \verb|odds[1]|, \verb|odds[2]| and \verb|odds[3]|.
The first of these is read ``odds sub one bus'' (``\ix{bus}'' is the opposite
of ``\ix{sub}'').  In a multi-dimensional multiple, two or more subscripts are
required to access a single element, the subscripts being separated
by commas.  For example, in the multiple
\begin{verbatim}
   [,]REAL rs = ((1.0, 2.0, 3.0),
                 (4.0, 5.0, 6.0))
\end{verbatim}
\noindent
\verb|rs[1,2]| yields \verb|2.0|.  Similarly, \verb|rs[2,3]| yields
\verb|6.0|.  Thus one can declare
\begin{verbatim}
   REAL rs12 = rs[1,2],
        rs23 = rs[2,3]
\end{verbatim}

Although, technically, a multiple with all its subscripts specified
is called a \bfix{slice}, the term is usually reserved for a multiple
with less than the maximum number of subscripts (in other words, at
least one of the dimensions does not have a subscript).  For example,
using \verb|rs| declared above, we can write
\begin{verbatim}
   []REAL srs = rs[1,]
\end{verbatim}
\noindent
which yields the multiple denoted by \verb|(1.0,2.0,3.0)|.
The \ix{comma} must be present in the slice on the right-hand side
otherwise the compiler will report an error of ``wrong number of
indices''.

Vertical slicing is also possible. The phrase \verb|rs[,2]| yields
the multiple \verb|(2.0,5.0)|. In the context of the declaration
\begin{verbatim}
   [,]CHAR rs2 = (("a","b","c","d"),
                  ("e","f","g","h"),
                  ("i","j","k","l"))
\end{verbatim}
\noindent
the slice \verb|rs2[,3]| yields the value \verb|"cgk"| with a mode of
\verb|[]CHAR|. Note, however, that \ix{vertical slicing} is only
possible for multiples with at least two dimensions.  The multiple
\verb|days|, declared in the previous section, is one-dimensional and
so cannot be sliced vertically.

In a 3-dimensional multiple, both 2-dimensional and 1-dim\-en\-sional
slices can be produced. Here are some examples:
\begin{verbatim}
   [,,]INT r3 = (((1,2),(3,4),((5,6),(7,8)));
   [,]INT r31 = r3[1,,],
          r32 = r3[,2,],
          r33 = r3[,,3];
   []INT r312 = r31[2,], r4 = r31[,2]
\end{verbatim}

\begin{exercise}
\item The declaration
\begin{verbatim}
   [,]INT r = (( 1, 2, 3, 4),
               ( 5, 6, 7, 8),
               ( 9,10,11,12),
               (13,14,15,16))
\end{verbatim}
\noindent
is in force for this and the following exercise.  Give the value of
the following slices:
\begin{subex}
\item \verb|r[2,2]| \subans \verb|6|
'
\item \verb|r[3,]| \subans \verb|(9,10,11,12)|
'
\item \verb|r[,2 UPB r]| \subans \verb|(4,8,12,16)|
'
\end{subex}
\item Write slices for the following values
\begin{subex}
\item \verb|10| \subans \verb|r[3,2]|
'
\item \verb|(5,6,7,8)| \subans \verb|r[2,]|
'
\item \verb|(3,7,11,15)| \subans \verb|r[,3]|
'
\end{subex}
\end{exercise}

\Section{Trimming}{rep-trim}
The bounds of a multiple can be changed using the \verb|@|
construction. For example, in the declaration
\begin{verbatim}
   []CHAR digits = "0123456789"[@0]
\end{verbatim}
\noindent
the bounds of \verb|digits| are \verb|[0:9]|. Bounds do not have to
be non-negative.  For example,
\begin{verbatim}
   [,]INT ii = ((1,2,3),(4,5,6));
   [,]INT jj = ii[@-3,@-50]
\end{verbatim}
\noindent
whence the bounds of \verb|jj| are \verb|[-3:-4,-50:-48]|.  Notice
that you cannot change the bounds of a row-display (except by using a
\textbf{cast}---see section \hyref{gram-slice}). For now, always
declare an identifier for the display, and then alter the bounds.
The bounds of a slice can be changed:
\begin{verbatim}
   [,]INT ij = ((1,3,5),(7,9,11),(13,15,17));
    []INT ij2 = ij[2,][@0]
\end{verbatim}
\noindent
The declaration for \verb|ij2| could also be written
\begin{verbatim}
   []INT ij2 = ij[2,@0]
\end{verbatim}
\noindent
\verb|@| can also be written \ixtt{AT}.

Wherever an integer is required in the above, any unit yielding an
integer will do. Thus it is quite in order to use the formula
\begin{verbatim}
   (a+b) UPB r
\end{verbatim}
\noindent
where the parentheses are necessary if \verb|a+b| is expected to
yield the dimension of \verb|r| under consideration (because the
\ix{priority} of \verb|UPB| is greater than the priority
of \verb|+|).

A \bfix{trimmer} uses the \ixtt{:} construction.  In the context of
the declaration of \verb|digits| above, the phrase \verb|digits[1:3]|
yields the value \verb|"123"| with mode \verb|[]CHAR|.  Again, using
the declaration of \verb|r| in the last set of exercises,
\verb|r[1:2,1]| yields \verb|(1,2)|, and \verb|r[1:2,1:2]| yields
\verb|((1,2),(5,6))|.

Trimming is particularly useful with values of mode \verb|[]CHAR|.
Given the declaration
\begin{verbatim}
   []CHAR quote = "Habent sua fata libelli"
\end{verbatim}
\noindent
(the quotation at the start of the acknowledgements in the ``Revised
Report''),
\begin{verbatim}
   quote[:6]
   quote[8:10]
   quote[12:15]
\end{verbatim}
\noindent
yield the first three words. Note that when the first subscript in a
trimmer is omitted, the lower bound for that dimension is assumed,
while omission of the second subscript assumes the corresponding
upper bound. Again, any unit yielding \verb|INT| may be used for the
trimmers. The context for a trimmer or a subscript is
\hx{\textbf{meek}}{context!meek}.

Omission of both subscripts yields the whole \ix{slice} with a lower
bound of~1. So, the upper bound of the phrase \verb|digits[:]| is
\verb|10| which is equivalent to \verb|digits[@1]|.

The lower bound of a \verb|trimmer| is, by default, \verb|1|, but may
be changed by the use of \verb|@|. For example, \verb|digits[3:6]|
has bounds \verb|[1:4]|, but \verb|digits[3:6@2]| has bounds
\verb|[2:5]|. The bounds of \verb|quote[17:]| mentioned above are
\verb|[1:7]|.

\begin{exercise}
\item Write an identity declaration for \verb|months| on the lines of
the declaration of \verb|days| in section 3.1.
\ans \ %
\begin{verbatim}
   [][]CHAR months=
      ("January","February","March",
       "April","May","June",
       "July","August","September",
       "October","November","December")
\end{verbatim}
'
\item Given the declarations
\begin{verbatim}
   [,]INT i = ((1,-2,3,4),(-5,6,7,8));
   []REAL r= (1.4,0,-5.4,3.6);
   []CHAR s= "abcdefghijklmnopqrstuvwxyz"
                                [@ ABS"a"]
\end{verbatim}
\noindent
what are the values of the following phrases?
\begin{subex}
\item \verb|2 UPB i + UPB s[@1]| \subans \verb|30|
'
\item \verb|r[2:3]| \subans \verb|(0.0,-5.4)|
'
\item \verb|i[2,2] - r[3]| \subans \verb|11.4|
'
\item \verb|i[2,2:]| \subans \verb|(6,7,8)|
'
\item \verb|s[ABS"p":ABS"t"]| \subans \verb|"pqrst"|
'
\end{subex}
\end{exercise}

\Section{Printing multiples}{rep-prmult}
We have already used \ixtt{print} to convert plain values to
characters displayed on your screen. In fact, \verb|print| can be
supplied with a row of values to be converted, so it is quite valid
to write
\begin{verbatim}
   []INT i1 = (2,3,5,7,11,13);  print(i1)
\end{verbatim}
\noindent
You can also present an actual \hx{row-display}{row!display}. Instead of using
\begin{verbatim}
   print(2); print(blank); print(3)
\end{verbatim}
\noindent
you can write \verb|print((2,blank,3))|. The doubled parentheses are
necessary: the outer pair are needed by \verb|print| anyway, and the
inner pair are part of the row-display. Notice that the modes of the
elements of the row-display are quite different. We shall learn in
chapter~8 how that can be so.

Here is a program which will print the answers to the last exercise.
\begin{verbatim}
   PROGRAM test CONTEXT VOID
   USE standard
   BEGIN
    [,]INT i = ((1,-2,3,4),(-5,6,7,8));
    []REAL r= (1.4,0,-5.4,3.6);
    []CHAR s= "abcdefghijklmnopqrstuvwxyz"
                                  [@ ABS"a"];
\end{verbatim}
\begin{verbatim}
    print(("i=",i,newline,
           "r=",r,newline,
           "s=[",s,"]",newline,
           "2 UPB i + UPB s[@1]=",
           2 UPB i+UPB s[@1],newline,
           "r[2:3]=",r[2:3],newline,
           "i[2,2] - r[3]=",
           i[2,2] - r[3],newline,
           "i[2,2:]=",i[2,2:],newline,
           "s[ABS""p"":ABS""t""]=",
           s[ABS"p":ABS"t"],
           newline))
   END
   FINISH
\end{verbatim}
\noindent
As you can see, \ixtt{print} will quite happily take values of modes
\verb|[]CHAR|, \verb|[,]INT|, \verb|[]REAL| and so on\footnote{but
\protect\hx{a68toc}{a68toc!dimensions} will only accept multiples of
upto three dimensions.}.  Notice also that in order to get quote
symbols in the last line to be printed, they are doubled.  A common
mistake is to omit a quote symbol or a closing comment symbol. If
your editor provides lexical highlighting (usually called ``syntax''
highlighting), an omitted quote or comment symbol will cause a large
part of your program to be highlighted as though it were a
\verb|[]CHAR| or a comment. The mistake will be very clear. If your
editor does not support lexical highlighting, you will get an odd
message from the compiler (usually to the effect that it has run out
of program!).

\begin{exercise}
\item Write short programs to print the answers to all the exercises
in this chapter from 3.2.1. You should insert multiples of
\verb|CHAR| at suitable points, as in the example above, so that you
can identify the printed answers. \ans This exercise is self-marking,
but here is a program to print the answer to the first exercise:-
\begin{verbatim}
   PROGRAM ex3 2 1 CONTEXT VOID
   USE standard
   BEGIN
     [,]INT m = ((1,2,3,4),(5,6,7,8),
              (9,10,11,12),(13,14,15,16));

     print(("m[2,2]=",m[2,2],newline,
          "m[3,]=",m[3,],newline,
          "m[,2 UPB m]=",m[,2 UPB m],
          newline))
   END
   FINISH
\end{verbatim}
'
\end{exercise}

\Section{Operators with multiples}{rep-opmult}
No operators are defined in the standard prelude for multiples whose
elements have modes \verb|INT| or \verb|REAL|. This is not a drawback
as you will learn in chapter~6. Nor are there any monadic operators
in the standard prelude for multiples of \verb|CHAR|. However,
multiples of \verb|CHAR| occur so often, that two dyadic operators
are available for them.

The operator \ixtt{+} is defined for all combinations of \verb|CHAR|
and \verb|[]CHAR|.  Thus, the formula
\begin{verbatim}
   "abc" + "d"
\end{verbatim}
\noindent
yields the value denoted by \verb|"abcd"|. With these operands,
\verb|+| acts as a \ix{concatenation} operator. The operator has a
\ix{priority} of~6 as before.

Multiplication of values of mode \verb|CHAR| or \verb|[]CHAR| is
defined using the operator \ixtt{*}. The other operand has mode
\verb|INT| and the yield has mode \verb|[]CHAR|. For example, in the
declaration
\begin{verbatim}
   []CHAR repetitions = "ab" * 3
\end{verbatim}
\noindent
\verb|repetitions| identifies \verb|"ababab"|.  The formula could
have been written with the integer as the left operand.  In both
cases, the operator only makes sense with a positive integer.

\begin{exercise}
\item Given the identity declarations
\begin{verbatim}
   []CHAR s = "Dog bites man",
          t = "aeiou"
\end{verbatim}
\noindent
what is the value of the following formul{\ae}?
\begin{subex}
\item \verb|"M"+s[UPB s-1:]+s[4:10]+"d"+s[2:3]|
\subans \verb|Man bites dog|
'
\item \verb|s[5]*3+2*s[6]| \subans \verb|bbbii|
'
\end{subex}
\end{exercise}

\Section{Ranges}{rep-range}
If you cast your mind back to the form of an Algol~68 program, you
will remember that it consists of a number of phrases enclosed by
\verb|BEGIN| and \verb|END| (or parentheses) preceded by a
\ixtt{PROGRAM} phrase with an
optional \ixtt{USE} phrase.  The part of the
program enclosed by \verb|BEGIN| and \verb|END| (including the
\verb|BEGIN| and \verb|END|) is called a
\hx{\textbf{closed clause}}{clause!closed}.  The important point
here is that a closed clause consists of one or more phrases
separated by semicolons\ixtt{;} (the last \ix{phrase} being a
\ix{unit}), surrounded by parentheses (or \verb|BEGIN| and
\verb|END|).  Since a declaration is not a unit, the last phrase
cannot be a declaration.  We say that the value of a
\hx{closed clause}{value!of a closed clause} is the value yielded
by the final unit.  As an example, here is a closed clause with a
value of mode \verb|INT|:
\begin{verbatim}
   BEGIN
      INT i = 43;
      print((i,newline));
      i
   END
\end{verbatim}

An important adjunct of a closed clause is that any identifiers
declared in the clause do not exist outside the clause.  We say that
the \bfix{range} of an \ix{identifier} is confined to that section of
the closed clause from its declaration to the end of the clause.

\Section{Program repetition}{rep-rep}
Having investigated the construction and use of multiple values, it
is now time to address \ix{repetition} of program actions.  For
example, suppose you wanted to output~8 blank lines.  You could write
\begin{verbatim}
   print((newline,newline,newline,newline,
          newline,newline,newline,newline))
\end{verbatim}
\noindent
A simpler way would be to write
\begin{verbatim}
   TO 8 DO print(newline) OD
\end{verbatim}
\noindent
The integer following the \ixtt{TO} can be any unit yielding an
integer (not necessarily positive) in a meek
\hx{context}{context!meek}.  If the value yielded is zero or
negative, then the ensuing clause enclosed by \verb|DO| and \verb|OD|
will not be elaborated at all.  The \verb|TO|~$\ldots$~\verb|OD|
construct is called a \textbf{loop clause} or, more simply, a loop.

If you omit the \verb|TO| integer construct, the loop will be
repeated indefinitely.  In that case, you would need some way of
terminating the program inside the loop.

A more useful form of the loop \hx{clause}{clause!loop} is shown
by the following example
\begin{verbatim}
   FOR i TO 10
   DO
      print((i,newline))
   OD
\end{verbatim}
\noindent
The \verb|i| is an \ix{identifier}, whose declaration occurs at that
point and whose mode is \ixtt{INT}.  The example will print the
numbers \verb|1| to \verb|10|, each on its own line.  The \ix{range} of
\verb|i| is the whole of the loop clause, but does not include the
unit following \verb|TO|.  Any identifier may be used in place of
\verb|i|. When the \verb|TO| part is omitted, it is as though
\verb|TO| $\infty$ had been written.

It is possible to modify the number of times the loop is obeyed. The
simplest way is to define the starting point using the \ixtt{FROM}
construct. Here is an example:
\begin{verbatim}
   FOR n FROM -10 TO 10 DO print((n,blank)) OD
\end{verbatim}
\noindent
This prints the numbers from \verb|-10| to \verb|+10| on the screen.
The integer after \verb|FROM| can be any \ix{unit} which yields a
value of mode \verb|INT| in a meek \hx{context}{context!meek}.
When \verb|FROM| is omitted, it is assumed that the first value of
the identifier following \verb|FOR| is~1.

This example prints the square of each of the numbers from \verb|0.2|
to \verb|0.9|:
\begin{verbatim}
   FOR number FROM 2 TO 9
   DO
      REAL value = number / 10;
      print((value," squared =",
             value * value,newline))
\end{verbatim}
\noindent
In these examples, the value of the identifier has always increased
by \verb|1|. The increase can be changed using the \ixtt{BY}
construct. For example, to print the cubes of the even numbers
between 30 and 50 inclusive, you could write\footnote{Unfortunately,
there is a bug in the \protect\hx{a68toc}{a68toc!BY@\texttt{BY}}
compiler which prevents you from using a denotation after
\texttt{BY}. Nor does \texttt{BY (2)} or \texttt{BY +2} work.
However, \texttt{BY -2} or \texttt{BY --2} does work. For other ways see chapter 6.}
\begin{verbatim}
   FOR n FROM 30 BY --2 TO 50
   DO
      print((n**3,newline))
   OD
\end{verbatim}
\noindent
The \verb|BY| construct is particularly useful for decreasing the
value of the identifier:
\begin{verbatim}
   []CHAR title =
          "Programming Algol 68 Made Easy";
\end{verbatim}
\begin{verbatim}
   FOR c FROM UPB title BY -1 TO LWB title
   DO
      print(title[c])
   OD
\end{verbatim}
\noindent
This last example shows how useful the loop
\hx{clause}{clause!loop} can be for accessing some of or all of
the elements of a multiple.  Here is another
\hx{example}{LWB@\texttt{LWB}}\hx{:}{UPB@\texttt{UPB}}
\begin{verbatim}
   []INT hh=(7,17,27,37,47);
   INT two=2;
\end{verbatim}
\begin{verbatim}
   FOR i BY --2 TO UPB hh
   DO
      print(hh[i] * hh[i])
   OD
\end{verbatim}
\noindent
which will print
\begin{verbatim}
        +49        +729       +2209
\end{verbatim}
\noindent
on one line.  Omitting the \verb|BY| construct assumes a default step
of~1.

Notice how use of the \verb|LWB| and \verb|UPB| operators ensures
that your program does not try to use a subscript outwith the bounds
of the multiple. If you try to access an element whose subscript is
greater than the upper bound (or less than the lower bound), the
program will fail at run-time with an appropriate error message.

An important use of the identity
\hx{declaration}{declaration!identity!optimisation} is that of
\ix{optimisation}. In the previous example, the computation of the
\ith{i}{th} element of \verb|hh| takes a little time,
and there is no point in repeating it.  In the following example, the
identity declaration computes the value of \verb|hh[i]| and the
\verb|print| statement uses the resulting value twice:
\begin{verbatim}
   FOR i BY --2 TO UPB hh
   DO
      INT hhi = hh[i];
      print((hhi * hhi,newline))
   OD
\end{verbatim}
\noindent
Everything said about multiples with elements of mode \verb|INT| or
\verb|CHAR| applies equally well to multiples whose elements have
mode \verb|REAL|. A \verb|FOR| loop yields no value (cf section
\hyref{rout-yname}).

\begin{exercise}
\item Write an Algol 68 program which will print the cubes of the numbers
from \verb|1| to \verb|25|. \ans \ %
\begin{verbatim}
   PROGRAM ex3 7 1 CONTEXT VOID
   USE standard
   BEGIN
      FOR num TO 25
      DO
         print((num^3,newline))
      OD
   END
   FINISH
\end{verbatim}
'
\item Write a program which will print the characters of the alphabet
backwards, all on one line. \ans \ %
\begin{verbatim}
   PROGRAM ex3 7 2 CONTEXT VOID
   USE standard
   BEGIN
      FOR c FROM ABS "Z" BY -1 TO ABS "A"
      DO
         print(REPR c)
      OD
   END
   FINISH
\end{verbatim}
'
\end{exercise}

\Section{Nested loops}{rep-nestloop}
When dealing with two-, and higher-dimensional multiples, it is often
necessary to run a subsidiary loop. For example, suppose we wanted to
print the square of each element in the multiple declared as
\begin{verbatim}
   [,]INT primes = (( 2, 3, 5, 7),
                    (11,13,17,19),
                    (23,29,31,37),
                    (41,43,47,53))
\end{verbatim}
\noindent
with each row on one line. Here is a piece of program which will do
it:
\begin{verbatim}
   FOR i FROM 1 LWB primes TO 1 UPB primes
   DO
      []INT pri=primes[i,];
\end{verbatim}
\begin{verbatim}
      FOR j FROM LWB pri TO UPB pri
      DO
         INT prij = pri[j];
         print(prij * prij)
      OD;
      print(newline)
   OD
\end{verbatim}
\noindent
Notice the optimisations. The first defines the \ith{i}{th}
``row'', and the second defines the \ith{j}{th} element in that
``row''. The point is that any piece of program can appear inside the
loop clause.  Loop clauses can be nested to any depth.  Because the
loop \hx{clause}{clause!loop} is an enclosed
\hx{clause}{clause!enclosed}, it must contain at least one
\ix{phrase}, and the last phrase must be a \ix{unit} (see chapter~10
for a thorough discussion of units).

\begin{exercise}
\item Using a nested loop, write a short program to display the first
25 letters of the alphabet on your screen in five rows of five
letters.  Separate each letter with a comma.
\ans The main difficulty lies in computing the letter to print. The
first solution uses numbers and \verb|REPR|:-
\begin{verbatim}
   PROGRAM ex3 8 1a CONTEXT VOID
   USE standard
   BEGIN
      FOR row TO 5
      DO
         FOR letter TO 4
         DO
            print((REPR((row-1)*5
                     +letter+ABS"@"),",")
         OD;

         print((REPR(row*5 + ABS "@"),
                newline))
      OD
   END
   FINISH
\end{verbatim}
\noindent
The second solution uses an actual alphabet and a modified inner
loop. Note that the formul\ae{} in the \verb|FROM| and \verb|TO|
constructs are elaborated once only: before the inner loop is
elaborated for the first time in each elaboration of the outer loop:-
\begin{verbatim}
   PROGRAM ex3 8 1b CONTEXT VOID
   USE standard
   BEGIN
      []CHAR alphabet =
       "ABCDEFGHIJKLMNOPQRSTUVWXYZ"[@1];

      FOR row TO 5
      DO
         INT row5 = row*5;

         FOR letter FROM row5-4 TO row5-1
         DO
            print((alphabet[letter],","))
         OD;

         print((alphabet[row5],newline))
      OD
   END
   FINISH
\end{verbatim}
'
\item Write a program to print the value of a 3-dimensional multiple
of real numbers which you have declared in your program.
\ans The most difficult part is in declaring the multiple.
\verb|print| will quite happily take the 3-dimensional multiple as
its parameter:
\begin{verbatim}
   PROGRAM ex3 8 2 CONTEXT VOID
   USE standard
   BEGIN
      [,,]REAL m=(((1e-7,1e-6),
                   (1e-5,1e-4)),
                  ((1e-3,1e-2),
                   (1e-1,1.0)));
      print(m)
   END
   FINISH
\end{verbatim}
'
\end{exercise}

\Section{Program structure}{rep-prog}
In chapter 1, it was mentioned that the basic structure of an
Algol~68 program consists of
\begin{verbatim}
   BEGIN
      phrases
   END
\end{verbatim}
\noindent
This is not strictly true. It is quite possible to write a program
consisting solely of a \verb|DO| loop! For example:\pagebreak
\begin{verbatim}
   PROGRAM dosum
   USE standard
   FOR i TO 5
   DO
      print((i**2,newline))
   OD
   FINISH
\end{verbatim}

\Section{The \texttt{FORALL} loop}{rep-FORALL}
The \hx{\texttt{FORALL} loop}{FORALL loop@\texttt{FORALL} loop} is
not part of Algol~68, but an extension introduced by the
\hx{a68toc}{a68toc!FORALL@\texttt{FORALL}} compiler. It is similar to
the \verb|FOR| loop, but the identifier has the mode of an element of
the multiple under consideration.  Look at this
\hx{example}{IN@\texttt{IN}}:
\begin{verbatim}
   []REAL r1 = (1.0,2.0,3.0,4.0,5.0);
   FORALL e IN r1 DO print(e * e) OD
\end{verbatim}
\noindent
In the \verb|FORALL| loop, \verb|e| takes the value of each element
in \verb|r1| and so has mode \verb|REAL|. The compiler generates more
efficient code using the \verb|FORALL| loop by avoiding the normal
overheads of the subscripting mechanism. However, the \verb|FORALL|
loop can only be used when all the elements of a dimension are
required. If you want to limit the processing to a few elements, you
can trim the multiple or use the \verb|FOR| loop.

The elements of more than one multiple can be combined
simultaneously. For example:
\begin{verbatim}
   []INT i = (1,2,3,4,5),
         j = (11,12,13,14,15);
   FORALL ii IN i, jj IN j
   DO
      print((ii * jj,newline))
   OD
\end{verbatim}
\noindent
The \ix{comma} between \verb|ii IN i| and
\verb|jj IN j| means that the constructs are elaborated
\hx{collaterally}{elaboration!collateral}.  The \ix{bounds} of
\verb|i| must be the same as the bounds of \verb|j|.

\verb|FORALL| clauses can be \ix{nested} as in the case of \verb|FOR|
clauses. If we use \verb|l| and \verb|m| declared in a previous
example, then
\begin{verbatim}
   FORALL ll IN l
   DO
      FORALL mm IN m
      DO
         print(ll * mm)
      OD
   OD
\end{verbatim}
\noindent
could be used to print the products of all the integers.

\Section{Summary}{rep-summ}
Modes of multiples start with brackets (\verb|[]|). A multiple of
characters has a special denotation. All multiples can be constructed
using a row-display. Rows have bounds and dimensions. Rows can be
sliced and trimmed, and their bounds can be changed using the
\verb|@| construct.

The \verb|FOR| loop has the form
\begin{verbatim}
   FOR id FROM a BY b TO c DO ... OD
\end{verbatim}
\noindent
where the default values of \verb|a|, \verb|b| and \verb|c| are
\verb|1|, \verb|1| and $\infty$ respectively, but may take any value
of mode \verb|INT| in a meek context. If \verb|c| is greater than or
equal to \verb|a| and \verb|b| is negative, the loop will not be
executed. If \verb|b| is zero, the loop will be executed
indefinitely. The range of \verb|id| excludes the units \verb|a|,
\verb|b| and \verb|c|.  The \verb|FORALL| loop has the form
\begin{verbatim}
   FORALL id1 IN row1 DO ... OD
\end{verbatim}

We have covered a good deal of ground in this chapter, so here are
some more exercises revising what you have learnt.  It is most
instructive to verify your answers by writing appropriate Algol~68
programs.
\newpage
\begin{exercise}
\item What is wrong with the following identity declarations?
\begin{subex}
\item \verb|[]REAL r1 = [2.5,-2.5,3.5]| \subans The brackets for the
row-display should be replaced by parentheses.
'
\item \verb|[,]INT i1 = ((1,2,3),(4,5,6,7))| \subans The number of
integers in each row should be the same.
'
\item \verb|[]CHAR s1 = "abcde'fg"| \subans Nothing. The denotation
of an apostrophe is not doubled.
'
\end{subex}
\item What are the upper and lower bounds of the following?
\begin{subex}
\item \verb|((10,20,30),(-10,-20,-30))| \subans \verb|[1:2,1:3]|
'
\item \verb|("a","b","c")| \subans \verb|[1:3]|
'
\item \verb|"abcdef"[3:4]| \subans \verb|[1:2]|
'
\end{subex}
\item If \verb|a| is declared as
\begin{verbatim}
   [,]INT a = ((9,8,7),
               (6,5,4),
               (3,2,1))
\end{verbatim}
\noindent
what is the value and mode of
\begin{subex}
\item \verb|a[2,]| \subans \verb|(6,5,4)| \verb|[]INT|
'
\item \verb|a[,2]| \subans \verb|(8,5,2)| \verb|[]INT|
'
\item \verb|a[:2,3]| \subans \verb|(7,4)| \verb|[]INT|
'
\item \verb|a[2:,:2]| \subans \verb|((6,5),(3,2))| \verb|[,]INT|
'
\end{subex}
\item What value does \verb|"abc"*3+"defg"| yield?
\ans \verb|"abcabcabcdefg"|
'
\item Write a program to display every fifth letter of the alphabet
all on one line. \ans Notice the means of avoiding the use of
\verb|BY|:
\begin{verbatim}
   PROGRAM ex3 10 5 CONTEXT VOID
   USE standard
   BEGIN
      []CHAR alphabet =
            "abcdefghijklmnopqrstuvwxyz";
      []INT by = (1,6,11,16,21,26);

      FOR c BY --5 TO UPB alphabet
      DO
         print(alphabet[c])
      OD
   END
   FINISH
\end{verbatim}
'
\end{exercise}
