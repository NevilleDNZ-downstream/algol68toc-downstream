% $Log: chapter2.tex,v $
% Revision 1.4  2002-06-20 11:49:28  sian
% mm removed, padding removed, ca68 added
%
\Chapter{Formul\ae}{chapii}
Formul{\ae} consist of \hx{\textbf{operators}}{operator} with
\hx{operands}{operand}.  Operators are predeclared pieces of program
which compute a value determined by their operands.  Algol~68 is
provided with a rich set of operators in the standard prelude and you
can define as many more as you want.  In this chapter, we shall
examine all the operators in the \ix{standard prelude} which can take
operands of mode \verb|INT|, \verb|REAL| or \verb|CHAR|.  In
chapter~6, we shall return to operators and look at what they do in
more detail, as well as how to define new ones.

Operators are written as a combination of one or more symbols, or in
capital letters like a \hx{mode indicant}{mode!indicant}. We
shall meet both kinds in this chapter.

\Section{Monadic operators}{form-monad}
Operators come in two flavours:
\hx{\textbf{monadic}}{operator!monadic} and
\hx{\textbf{dyadic}}{operator!dyadic}. A monadic operator has
only one operand, but a dyadic operator has two operands. A monadic
operator is written before its operand. For example, the monadic
minus \hx{\texttt{-}}{-@\texttt{-}!monadic} reverses the sign of
its operand:
\begin{verbatim}
   -3000
\end{verbatim}
\noindent
This could equally well be written \verb|- 3000| since spaces are,
generally speaking, not significant. There is, likewise, a monadic
\hx{\texttt{+}}{+@\texttt{+}!monadic} operator which doesn't do
anything to its operand, but is useful where you want to refer
expressly to a positive number.  It has been provided for the sake of
consistency.  You should note that \verb|-3000| is not a denotation,
but a formula consisting of a monadic operator operating on an
operand which is a \ix{denotation}.  We say that the monadic operator
\verb|-| takes an operand of mode \ixtt{INT} and
\hx{\textbf{yields}}{yield} a value of mode \verb|INT|.  It can also
take an operand of \hx{mode}{mode!INT@\texttt{INT}} \ixtt{REAL} when
it will yield a value of mode \verb|REAL|.

A formula can be used as the value part of an identity
\hx{declaration}{declaration!identity}.  Thus the following
identity declarations are both valid:
\begin{verbatim}
   INT  minus 2 = -2;
   REAL minus point five = -0.5
\end{verbatim}
\noindent
The operator \ixtt{ABS} takes an operand of mode \verb|INT| and
yields the absolute value again of mode \verb|INT|. For example,
\verb|ABS -5| yields the value denoted by~\verb|5|:
\begin{verbatim}
   INT five = ABS -5
\end{verbatim}
\noindent
Note that when two monadic operators are
\hx{combined}{operator!combining}, they are elaborated in
right-to-left order, as in the above
\hx{example}{elaboration!order of}.  That is, the \verb|-| acts
on the \verb|5| to yield \verb|-5|, then the \verb|ABS| acts on
\verb|-5| to yield \verb|+5|. This is just what you might expect.
\verb|ABS| can also take an operand of mode \verb|REAL| yielding a
value of mode \verb|REAL|.  For example:
\begin{verbatim}
   REAL x = -1.234;
   REAL y = ABS x
\end{verbatim}

Another monadic operator which takes an \verb|INT| operand is
\ixtt{SIGN}. This yields $-1$ if the operand is negative, $0$ if it
is zero, and $+1$ if it is positive.  Thus you can declare
\begin{verbatim}
   INT res = SIGN i
\end{verbatim}
\noindent
if \verb|i| has been previously declared.

\Section{Dyadic operators}{form-dyad}
A dyadic operator takes two operands and is written between them. The
simplest operator is dyadic \hx{\texttt{+}}{+@\texttt{+}!dyadic}.
Here is an identity declaration using it:
\begin{verbatim}
   INT one = 1;
   INT two = one + one
\end{verbatim}
\noindent
This operator takes two operands of mode \ixtt{INT} and yields a
result of mode \verb|INT|. It is also defined for two operands of
mode \ixtt{REAL} yielding a result of mode \verb|REAL|:
\begin{verbatim}
REAL x = 1.4e5 + 3.7e12
\end{verbatim}
\noindent
The \verb|+| operator performs an action quite different for
\verb|REAL| operands from that performed for \verb|INT| operands. Yet
the meaning is essentially the same, and so the same symbol is used
for the two operators.

Before we continue with the other dyadic operators, a word of caution
is in order. As we have seen, the maximum integer which the computer
can use is \verb|max int| and the maximum real is \verb|max real|.
The dyadic \verb|+| operator could give a result which is greater
than those two values. Adding two integers such that the sum exceeds
\verb|max int| is said to give
\hx{``integer overflow''}{overflow!integer}. Algol~68 contains no
specific rules
about what should happen in such a case.\footnote{The standard
prelude supplied with the Linux port of the \texttt{a68toc} compiler
provides a means of specifying what should be done if integer
overflow occurs. See section \hyref{stan-othop} for the details.
Likewise for ``floating-point overflow'' and ``floating-point
underflow'', see section \hyref{stan-fpov}.}

The dyadic \hx{\texttt{-}}{-@\texttt{-}!dyadic} operator can take
two operands of mode \verb|INT| or two operands of mode \verb|REAL|
and yields an \verb|INT| or \verb|REAL| result respectively:
\begin{verbatim}
   INT minus 4 = 3 - 7,
   REAL minus one point five = 1.9 - 3.4
\end{verbatim}
\noindent
Note that the dyadic \verb|-| is quite different from the monadic
\verb|-|. You can have both operators in the same formula:
\begin{verbatim}
   INT minus ten = -3 - 7
\end{verbatim}
\noindent
The first minus sign represents the monadic operator and the second,
the dyadic.

Since a formula yields a value of a particular mode, you can use it
as an \ix{operand} for another operator. For example:
\begin{verbatim}
   INT six = 1 + 2 + 3
\end{verbatim}
\noindent
The operators are elaborated in left-to-right order. First the
formula \verb|1+2| is elaborated, then the formula \verb|3+3|. What
about the formula \verb|1-2-3|?  Again, the first \verb|-| operator
is elaborated giving \verb|-1|, then the second giving the value
\verb|-4|.

Instead of saying ``the value of mode \verb|INT|'', we shall
sometimes say ``the \verb|INT| value'' or even ``the
\verb|INT|''---all these expressions are equivalent.

\begin{exercise}
\item Write an identity declaration for the \verb|INT| value
\verb|-35|. \ans \verb|INT minus thirty five = -35|
'
\item What is the value of each of the following formul{\ae}?
\begin{subex}
\item \verb|3 - 2| \subans \verb|1|
'
\item \verb|3.0 - 2.0| \subans \verb|1.0|
'
\item \verb|3.0 - -2.0| \subans \verb|5.0|
'
\item \verb|2 + 3 - 5| \subans \verb|0|
'
\item \verb|-2 + +3 - -4| \subans \verb|5|
'
\end{subex}
\item Given the following declarations
\begin{verbatim}
   INT a = 3, REAL b = 4.5
\end{verbatim}
\noindent
what is the value of the following formul{\ae}?
\begin{subex}
\item \verb|a+a| \subans \verb|6|
'
\item \verb|-a-a| \subans \verb|-6|
'
\item \verb|b+b+b| \subans \verb|13.5|
'
\item \verb|-b - -b + -b| \subans \verb|4.5|
'
\end{subex}
\end{exercise}

\Section{Multiplication}{form-mul}
The operand \ixtt{*} (often said "star") represents normal
arithmetic \ix{multiplication} and takes \verb|INT| operands yielding
an \verb|INT| result. For example:
\begin{verbatim}
   INT product = 45 * 36
\end{verbatim}
\noindent
Likewise, \verb|*| is also defined for multiplication of two values
of mode \verb|REAL|:
\begin{verbatim}
   REAL real product = 2.4e-4 * 0.5
\end{verbatim}
\noindent
It is important to note that although the actions of the two
operators are quite different, they both represent multiplication so
they both use the same symbol.

Like \verb|+| and \verb|-|, multiplication can occur several times:
\begin{verbatim}
   INT factorial six = 1 * 2 * 3 * 4 * 5 * 6
\end{verbatim}
\noindent
the order of elaboration being
\hx{left-to-right}{elaboration!order of}.

You can also combine multiplication with addition and subtraction.
For example, the value of the formula \verb|2+3*4| is \verb|14|.
At school, you were probably taught that multiplication should be
done before addition (your teachers may have used the mnemonic
\ix{BODMAS} to show the order in which operations are done.  It
stands for Brackets, Over, Division, Multiplication, Addition and
Subtraction).  In Algol~68, the same sort of thing applies and it is
done by operators having a \bfix{priority}.  The priority of
multiplication is higher than the priority for addition or
subtraction.  The priority of the dyadic \verb|+| and \verb|-|
operators is~6, and the priority of the \verb|*| operator is~7.

Here are identity \hx{declarations}{declaration!identity} using a
combination of multiplication and addition and subtraction:
\begin{verbatim}
   INT  i1 = 3, i2 = -7;
   INT  result1 = i1 * i2 - 8;
   REAL r1 = 35.2, r2 = -0.04;
   REAL result2 = r1 * -r2 + 12.67 * 10.0
\end{verbatim}
\noindent
In the elaboration of \verb|result2|, the multiplications are
elaborated first, and then the addition.

Remember from chapter 1 that \hx{widening}{coercion!widening} is
allowed in the context of the right-hand side of an identity
declaration, so the following declaration is valid:
\begin{verbatim}
   REAL a = 24 * -36
\end{verbatim}
\noindent
It is important to note that an operand is not in a strong
\hx{context}{context!strong}, so no widening is allowed. The
context of an operand is \textbf{firm}.  Because widening is not
allowed in a firm \hx{context}{context!firm}, it is possible for
the compiler to examine the modes of the operands of an operator and
determine which declaration of the operator is to be used in the
elaboration of the formula. This also applies to monadic operators
(see \hyref{rout-idops} for details).

Looking again at the above identity declaration, the context of the
denotation \verb|36| is firm (it is the operand of the monadic
\verb|-|), the contexts of the \verb|24| and the \verb|-36| are also
firm because they are the operands of the dyadic \verb|*|, but the
value \hx{yielded}{value!yield} by the formula is on the
right-hand side of the identity declaration, so it is in a strong
\hx{context}{context!strong}.  It is this value which is coerced
to a value of mode \verb|REAL| by the widening.  Note that the value
of the formula (which has mode \verb|INT|) does not change.  Instead,
it is replaced by the \ix{coercion} with a value of mode \verb|REAL|
whose whole number part has the same value as the \verb|INT| value.
It is worth saying that the value of the formula obtained by
elaboration is lost after the coercion. You could hang on to the
intermediate integer value by using another identity declaration:
\begin{verbatim}
    INT intermediate value = 24 * -36;
    REAL a = intermediate value
\end{verbatim}

\begin{exercise}
\item In this exercise, these declarations are assumed to be in force:
\begin{verbatim}
   INT  d1 = 12, d2 = -5;
   REAL d3 = 4.0 * 3.5, d4 = -3.0
\end{verbatim}
\noindent
What is the value of each of the following formul{\ae}?
\begin{subex}
\item \verb|ABS d2| \subans \verb|5|
'
\item \verb|- ABS d4 + d3 * d4| \subans \verb|-45.0|
'
\item \verb|d2 - d1 * 3 + d2 * 4| \subans \verb|-61|
'
\end{subex}
\end{exercise}

\Section{Division}{form-div}
In the preceding sections, all the operators mentioned yield results
which have the same mode as the operand or operands. In this and
following sections, we shall see that this is not always the case.

Division poses a problem because division by integers can have two
different meanings.  For example, $3\div2$ can be taken to mean
\verb|1| or \verb|1.5|.  In this case, we use two different operator
symbols.

Integer \hx{division}{integer division} is represented by the
symbol \verb|%|.  It takes operands of mode \verb|INT| and yields a
value of mode \verb|INT|.  It has the \ix{alternative representation}
\ixtt{OVER}.  The formula \texttt{7 \%\ 3} yields the value \verb|2|,
and the formula \texttt{-7 \%\ 3} yields the value \verb|-2|.  The
\ix{priority} of \verb|%| is~7, the same as multiplication.  Here
are some identity declarations using the operator:
\begin{verbatim}
   INT r = 23 OVER 4, s = -33 % 3;
   INT q = r * s % 2
\end{verbatim}
\noindent
Using the given values of \verb|r| and \verb|s|, the value of
\verb|q| is \verb|-27|. When a formula containing
\hx{consecutive}{consecutive operators} dyadic operators of the
same priority is elaborated, elaboration is always left-to-right, so
in this case the multiplication is elaborated first, followed by the
integer division.  Of course, \verb|%| can be combined with
subtraction as well as all the other operators already discussed.

The modulo \hx{operator}{operator!modulo} \ixtt{MOD} gives the
\ix{remainder} after integer division. It requires two operands of
mode \verb|INT| and yields a value also of mode \verb|INT|. Thus
\verb|5 MOD 3| yields \verb|2|, and \verb|12 MOD 3| yields \verb|0|.
It does work with negative integers, but the results are unexpected.
You can explore \verb|MOD| with negative integers in an exercise.
\verb|MOD| can also be written as \verb|%*|. The priority of
\verb|MOD| is~7.

Division of \hx{real}{division!real} numbers is performed by the
operator \ixtt{/}. It takes two operands of mode \verb|REAL| and
yields a \verb|REAL| result.  Thus the formula \verb|3.0/2.0| yields
\verb|1.5|.  Again, \verb|/| can be combined with \verb|*| and the
other operators already discussed.  It has a priority of~7. The
operator is also defined for integer operands.  Thus \verb|3/2|
yields the value \verb|1.5|.  No \hx{widening}{coercion!widening}
takes place here since the operator is defined to yield a value of
mode \verb|REAL| when its operands have mode \verb|INT|.

Here are some identity declarations using the operators described so
far:
\begin{verbatim}
   REAL pi by 2 = pi / 2,
        pm3 = pi - 3.0 * -4.1;
   INT c = 22 % 3 - 22 MOD 3;
   INT d = c MOD 6 + SIGN -36
\end{verbatim}

\begin{exercise}
\item What is the value yielded by each of the following formul{\ae}, and
what is its mode?
\begin{subex}
\item \verb|5 * 4| \subans \verb|20| \verb|INT|
'
\item \verb|5 % 4| \subans \verb|1| \verb|INT|
'
\item \verb|5 / 4| \subans \verb|1.25| \verb|REAL|
'
\item \verb|5 MOD 4| \subans \verb|1| \verb|INT|
'
\item \verb|5.0 * 3.5 - 2.0 / 4.0| \subans \verb|17.0| \verb|REAL|
'
\end{subex}
\item Write a short program to print the results of using \verb|MOD| with
negative integer operands. Try either operand negative, then both
operands negative. \ans Your answer should be something like this:-
\begin{verbatim}
   PROGRAM ex2 4 2 CONTEXT VOID
   USE standard
   BEGIN
      print(-7 MOD 3);
      print( 7 MOD -3);
      print(-7 MOD -3)
   END
   FINISH
\end{verbatim}
\noindent
This will display
\begin{verbatim*}
         +2          +1          +2
\end{verbatim*}
on your screen.
'
\item Give an identity declaration for the identifier
\verb|two pi|. \ans \verb|REAL two pi = 2 * pi|
'
\end{exercise}

\Section{Exponentiation}{form-exp}
If you want to compute the value of \verb|3*3*3*3| you can do so
using the multiplication operator, but it would be clearer and faster
if you used the exponentiation operator \ixtt{**}.  The mode of its
left operand can be either \verb|REAL| or \verb|INT|, but its right
operand must have mode \verb|INT|. If both its operands have the mode
\verb|INT|, the yield will have mode \verb|INT| (in this case the
right operand must not be negative), otherwise the yield will have
mode \verb|REAL|.  Thus the formula \verb|3**4| yields the value
\verb|81|, but \verb|3.0**4| yields the value \verb|81.0|.  Its
priority is~8.  In a formula involving exponentiation as well as
multiplication or division, the exponentiation is elaborated first.
For example, the formula \verb|3*2**4| yields \verb|48|, not
\verb|1296|.

Every \hx{dyadic}{operator!dyadic} operator has a priority of
between 1 and~9 inclusive, and all
\hx{monadic}{operator!monadic} operators bind more tightly than
all dyadic operators.  For example, the formula \verb|-2**2| yields
\verb|4|, not \verb|-4|.  Here the monadic minus is elaborated first,
followed by the exponentiation.

\begin{exercise}
\item Given these declarations:
\begin{verbatim}
   INT two = 2, m2 = -2;
   REAL x = 3.0 / 2.0, y = 1.0
\end{verbatim}
\noindent
what is the value and mode yielded by the following formul{\ae}?
\begin{subex}
\item \verb|two ** -m2| \subans \verb|4| \verb|INT|
'
\item \verb|x ** two + y ** two| \subans \verb|3.25| \verb|REAL|
'
\item \verb|3 * m2 ** two| \subans \verb|12| \verb|INT|
'
\end{subex}
\end{exercise}

\Section{Mixed arithmetic}{form-mix}
Up to now, the four basic \hx{arithmetic}{arithmetic!mixed}
operators have always had operands of the same modes.  In practice,
it is quite surprising how often you want to compute something like
\verb|2 * 3.0|.  Well, fortunately, the dyadic operators \verb|+|,
\verb|-|, \verb|*| and \verb|/| (but not \verb|%|) are also defined
for \hx{mixed modes}{operator!mixed modes}.  That is, any combination of
\verb|REAL| and \verb|INT| can be used.  With mixed modes the yield
is always \verb|REAL|.  Thus the following formul{\ae} are all valid:
\begin{verbatim}
   1+2.5    3.1*-4    2*3.5**3    2.4-2
\end{verbatim}
\noindent
The \ix{priority} of the mixed-mode operators is unchanged. As we
shall see later, the priority relates to the operator symbol rather
than the flavour of the operator in use.

\Section{Order of elaboration}{form-order}
Even though the \ix{order of elaboration} is dependent on the
priority of operators, it is often convenient to change the order.
This can be done by inserting \ix{parentheses} \ixtt{(} and \ixtt{)}
(or \ixtt{BEGIN} and \ixtt{END}): the formula inside the parentheses is
evaluated first.  Here are two formul{\ae} which differ only by the
insertion of parentheses:
\begin{verbatim}
   3 * 4 - 2
   3 *(4 - 2)
\end{verbatim}
\noindent
The first has the value \verb|10|, and the second \verb|6|.
Parentheses can be \hx{nested}{nesting} to any
\hx{depth}{parentheses!nesting of}.
\begin{verbatim}
   REAL a = (3*a3*(xmin+eps1)**2)/4;
   REAL alpha g=(ymax - ymin)/(xmax - xmin);
   INT p=BEGIN 2 * 3**4 % (13-2**3) END - 4.0
\end{verbatim}
\noindent
It is uncommon to find \verb|BEGIN| and \verb|END| in short
formul{\ae}. If you use \verb|BEGIN| at the start of a formula, you
must use \verb|END| to complete it even though these symbols and
parentheses are equivalent.

\Section{Changing the mode}{form-change}
We have seen that in a strong context, a value of mode \verb|INT| can
be coerced by \hx{widening}{coercion!widening} to a value of mode
\verb|REAL|. What about the other way round? Is it possible to coerce
a value of mode \verb|REAL| to a value of mode \verb|INT|?
Fortunately, it is impossible using coercion. The reason behind this
is related to the fact that real numbers can contain fractional
parts.  In replacing an integer by a real number there is no
essential change in the value, but when a real number is changed to
an integer, in general the \ix{fractional part} will be lost.  It is
undesirable that data should be lost without the programmer noticing.

If you want to convert a \verb|REAL| value to an \verb|INT|, you must
use one of the operators \ixtt{ROUND} or \ixtt{ENTIER}. The operator
\verb|ROUND| takes a single operand of mode \verb|REAL| and yields an
\verb|INT| whose value is the operand rounded to the nearest integer.
Thus \verb|ROUND 2.7| yields \verb|3|, and \verb|ROUND 2.2| yields
\verb|2|.  The same rule applies with negative numbers, thus
\verb|ROUND -3.6| yields \verb|-4|.  At the half way case, for
example, \verb|ROUND 2.5|, the value is \hx{rounded}{rounding}
away from zero if the whole number part is odd, and rounded toward
zero if it is even (zero, in this case, is taken to be an even
number). This ensures that rounding errors over a large number of
cases tend to cancel out.

The operator \verb|ENTIER| (French for ``whole'') takes a \verb|REAL|
operand and likewise yields an \verb|INT| result, but the yield is
the largest integer equal to or less than the operand. Thus
\verb|ENTIER 2.2| yields \verb|2|, \verb|ENTIER -2.2| yields
\verb|-3|.

The operator \ixtt{SIGN} can also be used with a \verb|REAL| operand.
Its yield has mode \verb|INT| with the same values as before, namely:
\verb|-1| if the operand is negative, \verb|0| if it is zero, and
\verb|+1| if it is positive. We shall see in subsequent chapters that
this property of \verb|SIGN| can be useful.

\begin{exercise}
\item What is the value and mode of the yield of each of the following
formul{\ae}?
\begin{subex}
\item \verb|ROUND(3.0 - 2.5**2)| \subans \verb|-3| \verb|INT|
'
\item \verb|ENTIER -4.5 + ROUND -4.5| \subans \verb|-9| \verb|REAL|
'
\item \verb|SIGN(ROUND 3.6 / 2.0) * 2.0| \subans \verb|2.0|
\verb|REAL|
'
\end{subex}
\item What is the value of the formula
\begin{verbatim}
(ENTIER -2.9 + 3**2)/4.0
\end{verbatim}
\indent \ans \verb|1.5|
'
\end{exercise}

\Section{Miscellaneous operators}{form-miscop}
The operators \ixtt{MAX} and \ixtt{MIN} are defined for any
combination of \verb|INT| and \verb|REAL| operands and yield the
maximum, or minimum, of two values. They can also be combined in the
same formula:
\begin{verbatim}
   INT max min = 345 MAX 249 MIN 1000
\end{verbatim}
\noindent
which yields \verb|345|.  Like \verb|+|, \verb|-|, and \verb|*|, they
only yield a value of mode \verb|INT| if both their operands are
\verb|INT|.  Otherwise, they yield a value of mode \verb|REAL|.  They
both have a priority of~9.

\Section{Operators using \texttt{CHAR}}{from-opchar}
This chapter has been rather heavy on arithmetic up to now.  You
might wonder whether operators can have operands of mode \verb|CHAR|.
The answer is yes.  Indeed, the \ixtt{+} and \ixtt{*} operators are
so declared, and we shall meet them in chapter~3. There are two
monadic operators which involve the mode \verb|CHAR|.  The operator
\ixtt{ABS} (which we have already met) can take a \verb|CHAR| operand
and yields the integer corresponding to that character.  For example,
\verb|ABS "A"| yields \verb|65| (the number associated with the
letter \verb|"A"| as defined by the \ix{ASCII} standard).  The
identifier \ixtt{max abs char} is declared in the
\ix{standard prelude} with the value \verb|255|.  Conversely, we can
convert an
integer to a character using the monadic operator \ixtt{REPR}.  The
formula
\begin{verbatim}
   REPR 65
\end{verbatim}
\noindent
yields the value \verb|"A"|.  \verb|REPR| can act on any integer in
the range \verb|0| to \verb|max abs char|.  \verb|REPR| is of
particular value in allowing access to control characters.  For
example, the tab character is declared in the standard prelude as
\ixtt{tab ch}. Consult section \hyref{stan-enqchar} for the details.

\Section{\texttt{print} revisited}{form-print}
In chapter 1, we used the \ixtt{print} phrase to convert internal
values to external characters.  We ought to say what \verb|print| is
and how it works, but we don't yet know enough about the language.
Just use it for the moment, and we shall learn more about it later.

Besides being able to convert internal values to external characters,
\verb|print| can take two \hx{parameters}{parameter} (see chapter~6
for the low-down on parameters) which can be used to format your
output.  \ixtt{newline} will cause following output to be displayed
on a new line, and \ixtt{newpage} will emit a form-feed character
(\verb|REPR 12|).  \verb|newline| and \verb|newpage| will be
described in detail in section \hyref{stan-trlay}.

If you want to print the characters emitted by your Algol~68 programs
you can use \ix{file redirection} to redirect your
output to a file, which you can later copy to the printer.  For
example, suppose you have compiled a program called \verb|tt|.  To
redirect its output to a file called \verb|tt.res|, which you can
later copy to the printer, you issue the command
\begin{verbatim}
   tt > tt.res
\end{verbatim}
\noindent
at the command line. Alternatively, you send the output directly to
the printer using the command
\begin{verbatim}
   tt | lpr
\end{verbatim}
\noindent
at the command line. Try compiling and running the following program:
\begin{verbatim}
   PROGRAM tt CONTEXT VOID
   USE standard
   BEGIN
      print(newpage);
      INT a = ENTIER (3.6**5);
      REAL p = 4.3 / 2.7;
      print(a);  print(newline);
      print(b);  print(newline)
   END
   FINISH
\end{verbatim}

\Section{Summary}{form-summ}
Operators combined with operands are called formul{\ae}. Operators are
monadic or dyadic. Monadic operators take a single operand, bind more
tightly than dyadic operators and when combined are elaborated from
right to left. Dyadic operators take two operands and have a priority
of 1 to~9. Successive dyadic operators having the same priority are
elaborated from left to right. Parentheses, or \verb|BEGIN| and
\verb|END|, may be used to alter the order of elaboration.

A summary of all the operators described in this chapter, together
with their priorities, can be found in chapter~13.

Here are some exercises which test you on what you have learned in
this chapter.  The exercises involving \verb|ABS| and \verb|REPR|
will need to be written as small programs and compiled and run.  In
fact, it would be a good idea to write all the answers as small
programs (or incorporate them all in one large program).  Don't
forget to use the \verb|print| phrase with \verb|newline| and
\verb|newpage| to separate your output.

\begin{exercise}
\item The following declarations are assumed to be in force for
these exercises:
\begin{verbatim}
   INT i = 13, j = -4, k = 7;
   CHAR s = "s", t = "T";
   REAL x = -2.4, y = 2.7, z = 0.0
\end{verbatim}
\noindent
What is the value of each of the following formul{\ae}?
\begin{subex}
\item \verb|(2 + 3) * (3 - 2)| \subans \verb|5|
'
\item \verb|j+i-k| \subans \verb|2|
'
\item \verb|3*ABS s| \subans \verb|345|
'
\item \verb|ABS"t"-ABS t| \subans \verb|32|
'
\item \verb|REPR(k**2)| \subans \verb|"1"|
'
\item \verb|ROUND(x**2-y/(x+1))| \subans \verb|8|
'
\item \verb|z**9| \subans \verb|0.0|
'
\end{subex}
\item Because of the kind of arithmetic performed by the compiler,
division of values of mode \verb|REAL| by zero does not cause a
program to fail (but see section \hyref{stan-fpov}). Write a program
containing the phrases \verb|REAL z=0.0/0.0;| and \verb|REAL iz=1/0;|
and see what happens.  In practice, it's probably a good idea to
check for division by zero. \ans The first \verb|print| phrase
displays
\begin{verbatim}
   $0.0000000000000000$
\end{verbatim}
\noindent
(16 zeros) and the second displays \verb|+infinity|.
'
\item Now try the phrase \verb|print(1%0)|. \ans The compiler detects
the error and rejects it.
'
\item What is wrong with the following formul{\ae}?
\begin{subex}
\item \verb|[4-j]*3| \subans The brackets should be replaced with
parentheses.
'
\item \verb|(((3-j)*x+3)*x+5.6| \subans There are more opening than
closing parentheses. The first opening parenthesis should be deleted.
'
\item \verb|ROUND "e"| \subans The operator \verb|ROUND| has not been
declared to use an operand with mode \verb|CHAR|.
'
\item \verb|ENTIER 4 + 3.0| \subans The operator \verb|ENTIER| has
not been declared for use with an operand with mode \verb|INT|. 
'
\end{subex}
\end{exercise}
