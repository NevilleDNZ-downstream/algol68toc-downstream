% $Log: chapter5.tex,v $
% Revision 1.4  2002-06-20 11:49:28  sian
% mm removed, padding removed, ca68 added
%
\Chapter{Names}{chapv}
Previous chapters dealt with values that have always been known when
the program was written.  If a program is to be able to react to its
environment, it must be able to convert external values into internal
values and then manipulate them.  Analogous to \verb|print|, the
conversion can be done by \ixtt{read} which constructs internal
values from external character sequences. In order to manipulate such
converted values, we need some way of referring to them. Algol~68 can
generate values which can refer to other values.  This kind of
value is called a \bfix{name}. Although a name has a value, it is
quite different from the value referred to. The difference is rather
like your name: your name refers to you, but is quite distinct from
you.

For example, suppose \verb|read| is presented with the character
sequence ``\verb|123G|'' and is expecting an integer. \verb|read| will
convert the digits into the number ``one hundred and twenty-three'',
held in a special internal form called
\hx{``2's-complement binary''}{2's-complement binary}.
To manipulate that value, a name must be generated to refer to it.
The mode of a name is called a ``reference mode''.

A name which can refer to a value of mode \verb|INT| is said to have
the mode \verb|REF INT|. Likewise, we can create names with modes
\begin{verbatim}
   REF BOOL    REF[]CHAR    REF[,]REAL
\end{verbatim}
\noindent
As you can see, \verb|REF| can precede any mode.  It can also include
a mode already containing \verb|REF|.  Thus it is possible to
construct modes such as
\begin{verbatim}
   REF REF INT
   REF[]REF REAL
   REF[]REF[]CHAR
   REF REF REF BOOL
\end{verbatim}
\noindent
but we shall defer discussion of these latter modes to chapter~11.

Names are created using \hx{\textbf{generators}}{generator}.  There
are two kinds of generator: \hx{local}{generator!local} and
\hx{global}{generator!global}.  The extent to which a name is valid
is called its \bfix{scope}.  The scope of a local
\hx{name}{local!name} is restricted to the smallest enclosing clause
which contains declarations.  The scope of a global name extends to
the whole program.  In general, values have scope, identifiers have
\ix{range}. We shall meet global generators in chapters~6 and~11.

The phrase \verb|LOC INT| generates a name of mode \verb|REF INT|
which can refer to a value of mode \verb|INT|.\footnote{Historically,
programmers were more interested in the value referred to than the
name (Algol~68 was the first language to distinguish clearly between
a name and the value referred to), so the generator is followed by
the mode of the value to which the name will refer.} The \ixtt{LOC}
stands for local. It is quite reasonable to write the phrase
\begin{verbatim}
   read(LOC INT)
\end{verbatim}
\noindent
Unfortunately, the created name is an anonymous name in the sense
that it has no identifier so that once the \verb|read| has completed,
the name disappears. We need some way of linking an identifier with
the generated name so that we can access the name after \verb|read|
has finished. This is done with an identity declaration. Here is an
identity \hx{declaration}{declaration!identity!LOC@\texttt{LOC}} with
a local \hx{generator}{REF@\texttt{REF}}:
\begin{verbatim}
   REF INT a = LOC INT
\end{verbatim}
\noindent
The value identified by \verb|a| has the mode \verb|REF INT| because
the phrase \verb|LOC INT| generates a name of mode \verb|REF INT|. Thus
it is a name, and it can refer to a value (as yet undefined) of mode
\verb|INT| (the value referred to always has a mode of one less
\verb|REF|). So now, we can write
\begin{verbatim}
   read(a)
\end{verbatim}
\noindent
After that phrase has been elaborated, \verb|a| identifies a name which
now refers to an integer.

Names can also be declared using a predeclared name on the right-hand
side of the identity declaration. Here is another identity declaration
using \verb|a|:
\begin{verbatim}
   REF INT b = a
\end{verbatim}
\noindent
In this declaration, \verb|b| has the mode \verb|REF INT| so it
identifies a name. \verb|a| also has the mode \verb|REF INT| and
therefore also identifies a name. The identity declaration makes
\verb|b| identify the same name as \verb|a|. This means that if the
name identified by \verb|a| refers to a value, then the name identified
by \verb|b| (the same name) will always refer to the same value.

\Section{Assignment}{name-assign}
The process of causing a name to refer to a value is called
\bfix{assignment}. Using the identifier declared above, we
can write
\begin{verbatim}
   a:=3
\end{verbatim}
\noindent
We say ``a assign 3''.  Note that the mode of the name identified by
\verb|a| is \verb|REF INT|, and the mode of the denotation \verb|3|
is \verb|INT|.  After the assignment, the name identified by \verb|a|
refers to the value denoted by \verb|3|.

Suppose now we want the name identified by \verb|a| to refer to the
value denoted by \verb|4| (this may seem pedantic, but as you will
see below, it is necessary to distinguish between the denotation of a
value and that value itself).  We write
\begin{verbatim}
   a:=4
\end{verbatim}
\noindent
Let us juxtapose these two assignments:
\begin{verbatim}
   a:=3
     ;
   a:=4
\end{verbatim}
\noindent
If you look carefully at the two assignments, a number of things
spring to mind. Firstly, an assignment consists of three parts: on
the left-hand side is an identifier of a name, in the middle is the
\bfix{assignment token}, and on the right-hand side is a denotation.
Secondly, the left-hand side of the two assignments is the same
identifier: \verb|a|.  Since the identifier is the same, the value
must be the same.\footnote{Provided that both identifiers appear in
the same range.}  That is, in the two assignments, \verb|a| is
synonymous with a value which \textit{does not change}. The value is a
\ix{name} and has the mode \verb|REF INT| (in this case).  Thus the
value of the left-hand side of an assignment is a name.

Thirdly, the values on the right-hand side of the two assignments
differ. Firstly, \verb|a| is assigned the value denoted by \verb|3|,
then (after the \ix{go-on} symbol), \verb|a| is assigned
the value denoted by \verb|4|.

After the second assignment, \verb|a| refers to \verb|4|.  Of course,
when we say ``\verb|a| refers to'', we mean ``the name identified by
\verb|a| refers to''. What has happened to the value \verb|3|? To
understand this, we need to look a little more closely at what we
mean by the value \verb|3|. The denotation \verb|3| represents the
number three. Now, of course, the number three exists independently
of a computer program. When the digit \verb|3| is elaborated in an
Algol~68 program, an \bfix{instance} of the number three is created.
Likewise, elaborating the digit \verb|4| creates an instance of the
number four.  When \verb|a| is assigned an instance of the value
four, the instance of the value three disappears. This property of
assignment is very important. Because an assignment causes data to
disappear, it is dangerous to use. You have to be careful that the
data which disappears is not data you wanted to keep. So the instance
of a value can disappear, but the value still exists (like the number
three).

It is worth reiterating that however many times a name is assigned a
value, the value of the name remains unchanged. It is the value
referred to which is superseded. Outwith the realm of computers, if
an individual is assigned to a department of an organisation, clearly
the department hasn't changed. Only its members have changed.

When an identifier for a name has been declared, the name can be made
to refer to a value immediately after the declaration.  For
\hx{example}{REF@\texttt{REF}}
\begin{verbatim}
   REF REAL x = LOC REAL := pi
\end{verbatim}
\noindent
where \verb|pi| is the value declared in the \ix{standard prelude}.
\verb|LOC REAL| generates a name of mode \verb|REF REAL|.

The right-hand side of an assignment is a strong
\hx{context}{context!strong} so \hx{widening}{coercion!widening} is
allowed. Thus we can write
\begin{verbatim}
   x:=3
\end{verbatim}
\noindent
where the \verb|3| is widened to \verb|3.0| before being assigned to
\verb|x|. In reality, the value denoted by \verb|3| is not changed to
the value denoted by \verb|3.0|: it is replaced by the new value.
There is an important principle here. It is called the
``\hx{principle of value integrity}{value integrity!principle of}'':
once an
instance of a value has been created, it does not change until such
time as it disappears.  Thus, in Algol~68, every value is a
\ix{constant}.  Every \ix{coercion} defined in Algol~68
replaces a \ix{value} of one mode with a related value of
another mode.
\newpage

\Subsection{Copying values}{name-copy}
Here is another identity declaration with an \ix{initial assignment}:
\begin{verbatim}
   REF INT c = LOC INT := 5
\end{verbatim}
\noindent
Using the identifier \verb|a| declared earlier, we can write
\begin{verbatim}
   a:=c
\end{verbatim}
\noindent
and say ``a assign c''. The name on the left-hand side of the
assignment has mode \verb|REF INT|, so a value which has mode
\verb|INT| is required on the right-hand side, but what has been
provided is a name with mode \verb|REF INT|. Fortunately, there is a
coercion which replaces a name with the value to which it refers. It
is called \hx{\textbf{dereferencing}}{coercion!dereferencing} and is
allowed in a \hx{strong context}{context!strong}.  In the above
assignment, the name identified by \verb|c| is dereferenced yielding
an instance of the value five which is a copy of the instance
referred to by \verb|c|.  That new instance is assigned to \verb|a|.
It is important to remember that the process of dereferencing yields
a \emph{new} instance of a value.

Try the following program:
\begin{verbatim}
   PROGRAM assign CONTEXT VOID
   USE standard
   BEGIN
      REF INT a = LOC INT,
              b = LOC INT:=7;
      print(("b=",b,newline));
      print("Please key 123G:"); read(b);
      a:=b;
      print(("a now refers to",a,newline,
             "b now refers to",b,newline))
   END
   FINISH
\end{verbatim}
\noindent
This should convince you that dereferencing involves copying.

Every construct in Algol 68 has a value except an identity
declaration. We said above that the value of the left-hand side of an
assignment is a name. In fact, the value of the whole of the
assignment is the value of the left-hand side. Because this is a
name, it can be used on the right-hand side of another assignment.
For example:
\begin{verbatim}
   a:=b:=c
\end{verbatim}
\noindent
You should note that an assignment is not an operator. The assignments
are performed from \emph{right to left}: firstly, \verb|c| is
dereferenced and the resulting value assigned to \verb|b|. Then
\verb|b| is dereferenced and the resulting value is assigned to
\verb|a|.

\Subsection{Assigning operators}{name-assops}
The following assignment
\begin{verbatim}
   a:=a
\end{verbatim}
\noindent
does not do anything useful, but serves to remind us that the name
identified by \verb|a| on the right-hand side of the assignment is
dereferenced, and the resulting value is assigned to \verb|a|. However,
\verb|a| now refers to a new instance of the value it previously
referred to and the previous instance has now disappeared.

Now consider the phrases
\begin{verbatim}
   c:=5; a:=c+1
\end{verbatim}
\noindent
The right-hand side of the second assignment is now a formula. The
name identified by \verb|c| is now in a firm
\hx{context}{context!firm} (it is the left-operand of the \verb|+|
operator).  Fortunately, \hx{dereferencing}{coercion!dereferencing}
is also allowed in a firm context.  Thus the value of \verb|c| (a
name with mode \verb|REF INT|) is replaced in the formula by a copy
of the value to which it refers (\verb|5|), which is added to
\verb|1|, and \verb|a| is assigned the new value (\verb|6|).  We say
``a is assigned c plus one''.

What about the phrase
\begin{verbatim}
   a:=a+1
\end{verbatim}
\noindent
In exactly the same way as the previous phrase, the name on the
right-hand side is dereferenced, the new value created is added to
\verb|1|, and then the same name is assigned the new value.

One of the features of assignment is that the elaboration of the two
sides is performed \hx{collaterally}{elaboration!collateral}. This
means that the order of elaboration is undefined.  This does not matter
in the last example because the value of the name identified by
\verb|a| is the same on the two sides of the assignment. Remember that
the value of \verb|a| is a name with mode \verb|REF INT|.  It is the
value to which \verb|a| referred which was superceded.

Assignments of this kind are so common that a special operator has been
devised to perform them. The above assignment can be
written
\begin{verbatim}
   a+:=1
\end{verbatim}
\noindent
and is read ``a plus-and-assign one''. The operator has the
alternative representation \ixtt{PLUSAB}.\footnote{\texttt{PLUSAB}
stands for ``plus-and-becomes''. When Algol 68 was first designed,
people were more concerned with the values referred to than the
names, so \texttt{PLUSAB} was intended to describe what happens to
the value referred to. Bearing in mind the principle of value
\protect\hx{integrity}{principle of value integrity}, the value
referred to by \texttt{a} does not become anything, but is replaced by
its value plus 1.} Note that the left-hand operand must be a name.
The right-hand operand must be any \ix{unit} which yields a value of
the appropriate mode in a \hx{firm}{context!firm} context.

The operator \verb|+:=| is defined for a left-operand of mode
\verb|REF| \verb|INT| or \verb|REF| \verb|REAL|, and a right-operand
of mode \verb|INT| or \verb|REAL| respectively.  The yield of the
operator is the value of the left-operand (the name).  If the
left-operand has mode \verb|REF REAL|, the right-operand can also
have mode \verb|INT|.  No widening occurs in this case, the operator
having been declared for operands having these modes. Because the
operator yields a name, that name can be used as the operand for
another assigning operator. For example
\begin{verbatim}
   x +:= 3.0 *:= 4.0
\end{verbatim}
\noindent
which results in \verb|x| referring to \verb|4.0*(x+3.0)|. The formula
is elaborated in left-to-right order because the operators have the
same priority. The operators are more efficient than writing out the
assignments in full.

There are four other operators like \verb|+:=|.  They are \verb|-:=|,
\verb|*:=|, \verb|/:=|,
\verb|%:=| and \verb|%*:=|.  Their alternative
representations are respectively \ixtt{MINUSAB}, \ixtt{TIMESAB},
\ixtt{DIVAB}, \ixtt{OVERAB} and \ixtt{MODAB}.  The operators
\verb|OVERAB| and \verb|MODAB| are only declared for operands with
modes \verb|REF| \verb|INT| and \verb|INT|.  The
\ix{priority} of all the operators is~1.

The \ix{assignment operators} are operators, not assignments
(although they perform an assignment), so that the previous example
is not an assignment, but a formula.

The right-hand side of an assignment can be any \ix{unit} which
yields a value whose mode has one less \verb|REF| than the mode of the
name on the left-hand side.  Names whose mode contains more than one
\verb|REF| will be considered in chapter~11.

\begin{exercise}
\item The following identity declarations
\begin{verbatim}
   REF CHAR s = LOC CHAR,
   REF INT  i = LOC INT,
   REF REAL r = LOC REAL
\end{verbatim}
\noindent
hold in this and the following exercises.\footnote{The
\protect\ix{a68toc} compiler requires that you write semicolons
instead of commas to separate these three declarations.} What is the
mode of \verb|i|? \ans \verb|REF INT|
'
\item After the assignment \verb|r:=-2.7| has been elaborated, what is
the mode of the value referred to by \verb|r|? \ans \verb|REAL|
'
\item What is wrong with the assignment \verb|i:=r| and how would you
correct it? \ans The right-hand side of the identity declaration
should yield a value of mode \verb|INT|. Insert \verb|ENTIER| or
\verb|ROUND| before the \verb|r|.
'
\end{exercise}

\Section{Assignments in formul\ae}{name-assform}
Since an assignment yields a name, it can be used in a formula.
However, the assignment must be converted into an enclosed
\hx{clause}{clause!enclosed} (using parentheses or \verb|BEGIN|
and \verb|END|) ensuring that the assignment is elaborated first. For
example, in
\begin{verbatim}
   3*(a:=c+4)+2
\end{verbatim}
\noindent
if \verb|c| refers to \verb|3|, the value of the formula will be
\verb|23| with mode \verb|INT|, \verb|a| will refer to \verb|7|, the
value of the assignment is a name of mode \verb|REF INT| and \verb|c|
will still refer to \verb|3|. Remember that assignment is not an
operator.

Here is an example of two assignments in a conditional clause:
\begin{verbatim}
   IF a<2 THEN x:=3.2 ELSE x:=-5.0 FI
\end{verbatim}
\noindent
This can be written with greater efficiency as
\begin{verbatim}
   x:=IF a < 2 THEN 3.2 ELSE -5.0 FI
\end{verbatim}

The left-hand side of an assignment has a
\hx{\textbf{soft}}{context!soft} context. In a soft context,
dereferencing is not allowed (it is the only context in which
dereferencing is not allowed). In the following phrase, the
conditional \hx{clause}{clause!conditional} on the left yields a
name which is then assigned the value of the right-hand side:
\begin{verbatim}
   IF a < 2 THEN x ELSE y FI := 3.5
\end{verbatim}
\noindent
In the next assignment, a conditional clause appears on both sides of the
assignment:
\begin{verbatim}
   (a<2|x|y):=(b<2|x|y)
\end{verbatim}
\noindent
The result depends on the values referred to by both \verb|a| and
\verb|b| as much as on the values referred to by both \verb|x| and
\verb|y|.

\begin{exercise}
\item What is wrong with the following program fragment?
\begin{verbatim}
   REF REAL x = LOC REAL,
            y = LOC REAL:=3.5;
   y:=4.2+x
\end{verbatim}
\indent\ans No value has been assigned to \verb|x| when the second
assignment is elaborated.
'
\item If \verb|x| refers to \verb|3.5| and \verb|y| refers to
\verb|-2.5|, what is the mode and value yielded by the following
phrases:
\begin{subex}
\item \verb|x:=-y| \subans A name with mode \verb|REF REAL|.
'
\item \verb|ABS y| \subans The real number denoted by \verb|2.5| with
mode \verb|REAL|.
'
\end{subex}
\item What does \verb|x| refer to after
\begin{verbatim}
   x:=1.5;  x PLUSAB 2.0 DIVAB 3.0
\end{verbatim}
\noindent
(try it in a small program). \ans \verb|1.166666|$\ldots$
'
\end{exercise}

\Section{Multiple names}{name-mult}
Here is an identity declaration for a name which can refer to a
\ix{multiple}\hx{:}{REF@\texttt{REF}}:
\begin{verbatim}
   REF[]INT i7 = LOC[1:7]INT
\end{verbatim}
\noindent
There are two things to notice about this declaration.  Firstly, the
mode on the left-hand side is known as a \bfix{formal-declarer}.  It
says what the mode of the name is, but it says nothing about how many
elements there will be in any multiple to be assigned, nor what its
bounds will be.  All the identity declarations for multiples in
chapter~3 used formal-declarers on the left-hand side.  In fact, only
formal-declarers are used on the left-hand side of \emph{any}
identity \hx{declaration}{declaration!identity}.

Secondly, the generator on the right-hand side is an
\bfix{actual-declarer}. It specifies how many elements can be
assigned. In fact, the \ix{trimmer} represents the \ix{bounds} of the
multiple which can be assigned.  If the lower bound is \verb|1| it
may be omitted, so the above declaration could well have been written
\begin{verbatim}
   REF[]INT i7 = LOC[7]INT
\end{verbatim}
\noindent
which can be read as ``ref row of int i7 equals loc row of seven
int''.  The bounds of a multiple do not have to start from \verb|1|
as we saw in chapter~3. In this identity declaration
\begin{verbatim}
   REF[]INT i7 at 0 = LOC[0:6]INT
\end{verbatim}
\noindent
the bounds of the multiple will be \verb|[0:6]|.

\Section{Assigning to multiple names}{name-assmult}
We can assign values to the elements of a multiple either
individually or collectively.

\Subsection{Individual assignment}{name-indiv}
You may remember from chapter 3 that we can access an individual
element of a multiple by specifying the subscript(s) of that element.
For example, suppose that we wish to access the third element of
\verb|i7| as declared in the last section. The rules of the language
state that a subscripted element of a multiple name is itself a name.
In fact, the elaboration of a slice of a multiple name creates a new
name. Thus the mode of \verb|i7[3]| is \verb|REF INT|. We can assign a
value to \verb|i7[3]| by placing the element on the left-hand side of
an assignment:
\begin{verbatim}
   i7[3]:=4
\end{verbatim}
\noindent
Unless you define a new identifier for the new name, it will cease to
exist after the above assignment has been elaborated (see below for
examples of this).

Since each element of \verb|i7| has an associated name (created by
slicing) of mode \verb|REF INT|, it can be used in a formula:
\begin{verbatim}
   i7[2]:=3*i7[i7[1]] + ENTIER(4.0/i7[3])
\end{verbatim}
\noindent
As you can see, an element was used to compute a \ix{subscript}. It
has been presumed that the value obtained after dereferencing lies
between \verb|1| and \verb|7| inclusive. If this were not so, a
run-time \hx{error}{error!run-time} would be generated. In the
above assignment, all three elements on the right-hand side of the
assignment would be \hx{dereferenced}{coercion!dereferencing}
before being used in the formula.  Note that subscripting (or slicing
or trimming) binds more tightly than any operator.  Thus, in the last
term in the above example, \verb|i7| would be sliced first, then the
yielded name dereferenced, and finally, the new value would be
divided into \verb|4.0|.

Here is a \verb|FOR| \hx{loop}{clause!loop} which assigns a value to
each element of \verb|i7| individually:
\begin{verbatim}
   FOR e FROM LWB i7 TO UPB i7
   DO
      i7[e]:=e**3
   OD
\end{verbatim}
\noindent
Using the bounds interrogation operators is useful because:
\begin{enumerate}
\item The fact that the lower bound of \verb|i7| is \verb|1| is
masked, but the formula \verb|LWB i7| ensures that the
correct value is used.
\item If the bounds of \verb|i7| are changed when the program is
being maintained, the loop clause can remain unchanged. This simplifies
the maintenance of Algol 68 programs.
\item The compiler can omit bounds checking. For large multiples,
this can speed up processing considerably.
\end{enumerate}

Here is a program which uses a name whose mode is\linebreak
\verb|REF[]BOOL|. It computes all the prime numbers less than 1000
and is known as \ix{Eratosthenes' Sieve}:
\begin{verbatim}
   PROGRAM sieve CONTEXT VOID
   USE standard
   BEGIN
      INT size = 1000;

      REF[]BOOL flags = LOC[2:size]BOOL;

      FOR i FROM LWB flags TO UPB flags
      DO
         flags[i] := TRUE
      OD;

      FOR i FROM LWB flags TO UPB flags
      DO
         IF flags[i]
         THEN
            FOR k
            FROM 2*i BY --i TO UPB flags
            DO
               flags[k] := FALSE
               CO Remove multiples of i CO
            OD
         FI
      OD;

      FOR i FROM LWB flags TO UPB flags
      DO
         IF flags[i] THEN print((i,blank)) FI
      OD
   END
   FINISH
\end{verbatim}

\Subsection{Collective assignment}{name-coll}
There are two ways of assigning values collectively.  Firstly, it can
be done with a \hx{row-display}{row!display} or a \verb|[]CHAR|
denotation. For example, using the declaration of \verb|i7| above:
\begin{verbatim}
   i7:=(4, -8, 11, ABS "K",
        ABS TRUE, 0, ROUND 3.4)
\end{verbatim}
\noindent
Notice that the bounds of both \verb|i7| and the row-display are
\verb|[1:7]|. In the assignment of a multiple, the bounds of the
multiple on the right-hand side must match the bounds of the multiple
name on the left-hand side. If they differ, a fault is generated. If
the bounds are known at compile-time, the compiler will generate an
error message. If the bounds are only known at run-time (see section
\hyref{name-dyn} on dynamic names), a run-time error will be
generated. The \ix{bounds} can be changed using a trimmer or the
\verb|@| symbol (or \ixtt{AT}). See chapter~3 for details.

The second way of assigning to the elements of a multiple collectively
is to use an identifier of a multiple with the required bounds. For
example:
\begin{verbatim}
   []INT i3 = (1,2,3);
   REF[]INT k = LOC[1:3]INT := i3
\end{verbatim}
\noindent
The right-hand side has been assigned to the multiple name \verb|k|.

As mentioned above, parts of a multiple can be assigned using
\hx{slicing}{slice} or \ix{trimming}. For example, given the
\hx{declarations}{REF@\texttt{REF}}
\begin{verbatim}
   REF[,]REAL x = LOC[1:3,1:3]REAL,
              y = LOC[0:2,0:2]REAL
\end{verbatim}
\noindent
and the assignment
\begin{verbatim}
   x:=((1,2,3),
       (4,5,6),
       (7,8,9))
\end{verbatim}
\noindent
we can write
\begin{verbatim}
   y[2,0]:=x[3,2]
\end{verbatim}
\noindent
The multiple name \verb|y| is sliced yielding a name of mode
\verb|REF INT|. Then\footnote{But because the two sides of an
assignment are elaborated collaterally, the RHS might be elaborated
before the LHS or even in parallel.} the multiple name \verb|x| is
sliced also yielding a name of mode \verb|REF INT| which is then
dereferenced yielding a new instance of the value to which it refers
(\verb|8|) which is then assigned to the new name on the LHS of the
assignment.  Here is an identity-declaration which makes the new name
permanent:
\begin{verbatim}
   REF INT y20 = y[2,0];  y20:=x[3,2]
\end{verbatim}
\noindent
which has its uses (see below).

Here are some examples of slicing with (implied) multiple assignments:
\begin{verbatim}
   y     := x[@0,@0];
   y[2,] := x[ 1,@0];
   y[,1] := x[ 2,@0]
\end{verbatim}
\noindent
In the first example, the right-hand side is a slice of a name whose
mode is \verb|REF[,]REAL|. Because the slice has no trimmers its mode
is also \verb|REF[,]REAL|. Using the \verb|@| symbol, the lower bounds
of both dimensions are changed to \verb|0|, ensuring that the bounds of
the multiple name thus created match the bounds of the multiple name
\verb|y| on the left. After the assignment (and the dereferencing),
\verb|y| will refer to a copy of the multiple \verb|x| and the name
created by the slicing will no longer exist.

In the second \ix{assignment}, the multiple \verb|x| has
been sliced yielding a name whose mode is \verb|REF[]REAL|.  It refers,
in fact, to the first ``row'' of \verb|x|.  The \verb|@0| ensures that
the \ix{lower bound} of the second dimension of \verb|x|
is \verb|0|. The left-hand side yields a name of mode \verb|REF[]REAL|
which refers to the last ``row'' of the multiple \verb|y|.  The name on
the right-hand side is dereferenced. After the assignment \verb|y[2,]|
will refer to a copy of the first ``row'' of \verb|x| and the name
produced by the slicing will no longer exist.

In the third assignment, the second ``row'' of \verb|x| is assigned to
the second ``column'' of \verb|y|.  Again, the \verb|@0| construction
ensures that the lower bound of the second dimension of \verb|x| is
zero. After the assignment, the name created by the slicing
will no longer exist.

Notice how the two declarations for \verb|x| and \verb|y| have a common
\ix{formal-declarer} on the left-hand side, with a
comma between the two declarations. This is a common abbreviation. The
comma means that the two declarations are elaborated collaterally (and
on a parallel processing computer, possibly in parallel).

It was stated in the section on names that names can be put on the
right-hand side of an identity declaration.  This is particularly
useful for accessing elements of rows.  Consider the following:
\begin{verbatim}
   REF[]INT r = LOC[100]INT;

   FOR i FROM LWB r TO UPB r DO r[i]:=i*i OD;

   FOR i FROM LWB r TO UPB r-1
   DO
      IF   REF INT ri=r[i], ri1=ri[i+1];
           ri > ri1
      THEN ri:=ri1
      ELSE ri1:=ri
      FI
   OD
\end{verbatim}
\noindent
This is another example of \ix{optimisation}, but in
this case, we need names because the \verb|THEN| and \verb|ELSE|
clauses contain assignments. Both \verb|ri| and \verb|ri1| are used
thrice in the conditional \hx{clause}{clause!conditional}, but the
multiple \verb|r| is only subscripted twice in each loop. In the
condition following the \verb|IF|, both \verb|ri| and \verb|ri1| would
be dereferenced (but not in the identity declarations).  The values of
\verb|ri| and \verb|ri1| remain constant: the names are assigned new
values.  You can see from the identity declarations that the modes of
the names \verb|ri| and \verb|ri1| are both \verb|REF INT|.

Here is a program fragment which uses a \verb|REF[]REAL| identity
declaration for \hx{optimisation}{REF@\texttt{REF}}:
\begin{verbatim}
   REF[,]REAL m = LOC[3,4]REAL;   read(m);
   
   FOR i FROM 1 LWB m TO 1 UPB m
   DO
      REF[]REAL mi = m[i,];
      FOR j FROM LWB mi TO UPB mi
      DO
         REF REAL mij = mi[j];
         mij*:=mij
      OD
   OD;
   
   print((m,newline))
\end{verbatim}
\noindent
As you can see, \ixtt{read} behaves just like
\ixtt{print} in that a whole multiple can be read
at one go (see chapter~3 for the use of \verb|print| with multiples).
The only difference between the way \verb|read| is used and the way
\verb|print| is used is that the values for \verb|read| must be names
(or identifiers of names) whereas \verb|print| can use denotations or
identifiers of names or identifiers which are not names.

\begin{exercise}
\item After the assignments of \verb|x| to \verb|y| discussed above,
what is the final value of \verb|y| (careful)? \ans A name with mode
\verb|REF[,]REAL|.
'
\item Given these declarations
\begin{verbatim}
   REF[,]INT m = LOC[3:5,-2:0]INT,
   REF[]INT n = LOC[1:3]INT:=(1,2,3)
\end{verbatim}
\indent\begin{subex}
\item What is wrong with the assignment \verb|m[1,]:=n|? \subans The
bounds of the slice on the left-hand side of the assignment are
\verb|[-2:0]|, but the bounds of \verb|n| are \verb|[1:3]|. The
assignment will cause a run-time error.
'
\item How would you assign the second ``column'' of \verb|m| to its
third ``row''? \subans You could write \verb|m[5,]:=m[,-1]|, but it
is unlikely that you would get what you wanted because the second
column overlaps the third row. Here is a solution guaranteed to work:-
\begin{verbatim}
   []INT temp = m[,-1];
   m[5,]:=temp[@-2]
\end{verbatim}
'
\end{subex}

\item Modify Eratosthenes' Sieve to compute the
\ith{365}{th} prime. \ans There is no known formula
which will tell you how big the sieve must be to find the
\ith{365}{th} prime; you just have to guess. A sieve with \verb|size|
equal to 5000 suffices. You need a counter for the primes. The
complete program is:-
\begin{verbatim}
   PROGRAM sieve CONTEXT VOID
   USE standard
   BEGIN
      INT size=5000;
      REF[]BOOL flags = LOC[2:size]BOOL;

      FOR i FROM LWB flags TO UPB flags
      DO
         flags[i]:=TRUE
      OD;

      FOR i FROM LWB flags TO UPB flags
      DO
         IF flags[i]
         THEN
            FOR k FROM i*2 BY --i TO UPB flags
            DO
               flags[k]:=FALSE
               CO Remove multiples of i CO
            OD
         FI
      OD;

      REF INT count = LOC INT:=0;

      FOR i FROM LWB flags TO UPB flags
      DO
         IF flags[i] ANDTH (count+:=1)=365
         THEN print(i)
         FI
      OD
   END
   FINISH                                     
\end{verbatim}
'
\end{exercise}

\Section{Flexible names}{name-flex}
In the previous section, we declared mutliple names. The bounds of
the multiple to which the name can refer are included in the
generator. In subsequent assignments, the bounds of the new multiple
to be assigned must be the same as the bounds given in the generator.
In Algol~68, it is possible to declare names which can refer to a
multiple of any number of elements (including none) and, at a later
time, can refer to a different number of elements. They are called
\bfix{flexible} names. Here is an identity
\hx{declaration}{declaration!identity!FLEX@\texttt{FLEX}} for a
flexible \hx{name}{REF@\texttt{REF}}:
\begin{verbatim}
   REF FLEX[]INT fn = LOC FLEX[1:0]INT
\end{verbatim}
\noindent
There are several things to note about this declaration.  Firstly,
the mode of the name is not \verb|REF[]INT|, but
\texttt{REF FLEX[]INT}.  The \ixtt{FLEX} means that the \ix{bounds} of the
multiple to which the name can refer can differ from one assignment
to the next.  Secondly, the bounds of the name generated at the time
of the declaration are \verb|[1:0]|. Since the upper bound is less
than the lower bound, the multiple is said to be
\hx{\textbf{flat}}{multiple!flat}; in other words, it has no elements
at the time of its declaration\footnote{The Revised Report mentions a
``ghost element'' in this context (see section
\protect\hyref{gram-flex} for details)}.  Thirdly, \verb|FLEX| is
present on both sides of the identity declaration (but in the last
section of this chapter we shall see a way round that).

We can now assign multiples of integers to \verb|fn|:
\begin{verbatim}
   fn:=(1,2,3,4)
\end{verbatim}
\noindent
The bounds of the multiple to which \verb|fn| now refers are
\verb|[1:4]|. Again, we can write
\begin{verbatim}
   fn:=(2,3,4)
\end{verbatim}
\noindent
Now the bounds of the multiple to which \verb|fn| refers are
\verb|[1:3]|. We can even write
\begin{verbatim}
   fn:=7
\end{verbatim}
\noindent
in which the right-hand side will be \hx{rowed}{coercion!rowing} to yield a
one-dimen\-sional multiple with bounds \verb|[1:1]|, and
\begin{verbatim}
   fn:=()
\end{verbatim}
\noindent
giving bounds of \verb|[1:0]|.

In the original declaration of \verb|fn| the bounds were \verb|[1:0]|. The
compiler will not ignore any bounds other than \verb|[1:0]|, but will
generate a name whose initial bounds are those given. So the
\hx{declaration}{FLEX@\texttt{FLEX}}
\begin{verbatim}
   REF FLEX[]INT fn1 = LOC FLEX[1:4]INT
\end{verbatim}
\noindent
will cause \verb|fn1| to have the bounds \verb|[1:4]| instead of
\verb|[1:0]|.

The lower bound does not have to be \verb|1|. In this example,
\begin{verbatim}
   REF[]INT m1 = LOC[-1:1]INT;
   FOR i FROM LWB m1 TO UPB m1 DO m1[i]:=i+3 OD;
   REF FLEX[]INT f1 = LOC FLEX[1:0]INT := m1
\end{verbatim}
\noindent
the bounds of \verb|f1| after the initial assignment are \verb|[-1:1]|.

If a flexible name is sliced or trimmed, the resulting name is called
a \bfix{transient name} because it can only exist so long as the
flexible name stays the same size. Such names have a restricted use
to avoid the production of names which could refer to nothing. For
example, consider the declaration and assignation
\begin{verbatim}
   REF FLEX[]CHAR c1 = LOC FLEX[1:0]INT;
    c1:="abcdef";
\end{verbatim}
\noindent
Suppose now we have the declaration
\begin{verbatim}
   REF[]CHAR lc1=c1[2:4]; #WRONG#
\end{verbatim}
\noindent
followed by this assignment:
\begin{verbatim}
   c1:="z";
\end{verbatim}
\noindent
It is clear that \verb|lc1| no longer refers to anything meaningful.
Thus transient names cannot be assigned without being dereferenced,
nor given identifiers, nor used as parameters for a routine (whether
operator or procedure).  However there is nothing to prevent them
being used in an assignment.  For example,
\begin{verbatim}
   REF FLEX[]CHAR s=LOC[1:0]CHAR:=
      "abcdefghijklmnopqrstuvwxyz";
   s[2:7]:=s[9:14]
\end{verbatim}
\noindent
where the name yielded by \verb|s[9:14]| is immediately dereferenced.
Note that the bounds of a trim are fixed even if the value trimmed is a
flexible name. So the assignment
\begin{verbatim}
   s[2:7]:="abc"
\end{verbatim}
\noindent
would produce a run-time \hx{fault}{error!run-time}.

\begin{exercise}
\item The declaration
\begin{verbatim}
   REF FLEX[]CHAR s = LOC FLEX[1:0]CHAR
\end{verbatim}
\noindent
applies to the following:
\begin{subex}
\item What is the value of \verb|s|? \subans A name of mode
\verb|REF FLEX[]CHAR|.
'
\item After the assignment
\begin{verbatim}
   s:="aeiou"
\end{verbatim}
\noindent
what are the \ix{bounds} of \verb|s|? \subans 1 and 5.
'
\end{subex}
\end{exercise}

\Section{The mode \texttt{STRING}}{name-string}
The mode \ixtt{STRING} is defined in the \ix{standard prelude} as
having the same mode as the expression
\hx{\texttt{FLEX[1:0]CHAR}}{FLEX@\texttt{FLEX}}. That is, the
identity \hx{declaration}{declaration!identity!REF@\texttt{REF}}
\begin{verbatim}
   REF STRING s = LOC STRING
\end{verbatim}
\noindent
has exactly the same effect as the declaration
\begin{verbatim}
   REF FLEX[]CHAR s = LOC FLEX[1:0]CHAR
\end{verbatim}
\noindent
You will notice that although the mode \hx{indicant}{mode!indicant}
\verb|STRING| appears on both sides of the identity declaration for
\verb|s|, in the second declaration the bounds are omitted on the
left-hand side (the mode is a \ix{formal-declarer}) and kept on the
right-hand side (the \ix{actual-declarer}).  Without getting into
abstruse grammatical explanations, just accept that if you define a
mode like \verb|STRING|, whenever it is used on the left-hand side of
an identity declaration the compiler will ignore the bounds inherent
in its definition.

We can now write
\begin{verbatim}
   s:="String"
\end{verbatim}
\noindent
which gives bounds of \verb|[1:6]| to \verb|s|.  We can
\ix{slice} that row to get a value with mode \verb|REF CHAR|
which can be used in a formula. If we want to change the bounds of
\verb|s|, we must assign a value which yields a value of mode
\verb|[]CHAR| to the whole of \verb|s| as in
\begin{center}
\texttt{s:="Another string"\quad\textrm{or}\quad s:=s[2:4]}
\end{center}
Wherever \verb|[]CHAR| appears in chapter~3, it may be safely
replaced by \verb|STRING|. This is because it is only
\hx{names}{name!flexible} which are flexible so the flexibility of
\verb|STRING| is only available in \verb|REF STRING| declarations.

There are two operators defined in the \ix{standard prelude} which
use an operand of mode \verb|REF STRING|: \ixtt{PLUSAB}, whose left
operand has mode \verb|REF STRING| and whose right operand has mode
\verb|STRING| or \verb|CHAR|, and \ixtt{PLUSTO}, whose left operand
has mode \verb|STRING| or \verb|CHAR| and whose right operand has
mode \verb|REF STRING|.  Using the \ix{concatenation} operator
\ixtt{+}, their actions can be summarised as follows:
\begin{eqnarray*}
   \hbox{\tt a PLUSAB b}&\equiv&\hbox{\enskip a:=a+b}\\
   \hbox{\tt a PLUSTO b}&\equiv&\hbox{\enskip b:=a+b}
\end{eqnarray*}
Thus \verb|PLUSAB| concatenates \verb|b| onto the end of \verb|a|,
and \verb|PLUSTO| concatenates \verb|a| to the beginning of \verb|b|.
Their alternative representations are \verb|+:=| and \verb|+=:|
respectively. For example, if \verb|a| refers to \verb|"abc"| and
\verb|b| refers to \verb|"def"|, after \verb|a PLUSAB b|, \verb|a|
refers to \verb|"abcdef"|, and after \verb|a PLUSTO b|, \verb|b|
refers to \verb|"abcdefdef"| (assuming the \verb|PLUSAB| was
elaborated first).

\begin{exercise}
\item Write a program which declares a name with mode \verb|REF|
\verb|STRING| and then consecutively assigns the rows of characters
\verb|"ab"|, \verb|"abc"|, upto the whole alphabet and prints each
row on a separate line. Use a \verb|FOR| loop clause.  \ans \ %
\begin{verbatim}
   PROGRAM ex5 7 11 CONTEXT VOID
   USE standard
   BEGIN
      REF STRING ss = LOC STRING;
      FOR c FROM ABS "a" TO ABS "z"
      DO
         ss:="a"-REPR c;  print((ss,newline))
      OD
   END
   FINISH
\end{verbatim}
'
\item Declare a flexible name which can refer to a 2-dimen\-sional row
whose elements have mode \verb|REAL|.  Assign a one-dimensional row
whose elements are
\begin{verbatim}
   5.0 10.0 15.0 20.0
\end{verbatim}
\noindent
Write the \verb|print| phrase which will display each bound on the screen
followed by a space, all on one line. \ans \ %
\begin{verbatim}
   REF FLEX[,]REAL f=
      LOC FLEX[1:0,1:0]REAL;
   f:=(5.0,10.0,15.0,20.0);
   print((1 LWB f,1 UPB f,
          2 LWB f,2 UPB f))
\end{verbatim}
'
\end{exercise}

\Section{Reference modes in transput}{name-trans}
Wherever previously we have used a value of mode \verb|INT| with
\verb|print|, we can safely use a name with mode \verb|REF INT|, and
similarly with all the other modes (such as \verb|[,]REAL|). This is
because the \hx{parameters}{parameter} for \ixtt{print} (the
identifiers or denotations used for \verb|print|) are in a firm
\hx{context}{context!firm} and so can be
\hx{dereferenced}{coercion!dereferencing} before being used.

In the preamble to this chapter, \verb|print|'s counterpart
\ixtt{read} was mentioned. It is now time to examine \verb|read| more
closely. Generally speaking, values displayed with \verb|print| can
be input with \verb|read|. The main differences are that firstly, the
parameters for \verb|read| must be names. For example, we may write
\begin{verbatim}
   REF REAL r = LOC REAL;
   read(r)
\end{verbatim}
\noindent
and the program will skip spaces, tabs and end-of-line and new-page
characters until it meets an optional sign followed by optional
spaces and at least one digit, when it will expect to read a number.
If an integer is present, it will be read, converted to the
\ix{internal representation} of an integer and then
\hx{widened}{coercion!widening} to a real.

Likewise, \verb|read| may be used to read integers.  The plus and minus
signs (\verb|+| and \verb|-|) can precede integers and reals.  Absence
of a \ix{sign} is taken to mean that the number is positive. Any
non-digit will terminate the reading of an integer except for a
possible sign at the start. Reals can contain \verb|e| as in
\verb|3.41e5|. It is best to ensure that each number is preceded by a
sign so that the reading of any preceding number will be terminated by
that sign.

For a name of mode \verb|REF CHAR|, a single character will be read,
\verb|newline| or \verb|newpage| being called if necessary. In fact,
tabs and any other control characters (whose absolute value is less
than \verb|ABS blank|) will also be skipped.

If \verb|read| is used to read a \verb|[]CHAR| with fixed bounds as in
\begin{verbatim}
   REF[]CHAR sf = LOC[36]CHAR;
   read(sf)
\end{verbatim}
\noindent
then the number of characters specified by the bounds will be read,
\ixtt{newline} and \ixtt{newpage} being called as needed. You can
call \verb|newline| and \verb|newpage| explicitly to ensure that the
next value to be input will start at the beginning of the next line
or page.

Just like \verb|print|, \verb|read| can take more than one parameter
by enclosing them in a \hx{row-display}{row!display}.

You should note that the end of a line or page will not terminate the
reading of a number. So if you want to read a number from the
keyboard, you should follow the number with a non-digit before
pressing ``Enter''. In this case, you don't have to read a newline as
well, but the ``Enter'' generates a newline and that newline will be
pending in the input.\footnote{Console input is better handled using
the \texttt{kbd channel} described in section
\protect\hyref{stan-trchan}.}

The only flexible name for which \verb|read| can be used is
\verb|REF| \verb|STRING|. When reading values for \verb|REF|
\verb|STRING|, the reading pointer will not go past the end of the
current line.\footnote{See section \protect\hyref{trans-term} for
details of string terminators.} If the reading position is already at
the end of the line, the row will have no elements.  When reading a
\verb|STRING|, \verb|newline| must be called explicitly for transput
to continue.  The characters read are assigned to the name.

\begin{exercise}
\item Write a program to read two real numbers and then print their sum
and product. \ans \ %
\begin{verbatim}
   PROGRAM ex5 8 1 CONTEXT VOID
   USE standard
   BEGIN
      REF REAL a = LOC REAL,
            b = LOC REAL;

      print(Enter two real numbers->");
      read((a,b,newline));
      print(("Their sum is",a+b,newline,
           "Their product is",a*b))
   END
   FINISH
\end{verbatim}
'
\item Write a program which will input text line by line (the lines
being of different length) and which will then write out each line
with the characters reversed. For example, the line
\verb|"and so on"| will be displayed as \verb|"no os dna"|. Continue
reading until a line of zero length is read. \ans \ %
\begin{verbatim}
   PROGRAM ex5 8 2 CONTEXT VOID
   USE standard
   BEGIN
      REF STRING line = LOC STRING;

      DO
        read((line,newline));
        IF UPB line = 0
        THEN stop #terminate the program#
        ELSE
          FOR i
          FROM UPB line BY -1 TO LWB line
          DO
            print(line[i])
          OD;
          print(newline)
        FI
      OD
   END
   FINISH
\end{verbatim}
'
\end{exercise}

\Section{Dynamic names}{name-dyn}
Hitherto, all the names which can refer to rows
were declared with bounds whose values were given by integer
denotations.  In fact, the bounds given on the right-hand side of the
identity declaration can be any unit which yields an integer in a
\textbf{meek} \hx{context}{context!meek}.  So it is quite reasonable
to write
\begin{verbatim}
   REF INT size = LOC INT;  read(size);
   REF[]INT a = LOC[1:size]INT
\end{verbatim}
\noindent
or even
\begin{verbatim}
   REF[]INT r=
      LOC[1:(REF INT i=LOC INT;
             read(i);
             i)]INT
\end{verbatim}
\noindent
since an enclosed serial clause has the value of its last unit.  The
value of the clause in the parentheses is a name of mode
\verb|REF INT| and since the context of the clause is meek,
\hx{dereferencing}{coercion!dereferencing} is allowed. The context is
passed on to the last unit in the clause.  Thus the integer read by
\verb|read| will be passed to the generator.

A \hx{\textbf{dynamic}}{name!dynamic} name is one which can refer to
a multiple whose bounds are determined at the time the program is
elaborated. It means that you can declare names referring to
multiples of the size you actually require, rather than the maximum
size that you might ever need.
\newpage

\begin{exercise}
\item Declare a name which can refer to a multiple of reals whose
upper bound is determined by reading an integer from the keyoard.
\ans \ %
\begin{verbatim}
REF[]REAL r=
   LOC[(REF INT s=LOC INT; read(s); s)]REAL
\end{verbatim}
'
\item Write a program which will read an integer which says how many
integers follow it. Compute the sum of all the integers and print it.
\ans \ %
\begin{verbatim}
   PROGRAM ex5 9 2 CONTEXT VOID
   USE standard
   BEGIN
      REF INT number=LOC INT;
      read(number);

      REF[]INT multiple=LOC[number]INT;
      read(multiple);
      REF INT sum=LOC INT:=0;

      FOR i TO number
      DO
        sum+:=multiple[i]
      OD;
      print(sum)
   END
   FINISH
\end{verbatim}
'
\end{exercise}

\Section{Loops revisited}{name-loop}
In section \hyref{rep-rep}, we introduced the loop clause whose
start, step and finish were specified by integer denotations.
Instead of an integer, a unit which yields a value of mode \verb|INT|
in a \hx{meek}{context!meek} context can be supplied.  The principle
coercions not available in a meek context are rowing and widening.
In practice, almost any unit yielding \verb|INT| will do. In
particular, a name with mode \verb|REF INT| can be given.

There is an extra construct which is extremely useful for controlling
the execution of the \verb|DO|~$\ldots$~\verb|OD| loop.  It is very
common to execute a loop while a particular condition holds.  For
example, while integers are negative:
\begin{verbatim}
   WHILE
      REF INT int=LOC INT; read(int); int < 0
   DO
      print((ABS int,newline))
   OD
\end{verbatim}
\noindent
In this example, no loop counter was needed and so the \verb|FOR id|
part was omitted.  The phrase following the \ixtt{WHILE} must be an
enquiry \hx{clause}{clause!enquiry} yielding \verb|BOOL|. In this
case, an integer is read each time the loop is elaborated until a
non-negative integer is read. The range of any declarations in the
enquiry clause extends to the \verb|DO|~$\ldots$~\verb|OD| loop.

It happens quite often that the \verb|WHILE| enquiry clause performs
all the actions which need repeating and nothing is required in the
\verb|DO| part. Since the loop clause must contain at least one unit,
\ixtt{SKIP} can be used as in
\begin{verbatim}
   FOR i FROM LWB a TO UPB a
   WHILE (sum+:=a[i]) <= max
   DO
      SKIP
   OD
\end{verbatim}
\noindent
The complete loop \hx{clause}{clause!loop} thus takes the form:
\begin{verbatim}
   FOR id FROM from-unit BY by-unit TO to-unit
   WHILE boolean-enquiry-clause
   DO
      serial clause
   OD
\end{verbatim}
\newpage

\begin{exercise}
\item Write a program which will read integers until zero is
encountered.  The program should print the sums of the negative and
positive integers. \ans \ %
\begin{verbatim}
   PROGRAM ex5 10 1 CONTEXT VOID
   USE standard
   BEGIN
      REF INT neg = LOC INT:=0,
              pos = LOC INT:=0;

      WHILE
         REF INT i=LOC INT;
         read((i.newline));
         i /= 0
      DO
         (i < 0|neg|pos) +:= i
      OD;

      print(("Sum of negative integers =",
             neg,newline,
             "Sum of positive integers =",
             pos,newline))
   END
   FINISH
\end{verbatim}
'
\item Write a program which will read lines from the keyboard and
then compute a unique code for each line as follows: if \verb|"did"|
is read, compute the value of
\begin{verbatim}
   ABS"d" + ABS"i"*2 + ABS"d"*3
\end{verbatim}
\noindent
Display the string and its corresponding number on the screen.
Terminate the program when a zero-length line has been read (if the
result exceeds \texttt{max int}, you will normally not get an error:
just erroneous results---see section \hyref{stan-othop}). \ans \ %
\begin{verbatim}
   PROGRAM ex5 10 2 CONTEXT VOID
   USE standard
   BEGIN
      REF STRING line = LOC STRING;

      WHILE
         read((line,newline));
         UPB line /= 0
      DO
        REF INT v=LOC INT:=0;

        FOR i TO UPB line
        DO
          v+:=ABS line[i]*i
        OD;

        print((line,v,newline))
      OD
   END
   FINISH
\end{verbatim}
'
\end{exercise}

\Section{Abbreviated declarations}{name-abbr}
You have now met many identity declarations. When declaring names, it
is apparent that much of the declaration is repeated on both sides.
For example:
\begin{verbatim}
   REF[]REAL r = LOC[10]REAL
\end{verbatim}
\noindent
Declarations of names are very common in Algol~68 programs and
abbreviated \hx{declarations}{declaration!abbreviated} are
available. The above declaration can be written
\begin{verbatim}
   LOC[10]REAL r
\end{verbatim}
\noindent
or, most commonly
\begin{verbatim}
   [10]REAL r
\end{verbatim}
\noindent
An abbreviated declaration uses the \ix{actual-declarer} (the
right-hand side of an identity declaration) followed by the
identifier; and if the actual-declarer contains the generator
\verb|LOC|, you can omit the \verb|LOC| (see
section~\hyref{rout-rout} which explains actual-declarers and
formal-declarers).

Here are some of the declarations given as examples in this chapter
rewritten in their abbreviated form:
\begin{verbatim}
   INT a;
   REAL x:=pi;
   CHAR s;
   [7]INT i7;
   [0:6]INT i7 at 0;
   [3]INT k:=(1,2,3);
   [3,3]REAL x; [0:2,0:2]REAL y;
   FLEX[1:0]INT fn;
   [36]CHAR sf;
   [(INT i; read(i); i)]INT r
\end{verbatim}
\noindent
It is important to note that identity declarations should not be
mixed with abbreviated name declarations because the modes are quite
different. For example, in
\begin{verbatim}
   REAL a:=2.4;
   REAL b = a+2.1
\end{verbatim}
\noindent
the mode of \verb|a| is \verb|REF REAL|, but the mode of \verb|b| is
\verb|REAL|. In the abbreviated declaration of a name, the mode given
is that of the value to which the name will refer (the
actual-declarer).

When you declare a new object, if you do not intend assigning to it,
use an identity declaration. Only declare it as a name if you intend
superseding the value to which it will refer. Remember that assignment
can be dangerous because values are superseded.

\begin{exercise}
\item Write abbreviated declarations for the following:\newline
\begin{subex}
\item \verb|REF[]CHAR rc = LOC[1000]CHAR|
\subans \verb|[100]CHAR rc|
'
\item \verb|REF FLEX[]INT fi = LOC FLEX[1:0]INT|
\subans \verb|FLEX[1:0]INT fi|
'
\item \verb|REF BOOL b = LOC BOOL := TRUE|
\subans \verb|BOOL b:=TRUE| 
'
\end{subex}
\item Write full identity declarations for the following:\newline
\begin{subex}
\item \verb|INT a,b,c|
\subans \verb|REF INT a=LOC INT, b=LOC INT, c=LOC INT|
'
\item \verb|REAL x;[5]CHAR y;[3,3]REAL z|
\subans \verb|REF REAL x=LOC REAL;|\newline
\verb|REF[]CHAR y=LOC[5]CHAR;|\newline
\verb|REF[,]REAL z=LOC[3,3]REAL|
'
\item \verb|FLEX[1:0]CHAR s|
\subans \verb|REF FLEX[]CHAR s=LOC FLEX[1:0]CHAR|
'
\end{subex}
\end{exercise}

\Section{Summary}{name-summ}
A name is a value whose mode always begins with the mode constructor
\verb|REF|. A name can refer to a value whose mode starts with one
less \verb|REF| than the mode of the name.  An assignment causes a
name to refer to a value.  The value to which a name refers can be
superseded using a further assignment.  An assignment is a kind of
unit and can appear in a formula if it is enclosed by parentheses (or
\verb|BEGIN| and \verb|END|).  Multiple assignments can be used to
assign the same value to more than one name.

A name can be generated using a local or global generator and can be
made to refer to a value in the same phrase in which it is declared.

Algol 68 provides flexible names as well as fixed names for
multiples.  The mode indicant for \verb|FLEX[]CHAR| is defined in the
standard prelude as \verb|STRING|.  Names for multiples can have
bounds determined at run-time.

\verb|read| will convert external character sequences into internal
values.  Its parameters must be names or \verb|newline| or
\verb|newpage|.

Name declarations may be written as identity declarations or in an
abbreviated form.

Before continuing with chapter~6, it would be wise to revise the
material in the first five chapters since these comprise the basis of
the language.

\begin{exercise}
\item Declare a name to refer to a multiple of 1000 integers, first as
an identity declaration, and secondly in abbreviated form.
\ans \verb|REF[]INT m=LOC[1000]INT;| \verb|[1000]INT m|
'
\item Write a program which will compute the average of a number of
salaries ({\it eg}, 1010.53) read from the keyboard until the number
\verb|-1| is read. Display the average on the screen. \ans \ %
\begin{verbatim}
   PROGRAM ex5 12 2 CONTEXT VOID
   USE standard
   BEGIN
      REAL sum:=0.0, salary, INT num:=0;

      WHILE read(salary);  salary /= -1.00
      DO
        sum+:=salary;  num+:=1
      OD;

      print(("Average salary=",sum/num))
   END
   FINISH
\end{verbatim}
'
\item Write a program which will read a line and then scan it,
writing out the individual words on one line apiece. The program
should read the line into a \verb|REF STRING| name, then remove
leading and trailing spaces and add a space to the end. Use a
boolean name called \verb|in word| and make it refer to \verb|FALSE|.
As you step along the line, make \verb|in word| refer to \verb|FALSE|
if you read a space and \verb|TRUE| otherwise. Keep a track of the
length of the current word. Whenever the value \verb|in word| changes
from \verb|TRUE| to \verb|FALSE|, extract the word using an
appropriate trimmer and print it. Allow for there being more than one
space between words. Ignore the possibility of commas, brackets etc.
\ans When writing a program as involved as this, do not expect to get
it right first time. In practice, a programmer adds fine details to a
program after she has designed the main structure.
\hylabel{name-ex5-12-3}
\begin{verbatim}
   PROGRAM ex5 12 3 CONTEXT VOID
   USE standard
   BEGIN
      BOOL in word:=FALSE,
      STRING line;
      INT line start, line finish;
      INT word start, word finish;

      read((line,newline));
      line start:=LWB line;
      line finish:=UPB line;

      WHILE line[line start]=blank
                   &
            line start<=UPB line
      DO
        line start+:=1
      OD;

      WHILE line[line finish]=blank
                   &
            line finish>=line start
      DO
        line finish-:=1
      OD;

      line:=line[line start:line finish]
                       +blank;

      FOR c FROM LWB line
      WHILE c <= UPB line
      DO
        CHAR lc = line[c];

        IF   lc /= blank & NOT in word
        THEN word start:=c;  in word:=TRUE
        ELIF lc = blank & NOT in word
        THEN SKIP
        ELIF lc /= blank & in word
        THEN SKIP
        ELSE #lc = blank & in word#
          word finish:=c-1;
          in word:=FALSE;
          print((line[
                  word start:word finish],
                 newline))
        FI
      OD
   END
   FINISH
\end{verbatim}
\indent Notice that both \verb|word start| and \verb|word finish| are
made to refer to new values before being used. This is a good check
that you are writing the program properly. Notice also that the four
possible states of the compound condition on line 26 are carefully
spelled out on lines~28, 30 and~32.
'
\end{exercise}
