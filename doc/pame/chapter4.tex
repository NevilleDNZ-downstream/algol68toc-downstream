% $Log: chapter4.tex,v $
% Revision 1.3  2002-06-20 11:49:28  sian
% mm removed, padding removed, ca68 added
%
\Chapter{Choice}{chapiv}
One of the essential properties of a computer program is its ability
to modify its actions depending on its circumstances and environment.
In other words, its behaviour is not predetermined, but can vary from
one execution to another. In this chapter, we shall introduce a new
plain mode, describe the operators using or yielding values of the
new mode, and then investigate the program structures which allow an
Algol~68 program to choose between alternatives.

\pagebreak
\Section{Boolean values}{choice-bool}
 The mode \ixtt{BOOL} is named after George
\hx{Boole}{Boole, George}, the distinguished nineteenth century
mathematician who developed the system of logic which bears his name.
There are only two values of mode \verb|BOOL|, and their denotations
are \ixtt{TRUE} and \ixtt{FALSE}.  Let us declare two identifiers:
\begin{verbatim}
   BOOL t = TRUE,
        f = FALSE
\end{verbatim}
\noindent
The \ixtt{print} phrase, when fed with \ix{Boolean} values prints
\verb|T| for \verb|TRUE|, and \verb|F| for \verb|FALSE|, with spaces
neither before nor after. Thus
\begin{verbatim}
   print((t,f,t,f,t))
\end{verbatim}
\noindent
produces \verb|TFTFT| on the screen.

\Section{Boolean operators}{choice-bops}
The simplest operator which has an operand of mode \verb|BOOL| is
\ixtt{NOT}. If its operand is \verb|TRUE|, it yields \verb|FALSE|.
Conversely, if its operand is \verb|FALSE|, it yields \verb|TRUE|.
The operator \ixtt{ODD} yields \verb|TRUE| if its operand is an odd
integer and \verb|FALSE| if it is even. The operators can be
combined, so
\begin{verbatim}
   NOT ODD 2
\end{verbatim}
\noindent
yields \verb|TRUE|.

\ixtt{ABS} converts its operand of mode \verb|BOOL| and yields an
integer: \verb|ABS TRUE| yields \verb|1|, \verb|ABS FALSE| yields
\verb|0|.

Boolean dyadic \hx{operators}{operator!dyadic} come in two
kinds: those that take operands of mode \verb|BOOL|, yielding
\verb|TRUE| or \verb|FALSE|, and those that operate on operands of
other modes.

Two dyadic operators are declared in the \ix{standard prelude} which
take operands of mode \verb|BOOL|. The operator \ixtt{AND}
(alternative representation \verb|&|) yields \verb|TRUE| if,
and only if, both its operands yield \verb|TRUE|, so that
\begin{verbatim}
   t AND f
\end{verbatim}
\noindent
yields \verb|FALSE| (\verb|t| and \verb|f| were declared earlier).
Both the operands are elaborated before the operator (but see the
section later on pseudo-operators).  The \ix{priority} of \verb|AND|
is~3.

The operator \ixtt{OR} yields \verb|TRUE| if at least one of its
operands yields \verb|TRUE|. Thus
\begin{verbatim}
   t OR f
\end{verbatim}
\noindent
yields \verb|TRUE|. It has no alternative representation. Again, both
operands are elaborated before the operator. The priority of
\verb|OR| is~2.

You will learn in chapter~6 how to define new operators if you need
them.

\Section{Relational operators}{choice-relops}
Values of modes \verb|INT|, \verb|REAL|, \verb|CHAR| and
\verb|[]CHAR| can be compared with each other.  The expression
\begin{verbatim}
   3 = 1+2
\end{verbatim}
\noindent
yields \verb|TRUE|.  Similarly,
\begin{verbatim}
   1+1=1
\end{verbatim}
\noindent
yields \verb|FALSE|.  The equals symbol \ixtt{=} can also be written
\verb|EQ|. Likewise, the formula
\begin{verbatim}
   35.0 EQ 3.5e1
\end{verbatim}
\noindent
should also yield \verb|TRUE|, but you should be chary of comparing
two \verb|REAL|s for equality or inequality because the means of
transforming the denotations into binary values may yield values
which differ slightly.  The operator is also defined for both
operands being \verb|CHAR| or \verb|[]CHAR|.  In the latter case, the
two multiples must have the same number of elements, and
corresponding elements must be equal if the operator is to yield
\verb|TRUE|.  Thus
\begin{verbatim}
   "a" = "abc"
\end{verbatim}
\noindent
yields \verb|FALSE|. Notice that the bounds do not have to be the
same.  So \verb|a| and \verb|b| declared as
\begin{verbatim}
   []CHAR a = "Dodo" [@0],
          b = "Dodo"
\end{verbatim}
\noindent
yield \verb|TRUE| when compared with the equals operator.  Because
the rowing \hx{coercion}{coercion!rowing} is not allowed in
formul{\ae}, the operator is declared in the \ix{standard prelude}
for mixed modes (such as \verb|REAL| and \verb|INT|).

The converse of \verb|=| is \ixtt{/=} (not equal). So the formula
\begin{verbatim}
   3 /= 2
\end{verbatim}
\noindent
yields \verb|TRUE|, and
\begin{verbatim}
   "r" /= "r"
\end{verbatim}
\noindent
yields \verb|FALSE|.  An alternative representation of \verb|/=| is
\ixtt{NE}.  The \ix{priority} of both \verb|=| and \verb|/=| is~4.
The operands of \verb|=| and \verb|/=| can be any combination of
values of mode \verb|INT| and \verb|REAL|.  No
\hx{widening}{coercion!widening} takes place, the operators being
declared for the \ix{mixed modes}.

The ordering operators \ixtt{<}, \ixtt{>}, \ixtt{<=} and \ixtt{>=}
can be used to compare values of modes \verb|INT|, \verb|REAL|,
\verb|CHAR| and \verb|[]CHAR| in the same way as \verb|=| and
\verb|/=|.  They are read ``less than'', ``greater than'', ``less
than or equal to'' and ``greater than or equal to'' respectively. The
formula
\begin{verbatim}
   3 < 3.1
\end{verbatim}
\noindent
yields \verb|TRUE|.

If the identifiers \verb|b| and \verb|c| are declared as having mode
\verb|CHAR|, then the formula
\begin{verbatim}
   c < b
\end{verbatim}
\noindent
will yield the same value as
\begin{verbatim}
   ABS c < ABS b
\end{verbatim}
\noindent
and similarly for the operator \verb|>|. The operators \verb|<=| and
\verb|>=| can both be used with equal values.  For example,
\begin{verbatim}
   24 <= 24.0
\end{verbatim}
\noindent
yields \verb|TRUE|.

For values of mode \verb|[]CHAR|, the formula
\begin{verbatim}
   "abcd" > "abcc"
\end{verbatim}
\noindent
yields \verb|TRUE|. Two values of mode \verb|[]CHAR| of different
length can be compared. For example, both
\begin{verbatim}
   "aaa" <= "aaab"
\end{verbatim}
\noindent
and
\begin{verbatim}
   "aaa" <= "aaaa"
\end{verbatim}
\noindent
yield \verb|TRUE|. Alternative representations for these operators
are \verb|LT| and \verb|GT| for \verb|<| and \verb|>| and \verb|LE|
and \verb|GE| for \verb|<=| and \verb|>=| respectively. The priority
of all four \ix{ordering operators} is~5.

Note that apart from values of mode \verb|[]CHAR|, no operators are
defined in the standard prelude for multiples.

\begin{exercise}
\item What is the value of each of the following formul{\ae}?
\begin{subex}
\item \verb|ABS NOT TRUE| \subans \verb|0|
'
\item \verb|3.4 + ABS TRUE| \subans \verb|4.4|
'
\item \verb|-3.5 <= -13.4| \subans \verb|FALSE|
'
\item \verb|2e10 >= 3e9| \subans \verb|TRUE|
'
\item \verb|"abcd" > "abc"| \subans \verb|TRUE|
'
\end{subex}
\item In the context of these declarations
\begin{verbatim}
   []INT i1 = (2,3,5,7);
   []CHAR t = "uvwxyz"
\end{verbatim}
\noindent
what is the value of each of the following?
\begin{subex}
\item \verb|UPB i1 < UPB t| \subans \verb|TRUE|
'
\item \verb|t[2:4] >= t[2:3]| \subans \verb|TRUE|
'
\item \verb|i1[3] < UPB t[2:]| \subans \verb|FALSE| (the
\verb|UPB t[2:]| is \verb|5|
'
\end{subex}
\end{exercise}

\Section{Compound Boolean formul{\ae}}{choice-comp}
Formul{\ae} yielding \verb|TRUE| or \verb|FALSE| can be combined. For
example, here is a formula which tests whether $\pi$ lies between
\verb|3| and \verb|4|
\begin{verbatim}
   pi > 3 & pi < 4
\end{verbatim}
\noindent
which yields \verb|TRUE|. The priorities of \verb|<|, \verb|>| and
\verb|&| are so defined that parentheses are unnecessary in this
case.  Likewise, we may write
\begin{verbatim}
   "ab" < "aac" OR 3 < 2
\end{verbatim}
\noindent
which yields \verb|FALSE|. More complicated formul{\ae} can be
written:
\begin{verbatim}
   3.4 > 2 & "a" < "c" OR "b" >= "ab"
\end{verbatim}
\noindent
which yields \verb|TRUE|. Because the priority of the operator
\verb|&| is higher than the \ix{priority} of \ixtt{OR}, the \verb|&|
in the above formula is elaborated first. The order of elaboration
can be changed using parentheses.

There does not seem much point to these formul{\ae} since everything
is known beforehand, but all will become clear in the next chapter.

Compound Boolean formul{\ae} can be confusing. Being aware of the
converse of a compound \hx{condition}{converse condition} helps
you to ensure you have considered all possibilities. For example, the
converse of the formula
\begin{verbatim}
   a < b & c = d
\end{verbatim}
\noindent
is the formula
\begin{verbatim}
   a >= b OR c /= d
\end{verbatim}
\noindent
One of the formul{\ae} would yield \verb|TRUE| and the other
\verb|FALSE|.

\begin{exercise}
\item What is the value of each of the following:
\begin{subex}
\item \verb|NOT ODD 3 OR 3 < 4| \subans \verb|TRUE|
'
\item \verb|3 > 2 & (5 > 12 OR 7 <= 8)| \subans \verb|TRUE|
'
\item \verb|(TRUE OR FALSE) AND (FALSE OR TRUE)| \subans \verb|TRUE|
'
\item \ %
\begin{verbatim}
   NOT("d">"e")
      AND
      FALSE
       OR
   NOT(ODD 5 & 3.6e12 < 0)
\end{verbatim}
\indent\subans \verb|TRUE|. It is inadvisable to created compuound
conditions with this sort of complexity simply because the condition
is so difficult to understand. You should particularly avoid compound
conditions with \verb|NOT| in front of the various parts.
'
\item \verb|3<4 & 4<5 & 5<6 & 6>7| \subans \verb|FALSE|
'
\end{subex}
\item For each condition, write out its converse:
\begin{subex}
\item \verb|FALSE| \subans \verb|TRUE|
'
\item \verb|4 > 2| \subans \verb|4 <= 2|
'
\item \verb|a > b AND b > c| \subans \verb|a <= b OR b <= c|
'
\item \verb|x = y OR x = z| \subans \verb|x /= y AND x /= z|
'
\end{subex}
\end{exercise}

\Section{Conditional clauses}{choice-cond}
Now we can discuss clauses which choose between alternatives. We have
met the \hx{enclosed clause}{clause!enclosed} consisting of at least
one \ix{phrase} enclosed by \ixtt{BEGIN} and \ixtt{END} (or
\ix{parentheses}) in the structure of an Algol~68 program, and also
in the \verb|DO|~$\ldots$~\verb|OD| loop of a \verb|FOR| or
\verb|FORALL| clause.  The part of the enclosed clause inside the
parentheses (or \verb|BEGIN| and \verb|END|) is called a
\hx{\textbf{serial clause}}{clause!serial} because,
historically, sequential elaboration used to be called ``serial
elaboration''.  The value of the serial clause is the value of the
last phrase which must be a unit.

There are two kinds of clause which enable programs to modify their
behaviour. They are called \textbf{choice clauses}.  The
\hx{\textbf{conditional clause}}{clause!conditional} allows a program
to elaborate code depending on the value of a
\hx{boolean serial clause}{clause!boolean}, called a \verb|BOOL|
\hx{\textbf{enquiry clause}}{clause!enquiry}. Here is a simple
example:
\begin{verbatim}
   IF   salary < 5000
   THEN 0
   ELSE (salary-allowances)*rate
   FI
\end{verbatim}
\noindent
The enquiry clause consists of the formula
\begin{verbatim}
   salary < 5000
\end{verbatim}
\noindent
which yields a value of mode \verb|BOOL|. Two serial clauses, both
containing a single unit can be elaborated. If the value yielded by
\verb|salary| is less than \verb|5000|, the value \verb|0| is
yielded.  Otherwise, the program calculates the tax. That is, if the
\verb|BOOL| enquiry clause yields \verb|TRUE|, the serial clause
following \ixtt{THEN} is elaborated, otherwise the serial clause
following \ixtt{ELSE} is elaborated. The \ixtt{FI} following the
\verb|ELSE| serial clause must be there.

The enquiry clause and the serial clauses may consist of single units
or possibly declarations and formul{\ae} and loops. However, the last
phrase in an enquiry clause must be a unit yielding \verb|BOOL|. The
\ix{range} of any identifiers declared in the enquiry clause extends
to the serial clauses as well.  The range of any identifiers declared
in either serial clause is limited to that serial clause.  For
example, assuming that \verb|a| and \verb|i| are predeclared, we
could write:
\begin{verbatim}
   IF   INT ai = a[i];  ai < 0
   THEN print((ai," is negative",newline))
   ELSE print((ai," is non-negative",newline))
   FI
\end{verbatim}
\noindent
The conditional clause can be written wherever a unit is permitted,
so the previous example could also be written
\begin{verbatim}
   INT ai = a[i];
   print((ai,IF   ai < 0
             THEN "is negative"
             ELSE "is non-negative"
             FI,newline))
\end{verbatim}
\noindent
The value of each of the serial clauses following \verb|THEN| and
\verb|ELSE| in this case is \verb|[]CHAR|. Here is an example with a
\hx{conditional clause}{clause!conditional} inside a loop:
\begin{verbatim}
   FOR i TO 100
   DO
      IF i MOD 10 = 0
      THEN print((i,newline))
      ELSE print((i,blank))
      FI
   OD
\end{verbatim}
\noindent
The \verb|ELSE| part of a conditional clause can be omitted. Thus the
above example could also be written
\begin{verbatim}
   FOR i TO 100
   DO
      print((i,blank));
      IF i MOD 10 = 0 THEN print(newline) FI
   OD
\end{verbatim}

The whole conditional clause can appear as a formula or as an
\ix{operand}.  The short form of the clause is often used for this:
\verb|IF| and \verb|FI| are replaced by \ixtt{(} and \ixtt{)}
respectively, and \verb|THEN| and \verb|ELSE| are both replaced by
the vertical bar \hx{\texttt{|}}{clause!conditional!short form}%
\footnote{Some editors insert a different character when you press
the key marked \texttt{|}. Check that the character produced is
accepted by the Algol~68 compiler.}. For example, here is an identity
declaration which assumes a previous declaration for \verb|x|:
\begin{verbatim}
   REAL xx = (x < 3.0|x**2|x**3)
\end{verbatim}
\noindent
If the \verb|ELSE| part is missing then its serial clause is regarded
as containing the single unit \ixtt{SKIP}.  In this case, \verb|SKIP|
will yield an undefined value of the mode yielded by the \verb|THEN|
serial clause.  This is an example of \bfix{balancing} (explained in
chapter~10).  This is particularly important if a conditional clause
is used as an operand.\footnote{The
\protect\hx{a68toc}{a68toc!ELSE SKIP@\texttt{ELSE SKIP}} compiler
generates code which will cause a run-time fault if your program
tries to execute an \texttt{ELSE} part which has been omitted. You can
get around that bug by explicitly writing \texttt{ELSE SKIP}.}

Since the right-hand side of an identity declaration is in a strong
\hx{context}{context!strong},
\hx{widening}{coercion!widening} is allowed. Thus, in
\begin{verbatim}
   REAL x = (i < j|3|4)
\end{verbatim}
\noindent
whichever value the conditional clause yielded would be widened to a
value of mode \verb|REAL|.

Since the enquiry \hx{clause}{clause!enquiry} is a serial clause,
it can have any number of phrases before the \verb|THEN|. For example:
\begin{verbatim}
   IF []CHAR line =
      "a growing gleam glowing green";
      INT sz = UPB line - LWB line + 1;
      sz > 35
   THEN
      ...
\end{verbatim}

Conditional clauses can be \hx{nested}{clause!conditional!nested}
\begin{verbatim}
   IF a < 4.1
   THEN
      IF b >= 35
      THEN print("yes")
      ELSE print("no")
      FI
   ELSE
      IF c <= 20
      THEN print("perhaps")
      ELSE print("maybe")
      FI
   FI
\end{verbatim}
\noindent
The \ixtt{ELSE IF} in the above clause could be replaced by
\ixtt{ELIF}, and the final \verb|FI FI| with a single \verb|FI|,
giving:
\begin{verbatim}
   IF a < 4.1
   THEN
      IF b >= 35
      THEN print("yes")
      ELSE print("no")
      FI
   ELIF c <= 20
   THEN print("perhaps")
   ELSE print("maybe")
   FI
\end{verbatim}
\noindent
Here is another contracted example:
\begin{verbatim}
   INT p = IF   c = "a" THEN 1
           ELIF c = "h" THEN 2
           ELIF c = "q" THEN 3
           ELSE 4
           FI
\end{verbatim}

The \ix{range} of any identifier declared in an enquiry
\hx{clause}{clause!enquiry} extends to any serial clause beyond
its declaration but within the overall conditional clause. Consider
this conditional clause:
\begin{verbatim}
   IF   INT p1 = ABS(c="a");  p1=1
   THEN  p1+2
   ELIF INT p2 = p1-ABS(c="h");  p2 = -1
   THEN INT i1 = p1+p2;  i1+p1
   ELSE INT i2 = p1+2*p2;  i2-p2
   FI
\end{verbatim}
\noindent
The range of \verb|p1| extends to the enclosing \verb|FI|; likewise
the range of \verb|p2|. The ranges of \verb|i1| and \verb|i2| are
confined to their serial clauses.

In the abbreviated form, \hx{\texttt{|:}}{"|:@\texttt{"|:}} can be used
instead of \verb|ELIF|.  For example, the above identity declaration
for \verb|p| could be written
\begin{verbatim}
   INT p = (c="a"|1|:c="h"|2|:c="q"|3|4)
\end{verbatim}
\noindent
In both identity declarations, the opening parenthesis is an
abbreviated symbol for \verb|IF|.

Sometimes it is useful to include a conditional clause in the
\verb|IF| part of a conditional clause. In other words, a \verb|BOOL|
enquiry clause can be a conditional clause yielding a value of mode
\verb|BOOL|. Here is an example with \verb|a| and \verb|b|
predeclared with mode \verb|BOOL|:
\begin{verbatim}
   IF IF a
      THEN NOT b
      ELSE b
      FI
   THEN print("First possibility")
   ELSE print("Second possibility")
   FI
\end{verbatim}

\Subsection{Pseudo-operators}{choice-pseud}
As was mentioned in chapter 2, both the operands of an operator are
elaborated before the operator is elaborated. The \verb|a68toc| compiler
implements the \bfix{pseudo-operator} \verb|ANDTH| which although it
looks like an operator, has its right-hand operand elaborated only if
its left-hand operand yields \verb|TRUE|. Compare \verb|ANDTH| (which
is read ``and then'') with the operator \ixtt{AND}.  The priority of
\verb|ANDTH| is~1. The phrase \verb|IF p ANDTH q THEN ... FI| is
equivalent to
\begin{verbatim}
   IF IF NOT p
      THEN FALSE
      ELIF q
      THEN TRUE
      ELSE FALSE
      FI
   THEN ...
   FI
\end{verbatim}

You should be chary of using \verb|ANDTH| in a compound boolean
\hx{expression}{compound expression}. For example, given the condition
\begin{verbatim}
      UPB s > LWB s
         ANDTH
      s[UPB s]="-"
         AND
   (CHAR c=s[UPB s-1];
    c>="a" & c<="z")
\end{verbatim}
\noindent
the intention of the compound condition is to determine whether a
terminating hyphen is preceded by a lower-case letter.  Clearly,
testing for a character which precedes the hyphen can only be
elaborated if there are at least two characters in \verb|s|.  The
first boolean formula (the left operand of \verb|ANDTH|) ensures that
the second formula (the right operand of \verb|ANDTH|) is only
elaborated if \verb|s| identifies at least two characters.
Unfortunately, because the priority of \verb|AND| is greater than the
priority of \verb|ANDTH| and because both operands of an operator
must be elaborated before the operator is elaborated, the right-hand
operand of \verb|AND| will be elaborated whatever the value of the
left operand of \verb|ANDTH|.  In order to achieve the above aim, the
compound condition should be written
\begin{verbatim}
   UPB s > LWB s
       ANDTH
  (s[UPB s]="-"
      AND
   (CHAR c=s[UPB s-1];
    c>="a" & c<="z"))
\end{verbatim}
\noindent
Note the additional parentheses which ensure that the boolean formula
containing \verb|AND| is treated {\it as a whole\/} as the right-hand
operand of the pseudo-operator \verb|ANDTH|.

There is another pseudo-operator \ixtt{OREL} (read ``or else'') which
is similar to the operator \ixtt{OR} except that its right-hand
operand is only elaborated if its left-hand operand yields
\verb|FALSE|.  Like \verb|ANDTH|, the priority of \verb|OREL| is~1.
The remarks given above about the use of \verb|ANDTH| in compound
boolean formul{\ae} apply equally to \verb|OREL|.

Neither \verb|ANDTH| nor \verb|OREL| are part of Algol 68.

\begin{exercise}
\item Write a conditional clause which tests whether a \verb|REAL|
value is less than $\pi$, and prints "Yes" if it is and "No"
otherwise. \ans \ %
\begin{verbatim}
   IF x < pi
   THEN print("Yes")
   ELSE print("No")
   FI
\end{verbatim}
'
\item Write a conditional clause inside a loop clause to display the
first 96 multiples of 3 (including~3) in lines of 16. Use the
operator \ixtt{MOD} for the test. \ans \ %
\begin{verbatim}
   FOR i TO 96
   DO
      print(i*3);

      IF i MOD 16 = 0
      THEN print(newline)
      FI
   OD
\end{verbatim}
'
\item Replace the operator \verb|OREL| in the following program with
a suitable conditional clause:
\begin{verbatim}
   PROGRAM p CONTEXT VOID
   USE standard
   IF INT a=3, b=5, c=4;
      a > b OREL b > c
   THEN print("Ok")
   ELSE print("Wrong")
   FI
   FINISH
\end{verbatim}
\indent\ans The second operand of \verb|OREL| is only elaborated if
the first yields \verb|FALSE|.
\begin{verbatim}
   PROGRAM p CONTEXT VOID
   USE standard
   BEGIN
      INT a = 3, b = 5, c = 4;

      IF
         IF a > b
         THEN TRUE
         ELSE b > c
         FI
      THEN print("Ok")
      ELSE print("Wrong")
      FI
   END
   FINISH
\end{verbatim}
'
\end{exercise}

\Section{Multiple choice}{choice-case}
Sometimes the number of choices can be quite large or the different
choices are related in a simple way. For example, consider the
following conditional clause:
\begin{verbatim}
   IF n = 1
   THEN action1
   ELIF n = 2
   THEN action2
   ELIF n = 3
   THEN action3
   ELIF n = 4
   THEN action4
   ELSE action5
   FI
\end{verbatim}
\noindent
This sort of choice can be expressed more concisely using the
\hx{\textbf{case clause}}{clause!case} in which the boolean enquiry
clause is replaced by an integer enquiry clause. Here is the above
conditional clause rewritten using a case clause:
\begin{verbatim}
   CASE n
   IN
      action1
         ,
      action2
         ,
      action3
         ,
      action4
   OUT
      action5
   ESAC
\end{verbatim}
\noindent
\hx{which}{CASE@\texttt{CASE}} \hx{could}{IN@\texttt{IN}}
\hx{be}{OUT@\texttt{OUT}}
\hx{abbreviated}{ESAC@\texttt{ESAC}} as
\begin{verbatim}
   (n|action1,action2,action3,action4|action5)
\end{verbatim}
\noindent
Notice that \verb|action1|, \verb|action2|, \verb|action3| and
\verb|action4| are separated by \hx{commas}{comma} (they are not
terminators). Each of \verb|action1|, \verb|action2| and
\verb|action3| is a unit, so that if you want more than one phrase
for each action, you must make it an enclosed
\hx{clause}{clause!enclosed} by enclosing the action in
\ix{parentheses} (or \ixtt{BEGIN} and \ixtt{END}). If the \verb|INT|
enquiry \hx{clause}{clause!enquiry} yields~\verb|1|, \verb|action1|
is elaborated, \verb|2|, \verb|action2| is elaborated and so on. If
the value yielded is negative or zero, or exceeds the number of
actions available, \verb|action5| in the \verb|OUT| part is
elaborated.  The \verb|OUT| part is a
\hx{serial clause}{clause!serial} so no enclosure is required if
there is more than one unit.

In the following case clause, the second unit is a conditional clause
to show you that any piece of program which happens to be a unit can
be used for one of the cases:
\begin{verbatim}
   CASE i IN 3,(x>3.5|4|-2),6 OUT i+3 ESAC
\end{verbatim}
\noindent
The first action yields \verb|3|, the second yields \verb|4| if
\verb|x| exceeds \verb|3.5| and \verb|-2| otherwise, and the third
action yields~\verb|6|.

Sometimes the
\hx{\texttt{OUT} clause}{OUT clause@\texttt{OUT} clause}
consists of another case clause. For example,
\begin{verbatim}
   CASE n MOD 4
   IN
      print("case 1"),
      print("case 2"),
      print("case 3")
   OUT
      CASE (n-10) MOD 4
      IN
         print("case 11"),
         print("case 12"),
         print("case 13")
      OUT
         print("other case")
      ESAC
   ESAC
\end{verbatim}
\noindent
Just as with \verb|ELIF| in a conditional clause, \ixtt{OUT CASE}
$\ldots$ \verb|ESAC| \verb|ESAC| can be replaced by \ixtt{OUSE} $\ldots$
\verb|ESAC|. So the above example can be rewritten
\begin{verbatim}
   CASE n MOD 4
   IN
      print("case 1"),
      print("case 2"),
      print("case 3")
   OUSE (n-10) MOD 4
   IN
      print("case 11"),
      print("case 12"),
      print("case 13")
   OUT print("other case")
   ESAC
\end{verbatim}
\noindent
Here is a case clause with embedded case clauses:
\begin{verbatim}
   CASE command
   IN
      action1,
      action2,
\end{verbatim}
\begin{verbatim}
      (subcommand1
      |subaction1,subaction2
      |subaction3)
   OUSE subcommand2
\end{verbatim}
\begin{verbatim}
   IN subaction4,
      subaction5,
\end{verbatim}
\begin{verbatim}
      subaction6
   OUT
      subaction7
   ESAC
\end{verbatim}
\noindent
Calendar computations, which are notoriously difficult, give examples
of case clauses:
\begin{verbatim}
   INT days = CASE month IN
               31,
               IF year MOD 4 = 0
                       &
                  year MOD 100 /= 0
                      OR
                  year MOD 400 = 0
               THEN 29
               ELSE 28
               FI,
               31,30,31,30,31,31,30,31,30,31
               OUT -1
              ESAC
\end{verbatim}
\noindent
And here is one in dealing cards:
\begin{verbatim}
   []CHAR suit=(i|"spades",
                  "hearts",
                  "diamonds",
                  "clubs"
                 |"")
\end{verbatim}

Like the conditional clause, if you omit the \verb|OUT| part, the
compiler assumes that you wrote \hxtt{OUT SKIP}{SKIP}.  In the
following example, when \verb|i| is \verb|4|, nothing gets
printed:\footnote{The a68toc compiler objects to this with a run-time
error.  Ensure that at least \protect\texttt{OUT SKIP} occurs in
every case clause.}
\begin{verbatim}
   PROGRAM prog CONTEXT VOID
   USE standard
   FOR i TO 5
   DO
      print((i MOD 4|"a","g","r"))
   OD
   FINISH
\end{verbatim}

\begin{exercise}
\item What is wrong with the following identity declaration, assuming
that \verb|p| has been predeclared as a value of mode \verb|BOOL|:
\begin{verbatim}
   INT i = (p|1,2,3|4)
\end{verbatim}
\indent\ans The right-hand side of the identity declaration is
clearly an abbreviated case clause, so \verb|p| must yield
\verb|INT|, not \verb|BOOL|.
'
\item Write a program consisting solely of a case clause which uses
the \verb|SIGN| operator to give three different actions depending on
the sign of a number of mode \verb|REAL|. \ans \ %
\begin{verbatim}
   PROGRAM ex4 6 2 CONTEXT VOID
   USE standard
   CASE SIGN x + 2
   IN
      print("x < 0.0"),
      print("x = 0.0"),
      print("x > 0.0")
   ESAC
   FINISH
\end{verbatim}
'
\end{exercise}

\Section{Summary}{choice-summ}
There are two values having mode \verb|BOOL|. Operators with operands
of mode \verb|BOOL| are predeclared in the standard prelude.  A
conditional clause uses an enquiry clause yielding a value of mode
\verb|BOOL|.  A case clause uses an enquiry clause yielding a value
of mode \verb|INT|.  Both conditional and case clauses can be
abbreviated.  Extended conditional and case clauses can be written
using \verb|ELIF| and \verb|OUSE| respectively.  Conditional clauses
and case clauses are sometimes grouped together and termed choice
clauses. Choice clauses are examples of enclosed clauses, and are
units.

Here are some exercises which test you on the material covered in this
chapter.

\begin{exercise}
\item Which values have the mode \verb|BOOL|? \ans \verb|TRUE| and
\verb|FALSE|
'
\item What is the value of each of the following formul{\ae}?
\begin{subex}
\item \verb|3 < 4| \subans \verb|TRUE|
'
\item \verb|4.0 >= 0.4e1| \subans \verb|TRUE|
'
\item \verb|2 < 3 & 3 > 2| \subans \verb|TRUE|
'
\item \verb|11 < 2 OR 10 < ABS TRUE| \subans \verb|FALSE|
'
\item \verb|NOT TRUE & ABS "A" < ABS "D"| \subans \verb|FALSE|
'
\item \verb|NOT(3 > 2 & 3 > 1 OR 10 < 6)| \subans \verb|FALSE|
'
\end{subex}
\item What is wrong with the following (\verb|m| is predeclared):
\begin{verbatim}
   IF m>4|print("ok")ELSE print(".")ESAC
\end{verbatim}
\indent\ans You cannot mix full and abbreviated conditional
clauses.  Replace the vertical bar with \verb|THEN|. Also replace the
\verb|ESAC| with \verb|FI|.
'
\item What would be displayed on your screen by the following:
\begin{verbatim}
   FOR i TO 10 DO print(ODD i) OD
\end{verbatim}
\indent\ans \verb|TFTFTFTFTF|
'
\item Use a conditional clause to print \verb|"Units"| if \verb|m|
(which has mode \verb|INT|) is less than \verb|10|, \verb|"Tens"| if
it is less than \verb|100|, \verb|"Hundreds"| if it is less
than~\verb|1000| and \verb|"Too big"| otherwise. \ans \ %
\begin{verbatim}
   IF m < 10
   THEN print("Units")
   ELIF m < 100
   THEN print("Tens")
   ELIF m < 1000
   THEN print("Hundreds")
   ELSE print("Too big")
   FI
\end{verbatim}
'
\item Use a case clause to print the value of a card in words. For
example, if it is a queen, print \verb|"Queen"|.
\ans \ %
\begin{verbatim}
   print((card|"Ace","two","three",
               "four","five","six",
               "seven","eight","nine",
               "ten","Jack","Queen",
               "King"))
\end{verbatim}
'
\end{exercise}
