% $Log: chapter11.tex,v $
% Revision 1.7  2004-11-07 16:00:10  sian
% Removed all references to PLUS, MINUS and TIMES
%
% Revision 1.6  2003/05/31 13:02:41  sian
% Removal of compiled libraries and binaries
%
% Revision 1.5  2003/04/23 08:57:31  sian
% Debian release 1.5
%
% Revision 1.4  2002/06/20 11:49:28  sian
% mm removed, padding removed, ca68 added
%
\Chapter{Advanced constructs}{chapxi}
We have now covered most of Algol 68.  All that remains is the
identity relation, the parallel clause, the mode \verb|BITS|,
completers, different precisions of numbers and means of accessing
operating system facilities. Most of these are deceptively simple.
The identity relation is used with modes containing multiple
\verb|REF|s which take up the greater part of this chapter.  The
\verb|BITS| mode and the parallel clause are introduced in later
sections. Different numerical precisions and access to operating
system facilities is mainly covered in chapter~12. We start with
access to the machine word.
\newpage

\Section{Bits, bytes and words}{adv-words}
In our discussion of \hx{plain values}{value!plain} (values of
modes \verb|CHAR|, \verb|INT|, \verb|REAL|, and \verb|BOOL|), we have
omitted saying how these values are stored in computer memory for one
important reason: Algol~68 is a high-level programming language. A
high-level programming language is one in which the concepts of
computer programming are not expressed in terms of a computer, its
instructions and its memory, but in terms of high-level concepts such
as values and modes. Basically, the manner in which integers and
characters and so on are stored in the computer are not our business.
However, since programs written in Algol~68 need to access the
operating system, it is useful to know something about \ix{memory},
whether the main memory of the computer or the storage memory found
on hard disks and other devices.

Computer memory consists of millions of \textbf{bits} (short for binary
digits) which are grouped together as \bfix{bytes} or \bfix{words}.
A bit can take two values: 0 and 1.  A word is 16, 24, 32, 36, 60, 64
or~72 bits ``wide'', and a byte is 6, 8 or~9 bits ``wide''. Almost
all microcomputers use 8-bit bytes.  Microcomputers using the Intel
Pentium processor (or compatibles) or later chips, use a 32-bit word
and an 8-bit byte.  Generally speaking, a byte is used to store a
character and a word is used to store an integer.  Real numbers are
much more complicated than integers and we shall not describe how
they are stored in memory.  Before we can understand about the
equivalences of values of mode \verb|CHAR| and bytes, and values of
mode \verb|INT| and words, we need to say something about
\hx{\textbf{radix arithmetic}}{radix!arithmetic}.  If this is
something you already know, please skip the next section.

\Subsection{Radix arithmetic}{adv-radix}
Our ordinary arithmetic uses the ten digits 0, 1, $\ldots$, 9 and
expresses numbers in powers of ten. Thus the number 1896 consists of
1 thousand, 8 hundreds, 9 tens and 6 units. This could be written
$$1896=1\times1000+8\times100+9\times10+6\times1$$
Remembering that 100 is ten squared ($10^2$) and 1000 is ten cubed
($10^3$), we could rewrite this equation as
$$1896=1\times10^3+8\times10^2+9\times10^1+6\times10^0$$
As you can see, the powers of ten involved are 3, 2, 1 and 0. When we
write whole numbers, we understand that the digits we use represent
powers of ten. We say that the base, or \bfix{radix}, of our
arithmetic is ten, which is why it is frequently referred to as
``decimal arithmetic'' (\ix{decimal} is derived from the Latin word
for ten).

Now it is quite meaningful to develop an arithmetic having a different
radix. For example, suppose we use two as the radix. We should express
our numbers in terms of powers of two and they would be written using
the digits 0 and 1 only. In an arithmetic of radix two, when we write a
number, each digit would represent a power of two. For example, the
number 101 would mean
$$101=1\times2^2+0\times2^1+1\times2^0$$
in an exactly analogous way to the number 1896 in decimal arithmetic.
In fact, the decimal equivalent of 101 would be $4+0+1=5$ (in decimal).
Here is another example:
\begin{eqnarray*}
1101&=1\times2^3+1\times2^2+0\times2^1+1\times2^0\\
    &=8+4+0+1\\
    &=13\,\hbox{(thirteen, in decimal)}
\end{eqnarray*}
We could then construct addition and multiplication tables as follows:
$$\hbox{%
    \begin{tabular}{|c|cr|} \hline
      $+$ &0&1\\*[2pt] \hline
      0   &0&1\\
      1   &1&10\\*[2pt] \hline
    \end{tabular}\qquad\qquad
    \begin{tabular}{|c|cc|} \hline
      $\times$ &0&1\\*[2pt] \hline
      0        &0&0\\
      1        &0&1\\*[2pt] \hline
    \end{tabular}}
$$
As you can see from the addition table, $1+1=10$ (take row~2 and
column~2).  When you read this equation, you must say ``one-zero'' for
the number after the equals symbol. ``Ten'' means ten$+$zero units
which this number definitely is not. The number 10 in radix 2 means
``two$+$0 units'' which is what you would expect for the sum of 1
and~1.

Two radices of particular use with computers are sixteen and two.
Arithmetic with a radix of sixteen is called \bfix{hexadecimal} and
arithmetic with a radix of two is called \bfix{binary}.

In \hx{hexadecimal arithmetic}{hexadecimal!arithmetic}, the digits 0
to 9 are used, but digits are also required for the numbers ten to
fifteen. The first six letters of the alphabet are used for the
latter six numbers. They are commonly written in upper-case, but in
Algol~68 they are written in lower-case for a reason which will
become apparent in a later section. Thus the ``digits'' used for
hexadecimal arithmetic are
$$\hbox{\verb|0,1,2,3,4,5,6,7,8,9,a,b,c,d,e,f|}$$

Addition and multiplication tables could be constructed for
hexadecimal arithmetic on the same lines as those for radix two
arithmetic. You should note that when writing a number containing
more than one digit with a radix other than ten, the radix is
commonly written (in decimal) as follows:
$$2\times3=12_4$$
Thus, in hexadecimal arithmetic, we could write
$$7\times9=3\hbox{f}_{16}$$
and there are some exercises at the end of this section in which you
can try your hand at hexadecimal and other arithmetics.  Writing
numbers in hexadecimal is sometimes called
``\hx{hexadecimal notation}{hexadecimal!notation}''.

A byte consists of eight binary digits and can take any value from
$00000000_2$ to $11111111_2$.  The equivalent decimal value can be
obtained by rewriting it as the sum of descending powers of two:
\begin{eqnarray*}
10011011_2&=&1\times2^7+1\times2^4+1\times2^3+1\times2^1+1\times2^0\\
          &=&128+16+8+2+1\,\hbox{(in decimal)}\\
          &=&155_{10}
\end{eqnarray*}
The exercises at the end of this section will give you some practice
in this kind of conversion.

If you compare the number of digits used to express the same number,
you will find that hexadecimal arithmetic uses the least. For
example, the decimal number 135 can be written
\begin{eqnarray*}
    135&=&87_{16}\\
       &=&2013_4\\
       &=&10000111_2
\end{eqnarray*}
\hx{When}{radix!conversion} converting numbers written in binary to
hexadecimal, the simplest way is to split the binary number into two
groups of 4 bits and then convert each group into one hexadecimal
digit.  Thus \verb|00101011| can be split into \verb|0010| and
\verb|1011|, and their hexadecimal equivalents are \verb|2| and
\verb|b|.  If you intend accessing machine words, it would certainly be
a good idea to learn the binary equivalents of the 16 hexadecimal
digits \verb|0-f|.  To help you, here is the procedure
\ixtt{itostr} which converts a positive value of
mode \verb|INT| to a value of mode \verb|STRING| (with minimum width)
using any radix from 2 to 16:
\begin{verbatim}
   []CHAR digits="0123456789abcdef"[@0];
   
   PROC itostr=(INT n#umber#,r#adix#)STRING:
   IF n < r
   THEN digits[n]
   ELSE itostr(n%r,r)+digits[n MOD r]
   FI
\end{verbatim}
\noindent
Notice how its recursive definition simplifies the code.
\newpage

\begin{exercise}
\item Using the procedure \verb|itostr|, write a program which will
display the 16 integers between 0 and 15 (decimal) in decimal,
hexadecimal and binary (the binary equivalent should be displayed as 4
bits) in three columns. \ans
\begin{verbatim}
   PROGRAM ex11 1 1 CONTEXT VOID
   USE standard
   BEGIN
      []CHAR digits =
            "0123456789abcdef"[@0];

      PROC itostr = (INT n,r#adix#)STRING:
      IF n<r
      THEN digits[n]
      ELSE itostr(n%r,r)+digits[n%*r]
      FI;

      print(("Table of numbers 0--15",
             newline,newline,
             "Dec. Hex. Binary",newline));

      FOR i FROM 0 TO 15
      DO
         STRING bin = itostr(i,2),
                dec = itostr(i,10),
                hex = digits[i];
                #only one digit#
         print(((4-UPB dec)*blank,
                dec,3*blank,hex,
                4*blank,(4-UPB bin)*"0",
                bin,newline))
      OD
   END
   FINISH
\end{verbatim}
'
\item For each of the following, rewrite the number in the given radix:
\begin{subex}
\item $94_{10}\Rightarrow16$ \subans \begin{eqnarray*}
94_{10}&=&5\times16^1+14\times16^0\\
       &=&5\hbox{e}_{16}
\end{eqnarray*}
'
\item $13_{10}\Rightarrow2$ \subans \begin{eqnarray*}
13_{10}&=&1\times2^3+1\times2^2+0\times2^1+1\times2^0\\
       &=&1101_2
\end{eqnarray*}
'
\item $1111\,1001_2\Rightarrow16$ \subans \begin{eqnarray*}
1111\,1001_2&=&\hbox{f}9_{16}
\end{eqnarray*}
'
\item $3\hbox{e}1_{16}\Rightarrow 10$ \subans \begin{eqnarray*}
3\hbox{e}1_{16}&=&3\times16^2+\hbox{e}\times16^1+1\times16^0\\
               &=&3\times256+14\times16+1\\
               &=&768+224+1\\
               &=&993_{10}
\end{eqnarray*}
'
\item $2\hbox{c}_{16}\Rightarrow2$ \subans $2\hbox{c}_{16}=0010\,1100_2$.
'
\item $10101_2\Rightarrow10$ \subans \begin{eqnarray*}
10101_2&=&1\times2^4+1\times2^2+1\times2^0\\
       &=&16+4+1\\
       &=&21_{10}
\end{eqnarray*}
'
\end{subex}
\item Express the value of each of the following using the radix of that
exercise:
\begin{subex}
\item $101_2+110_2$ \subans $1011_2$
'
\item $35_{16}+\hbox{ae}_{16}$ \subans $\hbox{e}3_{16}$
'
\item $17_8+37_8$ \subans $56_8$
'
\end{subex}
\end{exercise}

\Section{The mode \texttt{BITS}}{adv-BITS}
A value occupying a machine word has the mode \ixtt{BITS}. The number
of binary digits in one machine word is given by the
\ix{environment enquiry} (see section~\hyref{stan-enq})
\ixtt{bits width} which, for
the \hx{a68toc}{a68toc!bits width@\texttt{bits width}} compiler is
\verb|32|. A \verb|BITS| value can be denoted in four different ways
using \hx{denotations}{denotation!BITS@\texttt{BITS}} written with
radices of 2, 4, 8 or~16. Thus the declarations
\begin{verbatim}
   BITS a = 2r 0000 0000 0000 0000
               0000 0010 1110 1101
   BITS b = 4r 0000 0000 0002 3231
   BITS c = 8r  000 0000 1355
   BITS d = 16r 0000 02ed
\end{verbatim}
\noindent
are all equivalent because they all denote the same value.  Notice
that the \ix{radix} precedes the \verb|r| and is written in decimal.
Notice also that the numbers can be written with spaces, or newlines,
in the middle of the number. However, you cannot put a comment in the
middle of the number. Since a \ix{machine word} contains 32 bits,
each denotation should contain 32 digits in radix 2, 16 digits in
radix 4, 11 digits in radix 8 and 8 digits in radix 16, but it is
common practice to omit digits on the left of the denotation whose
value is zero.  Thus the declaration for \verb|d| could have been
written
\begin{verbatim}
   BITS d = 16r2ed
\end{verbatim}
\noindent
When discussing values of mode \verb|BITS| where the values of more
significant bits is important, full denotations like the above may be
more appropriate.

\subsection*{Monadic operators for \texttt{BITS}}\hylabel{adv-mbitops}
There are many operators for \verb|BITS| values.  Firstly, the
monadic operator \ixtt{BIN} takes an \verb|INT| operand and yields
the equivalent value with mode \verb|BITS|.  The operator \ixtt{ABS}
converts a \verb|BITS| value to its equivalent with mode \verb|INT|.
The \ixtt{NOT} operator which you first met in chapter~4
(section~\hyref{choice-bops}) takes a \verb|BITS| operand and yields a
\verb|BITS| value where every bit in the operand is reversed.  Thus
\begin{verbatim}
   NOT 2r 1000 1110 0110 0101
          0010 1111 0010 1101
\end{verbatim}
\noindent
yields
\begin{verbatim}
       2r 0111 0001 1001 1010
          1101 0000 1101 0010
\end{verbatim}
\noindent
Notice that spaces have been used to make these binary denotations
more comprehensible. \verb|NOT| is said to be a
\bfix{bit-wise operator} because its action on each bit is independent of the value
of other bits.

\subsection*{Dyadic operators for \texttt{BITS}}\hylabel{adv-dbitops}
\ixtt{AND} and \ixtt{OR} (both of which you also met in chapter~4)
both take two \verb|BITS| operands and yield a \verb|BITS| value.
They are both bit-wise operators and their actions are summarised as
follows:
$$\begin{tabular}{|c|c|c|c|} \hline
Left Operand&Right Operand&\tt AND&\tt OR\\*[2pt] \hline
      0     &       0     &    0  &   0\\*
      0     &       1     &    0  &   1\\*
      1     &       0     &    0  &   1\\*
      1     &       1     &    1  &   1\\*[2pt] \hline
\end{tabular}$$
For \verb|OR|, the yield of
\begin{verbatim}
   2r 100110 OR 2r 10101
\end{verbatim}
\noindent
is \verb|2r 110111|. The priority of \verb|AND| is 3 and the priority
of \verb|OR| is~2.

The \verb|AND| operator is particularly useful for extracting parts
of a machine word. For example, suppose you have a \ixtt{BITS} value
where the least-significant 8 bits are equivalent to a character. You
could write
\begin{verbatim}
   CHAR c = REPR ABS (b AND 16rff)
\end{verbatim}
\noindent
Here, the operators \ixtt{REPR} and \ixtt{ABS} do not generate
machine-code instructions, but merely satisfy the compiler that the
modes are correct. This sort of formula is, in fact, very efficient
in Algol~68.

It is possible to extract a single bit from a word using the operator
\ixtt{ELEM} which has the header
\begin{verbatim}
   (INT n,BITS t)BOOL:
\end{verbatim}
\noindent
For example, given the declaration
\begin{verbatim}
   BITS bi = 16r 394a 2716
\end{verbatim}
\noindent
then each hexadecimal digit represents 4 bits: the \verb|3| occupies
bit positions 1--4, the \verb|9| occupies bit positions 5--8, the
\verb|4|, bit positions 9--12, and so on. Suppose we want the third bit
(the leftmost bit is bit-1). The following declaration is valid:
\begin{verbatim}
   BOOL bit3 = 3 ELEM bi
\end{verbatim}
\noindent
Thus, if the third bit is a \verb|1|, the declaration will give the
value \verb|TRUE| for bit 3.  In fact, \verb|3| written in
\ix{binary} is $0011_2$, so bit 3 is \verb|1|. Thus
\begin{verbatim}
   2 ELEM bi
\end{verbatim}
\noindent
would yield \verb|FALSE|. The priority of \verb|ELEM| is~7.

Incidentally, notice that in Algol~68 the most significant bit in a
machine word is bit-1 and the least significant bit is bit-32. This
strongly suggests that computers in the 1960's were ``big-endian''.
Intel microprocessors and other compatible processors are
``little-endian''\footnote{These terms come from the book by Jonathan
Swift entitled ``Gulliver's Travels'' and they refer to the habit of
some people of eating boiled eggs at the ``big'' end or the
``little'' end!}. Because the \ix{a68toc} compiler translates
Algol~68 programs into~C programs, it is quite possible for the
Algol68toC system to be implemented on a ``big-endian''
microprocessor, such as the Motorola 68000-series. A ``big-endian''
processor stores a machine word as four bytes (each of 8-bits) with
the most significant byte at the lowest memory address.
``Little-endian'' processors store the least significant byte at the
lowest memory address. Whatever kind of microprocessor is used to
elaborate your programs, the most significant bit of the word is
bit-1 and the least significant bit is bit-32 in Algol~68.

\hx{The}{shift operators} dyadic operators \ixtt{SHL} and \ixtt{SHR}
shift a machine word to the left or to the right respectively by the
number of bits specified by their right operand. Their priority is
\verb|8|. To illustrate their action we shall suppose that they all
operate on the \verb|BITS| value \verb|16r 89ab cdef|.  Both the shifts
are by four bits which is equivalent to one hexadecimal digit (4 bits
is half a byte and is commonly called a \ix{nibble}: yes, software
engineers do possess a sense of humour!).

The result of shifting the above value by 4 bits is given by the
following table:
$$\begin{tabular}{|c|c|c|}
    \hline
    \multicolumn{3}{|c|}{Original value = \texttt{16r 89ab cdef}}
      \\ \hline\hline
    Operator&Bits shifted&Yield
      \\ \hline
    \texttt{SHL}&\ 4&\texttt{9abc def0}\\
    \texttt{SHL}&-4&\texttt{089a bcde}\\
    \texttt{SHR}&\ 4&\texttt{089a bcde}\\
    \texttt{SHR}&-4&\texttt{9abc def0}
      \\ \hline
  \end{tabular}$$
When shifting left (\verb|SHL|), bits shifted beyond the most
significant part of the word are lost. New bits shifted in from the
right are always zero. When shifting right (\verb|SHR|), the reverse
happens. Note that the number of bits shifted should be in the range
$[-32,+32]$. For \verb|SHL|, if the number of bits to be shifted is
negative, the \verb|BITS| value is shifted to the right and likewise
for \verb|SHR|.  The header for \verb|SHL| is
\begin{verbatim}
   OP SHL = (BITS b,INT i)BITS:
\end{verbatim}
\noindent
and correspondingly for \verb|SHR|. The value \verb|b| is the value
to be shifted and the integer \verb|i| is the number of bits to
shift.  \verb|UP| and \verb|DOWN| are synonyms for \verb|SHL| and
\verb|SHR| respectively. The priorities of \verb|SHL| and \verb|SHR|
are both 8.

As well as the operators \verb|=| and \verb|/=| (which have the usual
meaning), the operators \verb|<=| and \verb|>=| are also defined for
mode \verb|BITS|. The formula
\begin{verbatim}
   s >= t
\end{verbatim}
\noindent
yields \verb|TRUE| only if for all bits in \verb|t| that are \verb|1|,
the corresponding bits in \verb|s| are also \verb|1|. This is
sometimes regarded as ``s implies t''. Contrariwise, the
formula
\begin{verbatim}
   s <= t
\end{verbatim}
\noindent
yields \verb|TRUE| only if for all bits in \verb|t| which are \verb|0|,
the corresponding bits in \verb|s| are also \verb|0|. Likewise, this
is sometimes regarded as ``\verb|NOT t| implies \verb|s|''.

\begin{exercise}
\item Given the declarations
\begin{verbatim}
   BITS a = 16r 1111 1111,
        b = 16r 89ab cdef
\end{verbatim}
\noindent
what is the value of each of the following:
\begin{subex}
\item \verb|a AND b| \subans \verb|16r 0101 0101|
'
\item \verb|a OR b| \subans \verb|16r 99bb ddff|
'
\item \verb|NOT a OR b| [Hint: convert each value to radix 2 and then
combine] \subans \verb|16r 6745 2301|
'
\item \verb|a = b| \subans \verb|FALSE|
'
\end{subex}
\item What is the value of
\begin{subex}
\item \verb|16rab SHL 3| \subans \verb|16r 558|
'
\item \verb|16rba SHR 3| \subans \verb|16r 17|
'
\end{subex}
\end{exercise}

\Section{Overlapping slices}{adv-overlap}
\hx{What}{slice!overlapping}\hx{happens}{trimming} if two
trimmed multiples overlap? For example, consider the program
\begin{verbatim}
   PROGRAM slices CONTEXT VOID
   USE standard
   BEGIN
      [4]INT r;

      PROC res = VOID:
      FOR n FROM LWB r TO UPB r
      DO r[n]:=n OD;

      PROC mpr = ([]INT m)VOID:
      (
         FOR i FROM LWB m TO UPB m
         DO
            print((whole(m[i],0),blank))
         OD;
         print(newline)
      ); #mpr#

      res;
      print("Original contents:");  mpr(r);

      r[:UPB r-1]:=r[1+LWB r:];
      print((newline,"r[:3]:=r[2:]",newline,
             "Compiler does it: "));  mpr(r);
      res;

      FOR i FROM LWB r TO UPB r-1
      DO r[i]:=r[i+1] OD;
      print("Forwards loop:    "); mpr(r);
      res;

      FOR i FROM UPB r-1 BY -1 TO LWB r
      DO r[i]:=r[i+1] OD;
      print("Backwards loop:   "); mpr(r);
      res;  r[1+LWB r:]:=r[:UPB r-1];
      print((newline,"r[2:]:=r[:3]",newline,
             "Compiler does it: "));  mpr(r);
      res;

      FOR i FROM 1+LWB r TO UPB r
      DO r[i]:=r[i-1] OD;
      print("Forwards loop:    "); mpr(r);
      res;

      FOR i FROM UPB r BY -1 TO 1+LWB r
      DO r[i]:=r[i-1] OD;
      print("Backwards loop:   "); mpr(r)
   END
   FINISH
\end{verbatim}
\noindent
When compiled and executed, the program gives the following output:
\begin{verbatim}
   Original contents:1 2 3 4 
   
   r[:3]:=r[2:]
   Compiler does it: 2 3 4 4 
   Forwards loop:    2 3 4 4 
   Backwards loop:   4 4 4 4 
   
   r[2:]:=r[:3]
   Compiler does it: 1 1 2 3 
   Forwards loop:    1 1 1 1 
   Backwards loop:   1 1 2 3 
\end{verbatim}
\noindent
Notice that lines 5 and 8 of the results are definitely wrong, but that
the compiler gets it right both times. The lesson is, do not worry
about overlapping multiples:  the compiler will ensure you get the
effect you want.

A different matter is when you want to replace a column of a square
multiple with a row. Here, the overlap is more of a ``crossoverlap''.
In this case you need to be careful---see the next exercise.
\newpage

\begin{exercise}
\item Given a square two-dimensional multiple of integers, write a
procedure which uses trimmers (not necessarily overlapping) to
convert its columns to rows and its rows to columns.  For example:
\begin{verbatim}
   ((1,2,3),         ((1,4,7),
    (4,5,6),   =>     (2,5,8),
    (7,8,9))          (3,6,9))
\end{verbatim}
\noindent
Your procedure should cater for any size of square multiple.
\ans \ %
\begin{verbatim}
   PROC transpose=(REF[,]INT m)VOID:
   IF   1 UPB m - 1 LWB m
               =
        2 UPB m - 2 LWB m
   THEN #m is square#
      REF[,]INT mm=m[@1,@1]; #a precaution#

      FOR i TO 1 UPB mm - 1
      DO
         REF[]INT mr=mm[i,i+1:],
                  mc=mm[i+1:,i];
         []INT temp=mr;
         mr:=mc;  mc:=temp
      OD
   FI
\end{verbatim}
'
\end{exercise}

\Section{Completers}{adv-exit}
Sometimes it is desirable to have more than one possible end-point of
a serial clause. This often happens when a loop needs to be
prematurely terminated so that a surrounding serial clause can yield
a value which is unexpected. A \bfix{completer} is so-called because
it provides a completion point for a
\hx{serial clause}{clause!serial}.  A completer can be placed
wherever a semicolon (the \hx{go-on symbol}{;@\texttt{;}}) can appear
except in enquiry clauses (whether BOOL enquiry clauses or INT
enquiry clauses). It consists of the construct \ixtt{EXIT} followed
by a \bfix{label} and a colon (\verb|:|). A label is formed with the
same rules as for an identifier and should not be the same as any
identifier in the current \ix{range}. Here is an example of a
completer:
\begin{verbatim}
   EXIT label:
\end{verbatim}
\noindent
The label must be referenced by a
\hx{\texttt{GOTO}}{clause!GOTO@\texttt{GOTO}} clause within the same
serial clause in which the completer occurs, or in an inner clause
(not necessarily serial). Here is an example of such a completer:
\begin{verbatim}
   a:=(INT i;  read((i,newline));
       IF i < 0 THEN GOTO negative FI;
       sqrt(i) EXIT
   negative:
       sqrt(-i)
      )
\end{verbatim}
\noindent
The example is artificial, but serves to illustrate the use of a
completer.

A completer can sometimes save the declaration of a boolean name. For
example, here is a procedure without a completer:
\begin{verbatim}
   PROC is in str = (STRING t, CHAR c)BOOL:
   (
      BOOL found := FALSE;
   
      FOR n FROM LWB t TO UPB t
      WHILE NOT found
      DO
         found:=c = t[n]
      OD;
   
      found
   );
\end{verbatim}
\noindent
Here is the procedure with a completer:
\begin{verbatim}
   PROC is in str = (STRING t,CHAR c)BOOL:
   (
      FOR n FROM LWB t TO UPB t
      DO
         IF c = t[n] THEN GOTO found
      OD;
      FALSE EXIT
   found:
      TRUE
   )
\end{verbatim}

In fact, \verb|GOTO| clauses are valid almost anywhere in Algol~68.
They are particularly useful when it is required to jump out of nested
clauses. Let us reconsider the program \verb|echo| in
section~\hyref{trans-logi} with a \verb|GOTO| clause:
\begin{verbatim}
   PROGRAM echo CONTEXT VOID
   USE standard
   BEGIN
   FILE args;
   IF open(args,"",arg channel)/=0
   THEN
      put(stand error,
          ("Cannot access the arguments",
           newline));
      stop
   ELSE
      FILE ff:=args;
      on logical file end(
         ff,
         (REF FILE f)BOOL:
            close(f);  GOTO end;  FALSE));
      DO
         STRING arg;
         get(ff,(skip terminators,arg));
         print((arg,newline))
      OD;
   end:
      print(("End of arguments",newline))
   FI
   END
   FINISH
\end{verbatim}
\noindent
Use of \verb|GOTO| clauses should be confined to exceptions because
otherwise they can destroy the natural structure of your programs
making them much more difficult to understand and maintain.

\Section{References to names}{adv-refref}
The idea that a mode can contain more than one \ixtt{REF}, or that a
mode might be \verb|REF[]REF[]CHAR| was broached at the start of
chapter~5 and mentioned in section~\hyref{gram-deref}. The time has now
come to address this topic fully.

Any mode which starts with \verb|REF| is the mode of a
\ix{name}. The value to which a name refers has a mode with one
\verb|REF| less.  Since names are values in their own right, there is
no reason at all why a name should not refer to a name.  For example,
suppose we declare
\begin{verbatim}
   INT x,y
\end{verbatim}
\noindent
then the mode of both \verb|x| and \verb|y| is \verb|REF INT|. We
could also declare
\begin{verbatim}
   REF INT xx, yy
\end{verbatim}
\noindent
so that \verb|xx| and \verb|yy| both have the mode
\hx{\texttt{REF REF INT}}{REF REF@\texttt{REF REF}}.

Now, according to the definition of an \ix{assignment}
(see section~\hyref{gram-quat}), it is perfectly legitimate to write
\begin{verbatim}
   xx:=x
\end{verbatim}
\noindent
without any dereferencing because the identifier on the left has mode
\verb|REF REF INT| and the identifier on the right has mode
\verb|REF INT|.  Leaving aside for the moment of how useful such
declarations and assignments might be (and they are very useful,
essential even), let us give our attention to the mechanics. We could
assign \verb|y| to \verb|xx| and a value to \verb|y| with the double
assignment
\begin{verbatim}
   xx:=y:=3
\end{verbatim}
\noindent
Again, \hx{no dereferencing}{coercion!dereferencing} is involved.
Now, given that \verb|xx| refers to \verb|y| which refers to
\verb|3|, how could we make \verb|y| refer to \verb|4|, say? Simple.
Assign \verb|4| directly to \verb|y|. However, if the assignment to
\verb|xx| was
\begin{verbatim}
   xx:=(random>0.8|x|y)
\end{verbatim}
\noindent
we should not know which name \verb|xx| referred to. Finding out which
name \verb|xx| refers to is the subject of the next section.

You may remember that the context of the left-hand side of an
assignment \hx{is soft}{context!soft} so no dereferencing is
allowed. The way to coerce a name of mode \verb|REF REF INT| to a
name of mode \verb|REF INT| is to use a cast:
\begin{verbatim}
   REF INT(xx):=4
\end{verbatim}
\noindent
Note that the unit
\begin{verbatim}
   print(xx)
\end{verbatim}
\noindent
will yield \verb|4| with \verb|xx| being dereferenced twice. There is
nothing to stop us writing
\begin{verbatim}
   REF REF INT xxx:=xx
\end{verbatim}
\noindent
with assignments like
\begin{verbatim}
   REF REF INT(xxx):=x
   REF INT(xxx):=-2
\end{verbatim}
\noindent
and we shall see in a later section that names with modes
\verb|REF REF REF some-mode| have a use. Although you can use as many
\verb|REF|s as you like, there does not seem to be any need for more
than three.

Now consider the assignments
\begin{verbatim}
   xx:=yy:=x:=4
\end{verbatim}
\noindent
Both \verb|xx| and \verb|yy| refer to different
\hx{instances}{instance} of the name \verb|x|, but when those
instances are \hx{dereferenced}{coercion!dereferencing}, they both yield
\verb|4|. This means that if we assign \verb|5| to \verb|x|, when
\verb|xx| and \verb|yy| are dereferenced twice, they will both yield
\verb|5|. We can represent this relationship by the diagram
$$\vbox{\epsfbox{ch11-1}}$$
where \verb|RRI| and \verb|RI| stand for \verb|REF REF INT| and
\verb|REF INT| respectively. Thus, although strictly speaking \verb|xx|
and \verb|yy| refer to different instances of the name identified by
\verb|x|, we shall regard them as both referring to \verb|x|.

\begin{exercise}
\item Given the declaration
\begin{verbatim}
   REF REAL xx:=LOC REAL
\end{verbatim}
\noindent
how would you make the \ix{anonymous name} refer to
\verb|120.5|? \ans Use a cast: \verb|REF REAL(xx):=120.5|
'
\item Write a declaration for \verb|rrq| which has the mode
\verb|REF| \verb|REF| \verb|REF[]CHAR| and make it refer to an
anonymous name which refers to an anonymous name which refers to a
multiple of 10 characters. \ans
\begin{verbatim}
   REF REF[]CHAR rrq;
   []CHAR m = "ABCDEFGHIJ";
   rrq:=LOC REF[]CHAR:=LOC[10]CHAR:=m[@1];
\end{verbatim}
'
\item Write the declaration of a name which can refer to a
\ix{flexible name} which can refer to a row of integers. In a
separate assignment, assign the row-display \newline
\verb|(3,-2,4)| to your name. \ans \ \newline%
\begin{verbatim}
   REF FLEX[]INT rfi;
   rfi:=FLEX[1:0]INT:=(3,-2,4)
\end{verbatim}
'
\end{exercise}

\Section{Identity relations}{adv-idrel}
Consider the declarations of the last section:
\begin{verbatim}
   INT x,y; REF INT xx,yy
\end{verbatim}
\noindent
We had assigned a name to \verb|xx| with the assignment
\begin{verbatim}
   xx:=(random > 0.8|x|y)
\end{verbatim}
\noindent
and we wished to ascertain whether \verb|xx| referred to \verb|x| or
to \verb|y|. Unfortunately, we cannot use the equals operator
\verb|=| for this purpose because its operands would be completely
dereferenced and the underlying integers would be compared. Instead,
we use an \hx{\textbf{identity relation}}{identity!relation} which is
used exclusively for comparing names. The identity relation
\begin{verbatim}
   xx :=: x
\end{verbatim}
\noindent
yields \verb|TRUE| if \verb|xx| refers to \verb|x|. The alternative
representation of \ixtt{:=:} is \ixtt{IS}. The identity relation
\begin{verbatim}
   xx :/=: x
\end{verbatim}
\noindent
yields \verb|TRUE| if \verb|xx| does not refer to \verb|x|. The
alternative representation of \ixtt{:/=:} is
\ixtt{ISNT}. Here is a short program which
illustrates the difference between \verb|=| and \verb|IS|:
\begin{verbatim}
   PROGRAM test CONTEXT VOID
   USE standard
   BEGIN
      REF INT xx, INT x:=2,y:=3;
      TO 3
      DO
         xx:=(random>0.5|x|y);
         print(("xx :=: x =",
                (xx :=: x|"TRUE"|"FALSE"),
                newline,"xx = ",xx,newline))
      OD
   END
   FINISH
\end{verbatim}
\noindent
If you want to compare the names that both \verb|xx| and \verb|yy|
refer to, it is no good writing
\begin{verbatim}
   xx IS yy
\end{verbatim}
\noindent
This always yields \verb|FALSE| because the names that \verb|xx| and
\verb|yy| identify always differ (they were created using two local
generators so the names are bound to be different). The point is that
no automatic dereferencing takes place in an identity relation. To
compare the names that both \verb|xx| and \verb|yy| refer to, you
should place one side or both sides in a \ix{cast}:
\begin{verbatim}
   REF INT(xx) IS yy
\end{verbatim}
\noindent
This will ensure that the right-hand side (in this case) is
dereferenced to yield a name of the same mode as the left-hand side.
This is because an identity relation is subject to \ix{balancing}:
one side of the relation is in a \hx{soft context}{context!soft} and
the other side is in a \hx{strong context}{context!strong}.  Given
the cast on the left-hand side, the two sides of the
\hx{identity relation}{identity!relation} would yield \verb|REF INT|
and \verb|REF REF INT|.  Since no dereferencing is allowed in a soft
context, it can be seen that the left-hand side is in the soft
context and the right-hand side is in the strong context.

The \verb|IS| and \verb|ISNT| in the identity relation are not
operators. Since the identity relation is a \ix{quaternary} (see
section~\hyref{gram-quat}), remember to enclose it in
\ix{parentheses} if you want to use it in a formula:
\begin{verbatim}
   IF (field OF struct ISNT xx) & x>=-5
   THEN field OF struct = 0
   ELSE FALSE
   FI
\end{verbatim}

\begin{exercise}
\item The program fragment
\begin{verbatim}
   REF STRING ff, ss; STRING f, s;
   f:="Joan of Arc";
   s:="Robert Burns";
   ff:=(random<0.1|f|s);
   ss:=(ff IS f|s|f)
\end{verbatim}
\noindent
applies to this and the following exercises.

What are the modes of \verb|f| and \verb|ss|? \ans \verb|f| has the
mode \verb|REF STRING| and \verb|ss| has the mode
\verb|REF REF STRING|.
'
\item What does \verb|f| refer to? \ans The multiple of mode
\verb|STRING| whose value is \texttt{"Joan of Arc"}.
'
\item Write a formula which compares the \ith{3}{rd} and \ith{4}{th}
characters of the multiple \verb|f| refers to with the \ith{7}{th}
and \ith{8}{th} characters of the multiple \verb|s| refers to. What
are the modes of the operands of the operator?
\ans \verb|f[3:4]=s[7:8]|. The modes are both \verb|STRING|.
'
\item Write an expression which compares the name referred to by
\verb|ff| with the name referred to by \verb|ss|. \ans Here are three
possible answers:
\begin{verbatim}
   REF STRING(ff) IS ss
   ff IS REF STRING(ss)
   REF STRING(ff) IS REF STRING(ss)
\end{verbatim}
\noindent
You could also use \verb|ISNT|, \verb|:=:| or \verb|:/=:|.
'
\end{exercise}

\Section{The value \texttt{NIL}}{adv-NIL}
Sometimes it is desirable that a name of mode \verb|REF| \verb|REF|
\verb|whatever| should not refer to a definite name (see, for
example, the discussion of queues below). This can be arranged by
making it refer to \ixtt{NIL} which is the only denotation of a name.
The mode of \verb|NIL| is \verb|REF whatever|. For example, consider
\begin{verbatim}
   REF[]CHAR rc=NIL;
   REF INT ri=NIL
\end{verbatim}
\noindent
The first \verb|NIL| has the mode \verb|REF[]CHAR| and the second has
the mode \verb|REF INT|.

Given the declaration
\begin{verbatim}
   REF INT xx:=NIL
\end{verbatim}
\noindent
the mode of \verb|NIL| is \verb|REF INT|. However, although
\verb|NIL| is a name, you cannot assign to it. That is, the assignment
\begin{verbatim}
   REF INT(xx):=4
\end{verbatim}
\noindent
would cause the run-time error
\begin{verbatim}
   Segmentation fault
\end{verbatim}
\noindent
and, very likely, a core dump, when using the
\hx{a68toc}{a68toc!NIL@\texttt{NIL}} compiler.

Nor can you use \verb|NIL| in a formula if that would involve
dereferencing.  The only use of \verb|NIL| is for determining, by
using an identity relation, that a name refers to it.  However, we
shall see in the sections on queues and trees that this is a vital
function.

Now consider the declaration
\begin{verbatim}
   REF REF INT rrri;
\end{verbatim}
\noindent
where the mode of \verb|rrri| is \verb|REF REF REF INT|. We could make
\verb|rrri| refer to \verb|NIL| directly using the assignment
\begin{verbatim}
   rrri:=NIL
\end{verbatim}
\noindent
whence the mode of \verb|NIL| is \verb|REF REF INT|. Or we could use a
\verb|NIL| of mode \verb|REF INT| by using an anonymous name:
\begin{verbatim}
   rrri:=LOC REF INT:=NIL
\end{verbatim}
\noindent
whence the mode of the anonymous name is \verb|REF REF INT|. In the
identity relation
\begin{verbatim}
   rrri IS NIL
\end{verbatim}
\noindent
how can we tell which \verb|NIL| is in use? Of course, we could use a
cast for \verb|rrri|, but there is a simpler and more useful way. First
we declare
\begin{verbatim}
   REF INT nil ri = NIL
\end{verbatim}
\noindent
then \ix{balancing} will ensure that the
\hx{identity relation}{identity!relation}
\begin{verbatim}
   rrri IS nil ri
\end{verbatim}
\noindent
gives the required answer with \verb|rrri| being dereferenced twice.
Alternatively, with the declaration
\begin{verbatim}
   REF REF INT nil rri = NIL
\end{verbatim}
\noindent
we can ensure that the identity relation
\begin{verbatim}
   rrri IS nil rri
\end{verbatim}
\noindent
will also be elaborated correctly. We shall see in the sections on
queues and trees that the declaration of \verb|nil ri| is more useful.

Now consider the declarations
\begin{verbatim}
   INT x:=ENTIER(random * 6), y;
   REF INT xx,yy;
   PROC x or y = REF INT: (random>0.8|x|y)
\end{verbatim}
\noindent
and the identity relation
\begin{verbatim}
   CASE randint(3) IN xx,x or y, NIL ESAC
                   IS
   CASE y IN x, SKIP, yy ESAC
\end{verbatim}
\noindent
The balancing of the identity relation includes balancing of the
\hx{case clauses}{clause!case}. The modes yielded are
\begin{tabbing}
\indent\=\verb|x or y|\qquad\=\kill
\+\verb|xx|\>\verb|REF REF INT|\\
\verb|x or y|\>\verb|PROC REF INT|\\
\verb|NIL|\>\verb|REF whatever|\\
\verb|x|\>\verb|REF INT|\\
\verb|SKIP|\>who knows?\\
\verb|yy|\>\verb|REF REF INT|
\end{tabbing}
In a soft context, these modes become:
\begin{tabbing}
\indent\=\verb|x or y|\qquad\=\qquad\=\kill
\+\+\+\verb|REF REF INT|\\
\verb|REF INT|\\
\verb|REF whatever|\\
\verb|REF INT|\\
who knows?\\
\verb|REF REF INT|
\end{tabbing}
Thus the left-hand side is the soft context and the right-hand side (of
the identity relation) is the strong context (remember that \verb|SKIP|
is only allowed in a strong context), and the final modes are all
\verb|REF INT|. In practice, it is rare that identity relations are so
complicated.

\begin{exercise}
\item Given the declarations
\begin{verbatim}
   FILE f1:=stand in, f2;
   REF FILE cur file:=f2;
   PROC p = REF FILE:
      (cur file IS f1|f1|f2)
\end{verbatim}
\noindent
what is the value of
\begin{subex}
\item \verb|cur file:=f2| \subans A name of mode \verb|REF REF FILE|.
'
\item \verb|cur file :/=: stand in|
\subans \verb|TRUE|\qquad\verb|BOOL|.
'
\item \verb|p:=f1| \subans A name of mode \verb|REF FILE|.
'
\item \verb|p:=:f1| \subans \verb|FALSE|\qquad\verb|BOOL|.
'
\end{subex}
\item Given the declarations of exercise 1, what is the mode of
\verb|NIL| in
\begin{subex}
\item \verb|cur file:=NIL| \subans \verb|REF FILE|
'
\item \verb|REF REF FILE ff:=NIL| \subans \verb|REF REF FILE|
'
\end{subex}
\end{exercise}

\Section{Queues}{adv-queues}
\hx{Consider}{queues} the problem of representing a queue.  We shall
suppose that the queue is at a football match and that each fan in the
queue has a name, in the ordinary sense, and a ticket number.  Rather
than just present the solution to this problem, we shall discuss the
problem in detail and show why the solution is what it is.

A suitable mode for the fan would be\ixtt{FAN}:
\begin{verbatim}
   MODE FAN = STRUCT(STRING name,INT ticket)
\end{verbatim}
\noindent
but what would be a suitable declaration for a queue? At first sight,
it would appear that a \hx{flexible name}{name!flexible} which can
refer to a multiple of fans would be suitable:
\begin{verbatim}
   MODE QUEUE = FLEX[1:0]FAN
\end{verbatim}
\noindent
but there are difficulties. Firstly, the only way a new fan could be
added to the queue would be to assign a whole new multiple to a name
(in the Algol~68 sense) referring to the queue:
\begin{verbatim}
   QUEUE q; q:=q+FAN("Jim",1)
\end{verbatim}
\noindent
assuming that the operator \verb|+| has been declared with the header
\begin{verbatim}
   OP + = ([]QUEUE a,FAN b)[]QUEUE:
\end{verbatim}
\noindent
If the queue were long, this would be very inefficient. Secondly, given
a particular fan, how would we find the fan behind him or her? Knowing
the subscript of the fan would seem to be the answer, but what happens
if someone joins the queue somewhere in front of the fan in question?
Given that there might be several fans under consideration, the program
would have to update all the relevant subscripts and keep a record of
which subscripts would be relevant.

The solution is to represent a queue as a
\hx{recursive structure}{structure!recursive}:
\begin{verbatim}
   MODE QUEUE=STRUCT(FAN fan,REF QUEUE next)
\end{verbatim}
\noindent
Then a queue with two fans in it could be represented by the diagram
$$\vbox{\epsfbox{ch11-2}}$$
where the mode of each box is \ixtt{QUEUE} and
\verb|F| and \verb|RQ| stand for \verb|FAN| and \verb|REF QUEUE|
respectively.  Notice that the \verb|next| field of the first structure
refers to the structure on its right.  The \verb|next| field of the
second structure does not refer to anything.

From the declaration of the mode \verb|QUEUE|, we can see that the
\verb|next| field of the structure is a name with mode
\verb|REF QUEUE|. It would appear that it is possible to construct a
queue in the way depicted by the last diagram: each \verb|next| field
of each structure would refer to the next structure (of mode
\verb|QUEUE|) and the last \verb|next| field would have the mode
\verb|REF QUEUE| and value \verb|NIL|.

Now consider adding another \verb|QUEUE| to the right-hand end of the
queue. We immediately run into a difficulty. The value of the
\verb|next| field of the last \verb|QUEUE| is \verb|NIL| with mode
\verb|REF QUEUE|. However, we cannot assign to \verb|NIL|, nor can we
replace the \ix{name} \ixtt{NIL} with another
name to make it refer to a new \verb|QUEUE|. The reason is that a name
of mode \verb|REF QUEUE| can only be replaced by another name of mode
\verb|REF QUEUE| if the first name is referred to by a name of mode
\verb|REF REF QUEUE|. In other words, instead of making the structures
have mode \verb|QUEUE|, we should make them have mode \verb|REF QUEUE|.
In section~\hyref{struc-field}, on field selection, we pointed out that
the modes of the fields of a structure name are all preceded by a
\verb|REF|. This also applies to a recursively-defined structure. In
this case, the mode of the \verb|next| field becomes
\verb|REF REF QUEUE| and could refer to \verb|NIL| (with mode
\verb|REF QUEUE|) or to another structure of mode \verb|REF QUEUE|. We
can depict this as
$$\vbox{\epsfbox{ch11-3}}$$
where \verb|RQ|, \verb|RRQ| and \verb|RF| represent the modes
\verb|REF QUEUE|, \verb|REF| \verb|REF| \verb|QUEUE| and
\verb|REF FAN| respectively.

Now let us consider how such a queue could be created. Since the length
of the queue at the time the program is written is unknown (and will
change as fans join or leave the queue), it is not possible to have an
identifier for each structure forming the queue. Instead, we can create
anonymous names using a \ix{generator}. However, we must be
able to refer to the queue in order to manipulate it. Let us declare a
name, identified by \verb|head|, to refer to the beginning of the
queue:
\begin{verbatim}
   REF QUEUE head:=NIL
\end{verbatim}
\noindent
We have made it refer to \ixtt{NIL} (with mode
\verb|REF QUEUE|) because the queue is currently empty. Using the
suggestion of the last section, we shall declare
\begin{verbatim}
   REF QUEUE nilq = NIL;
   REF QUEUE head:=nilq
\end{verbatim}
\noindent
where \verb|head| has the mode \verb|REF REF QUEUE|.

Let us assign the first fan to the queue:
\begin{verbatim}
   head:=LOC QUEUE:=(("Jim",1),nilq)
\end{verbatim}
\noindent
We can represent this by the diagram
$$\vbox{\epsfbox{ch11-4}}$$
We can now assign another fan to the queue:
\begin{verbatim}
   next OF head:=LOC QUEUE:=(("Fred",2),nilq)
\end{verbatim}
\noindent
Let us be quite clear what is happening here. The mode of \verb|head|
is \verb|REF REF QUEUE|. It is a name which refers to a name so it has no
fields. We can select \hx{fields}{field selection} only from a
\verb|QUEUE| or a \verb|REF QUEUE|. However, the context of a
\ix{selection} is \hx{weak}{context!weak}(see
section~\hyref{gram-coercions}) and so only weak-dereferencing is
\hx{allowed}{coercion!weakly-dereferencing}. Thus in
\begin{verbatim}
   next OF head
\end{verbatim}
\noindent
\verb|head| is dereferenced to mode \verb|REF QUEUE| and the
\verb|next| field selected (with mode \verb|REF REF QUEUE|). The
anonymous name \verb|LOC QUEUE| has mode \verb|REF QUEUE|, so the
structure display (with mode \verb|QUEUE|) is assigned to it, and it
in turn is assigned to \verb|next OF head| without dereferencing.
This means that the \verb|nilq| which \verb|next OF head| referred to
after the first fan \verb|("Jim",1)| was added to the queue has been
replaced by the second \verb|LOC| \verb|QUEUE| which is what we
wanted. We can now depict the queue by
$$\vbox{\epsfbox{ch11-5}}$$
We could now extend the queue by writing
\begin{verbatim}
   next OF next OF queue:=LOC QUEUE
\end{verbatim}
\noindent
but since we do not know how long the queue might become, clearly we
cannot go on writing
\begin{verbatim}
   next OF next OF ...
\end{verbatim}
\noindent
What we need is some way of referring to the tail of the queue without
lots of selections. Because the tail of the queue always has mode
\verb|REF REF QUEUE| (it is the \verb|next| field of a
\verb|REF QUEUE|), what we need is a name of mode
\verb|REF REF REF QUEUE| (yes, three \verb|REF|s). Here it is:
\begin{verbatim}
   REF REF QUEUE tail;
\end{verbatim}
\noindent
but again we run into a difficulty (the last one). When the queue is
empty, \verb|head| refers to \verb|nilq|, but what does \verb|tail|
refer to since we cannot select from \verb|nilq| (because it is
\verb|NIL|)? The solution is to make \verb|head| have the mode
\verb|REF REF REF QUEUE| as well as \verb|tail| and generate a name
of mode \verb|REF REF QUEUE| to refer to \verb|nilq|! (bear with it,
we're almost there):
\begin{verbatim}
   tail:=head:=LOC REF QUEUE:=nilq
\end{verbatim}
\noindent
In this triple assignment, only \verb|head| is dereferenced. We can
depict this as
$$\vbox{\epsfbox{ch11-6}}$$
Now we can assign the first fan to the head of the queue:
\begin{verbatim}
   REF REF QUEUE(head):=LOC QUEUE:=
         (("Jim",1),nilq))
\end{verbatim}
\noindent
and make \verb|tail| refer to the tail of the queue with
\begin{verbatim}
   tail:=next OF head
\end{verbatim}
\noindent
in which \verb|head| is dereferenced twice, but the
\ix{selection} is not dereferenced at all. The queue can
now be depicted as shown below.
$$\vbox{\epsfbox{ch11-7}}$$
A queue is one example of what is called a
\bfix{linked-list}.

\begin{exercise}
\item Extend the queue by assigning another \verb|REF QUEUE| to
\verb|tail|. \ans \begin{verbatim}
REF REF QUEUE(tail):=
        LOC QUEUE:=(("Barbara",3),nilq)
\end{verbatim}
'
\item Now make \verb|tail| refer to the tail of the queue again.
\ans \verb|tail:=next OF tail|
'
\item Has the name referred to by \verb|head| changed after adding the
new \verb|REF QUEUE|? \ans No.
'
\end{exercise}

\Section{The procedure \texttt{add fan}}{adv-addfan}
We are now ready to develop a procedure to add a fan to the end of the
queue. Clearly, there are two different situations: an empty queue and
a non-empty queue. Although we only need \verb|tail| to extend the
queue, we need \verb|head| to determine whether the queue is empty. So
here is the header:
\begin{verbatim}
   PROC add fan = (REF REF REF QUEUE head,tail,
                   REF FAN fan)VOID:
\end{verbatim}
\noindent
Firstly, we need to test whether the queue is empty:
\begin{verbatim}
   IF head IS nilq
\end{verbatim}
\noindent
Remember that the mode of \verb|head| is \verb|REF REF REF QUEUE|, so
in the \hx{identity relation}{identity!relation} \verb|head| is
\hx{dereferenced}{coercion!dereferencing} twice.

Secondly, if the queue is empty, we assign an
\hx{anonymous}{name!anonymous}
\verb|REF QUEUE| to the name \verb|head| refers to and assign
\verb|(fan,nilq)| to the \verb|REF QUEUE|:
\begin{verbatim}
   THEN REF REF QUEUE(head):=
            LOC QUEUE:=(fan,nilq)
\end{verbatim}
\noindent
but this will not work because the scope of the \verb|LOC QUEUE| is
limited to the routine denotation. We must use a global generator:
\begin{verbatim}
   THEN REF REF QUEUE(head):=
            HEAP QUEUE:=(fan,nilq)
\end{verbatim}
\noindent
Then we have to ensure that \verb|tail| refers to the tail of the queue:
\begin{verbatim}
   tail:=next OF head
\end{verbatim}
\noindent
If the queue is not empty, we assign an anonymous \verb|REF|
\verb|QUEUE| to the name that \verb|tail| points to:
\begin{verbatim}
   ELSE REF REF QUEUE(tail):=
            HEAP QUEUE:=(fan,nilq)
\end{verbatim}
\noindent
and make \verb|tail| refer to the new tail:
\begin{verbatim}
   tail:=next OF tail
\end{verbatim}
\noindent
Here is the complete procedure:
\begin{verbatim}
   PROC add fan = (REF REF REF QUEUE head,tail,
                   REF FAN fan)VOID:
   IF head IS nilq
   THEN #the queue is empty#
      REF REF QUEUE(head):=
            HEAP QUEUE:=(fan,nilq);
      tail:=next OF head
   ELSE
      REF REF QUEUE(tail):=
            HEAP QUEUE:=(fan,nilq);
      tail:=next OF tail
   FI #add fan#
\end{verbatim}

\begin{exercise}
\item It looks as though \verb|add fan| could be
\hx{optimised}{optimisation}. Rewrite the body of \verb|add|
\verb|fan| so that the overall structure is
\begin{verbatim}
   tail:=next OF (REF REF QUEUE
      CO IF ... FI  plus two assignments CO
                 )
\end{verbatim}
\indent \ans \ %
\begin{verbatim}
   PROC add fan=(REF REF REF QUEUE
                    head,tail,
                 REF FAN fan)VOID:
      tail:=next OF (REF REF QUEUE
             (head IS nilq|head|tail):=
                HEAP QUEUE:=(fan,nilq))
\end{verbatim}
'
\item Write a program containing the necessary declarations and loop to
create a queue containing 1000 fans---alternate the names of the fans
between \verb|Iain| and \verb|Fiona| and increment the ticket
numbers by 1. Compile and run the program to check that there are no
errors (no output will be produced). \ans \ %
\begin{verbatim}
   PROGRAM ex11 9 2 CONTEXT VOID
   USE standard
   BEGIN
      MODE FAN = STRUCT(STRING name,
                        INT ticket),
         QUEUE = STRUCT(FAN fan,
                        REF QUEUE next);

      REF QUEUE nilq = NIL;

      PROC add fan=(REF REF REF QUEUE
                        head,tail,
                    REF FAN fan)VOID:
         tail:=next OF
                (REF REF QUEUE
                  (head IS nilq|head|tail)
                     :=HEAP QUEUE
                     :=(fan,nilq);

      REF REF QUEUE head,tail;
      head:=tail:=LOC REF QUEUE:=nilq;

      FOR q TO 1000
      DO
         add fan(head,tail,
                 LOC FAN:=(IF ODD q
                           THEN "Iain"
                           ELSE "Fiona"
                           FI,
                           q))
      OD
   END
   FINISH
\end{verbatim}
\noindent The generator \verb|LOC FAN| is used because \verb|add fan|
requires a parameter of mode \verb|REF FAN|. The scope of the
generated name is from the declarations of \verb|head| and
\verb|tail| to the end of the program because there are no identity
declarations in the \verb|FOR| loop clause (therefore it is not a
range).
'
\end{exercise}

\Section{More queue procedures}{adv-mqproc}
\hx{We}{queue procedures} can now address three more procedures: how
to insert a fan into a queue, how to remove a fan from the queue, and
how to print the queue. Let us take the printing procedure first. Here
it is:
\begin{verbatim}
   PROC print queue = (REF REF QUEUE head)VOID:
   IF head IS nilq
   THEN print(("NIL",newline))
   ELSE print((newline,
               "(",name OF fan OF head,",",
               whole(ticket OF fan OF head,0),
               ")=>"));
        print queue(next OF head)
   FI
\end{verbatim}
\noindent
By not using the triple \verb|REF| name for the head of the queue, we
can use \ix{recursion} to simplify the procedure. Recursion is common
in procedures for \hx{linked-lists}{linked-list}.

Inserting a fan is a little more difficult. There are several
possibilities: the queue can be empty or non-empty. If it is
non-empty, the fan can be inserted at the head of the queue, or if
there are at least two fans in the queue, the fan could be inserted
somewhere between the head and the tail. The question is, how many
parameters are required for the procedure? Clearly, we need
\verb|head| to determine whether the queue is empty, \verb|tail| to
be updated in case it is or if the fan is to be added to the end of
the queue. Here is a possible header:
\begin{verbatim}
   PROC insert fan=(REF REF REF QUEUE head,tail,
                    REF FAN fan)VOID:
\end{verbatim}
\noindent
We need a criterion for determining where in the queue a fan should be
inserted. Here is one: the fans should be inserted in the order of
ticket number (using a queue is not an efficient way of doing this, but
this criterion will do for our purposes). Here is \verb|insert fan|
with a diagram to help you understand it:
\begin{verbatim}
   PROC insert fan=(REF REF REF QUEUE head,tail,
                    REF FAN fan)VOID:
   IF   head IS nilq
   THEN #the queue is empty#
      REF REF QUEUE(head):=
            HEAP QUEUE:=(fan,nilq);
      tail:=next OF head
   ELIF ticket OF fan < ticket OF fan OF head
   THEN
      #insert the fan at the head of the queue#
      REF REF QUEUE(head):=
            HEAP QUEUE:=(fan,head)
   ELIF next OF head IS nilq
   THEN #add the fan after the head#
      REF REF QUEUE(tail):=
            HEAP QUEUE:=(fan,nilq);
      tail:=next OF tail
   ELIF REF QUEUE marker:=head;
      WHILE
         IF (next OF marker ISNT nilq)
         THEN
            ticket OF fan
                  >
            ticket OF fan OF next OF marker
         ELSE FALSE
         FI
      DO marker:=next OF marker OD;
      next OF marker IS nilq
   THEN
   #add the fan to the end of the queue#
      REF REF QUEUE(tail):=
            HEAP QUEUE:=(fan,nilq);
      tail:=next OF tail
   ELSE
   CO insert the fan between `marker'
      and `next of marker' CO
      next OF marker:=
            HEAP QUEUE:=(fan,next OF marker)
   FI
\end{verbatim}
\noindent
$$\vbox{\epsfbox{ch11-8}}$$
There are three lines where you need to look carefully at the modes and
values involved:
\begin{itemize}
\item the line which ends in \verb|(fan,head)|,
\item the line which ends in \verb|(fan,next OF marker)|,
\item the line containing the \verb|>| operator.
\end{itemize}
Discussion of this procedure completes our examination
\hx{of}{linked-list} \ix{queues}.
\begin{exercise}
\item In the procedure \verb|insert fan|, explain the circumstances
in which the loop will terminate. \ans Because \verb|marker| has mode
\verb|REF REF QUEUE|, it is made to refer to each \verb|REF QUEUE|
name in the linked-list. The condition
\begin{verbatim}
   next OF marker ISNT nilq
\end{verbatim}
\noindent
ensures that \verb|marker| is not currently referring to the last
\verb|REF QUEUE| in the list. The loop will terminate when
\verb|marker| refers to the last \verb|REF QUEUE| in the list or the
number of the ticket of the fan to be inserted in the queue does not
exceed the number of the ticket of the fan referred to by
\verb|marker|.
\medskip
If the operator \verb|AND| had been used, both operands would have been
elaborated before the operator; in which case, if the left operand
had yielded \verb|FALSE|, elaboration of the right operand would have
caused the run-time error \verb|"Selection from NIL"|.
'
\item Using the procedure \verb|print queue|, confirm that the
procedure \verb|insert fan| works. \ans This can best be done by
writing a program. Here is a possible solution:-
\begin{verbatim}
   PROGRAM ex11 10 2 CONTEXT VOID
   USE standard
   BEGIN
      MODE FAN = STRUCT(STRING name,
                        INT ticket),
         QUEUE = STRUCT(FAN fan,
                        REF QUEUE next);
      REF QUEUE nilq = NIL;

      PROC insert fan = . . .
      . . .
      PROC print queue = . . .
      . . .
      REF REF QUEUE head,tail;
      head:=tail:=LOC REF QUEUE:=nilq;
      INT max ticket = 1000;
      INT tickets issued:=0;
      [max ticket]BOOL ticket issued;

      FOR i
      FROM LWB ticket issued
      TO UPB ticket issued
      DO FALSE OD;

      WHILE tickets issued < max ticket
      DO
         INT i=random int(max ticket);
         IF REF BOOL ti=ticket issued[i];
            NOT ti
         THEN
            ti:=TRUE;
            insert fan(head,tail,HEAP FAN:=
                       ((ODD i
                        |"Iain"
                        |"Fiona"
                        ),i));
            tickets issued+:=1
         FI
      OD #fans added to the queue#;

      print queue(head)
   END FINISH
\end{verbatim}
\noindent Instead of sending the output to \verb|stand out|, it would be
better to direct it to an output book so that the results could be
examined at leisure.  Alternatively, command line redirection could
be used.  The use of \verb|ticket issued| ensures that unique ticket
numbers are added to the queue since \verb|insert fan| does not cater
explicitly for duplicate ticket numbers.
'
\item Write the procedure \verb|delete fan| which will delete a fan
with a given ticket number from the queue. It should yield the fan if
it has been deleted and \verb|FALSE| if it cannot be found. This
diagram should help you:
$$\vbox{\epsfbox{ch11-9}}$$
Include the procedure in a program and test it. \ans The procedure
has to find the fan concerned and must keep track of the reference to
that fan.
\begin{verbatim}
   PROC delete fan=(REF REF QUEUE q,
                    INT t#icket#
                   )UNION(REF FAN,BOOL):
   IF   q IS nilq
   THEN FALSE #empty queue#
   ELIF next OF q IS nilq
   THEN #last fan in the queue#
      IF   ticket OF q = t
      THEN REF FAN rf = q;
         q:=nilq;  #delete last fan#
         rf
      ELSE FALSE
      FI
   ELIF ticket OF next OF q < t
   THEN delete fan(next OF q,t)
   ELIF ticket OF next OF q > t
   THEN #not found# FALSE
   ELSE REF FAN rf = next OF q;
      next OF q:=next OF next OF q;
      rf
   FI #delete fan#;
\end{verbatim}
\noindent
In the assignment, the mode of \verb|next OF q| is \verb|REF|
\verb|REF| \verb|QUEUE|, so the mode of \verb|next OF next OF q| must
be \verb|REF| \verb|QUEUE|. Look at the required dereferencing to see
what is assigned to \verb|next OF q|.
'
\end{exercise}

\Section{Trees}{adv-trees}
Both queues and trees are examples of
\hx{recursive structures}{structure!recursive}. Queues contain only
one link between individual structures, trees contain at least two.
%The concordance program described in the next chapter uses trees for
%data storage.
Trees are another kind of \ix{linked-list} and are
interesting because they give more examples of how
\hx{recursive procedures}{procedure!recursive} are used to manipulate
recursively-defined data structures.

There are two principal kinds of \hx{trees}{tree} in common use:
B-trees and binary trees. \ix{B-trees} (sometimes called
\bfix{balanced trees}) are too advanced to be described here.

A binary tree consists of a number of forks, usually called
\bfix{nodes}, which are linked with two links per node.

Here is an example of a small tree:\par
\vfill\goodbreak
\noindent\vbox{\epsfbox{ch11-10}}
The topmost node is called the \bfix{root} (trees are
usually depicted upside-down\footnote{The remainder of the intriguing
limerick runs as follows:-
\protect\begin{verse}
Who ate sour apples and died.\protect\\
The apples fermented inside the lamented\protect\\
and made cider inside 'er inside.
\protect\end{verse}}). Each node consists of three parts: the
data which each node bears and left and right references which can
refer to other nodes. In the small tree shown above, there are seven
nodes on five levels. There are 4 nodes on the left branch of the root
and 2 on the right, so that the tree is unbalanced.

A binary tree is particularly suitable for the ordering of data: that
is, for arranging data in a predefined order\footnote{but only if the
tree is reasonably balanced}. In the previous section, in procedure
\verb|insert fan|, we considered inserting a fan into a queue in
ascending order of ticket number. This is an inefficient way of
ordering data. For example, suppose there are 100 fans in the queue.
Then, on average, we can expect to insert a fan halfway down the
queue; which means 50 comparisons of ticket numbers. If the fans were
stored as a balanced binary tree, the maximum number of comparisons
would be only 7 (because $2^6 < 100 < 2^7$). For larger numbers, the
difference between the two kinds of linked-list is even more marked.
For 1000 fans, a queue would need 500 comparisons on average, whereas
a balanced binary tree would need 10 at most. While it is true that
these figures are minima (they assume that the tree is
\hx{balanced}{tree!balanced}, that is, that there are as many nodes
to the left of the root as to the right), nevertheless, on average, a
binary tree is much more efficient than a queue for ordering
data.

Here is a typical mode declaration for a binary tree:
\begin{verbatim}
   MODE WORD = STRUCT(STRING wd,
                      INT ct,
                      REAL fq),
        TREE = STRUCT(REF WORD w,
                      REF TREE left,right);
\end{verbatim}
\noindent
The mode of the data in the declaration of \verb|TREE| is
\texttt{REF WORD} so that if an item of data is moved around, it is only the
reference which is moved. This is more efficient than moving the data
item itself.

We shall give two example tree procedures: adding an item of data to
the tree and printing the tree. We need to check whether the tree at
some node is empty. For this, we use the declaration
\begin{verbatim}
   REF TREE leaf = NIL
\end{verbatim}
\noindent
Here is the procedure \verb|add word|:
\begin{verbatim}
   PROC add word = (REF REF TREE root,
                    REF WORD w)VOID:
   IF   root IS leaf
   THEN root:=HEAP TREE:=(w,leaf,leaf)
   ELIF wd OF w < wd OF w OF root
   THEN add word(left OF root,w)
   ELIF wd OF w > wd OF w OF root
   THEN add word(right OF root,w)
   ELSE ct OF w OF root+:=1
   FI
\end{verbatim}
\noindent
The ordering relation in \verb|add word| is the alphabetical ordering
of the string in each data item. When the string in the data item to be
added to the tree has been found in the tree, the occurrence number is
incremented by~1 (see the \verb|ELSE| clause above).
%The procedure \verb|add word| can be found in the concordance program
%(see section~\hyref{dev-conc}).
Note the use of \ix{recursion}.

Printing the tree follows a similar pattern, but when the ``root'' under
consideration is a leaf, nothing happens:
\begin{verbatim}
   PROC print tree=(REF FILE f,
                    REF REF TREE root)VOID:
   IF   root ISNT leaf
   THEN print tree(f,left OF root);
        put(f,(wd OF w OF root,
               ct OF w OF root,
               newline));
        print tree(f,right OF root)
   FI
\end{verbatim}
\noindent
As you can see, recursion is vital here. Although it is true that
recursion can be avoided by using a loop, recursion is better because
it clarifies the logic.

The allocation and release of memory for linked-lists (including
trees) are quite transparent to the program.  When a tree is read,
and nodes possibly deleted, all the lost memory is collected every so
often by a \bfix{garbage collector}.  You do not have to worry about
the details of memory maintenance---it is all done for you by the
compiler and the run-time system.  If you write a program which
relies heavily on global generators, then you should allocate extra
memory to the heap (see the on-line information for details of how to
use the Algol 68 compilation \hx{system}{memory control}).

\begin{exercise}
\item Write a program which reads a text book and creates a binary tree
containing the number of occurrences of each of the letters A--Z and
a--z (that is, case is significant). Print a report with the frequency
of occurrence represented by a percentage of the total number of
letters in the book to 2 decimal places. You should print the letters
going downwards with 13 lines for each column: first the upper case
letters, then the lower case. Only print lines for those letters which
occur in the book (use \verb|mem channel| to build the complete table in
memory before printing).\ans \ \newline%
\begin{verbatim}
   PROGRAM ex11 11 1 CONTEXT VOID
   USE standard
   BEGIN
      MODE
      LETTER=STRUCT(CHAR c,INT o),
      TREE=STRUCT(REF LETTER l,
                  REF TREE left,right);

      REF TREE leaf=NIL;
      REF TREE root:=leaf;

      PROC get letter=(REF FILE f)
                        REF LETTER:
      IF   CHAR ch;  get(f,ch);
           ch>="A" & ch<="Z"
                  OR
           ch>="a" & ch<="z"
      THEN HEAP LETTER:=(ch,1)
      ELSE get letter(f) #skip non-letters#
      FI #get letter#;

      PROC add letter=
            (REF REF TREE root,
             REF LETTER let)VOID:
      IF   root IS leaf
      THEN root:=HEAP TREE:=(let,leaf,leaf)
      ELIF c OF l OF root > c OF let
      THEN add letter(left OF root,let)
      ELIF c OF l OF root < c OF let
      THEN add letter(right OF root,let)
      ELSE o OF l OF root+:=1
      FI #add letter#;

      FILE inf, arg;
      STRING in bk;

      INT max row=13;
      [max row,81]CHAR out page;
      INT row:=max row, col:=0;

      FOR i TO max row
      DO
         out page[i,:80]:=80*blank;
         out page[i,81]:=lf
      OD #initialise out page#;

      INT num letters:=0;

      PROC put letter=(REF LETTER let)VOID:
      BEGIN
         IF   row=max row
         THEN col+:=1;  row:=1
         ELSE row+:=1
         FI;
         FILE f;
         establish(f,
                   "",
                   mem channel,
                   1,1,20);
         put(f,(c OF let,
                fixed(o OF let/
                        num letters*100,
                      -7,2),blank*12));
         out page[row,(col-1)*20+1:col*20]
           :=file buffer(f);
         close(f)
      END #put letter#;

      PROC print tree=
            (REF REF TREE root)VOID:
      IF   root ISNT leaf
      THEN
         print tree(left OF root);
         IF o OF l OF root > 0
         THEN put letter(l OF root)
         FI;
         print tree(right OF root)
      FI #print tree#;

      IF open(arg,"",arg channel)/=0
      THEN
         put(stand error,
             ("Cannot access arguments",
              newline));
         stop
      ELIF
         on logical file end(arg,
          (REF FILE f)BOOL:
           (put(stand error,
             ("Usage: tt in-book",
              newline)); stop; SKIP));
         get(arg,(LOC STRING,
                  LOC CHAR,
                  in bk));
         open(inf,
              in bk,
              stand in channel)/=0
      THEN
         put(stand error,
             ("Cannot open book ",in bk,
              newline));
         stop
      ELSE
         on logical file end(inf,
          (REF FILE f)BOOL:
           (
            print tree(root);
            print((
               "Frequency of occurrence ",
               "of letters in the book ",
               idf(f),newline,
               newline,out page,newline,
               "Total letters read: ",
               whole(num letters,0),
               newline));
            stop;  SKIP
           ))
      FI;

      FOR i TO 26 #letters in the alphabet#
      DO
         add letter(
            root,
            HEAP LETTER:=
             (REPR(ABS("A")-1+i),0));
         add letter(
            root,
            HEAP LETTER:=
             (REPR(ABS("a")-1+i),0))
      OD #all letters are now in the tree#;

      DO
         add letter(root,get letter(inf));
         num letters+:=1
      OD
   END
   FINISH
\end{verbatim}
'
\end{exercise}

\Section{Parallel programming}{adv-par}
Unfortunately, the a68toc compiler does not provide parallel
programming, so this section has been removed for the third edition of
this book. It will be reinstated when an Algol~68 compiler is made
available incorporating parallel programming.
\newpage

\Section{Summary}{adv-summ}
A machine word is accessed using the mode \verb|BITS| and a number of
operators. A value of mode \verb|BITS| can be denoted using binary,
quaternary (radix 4), octal or hexadecimal digits.  Names which refer
to names form the basis of self-referential modes (via \verb|STRUCT|
and \verb|REF|) from which we can construct queues and trees.  Some of
the basic procedures were covered which manipulate these data
structures.
