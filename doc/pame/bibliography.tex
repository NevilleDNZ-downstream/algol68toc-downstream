\Chapter{Bibliography}{biblio}
For a thorough treatment of the language from a more old-fashioned point of
view, I can recommend this book:-
\begin{itemize}
\item{} Lindsey, C. H. and van der Meulen, S. G., Informal Introduction to
Algol 68, North-Holland (1977).
\end{itemize}
The original report is not for the faint-hearted, but it is the final arbiter
of what constitutes Algol~68. Do not make the mistake of the many detractors
of Algol~68 who confused the method of description (a two-level grammar) with the
language itself. If you have read as far as here, you will know that
Algol~68 is easier to learn than to describe:-
\begin{itemize}
\item{} van Wijngaarden, A., Mailloux, B. J., Peck, J. E. L., Koster, C. H.
A., Sintzoff, S., Lindsey, C. H., Meertens, L. G. L. T. and Fisker, R. G.
(eds), Revised Report on the Algorithmic Language Algol 68, Springer-Verlag
(1976).
\end{itemize}
This little book contains much wisdom about solving problems. It \emph{is}
geared towards mathematical problems, but you should not find it too difficult
to apply to a whole range of other problems. It used to be the set book for
the {\it Foundation Course\/} in Mathematics at the Open University:-
\begin{itemize}
\item{}P\'olya, G., How to Solve It, 2nd ed., Penguin Books (1985).
\end{itemize}
Jackson's original book is well worth reading if you are considering taking up
programming seriously or even if you are already a professional programmer:-
\begin{itemize}
\item{}Jackson, M. A., Principles of Program Design, Academic Press
(1975).
\end{itemize}
Details of the floating-point processor within the Intel Pentium
microprocessor were taken from the following books:-
\begin{itemize}
\item{}Intel Architecture Software Developer's Manual, \emph{Volume
I, Basic Architecture}, Intel Corporation, 1999.
\item{}Intel Architecture Software Developer's Manual, \emph{Volume
II, Instruction Set Reference}, Intel Corporation, 1999.
\end{itemize}
