% $Log: chapter6.tex,v $
% Revision 1.3  2002-06-20 11:49:28  sian
% mm removed, padding removed, ca68 added
%
\Chapter{Routines}{chapvi}
Routines consist of two types: operators and procedures. They have
much in common, so the first section covers their common aspects.
These are followed by a section on operators and a section on
procedures. The length of this chapter reflects the importance of
routines in the language.

\Section{Routines}{rout-rout}
A \bfix{routine} is a number of encapsulated actions
which can be elaborated in their entirety in other parts of the
program. A routine has a well-defined mode. The value of a routine is
expressed as a \hx{\textbf{routine denotation}}{denotation!routine}.
Here is an example:
\begin{verbatim}
   ([]INT a)INT:
   (
      INT sum:=0;

      FOR i FROM LWB a TO UPB a DO sum+:=i OD;
      sum
   )
\end{verbatim}
\noindent
In this example, the \bfix{header}\hx{ }{routine!header}of the
routine is given by
\begin{verbatim}
   ([]INT a)INT:
\end{verbatim}
\noindent
which could be read as ``with (parameter) row of \verb|INT| \verb|a|
yielding \verb|INT|''.  The \hx{mode}{mode!routine} of the routine is
given by the header, less the colon and any identifiers. So the mode
of the above routine is
\begin{verbatim}
   ([]INT)INT
\end{verbatim}
\noindent
We say that the routine takes one \textbf{parameter} of mode
\verb|[]INT| and \hx{yields}{routine!yield} a value of mode
\verb|INT|.

As you can see from the \hx{\textbf{body}}{routine!body} of the
routine (everything except the header), the routine yields the sum of
the individual elements of the \ix{parameter}.  The body of a routine
is a \ix{unit}. In this case, it is an enclosed
\hx{clause}{clause!enclosed}.

We have met parameters before in a different guise. The formal
definition of an
\hx{identity declaration}{declaration!identity!formal definition}
is
$$\hbox{\texttt{<formal-mode-param> = <actual-mode-param>}}$$
The \hx{formal-mode-param}{parameter!formal} consists of an
identifier preceded by a \ix{formal-mode-declarer} (referred to in
the last chapter as a \ix{formal-declarer}). An
\hx{actual-mode-param}{parameter!actual} is a piece of program which
yields an internal object which henceforth is identified by the
\ix{identifier}. For example, in the identity declaration
\begin{verbatim}
   []INT a = (2,3,5,7,11)
\end{verbatim}
\noindent
\verb|[]INT a| is the formal (mode) parameter, \verb|[]INT| is the
formal (mode) declarer, the identifier is \verb|a|, and the actual
(mode) parameter is the \hx{row-display}{row!display}
\verb|(2,3,5,7,11)|.  The word ``mode'' was placed in parentheses
because it is common usage to omit it.  Henceforth, we shall talk
about formal parameters and actual parameters.

In the header of the above routine, \verb|a| is declared as a formal
parameter.  The mode of \verb|a| is \verb|[]INT|.  At the time the
routine is declared, \verb|a| does not identify a value.  That is why
it is called a ``formal'' parameter.  It is only when the routine is
used that \verb|a| will identify a value.  We'll come to that later.
Any identifier may be used for the formal parameter of a routine.

In the body of the routine, \verb|a| is treated as though it has a
value. Since its mode is \verb|[]INT|, it is a multiple and so it
can be sliced to access its individual elements.

The body of the routine written above consists of an enclosed
\hx{clause}{clause!enclosed}. In this case, the enclosure consists
of the parentheses \verb|(| and \verb|)|, but it might well have been
written using \ixtt{BEGIN} and \ixtt{END}. Inside the enclosure is a
\hx{serial clause}{clause!serial} consisting of three phrases. The
first is a declaration with an \ix{initial assignment}.  Although an
assignment yields a name, an identity declaration with an initial
assignment, even an abbreviated one, does not.  This is the only
exception.

The second phrase is a \verb|FOR| \hx{loop clause}{clause!loop}
which yields \verb|VOID| (see section~\hyref{rout-VOID}).  The third
phrase consists of the identifier \verb|sum| which yields its name
of mode \verb|REF INT|.

Now, according to the header of the routine, the routine must yield a
value of mode \verb|INT|.  The \hx{context}{routine!context} of the
body of a routine is strong.  \hylabel{rout-cont}Although a serial
clause cannot be coerced, the context of the serial clause is passed
to the last phrase of that clause.  In this case, we have a value of
mode \verb|REF INT| which, in a \hx{strong context}{context!strong},
can be coerced to a value of mode \verb|INT| by
\hx{dereferencing}{coercion!dereferencing}.

\begin{exercise}
\item What is the formal definition of an identity declaration?
\ans An identity declaration is
\begin{verbatim}
<formal-mode-param> = <actual-mode-param>
\end{verbatim}
'
\item Why is the parameter of a routine denotation called a formal
parameter? \ans Because it is an identifier with a mode, but no
associated value.
'
\item In the routine denotation
\begin{verbatim}
   (REAL r)INT: ENTIER r;
\end{verbatim}
\indent\begin{subex}
\item What is the mode of the formal parameter? \subans \verb|REAL|
'
\item What is the mode of the value yielded? \subans \verb|INT|
'
\item What is the context of the body of the routine? \subans Strong
'
\item If the value of \verb|r| were \verb|-4.6|, what value
would the routine yield? \subans \verb|-5|
'
\end{subex}
\item Write a routine which takes a parameter of mode \verb|[]INT| and
yields a value of mode \verb|[]CHAR|, where each element of the result
yields the character equivalent of the corresponding element in the
parameter (use \verb|FOR| and \verb|REPR|). \ans Using a loop:-
\begin{verbatim}
   ([]INT i)[]CHAR:
   (
      [LWB i:UPB i]CHAR s;

      FOR n FROM LWB i TO UPB i
      DO
      s[n]:=REPR ii
      OD;
      s
   )
\end{verbatim}
'
\end{exercise}
\vfill

\Subsection{Routine modes}{rout-modes}
In general, a routine may have any number of parameters, including
none, as we shall see. The mode of the parameters may be any mode,
and the value yielded may be any mode. The modes written for the
parameters and the yield are always formal declarers, so no bounds
are used if the modes of the parameters or yield involve multiples.

Here is a possible header of a more complicated
\hx{routine}{routine!header}:
\begin{verbatim}
   (INT i,REF[,]CHAR c,REAL a,REAL b)BOOL:
\end{verbatim}
\noindent
A minor abbreviation would be possible in this case. The
\begin{verbatim}
   REAL a,REAL b
\end{verbatim}
\noindent
could be written \verb|REAL a,b| giving
\begin{verbatim}
   (INT i,REF[,]CHAR c,REAL a,b)BOOL:
\end{verbatim}
\noindent
Notice that the parameters are separated by commas.  This means that
when the routine is used, the actual parameters are evaluated
\hx{collaterally}{collateral!elaboration}.  We shall see later that
this is important when we consider
\hx{\textbf{side-effects}}{side-effect}.

The order in which parameters are written in the header is of no
particular significance.

The mode of the routine whose header is given above is
\begin{verbatim}
   (INT,REF[,]CHAR,REAL,REAL)BOOL
\end{verbatim}
\noindent
(``with int ref row row of car real real yielding bool'').

\Subsection{Multiples as parameters}{rout-multpar}
Since a \hx{formal parameter}{parameter!formal} which is a multiple
has no bounds written in it, any multiple having that mode could be
used as the \hx{actual parameter}{parameter!actual}. This means that
if you need to know the bounds of the actual multiple, you will need to
use the \ix{bounds interrogation} operators. For
example, here is a \hx{routine denotation}{denotation!routine} which
finds the smallest element in its multiple parameter:
\begin{verbatim}
([]INT a)INT:
(
   INT min:=a[LWB a];

   FOR i FROM LWB a TO UPB a
   DO
       min:=min MIN a[i]
   OD;
   min
)
\end{verbatim}

\Subsection{Names as parameters}{rout-namepar}
The second parameter in the more complicated routine header given in
section \hyref{rout-modes} had the mode \verb|REF[,]CHAR|. When a
parameter is a name, the body of the routine can have an assignment
which makes the name refer to a new value. For example, here is a
routine denotation which assigns a value to its parameter:
\begin{verbatim}
   (REF INT a)INT:  a:=2
\end{verbatim}
\noindent
Notice that the \ix{unit} in this case is a single \ix{phrase} and so
does not need to be enclosed. Here is a routine denotation which
has two parameters and which yields a value of mode \verb|BOOL|:
\begin{verbatim}
   (REF[]CHAR rc,[]CHAR c)BOOL:
   IF UPB rc - LWB rc /= UPB c - LWB c
   THEN FALSE
   ELSE rc[:]:=c[:];  TRUE
   FI
\end{verbatim}
\noindent
Here, the body is a \hx{conditional clause}{clause!conditional}
which is another kind of \hx{enclosed clause}{clause!enclosed}. Note
the use of trimmers to ensure that the \ix{bounds} of the multiples
on each side of the assignment match.

If a flexible name could be used as an actual parameter, then
the mode of the formal parameter must include the mode constructor
\ixtt{FLEX}. For example,
\begin{verbatim}
   (REF FLEX[]CHAR s)INT:
   (CO Code to compute the number of words CO)
\end{verbatim}
\noindent
Of course, in this example, the mode of \verb|s| could have been
given as \verb|REF STRING|.

\Subsection{The mode \texttt{VOID}}{rout-VOID}
A routine must yield a value of some mode, but it is possible to
throw away that value using the \hx{voiding}{coercion!voiding}
coercion. The mode \ixtt{VOID} has a single value whose denotation is
\ixtt{EMPTY}. In practice, because the context of the yield of a
routine is strong, use of \verb|EMPTY| is usually unnecessary (but
see section \hyref{union-proc}). Here is another way of writing the
last routine in the previous \hx{section}{stop@\texttt{stop}}:
\begin{verbatim}
   (REF[]CHAR rc,[]CHAR c)VOID:
   IF UPB rc - LWB rc /= UPB c - LWB c
   THEN
      print(("Bounds mismatch",newline));
      stop
   ELSE rc[:]:=c[:]
   FI
\end{verbatim}
\noindent
This version produces an emergency error message and terminates the
program prematurely (see section \hyref{stan-stop} for details of
\verb|stop|).  Since the yield is \verb|VOID|, any value the
conditional clause might yield will be thrown away. A \verb|FOR| loop
yields \verb|EMPTY| and a semicolon voids the previous unit.
Declarations yield no value, not even \verb|EMPTY|.

\Subsection{Routines yielding names}{rout-yname}
Since the \ix{yield} of a routine can be a value of any mode, a
routine can yield a \ix{name}, but there is a restriction: the name
yielded must have a scope larger than the body of the routine.  This
means that any names declared to be \emph{local}, cannot be passed
from the routine. Names which exist outwith the scope of the routine
can be passed via a parameter and yielded by the routine. For
example, here is a routine denotation which yields the name
passed by such a parameter:
\begin{verbatim}
   (REF INT a)REF INT:  a:=2
\end{verbatim}
\noindent
Compare this routine with the first routine denotation in
section~\hyref{rout-namepar}.

In chapter 5, we said that a new name can be declared using the
generator \verb|LOC|. For example, here is an identity declaration
for a name:
\begin{verbatim}
   REF INT x = LOC INT
\end{verbatim}
\noindent
The \ix{range} of the identifier \verb|x| is the smallest enclosed
clause in which it has been declared.  The \ix{scope} of the value it
identifies is limited to that smallest enclosed clause because the
generator used is the \hx{\textbf{local generator}}{generator!local}
\ixtt{LOC}. Here is a routine which tries to yield a name declared
within its body:
\begin{verbatim}
   (INT a)REF INT:
     (REF INT x = LOC INT:=a;  x) #wrong!#
\end{verbatim}
\noindent
This routine is wrong because the scope of the name identified by
\verb|x| is limited to the body of the routine. Note, however, the
\hx{a68toc}{a68toc!scope checking} Algol~68 compiler provides neither
compile-time nor run-time \ix{scope checking} so that it is possible
to yield a locally declared name. However, the rest of the program
would be undefined---you might or might not get meaningful things
happening.  When scopes are checked, this sort of error cannot occur.

However, there is a way of yielding a name declared in a routine.
This is achieved using a \hx{global generator}{generator!global}. If
\verb|x| above were declared \hx{as}{HEAP@\texttt{HEAP}}
\begin{verbatim}
   REF INT x = HEAP INT
\end{verbatim}
\noindent
or, in abbreviated form, \verb|HEAP INT x|, then the scope of the name
identified by \verb|x| would be from its declaration to the end of the
program even though the range of the \emph{identifier} \verb|x| is
limited to the body of the routine:
\begin{verbatim}
   (INT a)REF INT:  (HEAP INT x:=a;  x)
\end{verbatim}
\noindent
Notice that the mode of the yield is still \verb|REF INT|. All that
has changed is the scope of the value yielded. Of course, you would
not be able to identify the yielded name using \verb|x|, but we'll
come to that problem when we deal with how routines are used. Notice
that the global generator is written \verb|HEAP| instead of
\verb|GLOB| as you might expect. This is because global generators
use a different method of allocating storage for names with global
scope and, historically, this different method is recorded in the
mode constructor.

The difference between range and scope is that identifiers have
range, but values have scope. Furthermore,
\hx{denotations}{denotation} have global scope.

\begin{exercise}
\item Write the header of a routine with a parameter of mode
\verb|REF REAL| and which yields a value of mode \verb|REAL|.
\ans \verb|(REF REAL id)REAL:|
'
\item Write the header of a routine which takes two parameters each of
which is a name of mode \verb|REF CHAR|, and yields a name of mode
\verb|REF CHAR|. \ans \verb|(REF CHAR a,b)REF CHAR:|
'
\item Write a routine which takes a parameter of mode \verb|STRING| and
yields a value of mode \verb|[]STRING| consisting of the words of the
parameter (use your answer to exercise \hyref{name-ex5-12-3}).
\ans \ %
\begin{verbatim}
   (STRING s)[]STRING:
   BEGIN
      FLEX[1:0]STRING r:="";
         #rowing coercion#
      BOOL in word:=FALSE;
      INT st:=LWB s,fn:=UPB s;

      WHILE s[st]=blank & st<=UPB line
      DO
        st+:=1
      OD;

      WHILE s[fn]=blank & fn>=st
      DO
       fn-:=1
      OD;

      STRING ss:=s[st:fn]+blank;

      FOR c FROM LWB ss WHILE c <= UPB ss
      DO
         CHAR ssc=ss[c];

         IF   ssc/=blank & NOT in word
         THEN st:=c;  in word:=TRUE
         ELIF ssc=blank & NOT in word
         THEN SKIP
         ELIF ssc/=blank & in word
         THEN SKIP
         ELSE #ssc=blank & in word#
            fn:=c-1;  in word:=FALSE;
            [UPB r+1]STRING rr;
            rr[:UPB r]:=r;
            rr[UPB rr]:=ss[st:fn];
            r:=rr
        #The word has been added to r#
         FI
      OD;

      r[2:]  #Omit the null string#
   END
\end{verbatim}
'
\end{exercise}

\Subsection{Parameterless routines}{rout-nopar}
A routine can have no parameters. In the header, the parentheses
normally enclosing the \hx{formal parameter}{parameter!formal}
\textbf{list} (either one parameter, or more than one separated by
commas) are also omitted.  Here is a routine with no parameters and
which yields a value of mode \verb|INT|:
\begin{verbatim}
   INT: 2*3**4 - ENTIER 36.5
\end{verbatim}
\noindent
It would be more usual to use identifiers which had been declared in
some enclosing range. For example,
\begin{verbatim}
   INT: 2*a**4 - ENTIER b
\end{verbatim}
\noindent
Routines which have no parameters and yield no value are fairly
common.  For example,
\begin{verbatim}
   VOID:  print(2)
\end{verbatim}
\noindent
Strictly speaking, there is one value having the mode \ixtt{VOID},
but there's not a lot you can do with it. In practice, \verb|VOID|
routines usually use identifiers declared in an enclosing range (they
are called identifiers global to the routine). For
\hx{example}{identifier!global}:
\begin{verbatim}
   VOID: (x:=a;  x<=2|print(x)|print("Over 2"))
\end{verbatim}
\noindent
where the body is an abbreviated conditional clause, and \verb|x| and
\verb|a| have been declared globally with appropriate modes.

Assignment of values to names declared globally\footnote{The phrase
``names declared globally'' is intended to mean here that the names
have been declared in a range which encloses the routine, not that
\texttt{HEAP} has necessarily been used in the declaration. We use
the phrase ``a global name'' in the latter case.} to the routine is
known as a \bfix{side-effect}. We shall deal with side-effects when
we describe how routines are used, and we shall show why side-effects
are undesirable.  If you write parameterless routines, it is
preferable not to put any assignments to globally-declared names in
them. In fact, it would be safer to say: ``In a routine, don't assign
to names not declared in the routine or not provided as parameters''.
Side-effects are messy and are usually a sign of badly designed
programs.  It is better to use a parameter (or an extra parameter)
using a name, and then assign to that name.  This ensures that values
can only get into or out of your routines via the header, and results
in a much cleaner design.  Cleanly designed programs are easier to
write and easier to maintain.

\begin{exercise}
\item Write the header of a routine which yields a value of mode
\verb|REAL|, but takes no parameters. \ans \verb|REAL:|
'
\item Write a routine of mode \verb|VOID| which prints
\begin{verbatim}
   Hi, there
\end{verbatim}
\noindent
on your screen. \ans \verb|VOID: print("Hi, there")|
'
\end{exercise}

\Section{Operators}{rout-ops}
In the preamble to this chapter, it was mentioned that routines
consist of two kinds: procedures and operators. See
section~\hyref{rout-procs} for details of procedures.

An operator has a \hx{mode}{operator!mode} and a
\hx{value}{operator!value} (its routine
\hx{denotation}{denotation!routine}) and, if dyadic, a
\hx{priority}{operator!priority}.
The parameters to routines which are defined as operators are called
\hx{operands}{operand}. Monadic operators, while not having a
priority, behave as though they had a priority greater than any
\hx{dyadic operator}{operator!dyadic} and take one operand and
\hx{yield}{operator!yield} a value of some mode.

Here is an identity declaration of the
\hx{monadic operator}{operator!monadic} \verb|B|:
\begin{verbatim}
   OP(INT)INT B = (INT a)INT: a
\end{verbatim}
\noindent
There are several points to note.
\begin{enumerate}
\item The mode of the operator is \verb|OP(INT)INT|. That is,
it takes a single operand of mode \verb|INT| and yields a value of
mode \verb|INT|.
\item The \hx{symbol}{operator!symbol} for the operator looks
like a \hx{mode indicant}{mode!indicant}.  It isn't a mode
indicant, but obeys the same rules (starts with an uppercase letter
and possibly continues with uppercase letters or digits, and no
spaces are allowed inside the symbol).
\item The right-hand side of the identity
\hx{declaration}{declaration!identity!routine} is a routine
denotation.  A special identity declaration is used for operators:
instead of the mode of the operator, the mode constructor \ixtt{OP}
is used followed by the operator symbol. The abbreviated declaration
of the operator \verb|B| is
\begin{verbatim}
        OP B = (INT a)INT: a
\end{verbatim}
\end{enumerate}

Chapter 2 described how operators are used in formul\ae{}. A possible
formula using \verb|B| could be
\begin{verbatim}
   B 2
\end{verbatim}
\noindent
which would yield \verb|2|.\footnote{In chapter 3, we mentioned a bug
in the \protect\hx{a68toc}{a68toc!BY@\texttt{BY}} Algol~68 compiler
connected with the \texttt{BY} construct in loop clauses. Using the
\texttt{B} operator is another way of getting round the bug.}

\Subsection{Identification of operators}{rout-idops}
This section is more difficult than preceding sections and could be
omitted on a first reading. You are unlikely to fall afoul of what is
described here unless you are declaring many new operators.

One of the most useful properties of operators is that there can be
more than one declaration of the same operator symbol using an
operand having a different mode. This is called
\hx{``operator overloading''}{overloading}.
How does the compiler know which version of the operator to use?
Before answering this question, consider the following program
fragment:
\begin{verbatim}
    1 BEGIN
    2    OP D = (INT a)INT:   a+2;
    3    OP D = (REAL a)REAL: a+2.0;
    4    REAL x:=1.5, a:=-2.0; INT i:=4;
    5
    6    x:=IF  OP D = (REF REAL a)REF REAL:
    7                      a+:=2.0;
    8           ENTIER(D a:=x) > i
    9       THEN D i
   10       ELSE D x
   11       FI;
   12
   13    OP D = (REF REAL a)REF REAL:  a+:=3.0;
   14    x:=D a
   15 END
\end{verbatim}
\noindent
The numbers on the left-hand side are not part of the program. As you
can see, there are four declarations of \verb|D|: one with an
\verb|INT| operand, one with a \verb|REAL| operand and two with a
\verb|REF REAL| operand. If you try compiling this you will get the
error
\begin{verbatim}
        ERROR (146) more than one version of D
\end{verbatim}
\noindent
for the last declaration. There are two points to be made here.
\begin{enumerate}
\item Outside the conditional \hx{clause}{clause!conditional},
there are three declarations of \verb|D|: on lines 2, 3 and 13. Now,
an operator is used in a formula and the context of the operand of an
operator is firm. Of the coercions we have met so far, only one,
namely \hx{dereferencing}{coercion!dereferencing}, is allowed in
a \hx{firm context}{context!firm}.  If you look at the assignment
on line~14, you can see that the mode of the operand of \verb|D| is
\verb|REF REAL| (from the declaration of \verb|a| on line~4).  Now a
value of mode \verb|REF REAL| is \ix{firmly coercible} to \verb|REAL|
(by dereferencing).  So there are two declarations of \verb|D| which
could be used: the declaration on line~3 and the declaration on
line~13 (the \ix{range} of the declaration on line~6 is confined to
the conditional clause).  According to the rules for the
identification of operators (see below), the compiler would not be
able to distinguish between the two declarations.  Hence the error
message.
\item Why did the identical declaration of \verb|D| on line~6 not
cause a similar error message? Answer: because the declaration on
line~6 is at the start of a new range: the enclosed clause starting
on line~6 and extending to the \verb|FI| on line~11.  Since that is a
new range, any operator declarations with a mode which is
\bfix{firmly related} to the mode of an operator declared in an outer
range makes the declaration in the outer range inaccessible. Thus,
the assignment on line~8 will use the version of \verb|D| declared on
line~6, the \verb|D| on line~9 identifies the \verb|D| declared on
line~2, and the \verb|D| on line~10 again uses the \verb|D| declared
on line~6.
\end{enumerate}

Thus, in determining which operator to use, the compiler firstly
finds a declaration whose mode can be obtained from the operands in
question using any of the coercions allowed in a firm context
(chapter~10 will state all the coercions allowed). Secondly, it will
use the declaration in the smallest range enclosing the formula.

The declaration of an object is known as its
\hx{\textbf{defining occurrence}}{occurrence!defining}. Where the
object is used is called its
\hx{\textbf{applied occurrence}}{occurrence!applied}.  In
practice, it is rare to find like operator declarations in nested
ranges.
\newpage

\begin{exercise}
\item This and the following exercise use the following program
fragment:
\begin{verbatim}
    1 IF
    2    OP T = (INT a)INT:  a*a;
    3    OP T = (CHAR a)INT: ABS a * ABS a;
    4    INT p:=3, q:=4; CHAR c:=REPR 3;
    5    T p < T c
    6 THEN
    7    OP T = (REF INT a)REF INT: a*:=a;
    8    IF T 4 < T q
    9    THEN "Yes"
   10    ELSE T REPR 2
   11    FI
   12 ELSE T c > T q
   13 FI
\end{verbatim}
\noindent
There are 3 defining occurrences of the operator \verb|T| on lines 2,
3 and~7. There are 7 applied occurrences of the operator (on lines 5,
8, 10 and~12). Which defining occurrence is used for each applied
occurrence? \ans This table summaries the occurrences:-
\begin{center}
\begin{tabular}{|r|l|c|}\hline
      &\multicolumn{2}{c|}{Occurrences}\\ \hline
Line  &Applied&Defining\\ \hline
 5    &\texttt{T p}&2\\
 5    &\texttt{T c}&3\\
 8    &\texttt{T 4}&2\\
 8    &\texttt{T q}&7\\
10    &\texttt{T REPR 2}&3\\
12    &\texttt{T c}&3\\
12    &\texttt{T q}&2\\ \hline
\end{tabular}
\end{center}
'
\item What is the mode and value yielded by
\begin{subex}
\item \verb|T q| on line 8 \subans A name of mode \verb|REF INT|.
'
\item \verb|T q| on line 12 \subans The integer denoted by \verb|16| of
mode \verb|INT|.
'
\item \verb|T c| on line 12 \subans The integer nine of mode
\verb|INT|.
'
\item \verb|T REPR 2| on line 10 \subans The integer four of mode
\verb|INT|.
'
\end{subex}
\item What is wrong with these two declarations occurring in the same
range:
\begin{verbatim}
   OP TT = ([]INT a)[]INT:
           FOR i FROM LWB a TO UPB a
           DO print(a[i]*3) OD;
   OP TT = (REF[]INT a)[]INT:
           FOR i FROM LWB a TO UPB a
           DO print(a[i]*3) OD
\end{verbatim}
\indent\ans The two declarations are firmly related because, in a
firm context, a name of mode \verb|REF[]INT| can be dereferenced to a
multiple of mode \verb|[]INT|.
'
\end{exercise}

\Subsection{Operator usage}{rout-opuse}
Before we go on to dyadic operators, there is one more point to
consider. Given the operator declaration
\begin{verbatim}
   OP PLUS2 = (REAL a)REAL:  a+2.0
\end{verbatim}
\noindent
what is the mechanism by which the formal parameter gets its value?
Firstly, we must remember that a particular version of the operator is
chosen on the basis of firmly \hx{relatedness}{firmly related}. In
other words, only coercions allowed in a
\hx{firm context}{context!firm} can determine which declaration
of the operator to use.  Secondly, in elaborating the formula
\begin{verbatim}
   PLUS2 x
\end{verbatim}
\noindent
where \verb|x| has the mode \verb|REF REAL|, the compiler elaborates
the identity declaration
\begin{verbatim}
   REAL a = x
\end{verbatim}
\noindent
where \verb|REAL a| is the formal \hx{parameter}{parameter!formal}.
Since the context of the right-hand side of an identity declaration
is strong, any of the strong coercions would normally be allowed (all
coercions, including dereferencing).  However, because the version of
the operator was chosen on the basis of firmly relatedness, the
coercions available in a strong context which are not available in a
firm context (that is, widening and rowing) are not available in the
context of an operand.  If an operand of mode \verb|INT| is supplied
to an operator requiring a \verb|REAL|, the compiler will flag an
error: widening would not occur. This is the only exception to the
rule that the right-hand side of an identity declaration is a
\hx{strong context}{context!strong!exception}.

It was pointed out in section~\hyref{rout-yname} that a routine can
yield a name.  An operator does not usually yield a name because
subsequent use of the name usually involves dereferencing and using
the value the name refers to.  However, here is an operator
declaration which yields a name of a multiple which is used in a
subsequent phrase:
\begin{verbatim}
   OP NAME = (INT a)REF[]INT:
               (HEAP[2]INT x:=(a,a);  x);
   REF[]INT a = NAME 3
\end{verbatim}
\noindent
After the elaboration of the identity declaration, the name could be
accessed wherever necessary.
\newpage

\begin{exercise}
\item Given the declarations
\begin{verbatim}
   OP M3 = (INT i)INT:  i-3;
   OP M3 = ([]INT i)[]INT:
           FORALL n IN i DO n-3 OD;
   INT i:=1,[3]INT j:=(1,2,3)
\end{verbatim}
\noindent
which operator declarations would be used for the following
formul\ae{}
\begin{subex}
\item \verb|M3 i| \subans 1.
'
\item \verb|M3 j[2]| \subans 1.
'
\item \verb|M3 j| \subans 2.
'
\item \verb|M3 j[:2]| \subans 2.
'
\end{subex}
\end{exercise}

\Subsection{Dyadic operators}{rout-dyad}
The only differences between monadic and \hx{dyadic}{operator!dyadic}
operators are that the latter have a \hx{priority}{operator!priority}
and take two operands. Therefore the routine denotation used for a
dyadic operator has two formal parameters. The priority of a dyadic
operator is declared using the indicant \ixtt{PRIO}:
\begin{verbatim}
   PRIO HMEAN = 7; PRIO WHMEAN = 6
\end{verbatim}
\noindent
The declaration of the priority of the operator uses an integer
denotation in the range 1 to 9 on the right-hand side.

Consecutive priority \hx{declarations}{declaration!priority} do
not need to repeat the \verb|PRIO|, but can be abbreviated in the
usual way.  The priority declaration relates to the operator
\hx{symbol}{operator!symbol}.  Hence the same operator cannot
have two different priorities in the same range, but there is no
reason why an operator cannot have different priorities in different
ranges.  A priority declaration does not count as a declaration when
determining the \ix{scope} of a local name.

If an existing operator symbol is used in a new declaration, the
priority of the new operator must be the same as the old if it is in
the same range, so the priority declaration should be omitted.

The \hx{identification}{operator!dyadic!identification} of dyadic
operators proceeds exactly as for monadic operators except that the
most recently declared priority in the same range is used to
determine the order of \hx{elaboration}{elaboration!order of} of
operators in a formula.  Again, two operators using the same symbol
cannot be declared in the same range if they have firmly
\hx{related}{firmly related} modes (see section~\hyref{rout-idops}).

These declarations apply to the remainder of this section:
\begin{verbatim}
   PRIO HMEAN = 7, WHMEAN = 6;
   OP HMEAN  = (REAL a,b)REAL:
                  2.0/(1.0/a+1.0/b);
   OP WHMEAN = (REAL a,b)REAL:
                  2.0/(1.0/a+2.0/b)
\end{verbatim}
\noindent
If \verb|HMEAN| appears in the formula
\begin{verbatim}
   x HMEAN y
\end{verbatim}
\noindent
where \verb|x| and \verb|y| both have mode \verb|REF REAL|, the
compiler constructs the identity
\hx{declarations}{declaration!identity!routine call}
\begin{verbatim}
   REAL a = x, REAL b = y
\end{verbatim}
\noindent
Notice that the two identity declarations are elaborated
\hx{collaterally}{elaboration!collateral} (due to the comma
separating them), which could be important (see below).  If \verb|x|
refers to \verb|2.5| and \verb|y| refers to \verb|3.5|, the formula
will yield
\begin{verbatim}
   2.0/(1.0/2.5 + 1.0/3.5)
\end{verbatim}
\noindent
which is $2.91\dot6$. Likewise, the formula
\begin{verbatim}
   x WHMEAN y
\end{verbatim}
\noindent
would yield $2&#XB7;058\,823\,529\,411\,76$. Now consider the formula
\begin{verbatim}
   (x+:=1.0) WHMEAN (x+:=1.0)
\end{verbatim}
\noindent
which cause the value referred to by \verb|x| to be incremented twice
as a \ix{side-effect}. The resulting identity declarations are
\begin{verbatim}
   REAL a = (x+:=1.0), REAL b = (x+:=1.0)
\end{verbatim}
\noindent
The definition of Algol 68 says that the operands of a dyadic operator
should be elaborated collaterally, so the order of elaboration is
unknown. Suppose \verb|x| refers to \verb|1.0| before the formula is
elaborated. There are three cases:
\begin{enumerate}
\item The identity declaration for \verb|a| is elaborated first,
giving \verb|a=2.0| and \verb|b=3.0|. The formula will yield
$1.714\,285\,714$.
\item The identity declaration for \verb|b| is elaborated first,
giving \verb|b=2.0| and \verb|a=3.0|. The formula will yield $1.5$.
\item The identity declarations are elaborated in parallel. In this
case, the result could be \hx{indeterminate}{indeterminate result}.
\end{enumerate}
If you compile a program using a68toc with the declaration for
\verb|WHMEAN| and try to compute the formula given above, you get the
result \verb|+1.5000000000000000| which suggests that case~2
holds.

If \verb|x| refers to \verb|1.0|, then the formula
\begin{verbatim}
   1.0/(x+:=1.0) + 1.0/(x+:=1.0)
\end{verbatim}
\noindent
yields \verb|+.83333333333333339e +0| which is correct provided that
the two operands are elaborated sequentially. The moral of all this
is: avoid side-effects like the plague.

What happens if the identifier of an actual
\hx{parameter}{parameter!actual} is the same as the \ix{identifier}
of the formal \hx{parameter}{parameter!formal}? There is no clash.
Consider the identity declaration
\begin{verbatim}
   INT a = a
\end{verbatim}
\noindent
where the \verb|a| on the left-hand side is the formal parameter for
a routine denotation, and the \verb|a| on the right-hand side is an
actual parameter declared in some surrounding range. The formal
parameter occurs at the start of a new range. Within that range, the
identifier \verb|a| in the outer range becomes inaccessible, but at
the moment that the identity declaration is being elaborated, the
formal parameter is made to identify the value of the actual
parameter which, of course, is not an identifier. So go ahead and use
identical identifiers for formal parameters and actual parameters.

\Subsection{Operator symbols}{rout-opsym}
\hx{Most}{operator!symbol} of the operators described in chapters
2 to~5 used symbols rather than upper-case letters.  You may use any
combination of the \verb|<=>*/:| symbols (and any number of them)
except \verb|:=|, \verb|:=:| and \verb|:/=:| (the latter two are
described in chapter~11).  Any of the symbols \verb|+-?&%| can only
start a compound symbol. Of course, they can stand on their own for
an operator. In chapter~11, you will find the \verb|<<| and \verb|>>|
operators described as well as more declarations for existing
operators.  Here are some declarations of operators using the above
rules:
\begin{verbatim}
   OP *** = (INT a)INT: a*a*a;
   OP %< = (CHAR c)CHAR: (c<" "|" "|c);
   OP -:: = (CHAR c)INT: (ABS c-ABS"0")
\end{verbatim}

We have now covered everything about operators in the language.

\begin{exercise}
\item Why are side-effects undesirable? \ans These reasons are the
most important:-
\begin{enumerate}
\item Because their actions are not clear from the program code.
\item They can cause indeterminate states to occur.
\end{enumerate}
'
\item What is wrong with these operator symbols:
\begin{subex}
\item \verb|M*| \subans You cannot mix letters and symbols.
'
\item \verb|%+/| \subans The symbol should start with \verb|+| which
has already been declared as a monadic operator.
'
\item \verb|:=:| \subans This symbol is used for the identity
relation (see section \hyref{adv-idrel}) and is not an operator.
'
\end{subex}
\item Declare an operator using the symbol \verb|PP| which will add
\verb|1| to the value its \verb|REF INT| operand refers to, and which
will yield the name of its parameter.
\ans \verb|OP PP = (REF INT a)REF INT:  a+:=1|
'
\end{exercise}

\Section{Procedures}{rout-procs}
The second way of using routines is to declare them as
\hx{procedures}{procedure}. We have seen that an operator can be
declared and used, have a mode and a value (its routine denotation),
but apart from having an operator symbol, it cannot be identified with
an identifier in the way that a name or a denotation of a \verb|CHAR|
value can.  Procedures are quite different.

Firstly, here are some general remarks on the way procedures differ
from operators.  The \hx{mode}{procedure!mode} of a procedure
always starts with the mode constructor \ixtt{PROC}.  A procedure can
have any number of parameters, including none.  Two procedures having
the same identifier cannot be declared in the same range (so
``\ix{overloading}'' is not allowed).  When a procedure is used, its
parameters, if any, are in a strong context. This means that rowing
and widening are available.

Procedures are declared using the mode constructor \verb|PROC|. Here is
a procedure which creates a range of characters:
\begin{verbatim}
   PROC(CHAR,CHAR)[]CHAR range =
      (CHAR a,b)[]CHAR:
      BEGIN
         CHAR aa,bb;

         (a<=b|aa:=a; bb:=b|aa:=b; bb:=a);

         [ABS aa:ABS bb]CHAR r;

         FOR i
         FROM LWB r TO UPB r
         DO
            r[i]:=REPR i
         OD;
         r
      END
\end{verbatim}
\noindent
This procedure identity declaration resembles the declaration for a
multiple: much of the mode is repeated on the right-hand side and the
\ix{formal-declarer} on the left-hand side has no identifiers for the
modes of the parameters.  Notice that the modes of the parameters
must be repeated in the formal-declarer, but that the mode of the
procedure on the right-hand side can contain the usual abbreviation.
Here is the abbreviated header:
\begin{verbatim}
   PROC range = (CHAR a,b)[]CHAR:
\end{verbatim}

The formal-declarer is important for creating synonyms:
\begin{verbatim}
  PROC(REAL)REAL sine = sin
\end{verbatim}

Two or more procedure declarations can be separated by commas, even if
the procedures have different modes. Consider, for example:
\begin{verbatim}
   PROC pa = (INT i)INT:  i*i,
        pb = (INT i)CHAR: REPR(i*i),
        pc = (INT i)REAL: (i=0|0|1/i)
\end{verbatim}

\Subsection{Parameterless procedures}{rout-npprocs}
Procedures can have no parameters. Suppose the following names have
been declared:
\begin{verbatim}
   INT i,j
\end{verbatim}
\noindent
Here is a procedure with mode \verb|PROC INT| which yields an
\verb|INT|:
\begin{verbatim}
   PROC INT p1 = INT: i:=3+j
\end{verbatim}
\noindent
A procedure can be invoked or \hx{called}{procedure!call} by writing
its identifier.  For example, the procedure \verb|p1| would be called
by
\begin{verbatim}
   p1
\end{verbatim}
\noindent
or
\begin{verbatim}
   INT a = p1
\end{verbatim}
\noindent
The right-hand side of this identity declaration requires a value of
mode \verb|INT|, but it has been given a unit of mode
\verb|PROC INT|.  This is converted into a value of mode \verb|INT|
by the coercion known as \textbf{deproceduring}. This
\hx{coercion}{coercion!deproceduring} is available in every context
(even \hx{soft}{context!soft}).

Have you realised that \verb|print| must be the identifier of a
procedure? Well done! However, we cannot talk about its parameters yet
because we don't know enough about the language.

Here is another procedure which yields a name of mode \verb|REF INT|.
The mode of the procedure is \verb|PROC REF INT|:
\begin{verbatim}
   PROC p2 = REF INT: IF i < 0 THEN i ELSE j FI
\end{verbatim}
\noindent
and assumes that the names identified by \verb|i| and \verb|j| had
already been declared. Here is an identity declaration which uses
\verb|p2|:
\begin{verbatim}
   REF INT i or j = p2
\end{verbatim}
\noindent
Because \verb|p2| yields a name, it can be used on the left-hand side
of an assignment:
\begin{verbatim}
   p2:=4
\end{verbatim}
\noindent
Here, \verb|4| will be assigned to \verb|i| or \verb|j| depending on
the value \verb|i| refers to. The left-hand side of an assignment has
a soft \hx{context}{context!soft} in which only the deproceduring
coercion is allowed.

In procedures \verb|p1| and \verb|p2|, the identifier \verb|i| had
been declared globally to the procedures. Assignment to such an
identifier is, as already stated, a \ix{side-effect}.  Here is
another procedure of mode \verb|PROC INT| which uses a global
identifier, but does not assign to it:
\begin{verbatim}
   PROC p3 = REAL:
   (
      [i]REAL a;  read((a,newline));
      REAL sum:=0.0;

      FOR i FROM LWB a TO UPB a
      DO
         sum+:=a[i]
      OD;

      sum
   )
\end{verbatim}
\noindent
and here is a call of \verb|p3|:
\begin{verbatim}
   print(p3)
\end{verbatim}
\noindent
In the identity declaration
\begin{verbatim}
   REAL r = p2
\end{verbatim}
\noindent
\verb|p2| is deprocedured to yield a name of mode \verb|REF INT|,
dereferenced to yield an \verb|INT|, and then widened to yield a
\verb|REAL|.  All these coercions are available in a strong
\hx{context}{context!strong} (the right-hand side of an identity
declaration).

The call of a procedure can appear in a formula without parentheses.
Here is an example:
\begin{verbatim}
   p2:=p1 * ROUND p3
\end{verbatim}
\noindent
If we call the procedure \verb|p1|, declared above, its value does not
have to be used.  For example, in
\begin{verbatim}
   p1;
\end{verbatim}
\noindent
the value yielded by \verb|p1| has been \hx{voided}{coercion!voiding}
by the following semicolon after the procedure had been called.

In the section on routines, we introduced the mode \ixtt{VOID}. Here
is a procedure yielding \verb|VOID|:
\begin{verbatim}
   PROC p4 = VOID:  print(p3)
\end{verbatim}
\noindent
and a possible use:
\begin{verbatim}
   ; p4;
\end{verbatim}
\noindent
where the semicolons show that the call stands on its own.

When a parameterless \hx{procedure}{procedure!parameterless} yields
a multiple, the call of that procedure can be \hx{sliced}{slice}
to get an individual element.  For example, suppose we declare
\begin{verbatim}
   PROC p5 = [,]REAL:
   (
      [i,j]REAL a;
      read((a,newline));
      a
   )
\end{verbatim}
\noindent
where \verb|i| and \verb|j| were declared above, and then call
\verb|p5| in the formula
\begin{verbatim}
   REAL x = p5[i-3,j] * 2
\end{verbatim}
\noindent
When \verb|p5| is called, it yields a two-dimensional multiple of mode
\verb|[,]REAL| which is then sliced using the two
\hx{subscripts}{subscript} (assuming that \verb|i-3| is within the
bounds of the first dimension) to yield a value of mode \verb|REAL|,
which is then used in the formula.

Procedure \verb|p2|, declared above, yielded a name declared globally
to the procedure. As explained in the sections on routines, a
procedure cannot yield a locally-generated name. However, if the name
is generated using \ixtt{HEAP}, then the name can be yielded as in
\hx{\texttt{p6}:}{procedure!yield}
\begin{verbatim}
   PROC p6 = REF INT: (HEAP INT i:=3;  i)
\end{verbatim}
\noindent
Here is a call of \verb|p6| where the yielded name is captured with an
identity declaration:
\begin{verbatim}
   REF INT global int = p6
\end{verbatim}
\noindent
Then \verb|print(global int)| will display \verb|3|.

The yield of a procedure can be another procedure. Consider this
program fragment:
\begin{verbatim}
   PROC q2 = INT: max int % 2,
        q3 = INT: max int % 3,
        q4 = INT: max int % 4,
        q5 = INT: max int % 5;

   INT i;  read((i,newline));

   PROC q = PROC INT:
      CASE i+1 IN q2,q3,q4 OUT q5 ESAC
\end{verbatim}
\noindent
Procedure \verb|q| will yield one of the predeclared procedures
depending on the value of \verb|i|. Here, the yielded procedure will
not be deprocedured because the mode required is a procedure.

One parameterless procedure is provided in the \ix{standard prelude}.
Its identifier is \ixtt{random}, and when called returns the next
pseudo-random real number of a series. If called a large number of
times, the numbers yielded are uniformly distributed in the range
$\left[0,1\right)$.

\begin{exercise}
\item Write a procedure which assigns a value to a name declared
globally to the procedure. \ans \verb|PROC p = VOID:  a:=3|
'
\item Write a procedure which reads an integer from the keyboard, then
declares a dynamic name of a multiple of one dimension, and reads that
number of integers from the keyboard. Now compute the sum of all the
integers, and yield its value as the yield of the procedure.
\ans \ %
\begin{verbatim}
   PROC p = INT:
   BEGIN
      [(INT i; read((i,newline)); i)]INT a;
      read(a);
      INT sum:=0;

      FOR i TO UPB a DO sum+:=a[i] OD;
      sum
   END
\end{verbatim}
'
\item Write a procedure which yields the name of a two dimensional
multiple containing characters read from the keyboard. The mode of
the multiple should be \verb|REF[,]CHAR|. \ans \ %
\begin{verbatim}
   PROC p = REF[,]CHAR:
   (
      HEAP[3,20]CHAR n;
      read((n,newline));
      n
   )
\end{verbatim}
'
\end{exercise}

\Subsection{Procedures with parameters}{rout-parprocs}
\hx{Parameters}{procedure!parameters} of procedures can have any
mode (including procedures).  Unlike operators, procedures can have
any number of parameters.  The parameters are written as a
\hx{parameter}{parameter!list} <b>list</b> which consists of one
parameter, or two or more separated by commas.

Here is a procedure with a single parameter:
\begin{verbatim}
   PROC(INT)CHAR p7 = (INT i)CHAR: REPR(i>0|i|0)
\end{verbatim}
\noindent
This is a full identity declaration for \verb|p7|. It can be
abbreviated to
\begin{verbatim}
   PROC p7 = (INT i)CHAR: REPR(i>0|i|0)
\end{verbatim}
\noindent
The mode of \verb|p7| is \verb|PROC(INT)CHAR|. That is, \verb|p7| is a
procedure with a single integer parameter and yielding a character.
Here is a \hx{call}{procedure!call} of \verb|p7|:
\begin{verbatim}
   CHAR c = p7(-3)
\end{verbatim}
\noindent
Note that the single parameter is written between
\ix{parentheses}. Since the context of an actual
\hx{parameter}{parameter!actual} of a procedure is strong, a name of
mode \verb|REF INT| could be used:
\begin{verbatim}
   CHAR c = p7(i)
\end{verbatim}
\noindent
or
\begin{verbatim}
   CHAR c = p7(ai[j])
\end{verbatim}
\noindent
where \verb|ai| has mode \verb|REF[]INT| and \verb|j| has mode
\verb|INT| or \verb|REF INT| or \verb|PROC INT| (or even
\verb|PROC REF INT|).

Here is a procedure which takes three parameters:
\begin{verbatim}
   PROC char in string =
      (CHAR c,REF INT p,STRING s)BOOL:
   (
      BOOL found:= FALSE;
      FOR k FROM LWB s TO UPB s
      WHILE NOT found
      DO
         (c = s[k] | i:=k; found:= TRUE)
      OD;
      found
   )
\end{verbatim}
\noindent
\hylabel{rout-chinstr}
The procedure (which is in the standard prelude) tests whether a
character is in a string, and if it is, returns its position in the
parameter \verb|p|. The procedure yields \verb|TRUE| if the character
is in the string, and \verb|FALSE| if not. Here is a possible call of
the procedure:
\begin{verbatim}
   IF INT p;  char in string(char,"abcde",p)
   THEN ...
\end{verbatim}
\noindent
where \verb|char| was declared in an outer range. Notice that the
\verb|REF INT| parameter of \verb|char in string| is not assigned a
new value if the character is not found in the string.

When calling a procedure, the call must supply the same number
of actual parameters, and in the same order, as there are formal
\hx{parameters}{parameter!formal} in the procedure declaration.

If a multiple is one of the formal parameters, a
\hx{row-display}{row!display} can be supplied as an actual parameter
(remember that a row-display can only occur in a strong
\hx{context}{context!strong}).  In this case, the row-display counts
as a single parameter, but the number of elements in the row-display
can differ in successive calls since the bounds of the multiple can
be determined by the procedure using the bounds interrogation
operators.  Here is an example:
\begin{verbatim}
   PROC pb = ([]INT m)INT:
   (INT sum:=0;
    FOR i FROM LWB m TO UPB m DO sum+:= m[i] OD;
    sum)
\end{verbatim}
\noindent
and here are some calls of \verb|pb|:
\begin{verbatim}
   pb((1,2,3))   pb((2,3,5,7,11,13))
\end{verbatim}

Again, procedures with parameters can assign to, or use, globally
declared names and other values, but it is better to include the name
in the header of the procedure. Here is a procedure which reads data
into a globally declared multiple using that multiple as a
\hx{parameter}{read@\texttt{read}}:
\begin{verbatim}
   PROC rm = (REF[]REAL a)VOID:
      read((a,newline))
\end{verbatim}
\noindent
It could now be called by
\begin{verbatim}
   rm(multiple)
\end{verbatim}
\noindent
where \verb|multiple| had been previously declared as having mode
\verb|REF[]REAL|.

As described in section~\hyref{rout-namepar}, a \ix{flexible name} can
be used as an actual parameter provided that the formal parameter has
also been declared as being flexible.  For example, here is a
procedure which takes a single parameter of mode \verb|REF STRING|
and which yields an \hx{\texttt{INT}}{STRING@\texttt{STRING}}:
\begin{verbatim}
   PROC read line = (REF STRING s)INT:
   (
      read((s,newline));
      UPB s #LWB is 1#
   )
\end{verbatim}
\noindent
\verb|read line| reads the next line of characters from the keyboard,
assigns it to its parameter, which is a flexible name, and yields the
length of the line.

\begin{exercise}
\item Write a procedure which takes a \verb|REF REAL| parameter,
divides the value it refers to by $\pi$, multiplies it by~180,
assigns the final value to its parameter, and yields the parameter
(that is, its name). \ans \ %
\begin{verbatim}
   PROC p=(REF REAL r)REF REAL:
      r:=r/pi*180
\end{verbatim}
'
\item Write a procedure which takes two parameters: the first should
have mode \verb|STRING| and the second mode \verb|INT|. Display the
string on the screen the number of times given by the integer. If the
integer is negative, display a newline first and then use the
absolute value (use the operator \verb|ABS|) of the integer. Yield
the mode \verb|VOID|. \ans \ %
\begin{verbatim}
   PROC p = (STRING s,INT i)VOID:
   (
      INT ii = IF i < 0
               THEN print(newline); ABS i
               ELSE i
               FI;
      TO ii DO print(s) OD
   )
\end{verbatim}
'
\item Write a procedure, identified as \verb|num in multiple|, which
does for an integer what \verb|char in string| does for a character.
\ans \ %
\begin{verbatim}
   PROC num in multiple=(INT i,
                         []INT m,
                         REF INT p)BOOL:
   (
      INT pos:=LWB m - 1;

      FOR j FROM LWB m TO UPB m
      WHILE pos < LWB m
      DO
         (i=m[j]|pos:=j)
      OD;

      IF pos < LWB m
      THEN FALSE
      ELSE p:=pos; TRUE
      FI
   )
\end{verbatim}
'
\end{exercise}

\Subsection{Procedures as parameters}{rout-procpar}
Here is a procedure which takes a procedure as a parameter:
\begin{verbatim}
   PROC sum = (INT n,PROC(INT)REAL p)REAL:
   (
      REAL s:=0;
      FOR i TO n DO s+:=p(i) OD;
      s
   )
\end{verbatim}
\noindent
Notice that the mode of the procedure
\hx{parameter}{parameter!procedure} is a formal \hx{mode}{mode!formal} so no
identifier is required for its \verb|INT| parameter in the header of
the procedure \verb|sum|. In the loop clause, the procedure is called
with an actual parameter.

When a parameter must be a procedure, there are two ways in which it
can be supplied. Firstly, a predeclared procedure
\hx{identifier}{procedure!identifier} can be supplied, as in
\begin{verbatim}
   PROC pa = (INT a)REAL: 1/a;
   sum(34,pa)
\end{verbatim}
\noindent
Secondly, a \hx{routine denotation}{denotation!routine} can be
supplied:
\begin{verbatim}
   sum(34,(INT a)REAL: 1/a)
\end{verbatim}
\noindent
A routine denotation is a \hx{unit}{unit}.  In this case, the routine
denotation has the mode \verb|PROC(INT)REAL|, so it can be used in
the call of \verb|sum|.  Notice also that, because the routine
denotation is an actual parameter, its header includes the identifier
\verb|a|.  In fact, routine denotations can be used wherever a
procedure is required, so long as the denotation has the required
mode.  The routine denotation given in the call is on the right-hand
side of the implied identity declaration of the elaboration of the
parameter.  It is an example of an \bfix{anonymous} routine
denotation.

\begin{exercise}
\item Given the declaration of \verb|sum| in the text, what is
the value of:
\begin{subex}
\item \verb|sum(4,(INT a)REAL: a)| \subans \verb|10.0|
'
\item \verb|sum(2,(INT b)REAL: 1/(5*b))| \subans \verb|0.3|
'
\item \verb|sum(0,pa)| (\verb|pa| is declared in the text) \subans
\verb|0.0|.
'
\end{subex}
\end{exercise}

\Subsection{Recursion}{rout-rec}
One of the fun aspects of using procedures is that a procedure can
call itself. This is known as \bfix{recursion}. For example, here is
a simplistic way of calculating a factorial:
\begin{verbatim}
   PROC factorial = (INT n)INT:
      (n=1|1|n*factorial(n-1))
\end{verbatim}
\noindent
Try it with the call
\begin{verbatim}
   factorial(7)
\end{verbatim}
\noindent
Here is another recursively defined procedure which displays an
integer on the screen in minimum space:
\begin{verbatim}
   PROC ai = (INT i)VOID:
   IF   i < 0  THEN print("-");  ai(ABS i)
   ELIF i < 10 THEN print(REPR(i+ABS"0"))
   ELSE ai(i%10);  ai(i MOD 10)
   FI
\end{verbatim}
\noindent
In each of these two cases, the procedure includes a test which
chooses between a \ix{recursive call} and phrases
which do not result in a recursive call. This is necessary because,
otherwise, the procedure would never complete. Neither of these
procedures uses a locally declared value. Here is one which does:
\begin{verbatim}
   PROC new fact = (INT i)INT:
   IF INT n:=i-1;   n = 1
   THEN 2
   ELSE i*new fact(n)
   FI
\end{verbatim}
\noindent
The example is somewhat artificial, but illustrates the point. If
\verb|new fact| is called by, for example, \verb|new fact(3)|, then
in the first call, \verb|n| will have the value \verb|2|, and
\verb|new fact| will be called again with the parameter equal to
\verb|2|.  In the second call, \verb|n| will be \verb|1|, but this
\verb|n| this time round will be a new \verb|n|, with the first
\verb|n| inaccessible (it being declared in an \ix{enclosing range}).
\verb|new fact| will yield \verb|2|, and this value will be used in
the formula on line~4 of the procedure.  The first call to
\verb|new fact| will then exit with the value~\verb|6|.

Apart from being fun, recursive procedures can be an efficient way of
programming a particular problem. Chapter~11 deals with, amongst other
topics, recursive modes, and there, recursive programming comes into
its own.

A different form of recursion, known as \bfix{mutual recursion}, is
exemplified by two procedures which call each other. You have to
ensure there is no circularity. The principal difficulty of how to
use a procedure before it has been declared is overcome by first
declaring a procedure name and then assigning a routine denotation to
the procedure name after the other procedure has been declared. Here
is a simple example:\footnote{A compiler which implements the
Algol~68 defined by the Revised Report would not have to resort to
this device because the declaration of each procedure would be
available everywhere in the enclosing range (but see
section~\hyref{rout-other}).}
\begin{verbatim}
   PROC(INT)INT pb;
   PROC pa = (INT i)INT: (i>0|pb(i-1)|i);
   pb:=(INT i)INT: (i<0|pa(i+1)|i);
\end{verbatim}
\noindent
Then \verb|pa(4)| would yield \verb|3| and \verb|pa(-4)| would yield
\verb|-4|. Similarly, \verb|pb(4)| would yield \verb|4| and
\verb|pb(-4)| would yield \verb|-3|. Notice that the right-hand side of
the assignment is an anonymous routine denotation.

\begin{exercise}
\item Write a recursive procedure to reverse the order of letters in a
value of mode \verb|[]CHAR|. It should yield a value also of mode
\verb|[]CHAR|. \ans \ %
\begin{verbatim}
   PROC reverse = ([]CHAR s)[]CHAR:
   (SIZE s=1|s|s[UPB s]+reverse(s[:UPB s-1]))
\end{verbatim}
'
\item Write two mutually recursive procedures which take an
integer parameter and which yield an \verb|INT|. The first should call
the second if the parameter is odd, and the second should call the
first if the parameter is even. The alternative option should yield the
square of the parameter for the first procedure and the cube of the
parameter for the second procedure. Use \verb|square| and \verb|cube|
as the procedure identifiers. \ans \ %
\begin{verbatim}
   PROC(INT)INT cube;

   PROC square=(INT p)INT:
      (ODD p|cube(p)|p^2);
   cube:=(INT p)INT: (ODD p|p^3|square(p))
\end{verbatim}
\noindent
'
\end{exercise}

\Subsection{Standard procedures}{rout-stan}
The \ix{standard prelude} contains the declarations of more than 60
procedures, most of them concerned with transput (see chapter nine).
A number of procedures, all having the mode
\begin{verbatim}
   PROC(REAL)REAL
\end{verbatim}
\noindent
are declared in the standard prelude and yield the values of common
mathematical functions. These are \ixtt{sqrt}, \ixtt{exp}, \ixtt{ln},
\ixtt{cos}, \ixtt{sin}, \ixtt{tan}, \ixtt{arctan}, \ixtt{arcsin} and
\ixtt{arccos}.  Naturally, you must be careful to ensure that the
actual parameter for \verb|sqrt| is non-negative, and that the actual
parameter for \verb|ln| is greater than zero.  The procedures
\verb|cos|, \verb|sin| and \verb|tan| expect their \verb|REAL|
parameter to be in radians.

New procedures using these predeclared procedures can be declared:
\begin{verbatim}
   PROC sinh =
      (REAL x)REAL: (exp(x) + exp(-x))/2
\end{verbatim}
\noindent
A variety of pseudo-random numbers can be produced using
\verb|random int|. The mode of the procedure \ixtt{random int} is
\begin{verbatim}
   PROC(INT)INT
\end{verbatim}
\noindent
and yields a pseudo-random integer greater than or equal to one, and
less than or equal to its integer parameter.  For example, here is a
procedure which will compute the percentage of each possible die
throw in $10\,000$ such throws:
\begin{verbatim}
   PROC percentage = []REAL:
   (
      PROC throw = INT: random int(6);

      [6]REAL result:=(0,0,0,0,0,0);

      TO 10 000 DO result[throw]+:=1 OD;

      FOR i FROM LWB result TO UPB result
      DO result[i] /:= 10 000 OD;

      result
   )
\end{verbatim}
\noindent
Notice that \verb|percentage| has another procedure (\verb|throw|)
declared within it. There is no limit to such
\hx{nesting}{procedure!nesting}.

\Subsection{Other features of procedures}{rout-other}
Since a procedure is a value, it is possible to declare values whose
modes include a procedure mode. For example, here is a
multiple of \hx{procedures}{procedure!multiple}:
\begin{verbatim}
   []PROC(REAL)REAL pr = (sin,cos,tan)
\end{verbatim}
\noindent
and here is a possible call:
\begin{verbatim}
   pr[2](2)
\end{verbatim}
\noindent
which yields $-0.416\,146\,836\,5$. We could also declare a procedure
which could be called with the expression
\begin{verbatim}
   pr(2)[2]
\end{verbatim}
\noindent
but this is left as an exercise.

Similarly, names of procedures can be declared and can be quite
useful.  Instead of declaring
\begin{verbatim}
   PROC pc = (INT i)PROC(REAL)REAL: pr[i]
\end{verbatim}
\noindent
using \verb|pr| declared above, with a possible call of \verb|pc(2)|
we could write
\begin{verbatim}
   PROC(REAL)REAL pn:=pr[i]
\end{verbatim}
\noindent
and then use \verb|pn| instead of \verb|pc|. The advantage of this
would be that \verb|pr| would be subscripted only once instead of
whenever \verb|pc| is elaborated. Furthermore, another procedure
could be assigned to \verb|pn| and the procedure it refers to again
called.  Using \verb|pn| would usually involve
\hx{dereferencing}{coercion!dereferencing}.

There are scoping problems involved with procedure
\hx{names}{procedure!name}. Although the scope of a denotation is
global, procedure denotations may include an identifier whose range
is not global. For this reason, the scope of a procedure denotation
is limited to the smallest enclosing clause containing a declaration
of an identifier or mode or operator indicant which is used in the
procedure denotation.

For example, in this program fragment
\begin{verbatim}
   PROC REAL pp;  REAL y;
   BEGIN
      REAL x:=3.0;
      PROC p = REAL:  x:=4.0;
      print(p);
      pp:=p; CO wrong CO
      print(x)
   END;
   print(("pp=",pp)) #wrong#
\end{verbatim}
\noindent
the assignment in line~6 is wrong because the scope of the right-hand
side is less than the scope of the left-hand side. Unfortunately, the
\hx{a68toc}{a68toc!scope checking} compiler does not perform scope
checking and so will not flag the incorrect assignment.

There are times when \ixtt{SKIP} is useful in a procedure declaration:
\begin{verbatim}
   PROC p = REAL:
   IF x<0
   THEN print("Negative parameter"); stop; SKIP
   ELSE sqrt(x)
   FI
\end{verbatim}
\noindent
The yield of the procedure is \verb|REAL|, so each part of the
conditional \hx{clause}{clause!conditional} must yield a value of
mode \verb|REAL|.  The construct \verb|stop| yields \verb|VOID|, and
even in a strong \hx{context}{context!strong}, \verb|VOID| cannot be
coerced to \verb|REAL|.  However, \verb|SKIP| will yield an undefined
value of any required mode.  In this case, \verb|SKIP| yields a value
of mode \verb|REAL|, but the value is never used, because the program
is terminated just before.

Grouping your program into procedures helps to keep the logic simple
at each level.  Nesting procedures makes sense when the nested
\hx{procedures}{procedure!nesting} are used only within the outer
procedures.  This topic is covered in greater depth in chapter~12.

\Section{Summary}{rout-summ}
The fact that this is one of the longer chapters in the book reflects
the importance of routines in Algol~68 programs. Every formula uses
operators, and procedures enable a program to be written in small
chunks and tested in a piecewise manner.

A routine denotation forms the basis of both operators and
procedures.  Routine denotations have a well-defined mode, the value
being the denotation itself. A routine can declare identifiers within
its body, including other routines (whether operators or procedures).

Operators can have one or two operands (as the parameters are called)
and usually yield a value of some mode other than \verb|VOID|. Dyadic
operators have a priority of~1 to~9. Firmly related operators cannot be
declared in the same range. The operator symbol can be a bold indicant
(like a mode indicant) or one of or a combination of various symbols.

Procedures can have none or more parameters of any mode, and can
yield a value of any mode (including \verb|VOID|). Procedures can
call themselves: this is known as recursion.

Rows of procedures, names of procedures and other modes using
procedure modes can all be declared and, on occasion, can be useful.

Here are some exercises which cover some of the topics discussed in
this rather long chapter.

\begin{exercise}
\item At the time of the call of a procedure or operator, what is the
relationship between the formal parameters and the actual parameters?
\ans They form the two sides of an identity declaration.
'
\item Write an operator which will find the largest element in its
two-dimensional row-of-reals parameter. \ans \ %
\begin{verbatim}
   OP LARGEST = ([,]REAL a)REAL:
   (
      REAL largest:=a[1 LWB a,2 LWB a];

      FOR i FROM 1 LWB a TO 1 UPB a
      DO
         FOR j FROM 2 LWB a TO 2 UPB a
         DO
            largest:=largest MAX a[i,j]
         OD
      OD;
      largest
   )
\end{verbatim}
'
\item Write a procedure, identified by \verb|pr|, which can be called by the
phrase \verb|pr(2)[2]|. \ans \ %
\begin{verbatim}
   PROC pr = (INT n)REF[]INT: HEAP[n]INT
\end{verbatim}
'
\item Write a procedure which computes the length of a line read from the
keyboard. \ans \ %
\begin{verbatim}
   PROC leng = INT:
   (STRING s;
    read((s,newline));
    UPB s)
\end{verbatim}
'
\end{exercise}
