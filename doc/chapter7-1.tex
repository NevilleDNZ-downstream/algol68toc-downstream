% $Log: chapter7.tex,v $
% Revision 1.3  2002-06-20 11:49:28  sian
% mm removed, padding removed, ca68 added
%
\Chapter{Structures}{chapvii}
Structures are a powerful piece of Algol 68, particularly when
combined with the unions described in the next chapter.  In this
chapter, we shall meet another mode constructor, examine complex
numbers and their associated operators and learn how to construct
new modes.  In doing so, you will deepen your understanding of the
language.

\Section{Structure denotations}{struc-denot}
In chapter 3, we saw how a number of individual values can be
collected together to form a multiple whose mode was expressed as
``row of'' the base mode. The principal characteristic of multiples
is that all the elements have the same mode. A structure is another
way of grouping data elements, but in this case, the individual parts
can be, but need not be, of different modes. In general, accessing
the elements of a multiple is determined at run-time by the
elaboration of a slice. In a \ix{structure}, access to the individual
parts, called \bfix{fields}, are determined at compile time.
Structures are, therefore, an efficient means of grouping data
elements.

The mode constructor \ixtt{STRUCT} is used to create structure modes.
Here is a simple identity
\hx{declaration}{declaration!identity!STRUCT@\texttt{STRUCT}} of a
structure:
\begin{verbatim}
   STRUCT(INT i,CHAR c) s = (2,"e")
\end{verbatim}
\noindent
The mode of the structure is
\begin{verbatim}
   STRUCT(INT i,CHAR c)
\end{verbatim}
\noindent
and its identifier is \verb|s|. The \verb|i| and the \verb|c| are
called \hx{\textbf{field selectors}}{field selector} and are part of
the mode. They are not identifiers, even though the rule for
identifier construction applies to them, because they are not values
in themselves.  You cannot say that \verb|i| has mode \verb|INT|
because \verb|i| cannot stand by itself. We shall see in the next
section how they are used.

The expression to the right of the equals symbol is called a
\hx{\textbf{structure-display}}{structure!display}. Like
row-displays, structure-displays can only appear in a
\hx{strong context}{context!strong}.  In a strong context, the
compiler can determine which mode is required and so it can tell
whether a row-display or a structure-display has been provided.  We
could now declare another such structure:
\begin{verbatim}
   STRUCT(INT i,CHAR c) t = s
\end{verbatim}
\noindent
and \verb|t| would have the same value as \verb|s|.

Here is a structure declaration
\begin{verbatim}
   STRUCT(INT j,CHAR c) ss = (2,"e")
\end{verbatim}
\noindent
which looks almost exactly like the first structure declaration
above, except that the field selector \verb|i| has been replaced with
\verb|j|.  The structure \verb|ss| has a different mode from \verb|s|
because not only must the \hx{constituent modes}{mode!constituent} be
the same, but the field selectors must also be identical.

Structure names can be declared:
\begin{verbatim}
   REF STRUCT(INT i,CHAR c) sn =
             LOC STRUCT(INT i,CHAR c)
\end{verbatim}
\noindent
Because the field selectors are part of the mode, they appear on both
sides of the declaration. The abbreviated form is
\begin{verbatim}
   STRUCT(INT i,CHAR c) sn
\end{verbatim}
\noindent
We could then write
\begin{verbatim}
   sn:=s
\end{verbatim}
\noindent
in the usual way.

The modes of the fields can be any mode.  For example, we can declare
\begin{verbatim}
   STRUCT(REAL x,REAL y,REAL z) vector
\end{verbatim}
\noindent
which can be abbreviated to
\begin{verbatim}
   STRUCT(REAL x,y,z)vector
\end{verbatim}
\noindent
and here is a possible assignment:
\begin{verbatim}
   vector:=(1.3,-4,5.6e10)
\end{verbatim}
\noindent
where the value \verb|-4| would be \hx{widened}{coercion!widening}
to \verb|-4.0|.

Here is a structure with a \hx{procedure field}{structure!procedure field}:
\begin{verbatim}
   STRUCT(INT int,
          PROC(REAL)REAL p,
          CHAR char) method = (1,sin,"s")
\end{verbatim}
\noindent
Here is a name referring to such a structure:
\begin{verbatim}
   STRUCT(INT int,
          PROC(REAL)REAL p,
          CHAR char) m := method
\end{verbatim}
\noindent
A structure can even contain another
\hx{structure}{structure!nested}:
\begin{verbatim}
   STRUCT(CHAR c,
          STRUCT(INT i,j)s) double=("c",(1,2))
\end{verbatim}
\noindent
In this case, the inner structure has the \ix{field selector}
\verb|s| and \textit{its\/} field selectors are \verb|i| and \verb|j|.

\begin{exercise}
\item Declare a structure containing three integer values with field
selectors \verb|i|, \verb|j| and \verb|k|.
\ans \verb|STRUCT(INT i,j,k) s1 = (1,2,3)|
'
\item Declare a name which can refer to a structure containing an
integer, a real and a boolean using field selectors \verb|i|,
\verb|r| and \verb|b| respectively.
\ans \verb|STRUCT(INT i,REAL r,BOOL b)s2|
'
\end{exercise}

\Section{Field selection}{struc-field} 
\hx{The}{field selection} field-selectors of a
\hx{structure mode}{structure!mode} are used to ``extract'' the
individual fields of a structure.  For example, given this
declaration for the structure~\verb|s|:
\begin{verbatim}
   STRUCT(INT i,CHAR c) s = (2,"e")
\end{verbatim}
\noindent
we can select the first field of \verb|s| using the \bfix{selection}
\begin{verbatim}
   i OF s
\end{verbatim}
\noindent
The mode of the selection is \verb|INT| and its value is \verb|2|.
Note that the construct \ixtt{OF} is not an operator. The second
field of \verb|s| can be selected using the selection
\begin{verbatim}
   c OF s
\end{verbatim}
\noindent
whose mode is \verb|CHAR| with value \verb|"e"|. The field-selectors
cannot be used on their own: they can only be used in a selection.

A selection can be used as an operand. Consider the formula
\begin{verbatim}
   i OF s * ABS c OF s
\end{verbatim}
\noindent
In the structure \verb|method|, declared in the previous section, the
procedure in the structure can be selected by
\begin{verbatim}
   p OF method
\end{verbatim}
\noindent
which has the mode \verb|PROC(REAL)REAL|. For a reason which will be
clarified in chapter~10, if you want to call this procedure, you must
enclose the selection in \ix{parentheses}:
\begin{verbatim}
   (p OF method)(0.5)
\end{verbatim}
\noindent
Remembering that the context of the actual-parameters of a procedure
is strong, you could also \hx{write}{coercion!widening}
\begin{verbatim}
   (p OF method)(int OF method)
\end{verbatim}
\noindent
where \verb|int OF method| would be widened to a real number and the
whole expression would yield a value of mode \verb|REAL|.

The two fields of the structure \verb|double| (also declared in the
previous section), can be selected by writing
\begin{verbatim}
   c OF double
   s OF double
\end{verbatim}
\noindent
and their modes are \verb|CHAR| and \verb|STRUCT(INT i,j)|
respectively. Now the fields of the inner structure \verb|s| of
\verb|double| can be selected by writing
\begin{verbatim}
   i OF s OF double
   j OF s OF double
\end{verbatim}
\noindent
and both selections have mode \verb|INT|.

Now consider the structure name \verb|sn| declared by
\begin{verbatim}
   STRUCT(INT i,CHAR s) sn;
\end{verbatim}
\noindent
The mode of \verb|sn| is
\begin{verbatim}
   REF STRUCT(INT i,CHAR s)
\end{verbatim}
\noindent
This means that the mode of the selection
\begin{verbatim}
   i OF sn
\end{verbatim}
\noindent
is not \verb|INT|, but \verb|REF INT|, and the mode of the selection
\begin{verbatim}
   c OF sn
\end{verbatim}
\noindent
is \verb|REF CHAR|. That is, the modes of the fields of a structure
name are all preceded by \ixtt{REF} (they are all names).  This is
particularly important for recursively defined structures (see
chapter~11). Thus, instead of assigning a complete structure using a
\hx{structure-display}{structure!display}, you can assign values to individual fields.
That is, the assignment
\begin{verbatim}
   sn:=(3,"f")
\end{verbatim}
\noindent
is equivalent to the assignments
\begin{verbatim}
   i OF sn := 3;
   c OF sn := "f"
\end{verbatim}
\noindent
except that the assignments to the individual fields are separated by
the go-on symbol (the semicolon \ixtt{;}) and the
two units in the structure-display are separated by a comma and so
are elaborated \hx{collaterally}{elaboration!collateral}.

Given the declaration and initial assignment
\begin{verbatim}
   STRUCT(CHAR c,STRUCT(INT i,j)s)dn:=double
\end{verbatim}
\noindent
the selection
\begin{verbatim}
   s OF dn
\end{verbatim}
\noindent
has the mode \verb|REF STRUCT(INT i,j)|, and so you could assign
directly to~it:
\begin{verbatim}
   s OF dn:=(-1,-2)
\end{verbatim}
\noindent
as well as to one of its fields:
\begin{verbatim}
   j OF s OF dn:=0
\end{verbatim}

\begin{exercise}
\item Given the declarations
\begin{verbatim}
   STRUCT(STRUCT(CHAR a,INT b)c,
          PROC(STRUCT(CHAR a,INT b))INT p,
          INT d)st;
   STRUCT(CHAR a,INT b)sta
\end{verbatim}
\noindent
what is the mode of
\begin{subex}
\item \verb|c OF st| \subans \verb|REF STRUCT(CHAR a,INT b)|
'
\item \verb|a OF c OF st| \subans \verb|REF CHAR|
'
\item \verb|a OF sta| \subans \verb|REF CHAR|
'
\item \verb|(p OF st)(sta)| \subans \verb|INT|, provided that a
procedure had been assigned to \verb|p OF st|.
'
\item \verb|b OF c OF st * b OF sta| \subans \verb|INT|
'
\item \verb|sta:=c OF st| \subans \verb|REF STRUCT(CHAR a,INT b)|
'
\end{subex}
\item Declare a procedure which could be assigned to the selection
\verb|p OF st| in the last question. \ans \ %
\begin{verbatim}
   PROC p1=(STRUCT(CHAR a,INT b)s)INT:
      ABS a OF s * b OF s
\end{verbatim}
'
\end{exercise}

\Section{Mode declarations}{struc-mode}
\hx{Structure}{declaration!structure} declarations are very common
in Algol 68 programs because they are a convenient way of grouping
disparate data elements, but writing out their modes every time a
name needs declaring is error-prone.  Using the
\bfix{mode declaration}, a new mode \hx{indicant}{mode!indicant} can
be declared to act as an abbreviation.  For example, the mode
declaration
\begin{verbatim}
   MODE VEC = STRUCT(REAL x,y,z)
\end{verbatim}
\noindent
makes \verb|VEC| synonymous for the mode specification on the
right-hand side of the equals symbol. Henceforth, new values using
\verb|VEC| can be declared in the ordinary way:
\begin{verbatim}
   VEC vec = (1,2,3);
   VEC vn := vec;
   [10]VEC va;
   PROC(VEC v)VEC pv=CO a routine-denotation CO;
   STRUCT(VEC v,w,x) tensor
\end{verbatim}

Here is a mode declaration for a structure which contains a reference
mode:
\begin{verbatim}
   MODE RV = STRUCT(CHAR c,REF[]CHAR s)
\end{verbatim}
\noindent
but we shall consider such advanced modes in chapter~11. Using a mode
declaration, you might be tempted to declare a mode such as
\begin{verbatim}
   MODE CIRCULAR =
         STRUCT(INT i,CIRCULAR c)  CO wrong CO
\end{verbatim}
\noindent
but this is not allowed. However, there is nothing wrong with such
modes~as
\begin{verbatim}
   MODE NODE  = STRUCT(STRING s,
                       REF NODE next),
        PNODE = STRUCT(STRING s,
                       PROC(PNODE)STRING proc)
\end{verbatim}
\noindent
because the \verb|NODE| inside the \verb|STRUCT| of its declaration
is hidden by the \verb|REF|. Likewise, the \verb|PNODE| parameter for
\verb|proc| in the declaration of \verb|PNODE| is hidden by the
\verb|PROC|.

Suppose you want a mode which refers to another mode which hasn't
been declared, and the second mode will refer back to the first mode.
Both mode declarations cannot be first. In Algol 68 proper, you
simply declare both modes in the usual way. However, the
\hx{a68toc}{a68toc!recursive modes}
compiler is a single-pass compiler (it reads the source program once
only) and so all applied-occurrences must occur later in the source
program than the defining-occurrences. In this case, one of the modes
is declared using a \hx{\textbf{stub declaration}}{declaration!stub}.
Here is an example:
\begin{verbatim}
   MODE MODE2,
        MODE1 = STRUCT(CHAR c,REF MODE2 rb),
        MODE2 = STRUCT(INT i,REF MODE1 ra)
\end{verbatim}
\noindent
There is nothing circular about these declarations.  This is another
example of \ix{mutual recursion}.  Go ahead and experiment.

This raises the point of which modes are actually permissible. We
shall deal with this in chapter~10. For now, just ensure that you
don't declare modes like \verb|CIRCULAR|, and avoid modes which can
be strongly coerced into themselves, such as
\begin{verbatim}
MODE WRONG = [1:5]WRONG
\end{verbatim}
\noindent
If you inadvertently declare a disallowed mode, the compiler will
declare that the mode is not legal.

Mode declarations are not confined to structures. For example, the
mode \ixtt{STRING} is declared in the standard prelude as
\begin{verbatim}
   MODE STRING = FLEX[1:0]CHAR
\end{verbatim}
\noindent
and you can write declarations like
\begin{verbatim}
MODE FDES     = INT,
     MULTIPLE = [30]REAL,
     ROUTINE  = PROC(INT)INT,
     MATRIX   = [n,n]REAL
\end{verbatim}
\noindent
Notice that the mode declarations have been abbreviated (by omitting
\verb|MODE| each time and using commas). In the declaration of
\verb|ROUTINE|, notice that no identifier is given for the parameter of
the procedure. In the last declaration, the bounds will be determined
at the time of the declaration of any value using the mode
\verb|MATRIX|. Here, for example, is a small program using
\verb|MATRIX|:
\begin{verbatim}
   PROGRAM tt CONTEXT VOID
   USE standard
   BEGIN
   INT n;
   MODE MATRIX = [n,n]REAL;

   WHILE   
      print((newline,
             "Enter an integer ",
             "followed by a blank:"));
      read(n);
      n > 0
   DO
      MATRIX m;

      FOR i TO 1 UPB m
      DO
         FOR j TO 2 UPB m
         DO
            m[i,j]:=random*1000
         OD
      OD;

      FOR i TO 1 UPB m
      DO
         print((m[i,],newline))
      OD
   OD
   END
   FINISH
\end{verbatim}
\newpage

\begin{exercise}
\item Declare a mode for a structure containing two fields, one of mode
\verb|REAL| and one of mode \verb|PROC(REAL)REAL|.
\ans \ %
\begin{verbatim}
   MODE EX_7_3_1=STRUCT(REAL r,
                        PROC(REAL)REAL p)
\end{verbatim}
'
\item Declare a mode for a structure which contains three fields, the
first being the mode of the previous exercise, the second a procedure
which takes that mode as a parameter and yields \verb|VOID|, and the third
being of mode \verb|CHAR|.
\ans \ %
\begin{verbatim}
   MODE EX_7_3_2=
      STRUCT(EX_7_3_1 e,
             PROC(EX_7_3_1)VOID p,
             CHAR c)
\end{verbatim}
'
\item What is wrong with these two definitions?
\begin{verbatim}
   MODE AMODE = STRUCT(INT i,BMODE b),
        BMODE = STRUCT(CHAR c,AMODE a)
\end{verbatim}
\noindent
Try writing a program containing these declarations, with names
of modes \verb|AMODE| and \verb|BMODE| and finish the program with
the unit \verb|SKIP|.  \ans One of the \verb|BMODE| and \verb|AMODE|
structures is insufficiently shielded. You will get an error for
\verb|BMODE| saying it is not a legal mode and another error for the
declaration of a \verb|REF AMODE| saying that the mode \verb|AMODE|
has not been declared.
'
\end{exercise}

\Section{Complex numbers}{struc-compl}
This section describes the mode used to perform complex arithmetic.
This kind of arithmetic is useful to engineers, particularly electrical
engineers. Even if you know nothing about complex numbers, you may
still find this section useful.

The \ix{standard prelude} contains the mode
\hx{declaration}{declaration!mode}
\begin{verbatim}
   MODE COMPL = STRUCT(REAL re,im)
\end{verbatim}
\noindent
You can use values based on this mode to perform complex arithmetic.
Here are declarations for values of modes \ixtt{COMPL} and
\verb|REF COMPL| respectively:
\begin{verbatim}
   COMPL z1 = (2.4,-4.6);
   COMPL z2:=z1
\end{verbatim}
\noindent
Most of the operators you need to manipulate \ix{complex numbers}
have been declared in the standard prelude.

You can use the monadic operators \ixtt{+} and \ixtt{-} which have
also been declared for values of mode \verb|COMPL|.

The dyadic operator \ixtt{**} has been declared for a left-operand of
mode \verb|COMPL| and a right-operand of mode \verb|INT|.  The dyadic
operators \verb|+ - * /| have been declared for all combinations of
complex numbers, real numbers and integers, and so have the
boolean operators \ixtt{=} and \ixtt{/=}.  The assigning operators
\ixtt{TIMESAB}, \ixtt{DIVAB}, \ixtt{PLUSAB}, and \ixtt{MINUSAB} all
take a left operand of mode \verb|REF COMPL| and a right-operand of
modes \verb|INT|, \verb|REAL| or \verb|COMPL|.  In a strong
\hx{context}{context!strong}, a real number will be
\hx{widened}{coercion!widening} to a complex number.  So, for
example, in the following identity declaration
\begin{verbatim}
   COMPL z3 = -3.4
\end{verbatim}
\noindent
\verb|z3| will have the same value as if it had been
declared by
\begin{verbatim}
   COMPL z3 = (-3.4,0)
\end{verbatim}
\noindent
This is the only case where a real number can be widened into a
structure.

The dyadic operator \ixtt{I} takes left- and right-operands of any
combination of \verb|REAL| and \verb|INT| and yields a complex
number.  It has a priority of~9. For example, in a formula, the
context of operands is firm and so widening is not allowed.
Nevertheless, the yield of this formula is \verb|COMPL|:
\begin{verbatim}
   2 * 3 I 4
\end{verbatim}

Some operators act only on complex numbers. The monadic operator
\ixtt{RE} takes a \verb|COMPL| operand and yields its \verb|re| field
with mode \verb|REAL|. Likewise, the monadic operator \ixtt{IM} takes
an operand of mode \verb|COMPL| and yields its \verb|im| field with
mode \verb|REAL|. For example, given the declaration above of
\verb|z3|, \verb|RE z3| would yield \verb|-3.4|, and \verb|IM z3|
would yield \verb|0.0|. Given the complex number \verb|z| declared as
\begin{verbatim}
   COMPL z = 2 I 3
\end{verbatim}
\noindent
then \hxtt{CONJ z}{CONJ} would yield \verb|RE z I - IM z| or
\verb|(2.0,-3.0)|. The operator \ixtt{ARG} gives the argument of its
operand: \verb|ARG z| would yield $0.982\,793\,723\,2$, lying in the
interval $\left(-\pi,\pi\right]$.  The monadic operator \ixtt{ABS}
with a complex number may be defined as
\begin{verbatim}
   OP ABS = (COMPL z)REAL:
      sqrt(RE z**2 + IM z**2)
\end{verbatim}
\noindent
Remember that in the formula \verb|RE z**2|, the operator
\verb|RE| is monadic and so is elaborated first.

As described in the previous section, the mode \ixtt{COMPL} can be
used wherever a mode is required.  In particular, procedures taking
\verb|COMPL| parameters and yielding \verb|COMPL| values can be
declared.  Structures containing \verb|COMPL| can be declared as
above.

From the section on field selection, it is clear that in the
declarations
\begin{verbatim}
   COMPL z = (2.0,3.0);
   COMPL w:=z
\end{verbatim}
\noindent
the selection
\begin{verbatim}
   re OF z
\end{verbatim}
\noindent
has mode \verb|REAL| (and value \verb|2.0|), while the selection
\begin{verbatim}
   re OF w
\end{verbatim}
\noindent
has mode \verb|REF REAL| (and its value is a name). However, the
formula
\begin{verbatim}
   RE w
\end{verbatim}
\noindent
still yields a value of mode \verb|REAL| because
\ixtt{RE} is an operator whose single operand has mode
\verb|COMPL|. In the above phrase, the \verb|w| will be dereferenced
before \verb|RE| is elaborated. Thus it is quite legal to write
\begin{verbatim}
   im OF w:=RE w
\end{verbatim}
\noindent
or
\begin{verbatim}
   im OF w:=re OF w
\end{verbatim}
\noindent
in which case the right-hand side of the assignment will be
dereferenced before a copy is assigned.

\begin{exercise}
\item If the complex number \verb|za| has the mode \verb|COMPL| and the
value yielded by \verb|(2,-3)|, what is the value of
\begin{subex}
\item \verb|CONJ za| \subans \verb|(2.0,3.0)|
'
\item \verb|IM za * RE za * RE za| \subans \verb|-12.0|
'
\item \verb|ABS za| \subans Write a short program to get
\begin{verbatim}
   3.6055512754639891
\end{verbatim}
'
\item \verb|ARG za| \subans \verb|0.982 793 723 247 329 1|
'
\end{subex}
\item What is the value of the formula \verb|23 - 11 I -10|?
\ans The value denoted by \verb|(12.0,-10.0)|.
'
\item Given the declarations
\begin{verbatim}
   COMPL a = 2 I 3;
   COMPL b:= CONJ a
\end{verbatim}
\noindent
what is the mode and value of each of the following:
\begin{subex}
\item \verb|im OF b| \subans \verb|REF REAL|, a name.
'
\item \verb|IM b| \subans \verb|REAL  -3.0|
'
\item \verb|im OF a| \subans \verb|REAL  3.0|
'
\item \verb|IM a| \subans \verb|REAL  3.0|
'
\end{subex}
\end{exercise}

\Section{Multiples in structures}{struc-mult}
If multiples are required in a structure, the structure
\hx{declaration}{declaration!structure} should only contain the
required bounds if it is an \ix{actual-declarer}.  For example, we
could declare
\begin{verbatim}
   STRUCT([]CHAR forename,
                 surname,
                 title)
      lecturer=
         ("Nerissa","Leitch","Dr")
\end{verbatim}
\noindent
where the mode on the left is a \ix{formal-declarer} (remember that
the mode on the left-hand side of an identity
\hx{declaration}{identity!declaration!formal-declarer} is always a
formal-declarer).  We could equally well declare
\begin{verbatim}
   STRUCT([]CHAR forename,
                 surname,
                 title)
      student=
            ("Tom","MacAllister","Mr")
\end{verbatim}
\noindent
As you can see, the bounds of the individual multiples differ in the
two cases.

When declaring a name, because the mode preceding the name identifier
is an actual-declarer (in an abbreviated declaration), the bounds of
the required multiples must be included. A suitable declaration for a
name which could refer to \verb|lecturer| would be
\begin{verbatim}
   STRUCT([7]CHAR forename,
          [6]CHAR surname,
          [3]CHAR title)
            new lecturer := lecturer
\end{verbatim}
\noindent
but this would not be able to refer to \verb|student|. A better
declaration would use \ixtt{STRING}:
\begin{verbatim}
   STRUCT(STRING forename,surname,title)person
\end{verbatim}
\noindent
in which case we could now write
\begin{verbatim}
   person:=lecturer;
   person:=student
\end{verbatim}
\noindent
Using field selection, we can write
\begin{verbatim}
   title OF person
\end{verbatim}
\noindent
which would have mode \verb|REF STRING|. Thus, using field selection,
we can assign to the individual fields of \verb|person|:
\begin{verbatim}
   surname OF person:="McRae"
\end{verbatim}

When slicing a field which is a multiple, it is necessary to remember
that \hx{slicing}{slice} binds more tightly than selecting (see
chapter~10 for a detailed explanation). Thus the first character of
the surname of \verb|student| would be accessed by writing
\begin{verbatim}
   (surname OF student)[1]
\end{verbatim}
\noindent
which would have mode \verb|CHAR|. The parentheses ensure that the
selection is elaborated before the slicing. Similarly, the first five
characters of the forename of \verb|person| would be accessed as
\begin{verbatim}
   (forename OF person)[:5]
\end{verbatim}
\noindent
with mode \verb|REF[]CHAR|.

Consider the following program:
\begin{verbatim}
   PROGRAM t1 CONTEXT VOID
   BEGIN
      MODE AMODE = STRUCT([4]CHAR a,INT b);
      AMODE a = ("abcde",3);
      AMODE b:=a;
      SKIP
   END
   FINISH
\end{verbatim}
\noindent
In the identity declaration for \verb|a|, the mode required is a
\ix{formal-declarer}.  In this case, the
\hx{a68toc}{a68toc!ignoring bounds}
compiler will ignore the bounds in the declaration of \verb|AMODE|
giving the mode
\begin{verbatim}
   STRUCT([]CHAR a,INT b)
\end{verbatim}
\noindent
which explains why the structure-display on the right is accepted
(\verb|"abcde"| has bounds \verb|[1:5]|). However, although the
program compiles without errors, when it is run, it fails with the
error message
\begin{verbatim}
   Run time fault (aborting):
   ASSIGN2: bounds do not match in [] assignment
\end{verbatim}
\noindent
because the mode used in the declaration of the name \verb|b| is an
\ix{actual-declarer} (it contains the bounds given in the mode
\hx{declaration}{declaration!mode}) and you cannot assign a
\verb|[]CHAR| with bounds \verb|[1:5]| to a \verb|REF[]CHAR| with
bounds \verb|[1:4]|.

With more complicated structures, it is better to use a mode
declaration. For example, we could declare
\begin{verbatim}
   MODE EMPLOYEE =
      STRUCT(STRING name,
             [2]STRING address,
             STRING dept,ni code,tax code,
             REAL basic,overtime rate,
             [52]REAL net pay,tax);

   EMPLOYEE emp
\end{verbatim}
\noindent
and then read specific values from the keyboard (chapter 9 covers
reading data from files):
\begin{verbatim}
   read((name OF emp,newline,
         (address OF emp)[1],newline,
         (address OF emp)[2],newline,
   ...
\end{verbatim}
\noindent
The modes of
\begin{verbatim}
   name OF emp
   address OF emp
   net pay OF emp
\end{verbatim}
\noindent
are
\begin{verbatim}
   REF STRING
   REF[]STRING
   REF[]REAL
\end{verbatim}
\noindent
respectively. The phrase
\begin{verbatim}
   (net pay OF emp)[:10]
\end{verbatim}
\noindent
has the mode \verb|REF[]REAL| with bounds \verb|[1:10]| and
represents the net pay of \verb|emp| for the first 10 weeks. Note
that although the monetary values are held as \verb|REAL| values, the
accuracy with which a \verb|REAL| number is stored is such that no
rounding errors will ensue. See section~\hyref{dev-money} which
describes which modes are suitable for storing monetary values.

\begin{exercise}
\item Given the declaration of \verb|emp| in the text, what would be
the mode of each of the following:
\begin{subex}
\item \verb|address OF emp| \subans \verb|REF[]STRING|
'
\item \verb|basic OF emp| \subans \verb|REF REAL|
'
\item \verb|(tax OF emp)[12]| \subans \verb|REF REAL|
'
\item \verb|(net pay OF emp)[10:12]| \subans \verb|REF[]REAL|
'
\end{subex}
\item What are the bounds of the name in (d) above?
\ans \verb|[1:3]|.
'
\end{exercise}

\Section{Rows of structures}{struc-rows}
\hx{In}{structure!multiple} the last section, we considered
multiples in structures.  What happens if we have a multiple each of
whose elements is a structure?  No problem.  If we had declared
\begin{verbatim}
   [10]COMPL z4
\end{verbatim}
\noindent
then the selection \verb|re OF z4| would yield a name with mode
\verb|REF[]REAL| and bounds \verb|[1:10]|.\footnote{Unfortunately,
there is a bug in the \protect\hx{a68toc}{a68toc!selections} compiler
whereby this selection (and similar selections) are disallowed.} It
would be possible, because it is a name, to assign to it:
\begin{verbatim}
   re OF z4:=(1,2,3,4,5,6,7,8,9,10)
\end{verbatim}

Selecting the field of a sliced multiple of a structure is
straightforward. Since the multiple is sliced before the field is
selected, no parentheses are necessary. Thus the real part of the
third \verb|COMPL| of \verb|z4| above is given by the expression
\begin{verbatim}
   re OF z4[3]
\end{verbatim}

Now consider a multiple of a structure which contains a multiple. Here
is its declaration:
\begin{verbatim}
   [100]STRUCT(CHAR c,[5]INT i)s
\end{verbatim}
\noindent
Then the fourth integer in the \ith{25}{th} structure of \verb|s| is
given by
\begin{verbatim}
   (i OF s[25])[4]
\end{verbatim}
\noindent
and all the characters are given by the selection
\begin{verbatim}
   c OF s
\end{verbatim}
\noindent
with mode \verb|REF[]CHAR| and bounds \verb|[1:100]|.\footnote{But
this is disallowed by the \protect\hx{a68toc}{a68toc!selections} compiler.}

\begin{exercise}
\item Suppose a firm had 20 employees, and in writing one of the
programs in their payroll system, the modes of section 7.5 were used.
Suppose now that we have the declaration
\begin{verbatim}
   [20]EMPLOYEE employee;
\end{verbatim}
\noindent
What would be the mode of each of the following:
\begin{subex}
\item \verb|(dept OF employee[3])[3]| \subans \verb|REF CHAR|
'
\item \verb|dept OF employee[10:12]| \subans \verb|REF[]STRING|
'
\item \verb|ni code OF employee[1]| \subans \verb|REF STRING|
'
\item \verb|net pay OF employee[15]| \subans \verb|REF[]REAL|
'
\item \verb|(tax OF employee[2])[50:51]| \subans \verb|REF[]REAL|
'
\end{subex}
\end{exercise}

\Section{Transput of structures}{struc-trans}
The following program fragment will print the details of the name
\verb|emp| declared in section \hyref{struc-mult}:
\begin{verbatim}
   print((emp,newline))
\end{verbatim}
\noindent
For details of how this works, see the remarks on
``\ix{straightening}'' in section \hyref{trans-straight}. However,
the individual strings would be printed together and so, in this
case, it would be better to write the following:
\begin{verbatim}
   print((name OF emp,newline));
   FOR i TO UPB address OF emp
   DO
      print((address OF emp)[i],newline))
   OD;
   print((dept OF emp,newline,
          ni code OF emp,newline,
          tax code OF emp,
          basic OF emp,
          overtime OF emp,
          net pay OF emp,
          tax OF emp,newline))
\end{verbatim}
\noindent
In practice, it would be sensible to declare a procedure or an
operator which would print the structure and then call it as required.
\newpage

\Section{Summary}{struc-summ}
This chapter has significantly increased the number of different
modes we can construct.  Structures are constructed using the mode
constructor \verb|STRUCT|.  Complicated structures are best declared
using the mode declaration (using \verb|MODE|).  Structures can have
any number of fields from one up, and the fields can have any mode,
including the same modes.  The mode \verb|COMPL| has been declared in
the standard prelude together with the necessary operators to
manipulate complex numbers.

Structures can contain procedures and multiples and multiples of
structures can be declared. Although structures containing reference
modes can be declared, they are covered in chapter~11. Operators and
procedures which have structure parameters or yield can be declared.

Here are some exercises to check what you have learned.

\begin{exercise}
\item Write a suitable mode for a football team which contains the
names of its 11 members, the name of the team (ordinary name, not the
Algol~68 meaning), the number of games played, won and drawn, and the
number of goals scored for and against.
\ans \ %
\begin{verbatim}
   MODE TEAM=STRUCT([11]STRING name,
                    STRING team,
                    INT played, won, drawn,
                        for, against)
\end{verbatim}
'
\item Given the declaration
\begin{verbatim}
   STRUCT(INT i,[3]REAL r)s
\end{verbatim}
\noindent
explain why parentheses are needed in the phrase
\begin{verbatim}
   (r OF s)[2]
\end{verbatim}
\indent\ans Slicing binds more tightly than selecting, so the
selection must be enclosed in parentheses (see section
\hyref{gram-paren} for the full explanation).
'
\item Given the declaration
\begin{verbatim}
   [3]STRUCT(INT i,REAL r)s
\end{verbatim}
\noindent
explain why parentheses are not needed in the phrase
\verb|r OF s[2]|.  \ans The slicing takes place before the selection
so no parentheses are needed.
'
\item Given the declarations
\begin{verbatim}
   MODE S2,
        S1 = STRUCT([3]CHAR n,
                    PROC S2 p),
        S2 = STRUCT([3]CHAR m,
                    PROC(S1)S2 p);
   S1 s1;  S2 s2;
\end{verbatim}
\noindent
what are the modes of each of the following:
\begin{subex}
\item \verb|p OF s1| \subans \verb|REF PROC S2|
'
\item \verb|p OF s2| \subans \verb|REF PROC(S1)S2|
'
\item \verb|(n OF s1)[2:]| \subans \verb|REF[]CHAR|
'
\end{subex}
\end{exercise}
