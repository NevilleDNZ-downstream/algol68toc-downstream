% $Log: chapter12.tex,v $
% Revision 1.4  2004-11-07 16:00:10  sian
% Removed all references to PLUS, MINUS and TIMES
%
% Revision 1.3  2002/06/20 11:49:28  sian
% mm removed, padding removed, ca68 added
%
\Chapter{Program development}{chapxii}
Of course, there is more to writing programs than learning a
programming language. Although you will find many books on
programming languages, you will not find many on computer
\ix{programming} as such.  That is because it is very much a craft.
Be aware that this book does not, and cannot, train you to become a
professional programmer.  Only on-the-job training and experience can
do that---but after working through this chapter, you will have an
idea of some of the activities a professional programmer does.

In the computer industry, there is a widespread attitude that
\hx{program maintenance}{program!maintenance} helps build good
programmers. There are sound reasons for this. One is that reading
other people's programs helps you learn how to lay out programs, how
to organise the source, how to write structured code and how to solve
the sort of problems that a programmer meets daily. Another reason is
that program maintenance usually involves either removing errors
(usually called \bfix{bugs}) or making small changes to the program
to adapt it to changing requirements. You have to learn how a program
works before you change it and reading someone else's program means
that the philosophy of the program (the approach of the program to
solving a problem) is already there---you do not have to create it.

However, there is no substitute for writing your own programs. The
first section of this chapter is concerned with how to write your own
programs, from problem analysis to documentation. The next topic
discusses how to access operating system procedures.  This introduces
almost all those aspects of Algol~68 which involve direct machine
access apart from the mode \verb|BITS| and its associated operators
which were covered in chapter~11.

Next, we turn to the first aspect of program maintenance: how to
understand a program.  A small utility (\verb|lf|) is provided with
the \ix{Algol68toC} compilation system documentation.  This section looks
at \verb|lf| and analyses its functioning.

\Section{Writing programs}{dev-writ}
The first stage in the development of a new program consists of
analysing the problem that the program must solve. Unfortunately,
there is no known method or methodology which will solve any kind of
problem.  However, a particularly good book on \ix{problem solving}
was written by \hx{ George P\'olya}{Polya@P\'olya, George}(see
the Bibliography) and although the book is geared towards
mathematical problems, it will help you solve most technical problems.

\hx{Problem analysis}{problem analysis} is not usually taught to
beginners at computer programming because, so far as we know, it is
mainly an intuitive activity (it is a branch of \ix{Heuristics}).
Learning to analyse a problem with the intention of writing a
computer program is largely accomplished by writing simple programs
followed by programs of increasing sophistication---this is sometimes
called ``\ix{learning by doing}''.  When we start analysing actual
programs later in the chapter, each such analysis will be preceded by
a problem analysis.  You will be able to see how the program, as
presented, accords with that analysis.

Nevertheless, even though no definitive method can be given, there are
guidelines which help you to appreciate and analyse problems suitable
for computer solution. In the field of systems analysis, you will find
various methodologies (such as \ix{SSADM}). These are usually
geared towards large-scale systems and are designed to prevent systems
designers from forgetting details.  In the context of
\hx{program design}{program!design}, knowing the data to be used by
the program and the data to be produced by the program is the principal
guide to knowing what manipulations the program must perform.
\hx{Data knowledge}{data!knowledge} specifies the books accessed by
the program and usually constitutes a substantial part of the program's
documentation.

Once you know the data your program operates on, you can determine
the actual manipulations, or calculations, required. At this stage,
you should be able to determine which
\hx{data structures}{data!structure} are suitable for the solution of
your problem.  The data structures in turn lead you to the
\ix{mode declarations}.  The kind of data structure also helps to
determine the kind of \hx{procedures}{procedure} required.  Some
examples: if your data structures include a queue, then queue
procedures will be needed; or, if you are using multiples (repeated
data), then you will almost invariably be using loops.  Again, if an
input book contains structured data, such as an item which is
repeated many times, then again your program will contain a
processing loop.  The \hx{Jackson programming}{Jackson methodology}
methodology is a useful way of specifying procedures given the data
structures to be manipulated (see the bibliography).

\Subsection{Top-down analysis}{dev-top}
After you have determined suitable modes and procedures, you need to
analyse the problem in a top-down manner.  Basically,
\bfix{top-down analysis} consists of determining the principal
actions needed to perform a given action, then analysing each of the
principal actions in the same way.  For example, suppose we wished to
write a program to copy a book whose identifier is given on the
command line.  The topmost statement of the problem could be
\begin{verbatim}
   copy an identified book
\end{verbatim}

The next stage could be
\begin{verbatim}
   get the book identifier
   open the book
   establish the output copy book
   copy the input book to output
   close both books
\end{verbatim}
\noindent
At this stage, the process ``copy the input book to output'' will
depend on the structure of the input book. If it is text, with lines
of differing length, you could use a name of mode \verb|REF STRING|.
If the book contains similar groupings of data, called
\hx{\textbf{records}}{record}, then it would be more appropriate to
declare a structured mode and write appropriate input and output
procedures:
\begin{verbatim}
   DO
      get record from input book
      put record to output book
   OD
\end{verbatim}
\noindent
The analysis is continued until each action can be directly coded.

\Subsection{Program layout}{dev-layout}
Before you start coding the program (writing the actual Algol~68
source program), you should be aware of various programming
strategies besides the different means of manipulating data
structures.  The first to address is the matter of source
\hx{program layout}{program!layout}.

In the examples given in this book, code has been indented to reflect
program structure, but even in this matter, there are choices. For
example, some people indent the \verb|THEN| and \verb|ELSE| clauses of
an \verb|IF| \hx{clause}{code!indentation}:
\begin{verbatim}
   IF ...
     THEN ...
     ELSE ...
   FI
\end{verbatim}
\noindent
instead of
\begin{verbatim}
   IF   ...
   THEN ...
   ELSE ...
   FI
\end{verbatim}
\noindent
Others regard the parts of the \verb|IF| clause as some kind of
bracketing:
\begin{verbatim}
   IF
      ...
   THEN
      ...
   ELSE
      ...
   FI
\end{verbatim}
\noindent
Some people write a procedure as:
\begin{verbatim}
   PROC ...
      BEGIN
      ...
      END
\end{verbatim}
\noindent
Others never use \verb|BEGIN| and \verb|END|, but only use parentheses.

Another point is whether to put more than one phrase on the same
line. And what about \ix{blank lines}---these usually
improve a program's legibility.  Whatever you decide, keep to your
decision throughout the program (or most of the program) otherwise the
format of the code may prove confusing.  Of course, you will learn by
your mistakes and usually you will change your programming style over
the years.

\Subsection{Declarations}{dev-decs}
Another matter is whether to group declarations. Unlike many
programming languages, Algol~68 allows you to place declarations
wherever you wish. This does not mean that you should therefore
sprinkle declarations throughout your program, al\-though there is
something to be said for declarations being as local as possible.
There are also advantages in \hx{grouping}{declaration!grouping}
all your global declarations so that they can be found easily.
Generally speaking, it is a good idea to group all \ix{global names}
together (those in the outermost range) and within that grouping, to
declare together all names which use the same base mode (for example,
group declarations of modes \verb|CHAR|, \verb|[]CHAR| and
\verb|STRING|).  Some of the exercises in this book only declare
names when they are immediately followed by related procedures. If
your program needs many \hx{global names}{name!global}, it makes
sense to declare them near the beginning of the program, after mode
declarations, so that if subsequent changes are required, you know
that all the global name declarations are together and therefore you
are unlikely to miss any.

\Subsection{Procedures}{dev-procs}
The next consideration is breaking your code into
\hx{procedures}{procedure}.
As you analyse the problem, you will find that some of the processing
can be specified in a single line which must be analysed further
before it can be directly coded.  Such a line is a good indication
that that process should be written as a procedure.  Even a procedure
which is used once only is worth writing if the internal logic is
more than a couple of conditional clauses, or more than one
conditional clause even.

You also have to decide between repeating a procedure in a loop, or
placing the loop in the procedure.  Deciding the level at which
\hx{logic}{logic level} should be put in a procedure is largely the
product of experience---yours and other people's---another reason for
maintaining existing programs.

When you have decided where to use procedures, you should then consider
the \hx{interface}{procedure!interface} between the procedure and the
code that calls it.  What parameters should it have, what yield, should
you use a united mode for the yield, and so on.  Try to have as few
parameters as possible, but preferably use parameters rather than
assign to names global to the procedure.  The design of individual
procedures is similar to the design of a complete program.

When you are coding a procedure, be especially careful with compound
Boolean formul\ae. From experience, this is where most mistakes
arise. If you are writing a procedure which manipulates a linked
list, draw a diagram of what you are trying to do.  That is much
easier than trying to picture the structures in your head.

\Subsection{Monetary values}{dev-money}
Problems can arise when dealing with \hx{money}{monetary values} in
computer programs because the value stored must be exact.  For this
reason, it is usually argued that only integers should be used.  In
fact, real numbers can be used provided that the precision of the
mantissa is not exceeded.  Real numbers are stored in two parts: the
\ix{mantissa}, which contains the significant digits of the value,
and the \ix{exponent}, which multiplies that value by a power of 2.
In other words, using decimal arithmetic, the number
$3\cdot14159\times10^{-43}$ has $3\cdot14159$ as a mantissa and $-43$
as an exponent.  Because real numbers are stored in \ix{binary}
(radix~2), the mantissa is stored as a value in the range
$1\le\textrm{value}<2$ with the exponent adjusted appropriately.

There are a number of identifiers declared in the \ix{standard
prelude}, known as
\hx{\textbf{environment enquiries}}{environment enquiry}, which serve
to determine the range and precision of real numbers.  The
\ixtt{real precision} is the number of bits used to store the
mantissa, while the value \ixtt{max exp real} is the maximum exponent
which can be stored for a binary mantissa ({\it not\/} the number of
bits, although it is a guide to that number).  The \ixtt{real width}
and \ixtt{exp width} say how many decimal digits can be written for
the mantissa and the exponent.  The values \ixtt{max real} and
\ixtt{min real} are the maximum and minimum real numbers which can be
stored in the computer.  All these values are specified by the IEEE
754--1985 standard on
\hx{``Binary Floating-Point Arithmetic''}{floating-point standard}
which is implemented by most microprocessors today.

The value of \verb|real width| is 15 meaning that 15 decimal digits can be
stored accurately. Leaving a margin of safety, we can say that an integer
with 14 digits can be stored accurately, so that the maximum amount is
$$99,999,999,999,999$$
units. If the unit of currency is divided into
smaller units, such as the sterling pound into pence, or the dollar into
cents, then the monetary value should be stored in the smaller unit unless
it is known that the smaller unit is not required. Thus the greatest
sterling amount that can be handled would appear to be
\pounds 999,999,999,999.99.

However, Algol~68 allows arithmetic values to be stored to a lesser
or greater precision.  The modes \ixtt{INT}, \ixtt{REAL},
\ixtt{COMPL} and \ixtt{BITS} can be preceded by any number of
\ixtt{SHORT}s or \ixtt{LONG}s (but not both).  Thus
\begin{verbatim}
   LONG LONG LONG REAL r;
\end{verbatim}
\noindent
is a valid declaration for a name which can refer to an exceptionally
precise real.  When declaring identifiers of other precisions,
denotations of the required precision can be obtained by using a cast
with the standard denotation of the value as in
\begin{verbatim}
   LONG REAL lr = LONG REAL(1);
\end{verbatim}
\noindent
One alternative is to use \verb|LONG| with the denotation:
\begin{verbatim}
   LONG REAL lr = LONG 1.0;
\end{verbatim}
\noindent
Another is to use the \ixtt{LENG} operator, which converts a value of
mode \verb|INT| or \verb|REAL| to a value of the next longer
precision, as in
\begin{verbatim}
   LONG REAL lr = LENG 1.0;
\end{verbatim}
\noindent
\ixtt{SHORTEN}  goes the other way.
\begin{verbatim}
   SHORT SHORT INT ssi = SHORTEN SHORTEN 3;
\end{verbatim}
\noindent
All the arithmetic operators are valid for all the \verb|LONG| and
\verb|SHORT| modes. Although you can write as many \verb|LONG|s or
\verb|SHORT|s as you like, any implementation of Algol~68 will
provide only a limited number. The number of different precisions
available is given by some identifiers in the \ix{standard prelude}
called \hx{environment enquiries}{environment enquiry}. They are
\begin{itemize}
\item \ixtt{int lengths}
\item \ixtt{int shorths}
\item \ixtt{real lengths}
\item \ixtt{real shorths}
\item \ixtt{bits lengths}
\item \ixtt{bits shorths}
\end{itemize}
The values for complex numbers are the same as those for reals.
For integers, where \verb|int lengths| is
greater than \verb|1|, \ixtt{long max int}
and so on are also declared, and similarly for
\ixtt{short max int}.  If
\verb|int lengths| is \verb|1|, then only the mode \verb|INT| is
available.

For the \hx{a68toc}{a68toc!int lengths@\texttt{int lengths}}
\hx{compiler}{a68toc!int shorths@\texttt{int shorths}}
\begin{verbatim}
   int lengths=2
   int shorths=3
\end{verbatim}
\noindent
Thus it is meaningful to write
\begin{verbatim}
   LONG INT long int:=long max int;
   INT int:=max int;
   SHORT INT sh int:=short max int;
   SHORT SHORT INT sh sh int:=
            short short max int;
\end{verbatim}
\noindent
The same applies to the mode \verb|BITS|. Try writing a program which
prints out the values of the environment enquiries mentioned in this
section. The transput procedures \verb|get|, \verb|put|, \verb|get|
\verb|bin| and \verb|put bin| all handle the available \verb|LONG|
and \verb|SHORT| modes.

Although you can still write
\begin{verbatim}
   LONG LONG INT lli=LONG LONG 3;
\end{verbatim}
\noindent
the actual value created may not differ from \verb|LONG INT| depending
on the value of \verb|int lengths|. Note that you cannot transput a
value which is not covered by the available lengths/shorths. Use
\verb|LENG| or \verb|SHORTEN| before trying to transput.

For monetary values, \verb|LONG INT| is available with the value of
\verb|long max int| being
\begin{verbatim}
   9,223,372,036,854,775,807
\end{verbatim}
\noindent
which should be big enough for most amounts.

\Subsection{Optimisation}{dev-opt}
There are two well-known rules about optimisation:
\begin{enumerate}
\item Don't do it.
\item Don't do it now.
\end{enumerate}
However, often there is a great temptation to optimise
\hx{code}{optimisation!code}, particularly if two procedures are
very similar.  Using identity declarations is a good form of
optimisation because not only do they save some writing, they also
lead to more efficient code.  However, you should avoid procedure
optimisation like the plague because it usually leads to more
complicated or obscure code.  A good indicator of bad optimisation is
the necessity of extra conditional clauses.  In general, optimisation
is never a primary consideration: you might save a few milliseconds
of computer time at the expense of a few hours of programmer time.

\Subsection{Testing and debugging}{dev-test}
\hx{When}{testing} \hx{writing}{debugging} a program, there
is a strong tendency to write hundreds of lines of code and then test
it all at once.  Resist it.  The actual writing of a program rarely
occupies more than 30\% of the whole development time.  If you write
your overall logic, test it and it works, you will progress much
faster than if you had written the whole program.  Once your overall
logic works, you can code constituent procedures, gradually refining
your test data (see below) so that you are sure your program works at
each stage.  By the time you complete the writing of your program,
most of it should already be working.  You can then test it
thoroughly.  The added advantage of \ix{step-wise testing} is that
you can be sure of exercising more of your code.  Your test data will
also be simpler.

The idea behind devising \hx{test data}{testing!data} is not just
giving your program correct data to see whether it will produce the
desired results.  Almost every program is designed to deal with
exception conditions.  For example, the \verb|lf| program has to be
able to cope with blank lines (usually, zero-length lines) so the
test data should contain not one blank line, but also two consecutive
blank lines.  It also has to be able to cope with extra-long lines,
so the test data should contain at least one of those.  Programs
which check input data for validity need to be tested extensively
with erroneous data.

It is particularly important that you test your programs with data
designed to exercise \ix{boundary conditions}.  For example, suppose
the creation of an output book fails due to a full hard disk. Have
you tested it, and does your program terminate sensibly with a
meaningful error message? You could try testing your program with the
output book being created on a floppy disk which is full.

Sometimes a program will fault with a
\hx{run-time error}{error!run-time} such as
\begin{verbatim}
   Run time fault (aborting):
   Subscript out of bounds
\end{verbatim}
\noindent
or errors associated with slicing or trimming multiples. A good way of
discovering what has gone wrong is to write a monitor procedure on the
lines of
\begin{verbatim}
   PROC monitor=(INT a,
                 []UNION(SIMPLOUT,
                         PROC(REF FILE)VOID)r
                 )VOID:
   BEGIN
      print(("*** ",whole(a,0)));
      print(r)
   END
\end{verbatim}
\noindent
and then call \verb|monitor| with an identifying number and string at
various points in the program. For example, if you think a multiple
subscript is suspect, you could write
\begin{verbatim}
   monitor(20,("Subscript=",whole(subscript,0)))
\end{verbatim}
\noindent
By placing \ix{monitors} at judicious points, you can follow
the action of your program. This can be particularly useful for a
program that loops unexpectedly: monitors will tell you what has gone
wrong. If you need to collect a large amount of monitors, it is best to
send the output to a book. The disadvantage of this is that the
operating system does not register a book as having a size until it has
been closed after creating. This means that if your program creates a
monitoring book, writes a large amount of data to it and fails before
the book is closed, you will not be able to read any of the contents
because, according to most operating systems, there will not be any
contents. A way round this problem is to open the book whenever you
want to write to it, position the writing position at the end of the
book, write your data to it and then close the book. This will ensure
that the book will have all the executed monitors (unless, of course,
it is a monitor which has caused the program to fail!). The procedure
\ixtt{debug} given in section~\hyref{trans-bin} will
do this.

An alternative method of tracing the action of a program at run-time
is to use a \ix{source-level debugger}. The \verb|DDD| program can
help you debug the C~source program produced by the
\hx{a68toc}{a68toc!debugger} compiler, but unless you understand the
C~programming language and the output of the a68toc compiler, you
will not find it useful.  Monitors, although an old-fashioned
solution to program debugging, are still the best means of gathering
data about program execution.

Another proven method of \bfix{debugging} (the process of removing
bugs) is \bfix{dry-running}.  This involves acting as though you are
the computer and executing a small portion of program accordingly. An
example will be given in the analysis of the \verb|lf| program later.

\hx{Sometimes}{debugging!ploys}, no matter what you do, it just seems
impossible to find out what has gone wrong.  There are three ploys you can
try.  The first, and easiest, is to imagine that you are explaining your
program to a friend.  The second is to actually explain it to a friend!
This finds most errors.  Finally, if all else fails, contact the
author.

\Subsection{Compilation errors}{dev-errors}
You can trust the compiler to find grammatical errors in your program
if any are there. The compiler will not display an error message for
some weird, but legal, construction. If your program is syntactically
correct (that is, it is legal according to the rules of the
language), then it will parse \hx{correctly}{error!compilation}.

When compiling a program of more than a hundred lines, say, you can use
the parsing option (\verb|-check|) which will more than double the
speed of compilation. When your program parses without error, then it
is worth doing a straight compilation (see the online documentation for
program \verb|mm| in the \hx{a68toc}{a68toc!mm@\texttt{mm}} compilation system).

A definitive list of error messages can be found in the file
\begin{verbatim}
   algol68toc-1.2/a68toc/a/message.a68
\end{verbatim}
\noindent
You will find that most of the messages are easy to understand.
Occasionally, you will get a message which seems to make no sense at
all. This is usually because the actual error occurs much earlier in
your program. By the time the compiler has discovered something
wrong, it may well have compiled (or tried to compile) several
hundred lines of code. A typical error of this sort is starting a
comment and not finishing it, especially if you start the comment
with an opening brace (\verb|{|), which gives rise to the following
error message:
\begin{verbatim}
   ERROR (112) end of file inside
      comment or pragmat
\end{verbatim}
\noindent
If you start a comment with a sharp (\verb|#|) and forget to finish it
likewise, the next time a sharp appears at the beginning of another
comment, the compiler will announce all sorts of weird
errors.\footnote{One way of avoiding this sort of error is to use
``lexical'' highlighting with your favourite editor. A missing quote
or sharp will cause large amounts of your program to be displayed as
a string denotation or a comment.}

Another kind of troublesome error is to insert an extra closing
parenthesis or \verb|END|. This can produce lots of spurious errors.
For example:
\begin{verbatim}
   ERROR (118) FI expected here
                  (at character 48)
   ERROR (203) ELSE not expected here
                  (at character 4)
   ERROR (140) BOOL, INT or UNION required here,
                  not VOID
   ERROR (116) brackets mismatch
                  (at character 2)
   ERROR (159) elements of in-parts
                  must be units
   ERROR (117) FINISH expected here
                  (at character 3)
\end{verbatim}
\noindent
Omitting a semicolon, or inadvertently inserting one will also cause
the appearance of curious error messages. Messages about
\ixtt{UNION}s usually mean that you should use a
cast to ensure that the compiler knows which mode you mean. If, for
example, you have a procedure which expects a multiple of mode
\begin{verbatim}
   []UNION(STRING,[]INT)
\end{verbatim}
\noindent
and you present a parameter like
\begin{verbatim}
   ((1,2),(4,2),(0,4))
\end{verbatim}
\noindent
then the compiler will not know whether the display is a
\hx{row-display}{row!display} or a
\hx{structure-display}{structure!display}.  Either you should precede
it with a suitable mode, or modify your procedure to take a single
\verb|[]INT| and loop through it in twos.  Having to modify your
program because the compiler does not like what you have written is
rare however.

\pagebreak
\Subsection{Arithmetic overflow}{dev-overflow}
Sometimes your program will fail at the time of elaboration or
``run-time'' due to \hx{arithmetic overflow}{overflow!arithmetic}. If,
during a calculation, an intermediate result exceeds the capacity of an
\verb|INT|, no indication will be given other than erroneous results.

Overflow of \verb|REAL| numbers can be detected by the floating-point
unit. The standard prelude contains the value \verb|fpu| \verb|cw|
\verb|algol| \verb|68| \verb|round| of mode \verb|SHORT BITS| and the
procedure
\begin{verbatim}
   PROC set fpu cw = (SHORT BITS cw)VOID:
\end{verbatim}
\noindent
The small test program \verb|testov| (to be found with the
\hx{a68toc}{a68toc!test program} compilation system documentation)
illustrates testing for overflow both with integers and real numbers.

\Subsection{Documentation}{dev-docs}
The most tedious aspect of writing a program is documenting it.  Even
if you describe what the program is going to do before you write it,
but after you have designed it, documentation is not usually a
vitally interesting task. Large programming teams often have the
services of a technical writer whose job it is to ensure that
\hx{all}{documentation}
\hx{program documentation}{program!documentation} is
completed.\footnote{Various schemes have been developed for
documenting a program as it is written, They are often called
``literate programming''.}

Existing programs are usually documented and there is no doubt that
the best way of learning to document a program is to see how others
have done it. There are several documentation standards in use,
although most large companies have their own. Generally speaking, the
documentation for a program should contain at least the following
\begin{itemize}
\item the program name
\item the language used to write the program
\item a short description of what the program does
\item the details of all books used by, or produced by, the program,
including the screen and the printer
\item an analysis of how the program works, particularly any special
algorithms or data structures (queues and trees are examples)
\item who wrote the program, and when
\item the location of the source code
\item the latest listing of the source code
\end{itemize}
but not necessarily in the order given above. The aim of program
documentation is to make it easy to amend the program, or to use it
for a subsequent rewrite.

Lastly, it is worthwhile saying ``don't be rigid in program design''.
If, as you reach the more detailed stages of designing your program,
you discover that you have made a mistake in the high-level design, be
willing to backtrack and revise it. Design faults are usually
attributable to faulty analysis of the problem.

\Section{Non-canonical input}{dev-noncanon}
The \verb|noncanon| program provides a means of entering data via the
keyboard without echoing it to the screen. This is known as
non-canonical input mode, the usual echoing of input being
\ix{canonical input mode}. The general details of terminal control
are very complex, but simple access has been provided with the
\verb|kbd channel|.

Here is a sample program which may be used to test the effect of
\verb|kbd channel|:
\begin{verbatim}
   PROGRAM noncanon CONTEXT VOID
   USE standard
   BEGIN
      STRING password;
      FILE kbd;  open(kbd,"",kbd channel);
      WHILE
         CHAR ch;  get bin(kbd,ch);
         ch /= REPR lf
      DO
         password+:=ch;
         print("*")
      OD;
      close(kbd);
      print(("You entered [",
             password,"]",
             newline))
   END
   FINISH
\end{verbatim}
\noindent
Notice that the program cannot be aborted by pressing \verb|^C|. Ensure
you close the \verb|FILE| opened with the \verb|kbd channel| after use
otherwise you'll find all your commands at the command prompt unechoed.
If that happens, issue the following command at the prompt:
\begin{verbatim}
   $ stty sane
\end{verbatim}
\noindent
when normal echoing will be restored.

\Section{A simple utility}{dev-lf}
When you are writing computer programs, it is very useful to be able
to copy your Algol~68 source programs to a printer with line numbers.
Many editors, including \verb|vim|, \verb|Emacs| and \verb|FTE|, use
line numbers. When the Algol~68 compiler finds an error in your
program, it displays the offending line together with its number and
a descriptive message on the screen and the number of the character
in the line where the error occurred. However, it is insufficient to
merely copy the contents of a file to the printer (unless you are
using the spooling facility of a header file) because the output will
not contain any identifying information.

What is required is a small program which will optionally write line
numbers and which will write the name of the file being printed
together with the date and time at which the file was last modified.
A page number is another useful item as it prevents pages being lost
when the listing is made on separate sheets of paper. It would also
be very useful to be able to specify where in a file a listing should
start and where it should finish. Such a program is called a
\bfix{utility}. Notice that the program must be able to handle
zero-length lines as well as lines which are too long to be printed
on one line alone. Lastly, some editors allow you to insert tab
characters into your document, so the utility must be able to print
the file with the correct indentation.

The preceding problem analysis shows that we could write such a
program if we knew how to obtain the date and time of last
modification of a file from the operating system. In the directory
\verb|/usr/share/doc/a68toc/pame|, you will find the source of the
program \verb|lf| which solves the problem described above for the
Linux operating system. The source of \verb|lf| is 520 lines long.
Compile it and run it with the \ix{argument} \verb|-h|. The help
information displayed by the program should be displayed by every
program you write which is used at the command line: it prevents
accidental use from causing damage to your operating system files or
directories.

\Subsection{The source code}{dev-code}
There are many ways of tackling the understanding of a program, but
here is a method which does help with Algol 68 programs. In summary,
\begin{enumerate}
\item See what the program does.
\item Look at the principal processing.
\item Examine the mode declarations.
\item Examine the routines.
\item Repeat steps 2--4 for each routine.
\end{enumerate}

Stage one of examining a program is to see what it does. Examples of
its output, and possibly its input, help you to identify the actions
of various parts of the program. Documentation of the input and
output would suffice, but neither exists in this case because the
input is a plain text file and the output is better seen than
described. Compile the Algol 68 example program \verb|lf| in
\begin{verbatim}
   /usr/share/doc/a68toc/examples
\end{verbatim}
\noindent
and use it to list the file \verb|test.lf| (in the same directory)
with line numbers on your printer using the command
\begin{verbatim}
   lf -pg -n test.lf | lpr
\end{verbatim}
\noindent
to pipe the output to the printer unless you have a LaserJet 4 or 6L
when you can omit the \verb|-pg| argument.  Notice that the time and
date the file was last modified appears at the top of each page,
together with the identifier of the file and the page number. If you
used the \verb|-n| parameter to print the test file, each line will
be preceded by a line number and a colon.  If you did not list the
file with line numbers, do so now because the line numbers will
highlight another feature of the program. The first line in
\verb|test.lf| is too long to be printed on one line, so the program
breaks it into two parts.  The second part does not have a line
number since it is part of the same line in the input.

The second stage in understanding a program is to look at the
principal processing. Since procedures and other values must be
declared before use in the \hx{a68toc}{a68toc!declarations} compiler,
the last part of the program contains the \ix{main processing logic}.
Now print (or display) the source of \verb|lf.a68| using the command
\begin{verbatim}
   lf -n /usr/share/doc/a68toc/pame/lf.a68
\end{verbatim}
\noindent
In the source, the main processing logic is on lines 427--517.
Examine those lines now.

Before processing any command line arguments, the program defines the
actions to take when the last argument has been read. In other words,
what should be done when the logical end of file has been reached for
\verb|comm line|. The default action is to terminate the program
immediately with a suitable error message. In \verb|lf|, no
identification is given for \verb|comm line| in the \verb|open|
procedure, because it isn't relevant, but if you insert such an
identification, for example, \verb|command line file|, then any error
message issued by the transput system will include it. Notice that
although the anonymous procedure used as the second parameter for
\verb|on logical file end| on line 448 occurs within the
\verb|IF ... FI| clause, because it is a denotation (a procedure
denotation) it has global scope. That is one of the reasons why
anonymous procedures are so useful. Also note the use of \ixtt{SKIP} to
yield a value of mode \verb|BOOL|: in fact, it will never be used
because \ixtt{stop} is a synonym for \verb|GOTO end of program|.

In lines 442--517, the program processes the command line argument by
argument. If an argument starts with ``-'' it is assumed to be an
option otherwise it is assumed to be a filename. Note the use of
\verb|skip terminators| to skip spaces in the command line.  Options
that require a number (\verb|-s| and \verb|-t|) expect it to follow
the option directly (see lines 493 and 495). Lines 500--506 process a
solitary \verb|-| to mean ``list the standard input''. Lines 507--516
process a named file. As you examine the code, underline the
identifiers of all procedure calls.

The next stage in understanding a program is to look at all the
\ix{mode declarations}.  There are three in this program:
\verb|PRINTER|, \verb|SEC| and \verb|STAT|.  You should scan the
program to see what identifiers have that or a related mode and where
they are used.

\Subsection{Routines}{dev-rout}
Finally, you need to examine the routines declared. It is a good
idea, especially in a more complicated program, to list the
identifiers of all procedures with nested declarations of procedures
indented under their parent procedure identifiers. This helps to fix
the structure of the program in your mind. Then you should examine
the procedures used in the main processing loop.  In \verb|lf|, they
are:
\begin{verbatim}
   char in string       close
   disp error           get
   get mtime            get numeric arg
   get sections         help
   open                 print
   print file           process file name
   reset parameters     skip terminators
\end{verbatim}
\noindent
When you examine each procedure, do the same as you did for the whole
program: first the main logic, then the modes, then the procedures
and operators. You will need to backtrack several times in a large
program. If a lot of names are declared, prepare a list together with
a description of what each name is used for, where it is declared and
the places where it is used. A cross-reference program would be
really useful, but it is not a simple program to write for Algol~68.

The principle processing is performed by the procedure \verb|print|
\verb|file| on lines 258--322. Firstly, tab stops are set according
to the current value of \verb|tabs|, then \verb|lines| is initialised
and an initialisation string output to the printer. If letter quality
has been chosen (option \verb|-q|), a special string is sent to the
printer accordingly. Then the \verb|logical| \verb|file| \verb|end|
event procedure is set. Each section specified on the command line
(or the default section if no sections were specified) is then
printed using the procedure \verb|do line|. Each line is input using
\verb|get line| whose principal function is to expand tab characters
to the required number of spaces (3 unless set by the \verb|-t|
option). Lines are not output until the \verb|beg OF ss| line is
reached (1 unless set by the \verb|-s| option). Notice the code
following \verb|FROM| in the preamble to the inner \verb|DO ... OD|
loop (on lines 313--316) which ensures that the file is reset if the
sections to be printed are not ordered (the definition of ordered is
in the procedure \verb|get sections| (lines 381--425).

Similar to your list of nested procedures, prepare a list of
procedures where indented procedures identify procedures called by
the parent procedure. Here is part of the list for \verb|lf|:
\goodbreak
\begin{verbatim}
   fstat
      linux fstat
   help
      exit, newline, put
   reset parameters
   lf print
      ODD, print
   get mtime
      fstat, linux ctime
   get sections
      +:=
      add section
         char in string
         get numeric arg
      char in string
\end{verbatim}

\Subsection{Dry-running example}{dev-dry}
The procedure \verb|get line| (lines 232--250) and its associated
procedures \verb|set tabs| (lines 220--224) and \verb|tab pos| (lines
226--227) are worth examining in detail.  The best way to see how
they work is to dry-run them.  Take a blank sheet of paper and make a
vertical list of all the names, both local and global,
\hx{used}{dry-running} by the procedures. Opposite \verb|in line|,
write a piece of text containing tab characters (a piece of indented
program, for example).  Then work your way through the procedure,
marking the value referenced by each name as you complete each step.
You should also note the value of each non-name; for example, the
loop identifier \verb|i|.  Here is what your list could look like
after going 3 times round the outer loop (the inner loop is on lines
241--244):
\smallskip
\begin{tabbing}
\ttfamily
\quad\=tabstops\quad\=FFTFFTFFTFFTFFTFFTFFTFFT$\ldots$\\
\>line(ln)\>\verb*|    |T\\
\>in line\>$\rightarrow$\verb*| THEN ch:="A"|\\
\>op\>$\not1\not2\not3\not4\not5$ 6\\
\>i\>$\not1\not2$ 3\\
\>c\>$\not\rightarrow\not\hbox{\verb*| |}$ T
\end{tabbing}
\smallskip
\noindent Struck-out values have been superceded and \verb*| |
denotes a space. Dry-running is a very useful method, if laborious
and time-consuming, of finding bugs. \ixtt{tab ch} is declared in the
standard prelude.

This utility program (\verb|lf|) is quite short, but we have analysed
its working in detail so that you can see how it is done.

\Subsection{\texttt{ALIEN} procedures}{dev-ALIEN}
The utility \verb|lf| uses some of the extensions provided by the
\hx{a68toc}{a68toc!ALIEN@\texttt{ALIEN}} compiler, in particular, the
\verb|ALIEN| construct which provides access to procedures compiled
by other compilers. In this section we shall look at the
\verb|get cwd| and the \verb|fstat| procedures.

\subsubsection{The procedure \texttt{fstat}}\hylabel{dev-fstat}
The procedure \verb|fstat| is on lines 100--105. It depends on a call
of the \verb|linux fstat| procedure whose second parameter is a name
referring to a value of mode \verb|STAT|. The declaration of
\verb|STAT| is on lines 24--41.

If you investigate the file \verb|/usr/include/statbuf.h|, you will
find the C~definition of the \verb|stat| structure therein. The
\verb|STAT| mode accurately reflects this structure using \verb|LONG|
or \verb|SHORT| as appropriate. Briefly, a C~\verb|unsigned int| is
equivalent to an Algol 68 \verb|BITS|. For historical reasons, the~C
\verb|unsigned long int| has the same meaning as an
\verb|unsigned int| so \verb|BITS| could have been used for those
fields as well.  However, because the value is required as an integer
(and is stored as a positive integer), it is possible to regard them
as having mode \verb|INT|. Some of the C~modes\footnote{C people call
them \texttt{types}.} are hidden by further mode
declarations\footnote{\texttt{typedef}s}, but if you hunt for
\verb|__dev_t| you will find it is an \verb|unsigned long long int|
which is equivalent to the Algol 68 \verb|LONG BITS| or, as is used
in \verb|STAT|, \verb|LONG INT|.

Now look at the declaration of \verb|linux fstat| on lines 85--89.
Most of this construction is C~source code. The \verb|ALIEN|
construct may be written as
\begin{verbatim}
   <mode> <identifier> = ALIEN "<symbol>"
      "<C source code>";
\end{verbatim}
\noindent
where the angle brackets denote items to be replaced. In the
declaration for \verb|linux fstat| we have
\begin{itemize}
\item \verb|<mode> = PROC(INT,REF STAT)INT|
\item \verb|<identifier> = linux fstat|
\item \verb|<symbol> = FSTAT|
\end{itemize}
followed by three lines of C~source code. It is not my intention to
delve into the mysteries of C. If you don't understand that language,
consult someone who does. However, the point of the declaration is to
map the Algol 68 modes onto the C~equivalents. The C~procedure
\verb|fstat| takes two parameters: the first has mode \verb|int|
(equivalent to \verb|INT|) and the second of mode \verb|struct stat*|
which is equivalent to \verb|REF STAT|. The cast in C~consists of a
mode in parentheses (compare with the Algol 68 cast in
section~\hyref{gram-prim}) so the third line of C~code ensures that
the second parameter of the Algol 68 procedure \verb|linux fstat| has
the right mode. The \verb|A_int_INT(...)| construct is a C~language
macro\footnote{A synonym for another piece of text which is expanded
by the C~preprocessor} for a cast which ensures that the yielded
C~integer is equivalent to the Algol 68 \verb|INT|.  If you want to
see what the \hx{a68toc}{a68toc!FSTAT@\texttt{FSTAT}} compiler
generates, look for \verb|FSTAT| in the file \verb|lf.c|.

Reverting to line 102, the field \verb|sys file OF f| has the correct
mode for use as the ``file descriptor'' for \verb|fstat|. You should
check the manual page of \verb|fstat| (in section 2 of the Linux
Programming Manual) for details of its functioning and yield.

\subsubsection{The procedure \texttt{get cwd}}\hylabel{dev-getcwd}
The procedure \verb|get cwd| is more complicated because it uses
several facilities provided by the standard prelude as well as
another extension provided by the
\hx{a68toc}{a68toc!VECTOR@\texttt{VECTOR}} compiler. Firstly, look at
the \verb|ALIEN| declaration of \verb|linux getcwd| on lines 91--93.
The mode \verb|VECTOR[]CHAR| is similar to the mode \verb|[]CHAR|,
but the lower bound is always 1 and is omitted from the generated
construct. In fact, a68toc translates this mode into the C~equivalent
of
\begin{verbatim}
   STRUCT(REF CHAR data, INT gc, upb)
\end{verbatim}
\noindent
The \verb|gc| field is an integer provided for the garbage-collector
(the run-time memory management system which looks after the heap).
The \verb|data| field is a reference to the actual data (in fact it
is a memory address)\footnote{The \texttt{VECTOR} mode is not limited
to \texttt{CHAR}. You can use it for any mode. See section
\protect\hyref{stan-ctmodes} for details}.

The C~procedure \verb|getcwd| requires two parameters: a reference to
an area which it can use to return the full path of the current working
directory and an integer which states how big that area is. The
C~source code in the declaration for \verb|linux getcwd| contains the
\ix{C macro}
\begin{verbatim}
   A_VC_charptr(buf)
\end{verbatim}
\noindent
which expands into \verb|buf.data| (equivalent to the Algol~68
expression \verb|data OF buf|) and the C~macro \verb|A_INT_int| which
converts an Algol~68 \verb|INT| into a C~\verb|int| (directly
equivalent on Linux).

The yield of \verb|linux getcwd| is a reference to the area in which the
current working directory path has been put. Strictly speaking, this is
identical to the first parameter of the C~procedure \verb|getcwd|, but
the GNU C~compiler complains if it is used as such. To get around this,
the author used the cast \verb|(void *)| which effectively causes the
reference to be a reference to an anonymous piece of memory. The
Algol~68 equivalent is \verb|CPTR| which is defined in the standard
prelude as \verb|REF BITS|.

Now comes the clever bit. Look at line 98. The value of mode
\verb|CPTR| (\verb|REF BITS|) is converted by the operator
\verb|CPTRTOCSTR| into a value of mode \verb|CSTR| (declared in the
standard prelude as \verb|REF STRUCT 16000000 CHAR|). Now look at the
definition of that operator (on line 95)! \verb|BIOP| stands for
``built-in operator'' and \verb|BIOP 99| is the only built-in
operator implemented by the a68toc translator. \verb|BIOP 99| maps
its parameter (of one mode) onto its yield (of another mode). It
effectively acts as a cast (in this case) from one \verb|REF| mode to
another \verb|REF| mode. Have a look at the C~source code in
\verb|lf.c| if you are interested in the details. Then the value of
mode \verb|CSTR| is converted using the operator \verb|CSTRTORVC| to
a value of mode \verb|REF VECTOR[]CHAR| which is dereferenced and
then coerced to a value of mode \verb|STRING|. In fact, the a68toc
compiler will silently coerce values of mode \verb|REF STRUCT i MODE|
to mode \verb|REF| \verb|VECTOR[]MODE| and thence to
\verb|REF[]MODE|.  Notice that you cannot coerce a value of mode
\verb|REF| \verb|VECTOR[]MODE| to \verb|REF FLEX[]MODE|. The mode
\verb|STRING| has no flexibility (it is equivalent to \verb|[]CHAR|).

Lastly, note that the parameter of \verb|linux getcwd| is an
\hx{anonymous}{generator!anonymous} \verb|VECTOR[]CHAR| whose scope
is limited to the scope of \verb|get cwd| (the Algol~68 procedure).

If you want to examine the other macros used for the translated
C~source, have a look at the files in the directories
\begin{verbatim}
   /usr/share/a68toc/Linux
   /usr/share/a68toc/include
\end{verbatim}

\Section{Summary}{dev-summ}
In this chapter, we have covered most of the activities relating to
program development, whether it be the maintenance of existing
programs or the development of new programs. The constructor
\verb|ALIEN| is used to introduce procedures compiled by other
compilation systems (such as C). We have described one program
and have shown how to analyse the workings of a program.
