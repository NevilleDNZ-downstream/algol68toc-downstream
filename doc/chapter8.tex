% $Log: chapter8.tex,v $
% Revision 1.3.2.1  2004/09/04 17:36:11  teshields
% LCSE Algol68toC initial revision
% Default installation directory is '/usr/local/'
%
% Revision 1.3  2004/09/04 02:42:17  teshields
% Algol68toC v1.8 Public Baseline
%
% Revision 1.3  2002/06/20 11:49:28  sian
% mm removed, padding removed, ca68 added
%
\Chapter{Unions}{chapviii}
From time to time, you have been using the procedure \ixtt{print} to
display values on your screen. You must have noticed that it seems to
be able to take a large variety of values of different modes and that
it can process more than one value in one call. You may therefore be
wondering how the parameter of \verb|print| is specified. It cannot
be a structure because a structure has a fixed number of fields, but
if it is a row, how can a row have different modes for its elements?
Although the elements of a row must each have the same mode, the
explanation is that \verb|print| takes one parameter which is a row
of a \textbf{united} \hx{mode}{mode!united}.

This very short chapter introduces the final mode constructor available
in Algol~68. It shows the principles behind the construction and use of
united modes. It does not and cannot show all the possible usages.

\Section{United mode declarations}{union-mode}
\ixtt{UNION} is used to create a united mode. Here is a declaration
for a simple united mode:
\begin{verbatim}
   UNION(INT,STRING) u = (random < .5|4|"abc")
\end{verbatim}
\noindent
The first thing to notice is that, unlike structures, there are no
field selectors.  This is because a united mode does not consist of
constituent parts. The second thing to notice is that the above mode
could well have been written
\begin{verbatim}
   UNION(STRING,INT) u = (random < .5|4|"abc")
\end{verbatim}
\noindent
The order of the \hx{modes}{order of modes} in the union is
irrelevant.\footnote{Unfortunately, for the
\protect\hx{a68toc}{a68toc!UNION@\texttt{UNION}} compiler, this is
not true.} What is important is the actual modes present in the
union. Moreover, a constituent mode can be repeated, as in
\begin{verbatim}
   UNION(STRING,INT,STRING,INT) u =
      (random < .5|4|"abc")
\end{verbatim}
\noindent
This is equivalent to the previous declarations.\footnote{But not
for the a68toc compiler.}

Like structured modes, united modes are often declared with the mode
\hx{declaration}{declaration!mode}. Here is a suitable declaration of
a united mode containing the constituent modes \verb|STRING| and
\verb|INT|:
\begin{verbatim}
   MODE STRINT = UNION(STRING,INT)
\end{verbatim}
\noindent
We could create another mode \verb|STRINTR| in two ways:
\begin{verbatim}
   MODE STRINTR = UNION(STRINT,REAL)
\end{verbatim}
\noindent
or
\begin{verbatim}
   MODE STRINTR = UNION(STRING,INT,REAL)
\end{verbatim}
\noindent
Using the above declaration for \verb|STRINT|, we could declare
\verb|u| by
\begin{verbatim}
   STRINT u = (random < .5|4|"abc")
\end{verbatim}
\noindent
In this identity declaration, the mode yielded by the right-hand side
is either \verb|INT| or \verb|STRING|, but the mode required is
\verb|UNION(STRING,INT|).  The value on the right-hand side is
coerced to the required mode by \hx{\textbf{uniting}}{coercion!uniting}.

The united mode \verb|STRINT| is a mode whose values either have mode
\verb|INT| or mode \verb|STRING|.  It was stated in chapter~1 that
the number of values in the set of values defined by a mode can be
zero.  Any value of a united mode actually has a mode which is one of
the constituent modes of the union. So there are no \textbf{new}
values for a united mode.  \verb|u| identifies a value which is
either an \verb|INT| or a \verb|STRING|.  Because \ixtt{random}
yields a pseudo-random number, it is not possible to determine when
the program is compiled (that is, at  \textbf{compile-time}) which
mode the conditional clause yields.  As a result, all we can say is
that the underlying mode of \verb|u| is either \verb|INT| or
\verb|STRING|.  We shall see later how to determine that underlying
mode.\footnote{Note that an Algol~68 union is quite different from a
C~union. The latter is simply a remapping of a piece of memory. In an
Algol~68 union, where the underlying value is kept is the business of
the compiler and it cannot be remapped by the programmer.}

Because a united mode does not introduce new values, there are no
denotations for united modes, although denotations may well exist for
the constituent modes.

Almost any mode can be a constituent of a united mode.  For example,
here is a united mode containing a procedure mode and \verb|VOID|:
\begin{verbatim}
   MODE PROID = UNION(PROC(REAL)REAL,VOID)
\end{verbatim}
\noindent
and here is a declaration using it:
\begin{verbatim}
   PROID pd = EMPTY
\end{verbatim}
\noindent
The only limitation on united modes is that none of the constituent
modes may be \ix{firmly related} (see the section \hyref{rout-idops})
and a united mode cannot appear in its own declaration. The following
declaration is wrong because a value of one of the constituent modes
can be deprocedured in a firm \hx{context}{context!firm} to yield a
value of the united mode:
\begin{verbatim}
   MODE WRONG = UNION(PROC WRONG,INT)
\end{verbatim}

Names for values with united modes are declared in exactly the same way
as before. Here is a declaration for such a name using a local
\hx{generator}{generator!local}:
\begin{verbatim}
   REF UNION(BOOL,INT) un = LOC UNION(BOOL,INT);
\end{verbatim}
\noindent
The \hx{abbreviated declaration}{declaration!abbreviated} gives
\begin{verbatim}
   UNION(BOOL,INT) un;
\end{verbatim}
\noindent
Likewise, we could declare a name for the mode \verb|STRINT|:
\begin{verbatim}
   STRINT sn;
\end{verbatim}
\noindent
In other words, objects of united modes can be declared in the same way
as other objects.

\begin{exercise}
\item Write a mode declaration for the united mode \verb|BINT| whose
constituent modes are \verb|BOOL| and \verb|INT|.
\ans \verb|MODE BINT = UNION(BOOL,INT)|
'
\item Write an identity declaration for a value of mode \verb|BINT|.
\ans \verb|BINT b = TRUE|
'
\item What is wrong with the mode declaration
\begin{verbatim}
   MODE UB = UNION(REF UB,INT,BOOL)
\end{verbatim}
\indent \ans One of the constituent modes of the union is
firmly-related to the united mode. In other words, in a firm context,
\verb|REF UB| can be dereferenced to \verb|UB|.
'
\item Declare a name for a mode united from \verb|INT|, \verb|[]INT| and
\verb|[,]INT|. \ans \verb|UNION(INT,[]INT,[,]INT) mint|
'
\end{exercise}

\Section{United modes in procedures}{union-proc}
We can now partly address the problem of the parameters for
\ixtt{print} and \ixtt{read}.  If we extend the answer to the last
exercise, we should be able to construct a united mode which will
accept all the modes accepted by those two procedures.  In fact, the
united modes used are almost the same as the two following
declarations:
\begin{verbatim}
   MODE SIMPLOUT = UNION(CHAR, []CHAR,
                         INT,  []INT,
                         REAL, []REAL,
                         COMPL,[]COMPL,
                         BOOL, []BOOL,
                        ),

        SIMPLIN  = UNION(REF CHAR, REF[]CHAR,
                         REF INT,  REF[]INT,
                         REF REAL, REF[]REAL,
                         REF COMPL,REF[]COMPL,
                         REF BOOL, REF[]BOOL,
                        );
\end{verbatim}
\noindent
As you can see, the mode \ixtt{SIMPLIN} used for \verb|read| is
united from modes of names.

The modes \ixtt{SIMPLOUT} and \verb|SIMPLIN| are
a little more complicated than this because they include modes we have
not yet met (see chapters~9 and~11), but you now have the basic idea.

The uniting coercion is available in a
\hx{firm context}{context!firm}.  This means that operators which
accept operands with united modes will also accept operands whose
modes are any of the constituent modes.  We shall return to this in
the next section.

Here is an example of the uniting coercion in a call of the procedure
\verb|print|.  If \verb|a| has mode \verb|REF INT|, \verb|b| has mode
\verb|[]CHAR| and \verb|c| has mode \verb|PROC REAL|, then the call
\begin{verbatim}
   print((a,b,c))
\end{verbatim}
\noindent
causes the following \hx{to}{coercion!deproceduring}
\hx{happen}{coercion!dereferencing}:
\begin{enumerate}
\item \verb|a| is dereferenced to mode \verb|INT| and then united to
mode \verb|SIMPLOUT|.
\item \verb|b| is united to mode \verb|SIMPLOUT|.
\item \verb|c| is deprocedured to produce a value of mode \verb|REAL|
and then united to mode \verb|SIMPLOUT|.
\item The three elements are regarded as a row-display for a
\verb|[]SIMPLOUT|.
\item \verb|print| is called with its single parameter.
\end{enumerate}
\verb|print| uses a \hx{\textbf{conformity clause}}{clause!conformity}
(see next section) to extract the actual value from each element in
the row.

In section \hyref{rout-chinstr}, we gave the declaration of a
procedure identified as \verb|char in string|.  The header of that
procedure was
\begin{verbatim}
   PROC char in string=
      (CHAR ch,REF INT pos,[]CHAR s)BOOL:
\end{verbatim}
\noindent
The procedure yielded \verb|TRUE| if \verb|ch| was present in \verb|s|,
in which case \verb|pos| contained the position. Otherwise, the
procedure yielded \verb|FALSE|. The same procedure could be written to
yield the position of \verb|ch| in \verb|s| if it is present, and
\ixtt{VOID} if not:
\begin{verbatim}
   PROC ucis = (CHAR ch,[]CHAR s)
                    UNION(INT,VOID):
\end{verbatim}
\noindent
The body of the procedure has been left as an exercise.

\begin{exercise}
\item A procedure has the header
\begin{verbatim}
   PROC pu = ([]UNION(CHAR,[]CHAR) up)VOID:
\end{verbatim}
\noindent
Explain what happens to the parameters if it is called by the phrase
\begin{verbatim}
   pu((CHAR: REPR(ABS"a"+1),LOC[4]CHAR))
\end{verbatim}
\indent \ans The first parameter is deprocedured to mode \verb|CHAR|
before being united.  The second is dereferenced to mode
\verb|[]CHAR| and then united.  The two values of the united mode are
regarded as a row-display and the procedure is then called.  The
second parameter is an example of an anonymous name---no identifier
is attached.
'
\item Write the body of the procedure \verb|ucis|
given in the text.  \ans \ %
\begin{verbatim}
   PROC ucis=(CHAR ch,[]CHAR s)
                  UNION(INT,VOID):

   IF   INT p = ch FIND s;  p >= LWB s
   THEN p
   ELSE EMPTY
   FI
\end{verbatim}
'
\end{exercise}

\Section{Conformity clauses}{union-conf}
In the last section, we discussed the consequences of the
\hx{uniting}{coercion!uniting} coercion; that is, how values of
various modes can be united to values of united modes.  This raises
the question of how a value of a united mode can be extracted since
its constituent \hx{mode}{mode!constituent} cannot be determined at
compile-time.  There is no de-uniting coercion in Algol~68.  The
constituent mode or the value, or both, can be determined using a
variant of the case clause discussed in chapter~4
(see section~\hyref{choice-case}). It is called a
\hx{\textbf{conformity clause}}{clause!conformity}. For our
discussion, we shall use the declaration of \verb|u| in
section~\hyref{union-mode} (\verb|u| has mode \verb|STRINT|).

The constituent mode of \verb|u| can be determined by the following:
\begin{verbatim}
   CASE u IN
      (INT):    print("u is an integer")
      ,
      (STRING): print("u is a string")
   ESAC
\end{verbatim}
\noindent
If the constituent mode of \verb|u| is \verb|INT|, the first case
will be selected. Notice that the mode \hx{selector}{mode!selector}
is enclosed in parentheses and followed by a colon.  Otherwise, the
conformity clause is just like the case clause (in fact, it is
sometimes called a \ix{conformity case clause}).  This particular
example could also have been written
\begin{verbatim}
   CASE u
   IN
      (STRING): print("u is a string")
   OUT
      print("u is an integer")
   ESAC
\end{verbatim}
\noindent
This is the only circumstance when a \verb|CASE| clause can have one
choice. Usually, however, we want to extract the value. A slight
modification is required:
\begin{verbatim}
   CASE u IN
      (INT i):
         print(("u identifies the value",i))
      ,
      (STRING s):
         print(("u identifies the value ",s))
   ESAC
\end{verbatim}
\noindent
In this example, the mode selector and \ix{identifier} act as the
left-hand side of an
\hx{identity declaration}{declaration!identity!CASE@\texttt{CASE}}.
The identifier can be used in the following \ix{unit} (or
\hx{enclosed clause}{clause!enclosed}).

The two kinds of conformity clause can be mixed. For example, here is
one way of using the procedure \verb|ucis|:
\begin{verbatim}
   CASE ucis(c,s) IN
      (VOID):
         print("The character was not found"),
      (INT p):
         print(("The position was",p))
   ESAC
\end{verbatim}

We mentioned in the last section that operators with united-mode
operands can be declared. Here is one such:
\begin{verbatim}
   MODE IC = UNION(INT,CHAR);
   OP * = (IC a,b)IC:
   CASE a IN
      (INT ai):
         (b|(INT bi): ai*bi,
            (CHAR bc): ai*bc),
      (CHAR ac):
         (b|(INT bi): ac*bi,
            (CHAR bc): ABS ac*ABS bc)
   ESAC
\end{verbatim}
\noindent
In each of the four cases, the resulting product is united to the mode
\verb|IC|.

You can have more than one mode in a particular case. For example:
\begin{verbatim}
   MODE ICS = UNION(INT,CHAR,STRING);
   OP * = (ICS a,INT b)ICS:
   CASE a
   IN
      (UNION(STRING,CHAR) ic):
         (ic|(CHAR c): c*b,(STRING s): s*b),
      (INT n): n*b
   ESAC
\end{verbatim}
\noindent
Note that conformity clauses do not usually have an \verb|OUT| clause
because the only way of extracting a value is by using the
\verb|(MODE id):| construction. However, they do have their uses. See
the standard prelude for more examples of conformity clauses.

Although \ixtt{read} and \ixtt{print} use united modes in their call,
you cannot read a value of a united mode or print a value of a united
mode (remember that united modes do not introduce new values). You
have to read a value of a constituent \hx{mode}{mode!constituent}
and then unite it, or extract the value of a constituent mode and
print it.

\begin{exercise}
\item The modes
\begin{verbatim}
   MODE IRC = UNION(INT,REAL,COMPL),
        MIRC= UNION([]INT,[]REAL,[]COMPL)
\end{verbatim}
\noindent
are used in this and the following exercises.

Write a procedure which takes a single parameter of mode \verb|MIRC|
and which yields the sum of all the elements of its parameter as a
value with mode \verb|IRC|.
\ans \ %
\begin{verbatim}
   PROC p = (MIRC m)IRC:
   CASE m IN
      ([]INT i): (INT sum:=0;
                  FOR j FROM LWB i TO UPB i
                  DO sum+:=i[j] OD;
                  sum),
      ([]REAL r):(REAL sum:=0;
                  FOR j FROM LWB r TO UPB r
                  DO sum+:=r[j] OD;
                  sum),
      ([]COMPL c):(COMPL sum:=0;
                  FOR j FROM LWB c TO UPB c
                  DO sum+:=c[j] OD;
                  sum)
   ESAC
\end{verbatim}
'
\item Write the body of the operator \verb|*| whose header is
declared as
\begin{verbatim}
   OP * = (IRC a,b)IRC:
\end{verbatim}
\noindent
Use nested conformity clauses. Remember that there are 9 separate
cases. \ans \ %
\begin{verbatim}
   OP * = (IRC a,b)IRC:
   CASE a IN
      (INT i):  CASE b IN
                   (INT j):   i*j,
                   (REAL j):  i*j,
                   (COMPL j): i*j
                ESAC,
      (REAL i): CASE b IN
                   (INT j):   i*j,
                   (REAL j):  i*j,
                   (COMPL j): i*j
                ESAC,
      (COMPL i):CASE b IN
                   (INT j):   i*j,
                   (REAL j):  i*j,
                   (COMPL j): i*j
                ESAC
   ESAC
\end{verbatim}
'
\end{exercise}

\Section{Summary}{union-summ}
United modes introduce no new values. A united mode can have any mode
as one of its constituents except a mode which can be firmly coerced to
itself. The uniting coercion is available in firm contexts. Because the
values supplied to \verb|print| or \verb|read| are united, the context
of the parameter of those procedures is firm. A conformity clause is
used to extract the constituent mode or value. The mode \verb|VOID| can
be one of the constituents of a united mode and is useful to signal an
exceptional yield from a procedure. United modes are used in a variety
of ways.

\begin{exercise}
\item Write a declaration for the united mode \verb|CRIB| whose constituent
modes are \verb|CHAR|, \verb|REAL|, \verb|INT| and \verb|BOOL|.
\ans \verb|MODE CRIB = UNION(CHAR,REAL,INT,BOOL)|
'
\item Write a declaration for the operator \verb|UABS| which has a single
operand of mode \verb|CRIB| and which yields the absolute value of its
operand. \ans \ %
\begin{verbatim}
   OP UABS = (CRIB c)UNION(INT,REAL):
   CASE c IN
      (CHAR a):   ABS a,
      (REAL a):   ABS a,
      (INT a):    ABS a,
      (BOOL a):   ABS a
   ESAC
\end{verbatim}
'
\item Write four formul{\ae} which use \verb|UABS| and a denotation for each
of the constituent modes of \verb|CRIB|.
\ans \verb|UABS "c"; UABS -4.0; UABS -3; UABS TRUE|
'
\end{exercise}
