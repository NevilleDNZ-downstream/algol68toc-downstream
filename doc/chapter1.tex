% $Log: chapter1.tex,v $
% Revision 1.4  2004/09/04 02:37:16  teshields
% Algol68toC v1.8 Public Baseline
%
% Revision 1.4  2003/04/23 08:57:31  sian
% Debian release 1.5
%
% Revision 1.3  2002/06/20 11:49:28  sian
% mm removed, padding removed, ca68 added
%
\Chapter{Introduction}{chapi}
Algol 68 is a high-level, general-purpose programming language ideally
suited to modern operating systems.  This book will teach you Algol 68
plus the necessary development skills to enable you to write
substantial programs which can be executed from the command line.

In principle, you can solve any computable problem with Algol~68.
You can write programs which perform word processing, perform
complicated calculations with matrices, design graphs or bridges,
process pictures, predict the weather, and so on.  Or you can write
simple programs which count the number of words in a file or list a
file with line numbers.

Algol 68 is a powerful language.  There are many constructs which
enable you to manipulate complicated data structures with ease, and
yet it is all easy to understand because one of the guiding
principles of Algol~68 was that it was designed to be
\hx{\textbf{orthogonal}}{orthogonality}. This means that the
language is based on a few independent ideas which are developed and
applied with generality.  The language was designed in such a way
that it is impossible to write ambiguous programs.  The design is
also difficult to describe until it has been fully described, which
means that some concepts have to be introduced in a superficial
manner, but later reading will deepen your understanding.

You need to have a thorough grasp of the basic ideas if you are going
to write powerful programs in Algol 68: these ideas unfold in the
first five chapters.  The chapters should be read in order, but
chapter~5 is a watershed---it forms the basis of much of the computer
programming performed in the world today.  Its ideas should be
mastered before continuing.

Chapter 10 draws together all the various references to grammatical
points and clarifies the limitations of the language---you will need
to know these if you want to squeeze the last ounce of power out of
the language. Chapters~11 and~12 deal with advanced topics which
should not be touched until you have mastered preceding material.
Chapter~13 describes the standard prelude which, besides providing
means of determining the characteristics of an Algol 68
implementation, also provides the transput facilities whose power are
characteristic of Algol 68.

In this chapter, some aspects of Algol 68 grammar are described.
Don't worry if they seem confusing; all will become clear later in
the book.  It also covers denotations and the identity declaration,
the latter having crucial importance in the language.

\Section{What you will need}{intro-need}
The language described in this book is that made available by the
\hx{a68toc}{a68toc!requirements} Algol~68 compiler developed by the
\emph{Defence Research Agency} (see section~\hyref{intro-co} for more
information about what a compiler does).  It implements almost all of
the language known as Algol~68, and extends that language in minor
respects.  To run the programs described in this book you will need a
microcomputer with a Linux system. The source package will occupy
$\approx12$Mb on the hard disk while the binary package will also need
$\approx9$Mb of space on the hard disk. The source package may be
deleted once the binary package has been installed.

The book expects you to be familiar with the usual commands for
manipulating files. You will need to know how to use an editor for
plain text files (not a word processor).  No programming expertise is
presumed.

Much program development work on Linux takes place at the \ix{command
line} because a graphical user interface is usually too cumbersome to
cope with the myriad commands issued by the programmer.  See the
manual pages for \verb|ca68| and \verb|a68toc| for details of how to
use the \verb|Algol68toC| program development system.

\Section{Terminology}{intro-term}
In describing Algol 68, it is necessary to use a number of technical
terms which have a specialist meaning. However, the number of terms
used has been reduced to a minimum. Whenever a term is introduced it
will be written in \textbf{bold}. Parts of programs are printed as though
they had been produced by a typewriter like this:
\begin{verbatim}
   BEGIN
\end{verbatim}
\noindent
Some of the terminology may seem pedantic. Describing the parts of an
Algol~68 program can, and should be, precise. The power of Algol~68
derives as much from the precision as from the generality of its ideas.

\Section{Values and modes}{intro-values}
Two of the guiding principles of Algol~68 are the concepts of
\bfix{value} and \bfix{mode}.  Typically, an Algol~68 program
manipulates values to produce new values, and, in the process, does
useful work (such as word-processing). Values are such entities as
numbers and letters, but you will see in later chapters that values
can be very complicated and, indeed, can be things that you would not
normally think of as a value.

A value is characterised by its mode.  Every value has only one mode,
and cannot change its mode.  Therefore, if you have a mode change you
must have a new value as well (but see chapter~8).  A mode defines a
set of values. The number of values in the set depends on the mode
and there can be from none to potentially infinity.  For example, the
whole number represented by the digits 37 has mode \ixtt{INT}.  The
symbol \verb|INT| is called a \textbf{mode indicant}.  You will be
meeting many more mode indicants in the course of this book and they
are all written in capital letters and sometimes with digits.  The
strict definition of a mode indicant is that it consists of a series
of one or more characters which starts with a capital letter, and is
continued by capital letters or digits.  No intervening spaces are
allowed.  There is no limit to the length of a mode indicant although
in practice it is rare to find mode indicants longer than 16
characters.  Here are some more mode indicants which you will meet in
this and later chapters\hx{:}{mode!indicant!definition}
\begin{verbatim}
   BOOL    CHAR    COMPL    FILE    HMEAN
\end{verbatim}
\noindent
Section~\hyref{struc-mode} explains how you can define your own mode
indicants. Although you can use any sequence of valid characters,
meaningful mode indicants can help you to understand your programs.
\begin{exercise}
\item Is there anything wrong with the following mode indicants?
\begin{subex}
\item \verb|RealNumber| \subans Yes, it contains lower-case letters.
'
\item \verb|2NDINT| \subans Yes, it starts with a digit.
'
\item \verb|COMPL| \subans No.
'
\item \verb|UPPER CASE| \subans Yes, a space is included.
'
\item \verb|ONE.TWO| \subans Yes, a full stop is included.
'
\end{subex}
\item What is the definition of a mode indicant?  \ans It starts with
a capital letter and continues with capital letters, digits or
underscores with no intervening spaces, tab characters or newline
characters.
'
\end{exercise}

\Section{Integers}{intro-ints}
Although, strictly speaking, there is no largest positive
\ix{integer}, the \hx{largest positive integer}{integer!largest
positive} which can be manipulated by \verb|a68toc| is
$2\,147\,483\,647$, and the \hx{largest negative
integer}{integer!largest negative} is $-2\,147\,483\,647$ (the first
is $2^{31}-1$ and the second is $-2^{31}+1$). The representation of a
value in an Algol~68 program is called a \textbf{denotation} because
it denotes the value. It is important to realise that the
\ix{denotation} of a value is not the same as the value itself. To be
precise, we say that the denotation of a value represents an
\hx{instance}{value!instance} of that value.  For example, three
separate instances of the value denoted by the digits \verb|31| occur
in this paragraph.  All the instances denote the same value.

If you want to write the denotation of an integer in an Algol~68
program, you must use any of the digits \verb|0| to \verb|9|.  No
signs are \hx{allowed}{denotation!integer}.  This means that you
cannot write denotations for negative integers in Algol~68 (but this
is not a problem as you will see).  Although you cannot use commas or
decimal points, spaces can be inserted anywhere.  Here are some
examples of denotations of integers separated by commas (the commas
are not part of the denotations):
\begin{verbatim}
   0 , 3 , 03 , 3000000 , 2 147 483 647
\end{verbatim}
\noindent
Note that \verb|3| and \verb|03| denote the same value because the
\ix{leading zero} is not significant.  However, the zeros in the
three million are significant. The \ix{mode} of each of the five
denotations is \ixtt{INT}. The following are incorrect denotations:
\begin{verbatim}
   3,451    -2    1e6
\end{verbatim}
\noindent
The first contains a comma, the second is a \ix{formula}, and the
third contains the letter \verb|e|.  You will see later on that the
third expression denotes a number, but by definition this denotation
does not have mode \verb|INT|.

\begin{exercise}
\item Write a denotation for thirty-three. \ans 33
'
\item What is wrong with the following integer denotations?
\begin{subex}
\item \verb|1,234,567| \subans It contains commas.
'
\item \verb|5.| \subans It contains a decimal point.
'
\item \verb|-4| \subans It is not a denotation: it is a formula (see
chapter 2).
'
\end{subex}
\end{exercise}

\Section{Identity declarations}{intro-iddecs}
Suppose you want to use the integer whose denotation is
\begin{verbatim}
   48930767
\end{verbatim}
\noindent
in various parts of your program. If you had to
write out the denotation each time you wanted to use it, you would
find that
\begin{itemize}
\item you would almost certainly make mistakes in copying the value,
and
\item the meaning of the integer would not be at all clear
\end{itemize}
It is imperative, particularly with large programs, to make the
meaning of the program as clear as possible. Algol~68 provides a
special construct which enables you to declare a synonym for a value
(in this case, an integer denotation). It is done by means of the
construct known as an
\hx{\textbf{identity declaration}}{declaration!identity} which is
used widely in the language.  Here is an identity declaration for the
integer mentioned at the start of this paragraph:
\begin{verbatim}
   INT special integer = 48930767
\end{verbatim}
\noindent
Now, whenever you want to use the integer, you write
\begin{verbatim}
   special integer
\end{verbatim}
\noindent
in your program.

An identity declaration consists of four parts:
\begin{verbatim}
   <mode indicant> <identifier> = <value>
\end{verbatim}
\noindent
You have already met the \verb|<mode indicant>|. An \bfix{identifier}
is a sequence of one or more characters which starts with a
lower-case letter and continues with lower-case letters or digits or
underscores. It can be broken-up by spaces, newlines or tab
characters.  Here are some examples of valid identifiers (they are
separated by commas to show you where they end, but the commas are
not part of the identifiers):
\begin{verbatim}
   i,  algol,  rate 2 pay,  eigen value 3
\end{verbatim}
\noindent
The following are wrong:
\begin{verbatim}
   2pairs   escape.velocity   XConfigureEvent
\end{verbatim}
\noindent
The first starts with a digit, the second contains a character which is
neither a letter nor a digit nor an underscore, and the third contains
capital letters.

An identifier looks like a name, in the ordinary sense of that word,
but we do not use the term ``name'' in this sense because it has a
special meaning in Algol~68 which will be explained in Chapter~5. The
identifier can abut the mode indicant as in
\begin{verbatim}
   INTa = 4
\end{verbatim}
\noindent
but this is unusual. For clarity in your programs, ensure that a
mode indicant followed by an identifier is separated from the latter
by a space.

The third part is the equals symbol \ixtt{=}. The fourth part (the
right-hand side of the equals symbol) requires a value. You will see
later that the value can be any piece of program which yields a
\ix{value} of the mode specified by the
\hx{mode indicant}{mode!indicant}. So far, we have only met integers,
and we can only denote positive integers.

There are two ways of declaring identifiers for two integers:
\begin{verbatim}
   INT i = 2 ; INT j=3
\end{verbatim}
\noindent
The semicolon \verb|;| is called the
\hx{\textbf{go-on symbol}}{;@\texttt{;}} because it means ``throw
away the value yielded by the previous \ix{phrase}, and go on to the
next phrase''.  If this statement seems a little odd, just bear with
it and all will become clear later.  We can abbreviate the
declarations as follows:
\begin{verbatim}
   INT i=2, j = 3
\end{verbatim}
\noindent
The \ix{comma} separates the two declarations, but does not mean that
the \verb|i| is declared first, followed by the \verb|j|.  On the
contrary, it is up to the compiler to determine which
\ix{declaration} is elaborated first.  They could even be done in
parallel on a \hx{parallel processing}{parallel!processing} computer.
This is known as \hx{collateral elaboration}{elaboration!collateral},
as opposed to \textbf{sequential} elaboration determined by the
\hx{go-on symbol}{elaboration!sequential} (the semicolon). We shall
be meeting collateral elaboration again in later chapters.
Elaboration means, roughly, execution or ``working-out''.  The
compilation system translates your Algol~68 program into
\ix{machine code}.  When the machine code is obeyed by the computer,
your program is elaborated.  The
\hx{sequence of elaboration}{elaboration!sequence of} is determined
by the \ix{compiler} as well as by the structure of your
\hx{program}{program!structure}.  Note that spaces are allowed almost
everywhere in an Algol~68 program.

Some values are predefined in what is called the
\bfix{standard prelude}. You will be learning more about it in
succeeding chapters.  One integer which is predefined in Algol~68 has
the identifier \ixtt{max int}. Can you guess its value?

\begin{exercise}
\item What is wrong with the following identifiers?
\begin{subex}
\item \verb|INT| \subans It is not an identifier: it is a
mode-indicant.
'
\item \verb|int| \subans Nothing---it's all right.
'
\item \verb|thirty-four| \subans It contains a minus symbol.
'
\item \verb|AreaOfSquare|
\subans It contains upper-case letters.
'
\end{subex}
\item What is wrong with the following identity declarations?
\begin{subex}
\item \verb|INT thirty four > 33| \subans The \verb|>| symbol should
be \verb|=|.
'
\item \verb|INT big int = 3 000 000 000| \subans The integer
denotation is larger than the largest integer that the compiler can
handle.
'
\end{subex}
\item Write an identity declaration for the largest integer which the
Algol~68 compiler can handle. Use the identifier\newline
\verb|max int|. \ans \verb|INT max int = 2 147 483 647|
'
\end{exercise}

\Section{Characters}{intro-chars}
All the \ix{symbols} you can see on a computer, and some you
cannot see, are known as \ix{characters}.  The alphabet
consists of the characters \verb|A| to \verb|Z| and \verb|a| to
\verb|z|.  The digits comprise the characters \verb|0| to \verb|9|. 
Every computer recognises a particular set of characters.  The
\ix{character set} recognised by \verb|a68toc|
is \ix{ASCII} (which stands for American
Standard Code for Information Interchange). The mode of a character is
\ixtt{CHAR} (read ``car'' because it is short for
character).  A character is denoted by placing it between quote
characters.  Thus the denotation of the lower-case a is \verb|"a"|. 
Here are some \hx{character denotations}{denotation!character}:
\begin{verbatim}
   "a"  "A"  "3"  ";"  "\"  "'"  """"  " "
\end{verbatim}
\noindent
Note that quote characters are doubled in their denotations.  The
third denotation is \verb|"3"|.  This value has mode \verb|CHAR|.
The denotation \verb|3| has mode \verb|INT|: the two values are quite
distinct, and one is not a synonym for the other.  The last
denotation is that of the space character.

Here are some identity declarations for values of mode \verb|CHAR|:
\begin{verbatim}
   CHAR a = "A", zed = "z";  CHAR tilde = "~"
\end{verbatim}
\noindent
Note that the two sets of identity declarations are separated by a
semicolon, but the declaration for \verb|tilde| is not followed by a
semicolon.  This is because the semicolon \ixtt{;} is not a
terminator; it is an action. Identity declarations do not
\bfix{yield} any value. An identity declaration is a \bfix{phrase}.
Phrases are either identity declarations or \textbf{units}.  When a
phrase is elaborated, if it is a \ix{unit}, it will yield a value.
That is, after elaboration, a value will be available for further use
if required.  Again, this may not make much sense now, but it will
become clearer as you learn the language.

Here is a piece of program with identity declarations for an
\verb|INT| and a \verb|CHAR|:\footnote{The
\protect\hx{a68toc}{a68toc!declarations}
compiler insists on a semicolon between identity declarations for
different modes. In the above case, you would have to write
\texttt{INT ninety nine=99 ; CHAR x = "X"}}
\begin{verbatim}
   INT ninety nine=99 , CHAR x = "X"
\end{verbatim}

The compiler recognises 512 distinct values of mode \verb|CHAR|, but
most of them can only occur in denotations. The space is declared as
\ixtt{blank} in the \ix{standard prelude}.

\begin{exercise}
\item Write the denotations for the full-stop, the comma and the
digit \verb|8| (not the integer 8).  \ans \verb|"." "," "8"|
'
\item Write a suitable identity declaration for the question mark.
\ans \verb|CHAR question mark = "?"|
'
\end{exercise}

\Section{Real numbers}{intro-reals}
Numbers which contain fractional parts, such as $3.5$ or\newline
$0.0005623956$, or numbers expressed in scientific notation, such as
$1.95\times10^{34}$ are values of mode \ixtt{REAL}.  Reals are
denoted by digits and one at least of the decimal point (which is
denoted by a full stop), or the letter \verb|e|.  The \verb|e| means
$\times10^{\hbox{\scriptsize some power}}$.  Just as with integers,
there are no denotations for negative reals.  When the exponent is
preceded by a minus sign, this does not mean that the number is
negative, but that the decimal point should be shifted leftwards.
For example, in the following \verb|REAL|
\hx{denotations}{denotation!real}, the third denotation has the
same value as the fourth (again, the denotations are separated by
commas, but the commas are not part of the denotations):
\begin{verbatim}
   4.5,  .9,  0.000 000 003 4,  3.4e-9,  1e6
\end{verbatim}
\noindent
Although the second denotation is valid, it is advisable in such a
case to precede the decimal point with a zero: \verb|0.9|.  This is
better because a decimal point not preceded by an integer can be
missed easily.  Here are some identity declarations for values of
mode \verb|REAL|:
\begin{verbatim}
   REAL e = 2.718 281 828,
        electron charge = 1.602 10 e-19,
        monthly salary = 2574.43
\end{verbatim}
\noindent
The largest \verb|REAL| which the compiler can handle is declared in
the \ix{standard prelude} as \ixtt{max real}. Its value is
$$1.79769313486231571e308$$. The value of $\pi$ is declared in the
standard prelude with the identifier \ixtt{pi} and a value of
\begin{verbatim}
   REAL pi = 3.141592653589793238462643
\end{verbatim}

It was mentioned above that in an
\hx{identity declaration}{declaration!identity}, any piece of
program yielding a value of the required mode can be used as the
value. Here, for example, is an identity declaration where the value
has mode \verb|INT|:
\begin{verbatim}
   REAL a = 3
\end{verbatim}
\noindent
However, the mode required is \verb|REAL|. In certain circumstances,
a value of one mode can be coerced into a value of another mode.
These circumstances are known as \textbf{contexts}.  There are five
contexts defined in the language.  Each \ix{context} will be
mentioned as it occurs.  The right-hand side of an identity
declaration has a \textbf{strong} context.  In a
\hx{strong context}{context!strong}, a value and its mode can be
changed according to six rules, known as \textbf{coercions}, defined
in the language.  Again, each \ix{coercion} will be explained as it
occurs.  The coercion which replaces a value of mode \verb|INT| with
a value of mode \verb|REAL| is known as
\hx{\textbf{widening}}{coercion!widening}.  You will
meet a different kind of widening in section~\hyref{struc-compl}.

You can even supply an identifier yielding the required mode on the
right-hand side. Here are two identity declarations:
\begin{verbatim}
   REAL one = 1.0;
   REAL one again = one
\end{verbatim}
\noindent
You cannot combine these two declarations into one with a \ix{comma}
as in
\begin{verbatim}
   REAL one = 1.0, one again = one
\end{verbatim}
\noindent
because you cannot guarantee that the identity declaration for
\verb|one| will be elaborated before the declaration for
\verb|one again| (because the comma is not a \textbf{go-on}
symbol).\footnote{The \protect\hx{a68toc}{a68toc!declarations}
compiler does permit a subsequent declaration to use the value of a
previous value, but it is strictly non-standard. You would be wise to
restrict your programs to Algol~68 syntax because other Algol~68
compilers will not necessarily be so lax.}

Values of modes \verb|INT|, \verb|REAL| and \verb|CHAR| are known as
plain values. We shall be meeting another mode having plain values in
chapter~4, and modes in chapter~3 which are not plain.  Complex
numbers are dealt with in chapter~7.
\newpage

\begin{exercise}
\item Is there anything wrong with the following identity
declarations?
\begin{verbatim}
   REAL x = 5.,
        y = .5;
        z = 100
\end{verbatim}
\indent\ans The \verb|5.| should be \verb|5.0|. Either the semicolon
should be replaced by a comma, or \verb|z| should be preceded by
\verb|REAL| or \verb|INT|.
'
\item Given that light travels $2.997\,925\times10^8$ metres per
second in a vacuum, write an identity declaration for the identifier
\verb|light year| in terms of metres to an accuracy of 5 decimal
places (use a calculator).
\ans \verb|REAL light year = 9.454 26 e15|\newline
(assuming 365 days per year).
'
\end{exercise}

\Section{Program structure}{intro-prog}
Algol~68 programs can be written in one or more
\hx{parts}{program!structure} Here is a valid Algol~68
\ix{program}:
\begin{verbatim}
   PROGRAM firstprogram CONTEXT VOID
   USE standard
   BEGIN
      print(20)
   END
   FINISH
\end{verbatim}
\noindent
Only the three lines starting with \verb|BEGIN| and ending with
\verb|END| are strictly part of the Algol 68 program. The first,
second and last lines are specific to the
\hx{a68toc}{a68toc!directives} compiler. The
first line gives the identification of the program as
\texttt{firstprogram} and the fact that this file contains a program.
The \verb|CONTEXT VOID| phrase specifies that the program stands on
its own instead of being embedded in other parts. The phrase is a
vestige of the modular compilation system originally provided by the
compiler at the heart of \verb|a68toc|.

A full explanation of the \verb|print| \ix{phrase} will be found in
chapter~9 (Transput).  For now, it is enough to know that it causes
the value in the parentheses to be displayed on the
screen.\footnote{When Algol 68 was first implemented there were few
monitors around, so \texttt{print} literally printed its output onto
paper.} The standard prelude must be \ixtt{USE}d if you want to use
\ixtt{print}. You can use any identifier for the operating system
file in which to store the Algol~68 \hx{\textbf{source
code}}{code!source} of the
program. Although it does not have to be the same as the identifier
of the module, it is advisable to make it so.

Both the \verb|print| phrase and the \ix{denotation} are
\hx{units}{unit}.  Chapter 10 will explain units in detail.  Phrases
are separated by the go-on symbol (a semicolon \ixtt{;}).  Because
there is only one phrase in \verb|firstprogram|, no go-on symbols are
required.  Here is another valid program:
\begin{verbatim}
   PROGRAM prog CONTEXT VOID
   USE standard
   BEGIN
      INT special integer = 48930767;
      print(20) ; print(special integer)
   END
   FINISH
\end{verbatim}
\noindent
The semicolon between the two \verb|print| phrases is not a
terminator: it is a separator.  It means ``throw away any value
yielded by the previous phrase and go on with the succeeding
phrase''.  That is why it is called the
\hx{go-on symbol}{;@\texttt{;}}.  Notice that there is no semicolon
after the third phrase.

Algol 68 programs are written in \ix{free format}. This means that
the meaning of your program is independent of the layout of the
source program. However, it is sensible to lay out the code so as to
show the structure of the program. For example, you could write the
first program like this:
\begin{verbatim}
   PROGRAM firstprogram CONTEXT VOID
   USE standard(print(20))FINISH
\end{verbatim}
\noindent
which is just as valid, but not as comprehensible. Notice that
\verb|BEGIN| and \verb|END| can be replaced by \verb|(| and \verb|)|
respectively.  How you lay out your program is up to you, but writing
it as shown in the examples in this book will help you write
comprehensible programs.

\begin{exercise}
\item What is wrong with this sample program?
\begin{verbatim}
   PROGRAM test
   BEGIN
      print("A"))
   END
   FINISH
\end{verbatim}
\indent\ans The print phrase has one opening parenthesis and two
closing ones and there is no \verb|CONTEXT| \verb|VOID| \verb|USE|
\verb|standard| preceding the \verb|BEGIN|.
'
\item Using an editor, key in the two sample programs given in this
section, and compile and execute them. What do they display on your
screen?  \ans The first displays \verb|20| at the start of the line.
The second displays \verb*|        +20   +48930767| on one line.
'
\end{exercise}

\Section{Comments}{intro-co}
When you write a program in Algol 68, the pieces of program which do
the work are called \hx{``source code''}{code!source}. The
Algol~68 compiler translates this source code into C source code
which is then translated by the GNU C compiler into \hx{``object
code''}{code!object}.  This is then converted by a program called a
\ix{linker} into \hx{machine code}{code!machine}, which is
understood by the computer.  You then execute the program by typing
its name at the command-line plus any arguments needed.  This is
called \hx{\textbf{running}}{program!running} the program.

When you write the program, it is usually quite clear to you what the
program is doing.  However, if you return to that program after a gap
of several months, the source code may not be at all clear to you.
To help you understand what you have written in the program, it is
possible, and recommended, to write \hx{comments}{comment} in the
source code.  Comments can be put almost anywhere, but not in the
middle of \hx{mode indicants}{mode!indicant} and not in the middle of
denotations.  A comment is ignored by the compiler, except that
comments can be nested.  A comment is surrounded by one of the
following pairs:
\begin{verbatim}
   COMMENT ... COMMENT
   CO ... CO
   #...#
   {...}
\end{verbatim}
\noindent
where the $\ldots$ represent the actual comment. The paired braces
are peculiar to the \hx{a68toc}{a68toc!comments} compiler. Other
compilers may not accept them. Here is an Algol~68 program with
comments added:
\begin{verbatim}
   PROGRAM prog CONTEXT VOID
   USE standard
   BEGIN
      INT i = 23,  # My brother's age #
          s = 27; CO My sister's age CO
      CHAR z = "&",  COMMENT An ampersand
      COMMENT  y{acht}="y";
      REAL x = 1.25;
      print(i); print(s); print(z);
      print(y); print(x)
   END
   FINISH
\end{verbatim}
\noindent
There are four comments in the above program. If you start a comment
with \verb|CO| then you must also finish it with \verb|CO|, and
likewise for the other comment symbols (except the braces). Here is a
program with a bit of source code ``commented out'':
\begin{verbatim}
   PROGRAM prog CONTEXT VOID
   USE standard
   BEGIN
      INT i = 1, j = 2 #, k = 3#;
      print(i); print(j)
   END
   FINISH
\end{verbatim}
\noindent
The advantage of commenting out source is that you only have to
remove two characters and that source can be included in the program
again.  You can use any of the comment symbols for commenting out.
Here is another program with a part of the program containing a
comment commented-out:
\begin{verbatim}
   PROGRAM prog CONTEXT VOID
   USE standard
   BEGIN
      INT i = 1;
      COMMENT
      REAL six = 6.0, # Used subsequently #
           one by 2 = 0.5;COMMENT
      CHAR x = "X";
      print(i); {print(six);} print(x)
   END
   FINISH
\end{verbatim}
\noindent
This is an example of nested comments. You can use any of the comment
symbols for this purpose as long as you finish the comment with the
matching symbol. However, if the part of your program that you want
to comment out already contains comments, you should ensure that the
enclosing comment symbols should be different. One way of using
comment symbols is to develop a standard method. For example, the
author uses the \verb|#...#| comment symbols for one line comments in
the code, \verb|CO| symbols for multiline comments and \verb|COMMENT|
symbols for extensive comments required at the start of programs or
similar code chunks.

\begin{exercise}
\item Write a short program which will print the letters of your
first name. You should declare an identifier of mode \verb|CHAR| for
each letter, and write a \verb|print| phrase for each letter.
Remember to put semicolons in the right places. Add comments to your
program to explain what the program does. \ans It should display your
name without quote symbols on the screen.  Here is an example
program:-
\begin{verbatim}
   PROGRAM ex1 9 1 CONTEXT VOID
   USE standard
   BEGIN
      CHAR s="S", i="i", a="a", n="n";
      CO Letters of my first name CO
      print(s); print(i);
      print(a); print(n)
   END
   FINISH
\end{verbatim}
\noindent
which will display \verb|Sian| on the screen.
'
\end{exercise}

\Section{External values}{intro-ext}
Values denoted or manipulated by a program are called
\hx{\textbf{internal}}{value!internal} values.  Values which
exist outside a program and which are data used by a program or data
produced by a program (or both) are known as
\hx{\textbf{external}}{value!external} values.

In the previous sections we have been learning how plain values are
denoted in Algol~68 programs. This internal display of
\hx{values}{value!displaying}
is not necessarily the same as that used for external values. If
you copy the following program into a file and compile and run it
you will get\par
\verb*|        +10A +.15000000000000000e +1|\par\noindent
output on your screen.
\begin{verbatim}
   PROGRAM test CONTEXT VOID
   USE standard
   BEGIN
      print(10);  print("A");  print(1.5)
   END
   FINISH
\end{verbatim}
\noindent
Notice that although the \ix{denotation} for the first letter of the
alphabet is surrounded by quote characters, when it is displayed on
your screen, the quote characters are omitted.  The rules for numbers
are as follows: if a number is not the first value in the line it is
preceded by a space.  Integers are always printed in the space
required by \ixtt{max int} plus one position for the sign. Both
positive and negative integers have a sign.  A real number is always
printed using the print positions required by \ixtt{max real}, plus a
sign for the number. The exponent is also preceded by a sign.  If you
want extra spaces, you have to insert them.

Try the following program:
\begin{verbatim}
   PROGRAM print2 CONTEXT VOID
   USE standard
   BEGIN
      print(10); print(blank); print("A");
      print(0.015); print(0.15); print(1.5);
      print(15.0); print(150.0); print(1500.0);
      print(15e15)
   END
   FINISH
\end{verbatim}

\Section{Summary}{intro-summ}
An Algol 68 program manipulates values. A value is characterised by
its mode. A mode is indicated by a mode indicant. Plain values can be
denoted. Values occur in contexts, and can sometimes be coerced into
values of different modes. Identifiers can be linked to values using
identity declarations.  The values manipulated by a program are
called internal values.  External values are data used by, or
produced by, a program. Comments describe a program, but add nothing
to its elaboration.

Finally, here are some exercises which test you on concepts you have
met in this chapter.
\newpage

\begin{exercise}
\item Give denotations of the following values:
\begin{subex}
\item one thousand nine hundred and ninety six. \subans \verb|1996|
'
\item The fifth letter of the lower-case Roman alphabet.
\subans \verb|"e"|
'
\item The fraction $1\over7$ expressed as a decimal fraction to 6
decimal places. \subans \verb|0.142857|
'
\end{subex}
\item Is there anything wrong with the following mode indicants?
\begin{subex}
\item \verb|C H A R| \subans Yes, it contains spaces.
'
\item \verb|INT.CHAR| \subans Yes, it contains a decimal point.
'
\item \verb|THISISANEXTREMELYLONGMODEINDICANT| \subans No.
'
\item \verb|2CHAR| \subans Yes, it starts with a digit.
'
\end{subex}
\item Write suitable identity declarations for the following
identifiers:
\begin{subex}
\item \verb|fifty five| \subans \verb|INT fifty five = 55|
'
\item \verb|three times two point seven|
\subans \verb|REAL three times two point seven = 8.1|
'
\item \verb|colon| \subans \verb|CHAR colon=":"|
'
\end{subex}
\item Is there anything wrong with the following identity declarations?
\begin{verbatim}
   REAL x = 1.234,
        y = x;
\end{verbatim}
\indent\ans Yes, you cannot guarantee that the declaration for
\verb|x| will be elaborated before the declaration of \verb|y|. The
declarations should be written
\begin{verbatim}
   REAL x = 1.234;
   REAL y = x
\end{verbatim}
'
\item What is the difference in meaning between \verb|0| and
\verb|0.0|?  \ans \verb|0| denotes an integer with mode \verb|INT|,
\verb|0.0| denotes a real number with mode \verb|REAL|.
'
\item Write a program containing \verb|print| phrases to print the
following values on your screen, separated by one space between each
value:
\begin{verbatim}
   0.5   "G"   1   ":"   34000000
\end{verbatim}
\indent\ans \ %
\begin{verbatim}
   PROGRAM ex1 11 6 CONTEXT VOID
   USE standard
   BEGIN
      print(0.5);  print(blank);
      print("G");  print(1);
      print(blank);print(":");
      print(34 000 000)
   END
   FINISH
\end{verbatim}
'
\end{exercise}
