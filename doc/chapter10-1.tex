% $Log: chapter10.tex,v $
% Revision 1.4  2003-04-23 08:57:31  sian
% Debian release 1.5
%
% Revision 1.3  2002/06/20 11:49:28  sian
% mm removed, padding removed, ca68 added
%
\newcommand{\co}{$\;\Longrightarrow\;$} % Used in coercion displays
\Chapter{Units}{chapx}
The aim of this chapter is to describe the grammar of units in a
fairly rigorous manner.  The chapter covers units, contexts and
coercions, as well as a number of lesser, but still important, ideas
such as casts and balancing.  In describing some of the grammatical
aspects of the language in previous chapters, it has been necessary
to gloss over or distort some of the facts.  The definitive truth
about such matters is in this chapter.

An Algol~68 program consists of a closed \verb|VOID| clause which
means that any value yielded by the closed clause will be voided. Any
closed clause can be used including conditional and loop clauses. It
is unusual to write a program which starts other than with
\verb|BEGIN| (and ends other than with \verb|END|), but there is
nothing in the definition of the language to preclude it. On our
round tour of units, we shall start at the bottom and work up.

\Section{Phrases}{gram-phrases}
A \ix{phrase} is a \ix{declaration} or a \ix{unit}. Declarations
yield no value, even if they include an initial
\hx{assignment}{assignment!initial}. Units are the parts of the
language which actually manipulate values. There are 22 different
kinds of unit which can be subdivided into 5 classes arranged in a
hierarchy:
\begin{verbatim}
   Quaternaries
      Tertiaries
         Secondaries
            Primaries
               Enclosed clauses
\end{verbatim}
\noindent
where each class includes the lower class. For example, all enclosed
clauses are primaries, but not all primaries are enclosed clauses.

The distinctions between different classes of units prevent the
writing of ambiguous programs and help to provide the meaning you
might expect.

The units in each class are as follows:
\begin{itemize}
\item Quaternaries
\begin{itemize}
\item assignments
\item \hx{identity relations}{identity!relation}
\item \hx{routine denotations}{denotation!routine}
\item \ixtt{SKIP}
\end{itemize}
\item Tertiaries
\begin{itemize}
\item formul\ae{}
\item \ixtt{NIL}
\end{itemize}
\item Secondaries
\begin{itemize}
\item generators
\item selections
\end{itemize}
\item Primaries
\begin{itemize}
\item applied-identifiers
\item calls
\item casts
\item \hx{denotations}{denotation} (except routine denotations)
\item slices
\end{itemize}
\item Enclosed clauses
\begin{itemize}
\item case clauses
\item closed clauses
\item collateral clauses
\item conditional clauses
\item conformity clauses
\item loop clauses
\item parallel clauses
\item row-displays
\item structure-displays
\end{itemize}
\end{itemize}

\Section{Contexts}{gram-contexts}
The circumstances which allow certain coercions are called
contexts. Each context has an intrinsic strength. There are five
contexts called \textbf{strong}, \textbf{firm}, \textbf{meek},
\textbf{weak} and \textbf{soft}. The places in a program which have
these contexts are:
\begin{itemize}
\item \hx{Strong contexts}{context!strong}
\begin{itemize}
\item The actual-parameters of calls
\item The enclosed clauses of casts
\item The right-hand side of assignments
\item The right-hand side of identity declarations
\item The right-hand side of initialised name declarations
\item The units of routine denotations
\item \verb|VOID| units
\item All constituents except one of a balanced clause
\item One side of an \hx{identity relation}{identity!relation}
\end{itemize}
\item \hx{Firm contexts}{context!firm}
\begin{itemize}
\item Operands of formul\ae{}
\item The actual parameters of transput calls
\end{itemize}
\item \hx{Meek contexts}{context!meek}
\begin{itemize}
\item Enquiry-clauses (including \verb|WHILE|)
\item Primaries of calls
\item The units following \verb|FROM|, \verb|BY| and \verb|TO| in a
loop clause
\item Trimmers, subscripts and bounds (must yield an \verb|INT|)
\end{itemize}
\item \hx{Weak contexts}{context!weak}
\begin{itemize}
\item Primaries of slices
\item Secondaries of selections
\end{itemize}
\item \hx{Soft contexts}{context!soft}
\begin{itemize}
\item The left-hand side of assignments
\item The other side of an \hx{identity relation}{identity!relation}
(see strong context)
\end{itemize}
\end{itemize}

\Section{Coercions}{gram-coercions}
There are seven coercions in the language, namely
\begin{itemize}
\item \textbf{voiding}
\item \textbf{rowing}
\item \textbf{widening}
\item \textbf{uniting},
\item \textbf{deproceduring}
\item \textbf{dereferencing}
\item \textbf{weakly-dereferencing}
\end{itemize}
Roughly speaking, the coercions can be
arranged in a hierarchy within the hierarchy of contexts thus:
\begin{itemize}
\item Strong context
\begin{itemize}
\item \hx{deproceduring}{coercion!deproceduring}
\item \hx{rowing}{coercion!rowing}
\item \hx{voiding}{coercion!voiding}
\item \hx{widening}{coercion!widening}
\end{itemize}
\item Firm context
\begin{itemize}
\item \hx{uniting}{coercion!uniting}
\end{itemize}
\item Meek context
\begin{itemize}
\item \hx{dereferencing}{coercion!dereferencing}
\end{itemize}
\item Weak context
\begin{itemize}
\item \hx{weakly-dereferencing}{coercion!weakly-dereferencing}
\end{itemize}
\item Soft context
\begin{itemize}
\item deproceduring
\end{itemize}
\end{itemize}

The only coercion not yet met is weakly-dereferencing. However, it is
useful to describe all the coercions here. Before we do so, it should
be noted that one of the limitations of the language is that you
cannot specify the kind of context. Thus if you have a weak context
and you would like a firm context, you cannot specify it. However, in
any context, you can use a \bfix{cast} (see the section on primaries
below) which will always make a context strong and because all
coercions are available in a strong context, you can use the cast to
specify the mode you require.

\Subsection{Deproceduring}{gram-deproc}
\hx{This}{coercion!deproceduring} coercion is available in all
contexts.  Deproceduring is the process by which a parameterless
procedure is called.  For example, the procedure \ixtt{random},
declared in the \ix{standard prelude} as having mode
\verb|PROC REAL|, when called yields a \verb|REAL|.  We can represent
the coercion by
$$\hbox{\verb|PROC REAL|\co\verb|REAL|}$$
The \verb|PROC| is ``removed'', which is why it is called
deproceduring.

There are occasions when the identifier of a procedure can be written
without the procedure being called. In the program fragment
\begin{verbatim}
   PROC REAL rnd:=random
\end{verbatim}
\noindent
the right-hand side of the assignment requires the mode \verb|PROC|
\verb|REAL| because the mode of the name identified by \verb|rnd| is
\verb|REF| \verb|PROC| \verb|REAL|.  Clearly, \verb|random| is not
called here.

The only possible ambiguities with deproceduring are those of
assignments and casts. For example, having declared \verb|rnd| above,
the subsequent assignment
\begin{verbatim}
   rnd:=random;
\end{verbatim}
\noindent
yields a value of mode \verb|REF PROC REAL|, because the value of an
assignment is the value of the left-hand side (see section
\hyref{gram-quat}). However, the following ``go-on symbol'' indicates
that the assignment should now be voided. It is a rule of the
language that voiding takes place before deproceduring if the unit
being voided is an assignment. If, however, \verb|rnd| had been used
on its own, as in
\begin{verbatim}
   rnd;
\end{verbatim}
\noindent
then it would have been dereferenced, then deprocedured and the
resulting \verb|REAL| value voided. This would ensure that any
side-effects (see sections \hyref{rout-nopar} and \hyref{rout-dyad})
would take effect.

Similarly, in the unit
\begin{verbatim}
   PROC REAL(rnd);
\end{verbatim}
\noindent
\verb|rnd| (with mode \verb|REF PROC REAL|) will be dereferenced, but
the resulting value of mode \verb|PROC REAL| will be voided
immediately since it is clear that a \verb|REAL| value is not
required. Note that all the code examples using a go-on symbol could
have been written with \verb|END| or \verb|FI| etc, provided that the
resulting context would have resulted in voiding.

When writing a program, it is common to make mistakes\footnote{You
should expect to make one mistake every 20 lines.  Congratulate
yourself if you do better!}, and one mistake is to write the
identifier of a procedure without its parameters (the primary of a
call).  This is not, strictly speaking, an error. At least, not a
grammatical error. However, in such a case, the
\hx{a68toc}{a68toc!voiding error} compiler will issue a warning:
\begin{verbatim}
   Proc with parameters voided,
   parameters of call forgotten perhaps
\end{verbatim}
\noindent
in which case the mistake should be obvious. Suppose you write the
identifier of a procedure in a formula without its parameters, as in
\begin{verbatim}
   PROC p1 = (INT n)INT:  n**2+3;
   INT a:=4;  a:=4+p1;
\end{verbatim}
\noindent
then the \hx{a68toc}{a68toc!OP error@\texttt{OP} error} compiler will
issue the message
\begin{verbatim}
   op + not declared for INT and PROC (INT)INT
\end{verbatim}
\noindent
The error message for a procedure identifier on the right-hand side of
an assignment is
\begin{verbatim}
   PROC (INT)INT cannot be coerced to INT
\end{verbatim}

Deproceduring only occurs with parameterless procedures.

\Subsection{Dereferencing}{gram-deref}
This is the process of moving from a name to the value to which it
refers (which could also be a name---see chapter~11). For example, if
\verb|x| has mode \verb|REF REAL|, then in the
\hx{formula}{coercion!dereferencing}
\begin{verbatim}
   x * 3.5
\end{verbatim}
\noindent
the name \verb|x| will be dereferenced to yield a new instance of the
\verb|REAL| referred to by \verb|x|. The coercion can be represented by
$$\hbox{\verb|REF REAL|\co\verb|REAL|}$$
If \verb|rx| has mode \verb|REF REF REAL| (that is, \verb|rx| can
refer to a name of mode \verb|REF REAL|), then the formula
\begin{verbatim}
   rx * 3.5
\end{verbatim}
\noindent
will result in \verb|rx| being dereferenced twice. In this case, the
coercion could be represented as
$$\hbox{\verb|REF REF REAL|\co\verb|REAL|}$$
Dereferencing is available in all contexts except soft.

When a name, such as \verb|rx|, is dereferenced twice, new instances
of both the values referred to (in the case of \verb|rx|, the
\verb|REF REAL| and the \verb|REAL| values) are created. However, the
new instance of the \verb|REF REAL| value is discarded after the
creation of the \verb|REAL| value. This has no effect on the
elaboration of the program.

\Subsection{Weakly-dereferencing}{gram-wderef}
\hx{This}{coercion!weakly-dereferencing} is a variant of the dereferencing
coercion in which any number of \verb|REF|s can be removed except the
last. Thus, in the case of \verb|rx| above, weakly-dereferencing would
yield a mode of \verb|REF REAL| and could be represented by
$$\hbox{\verb|REF REF REAL|\co\verb|REF REAL|}$$
This coercion is only available in \hx{weak contexts}{context!weak}.
It is particularly useful in the selection of
\hx{secondaries}{secondary} of structure modes which contain fields
whose mode starts with \verb|REF| (see section~\hyref{gram-sec} and
chapter~11).

\Subsection{Uniting}{gram-unite}
\hx{In}{uniting} this coercion, the mode of a value becomes a
united-mode. For example, if \verb|OO| is an operator both of whose
operands are \verb|UNION(INT,REAL)|, then in the formula
\begin{verbatim}
   3.0 OO -2
\end{verbatim}
\noindent
both operands will be united to \verb|UNION(INT,REAL)| before the
operator is elaborated. These coercions can be represented by
$$\left.
    \begin{array}{l}
      \hbox{\tt INT}\\
      \hbox{\tt REAL}
    \end{array}
  \right\}\quad\Longrightarrow\quad\hbox{\tt UNION(INT,REAL)}$$
Uniting is available in \hx{firm}{context!firm} and
\hx{strong}{context!strong} contexts and must precede rowing.

\Subsection{Widening}{gram-wid}
In a strong context, an integer can be replaced by a real number and
a real number replaced by a complex number, depending on the mode
required. This can be represented by
\begin{eqnarray*}
 \hbox{\tt INT}\,&\Longrightarrow\,&\hbox{\tt REAL}\\
 \hbox{\tt REAL}\,&\Longrightarrow\,&\hbox{\tt COMPL}
\end{eqnarray*}
Widening is not available in formul\ae{} (firm contexts).
\newpage

\Subsection{Rowing}{gram-row}
\hx{If}{coercion!rowing}, in a strong context, a multiple is required
and a value is provided whose mode is the \hx{base mode}{mode!base} of the
multiple, then the value will be rowed to provide the required
multiple.  There are two cases to consider:
\begin{enumerate}
\item If the mode required is not a name and the base-mode of the
multiple is the mode of the value given, then the value will be rowed
to give \verb|[]base-mode|.  For example, if the required mode is
\verb|[]INT|, then the base-mode is \verb|INT|.  In the identity
declaration
\begin{verbatim}
   []INT i = 3
\end{verbatim}
\noindent
the value yielded by the right-hand side (an integer) will be rowed and
the coercion can be expressed as
$$\hbox{\verb|INT|\co\verb|[]INT|}$$
If the value given is a row mode, such as \verb|[]INT|, then there are
two possible rowings that can occur.
\begin{enumerate}
\item In the identity declaration
\begin{verbatim}
   [,]INT a = i
\end{verbatim}
\noindent
where \verb|i| was declared above with mode \verb|[]INT|, the coercion
can be expressed as
$$\hbox{\verb|[]INT|\co\verb|[,]INT|}$$
In this case, an extra dimension is added to the multiple.
\item If the required mode is \verb|[][]INT| as in
\begin{verbatim}
   [][]INT r = i
\end{verbatim}
\noindent
then the value on the right-hand side is rowed to yield a
one-dimensional multiple whose base-mode is \verb|[]INT|. This coercion
can be represented as
$$\hbox{\verb|[]INT|\co\verb|[][]INT|}$$
\end{enumerate}
\item If the multiple required is a \ix{name}, then a name of
a non-multiple can be supplied. For example, if the value supplied is
a name with mode \verb|REF INT|, then a name with mode
\verb|REF[]INT| will be created. In this identity
\hx{declaration}{declaration!identity}
\begin{verbatim}
   REF[]INT ni = LOC INT
\end{verbatim}
\noindent
the local \hx{generator}{generator!local} yields a name with mode
\verb|REF INT| and the rowing coercion yields a name with mode
\verb|REF[]INT| and bounds \verb|[1:1]|. The coercion can be
represented by
$$\hbox{\verb|REF INT|\co\verb|REF[]INT|}$$
The first kind of rowing could also occur. The identity-declaration
\begin{verbatim}
   []REF INT rri = LOC INT
\end{verbatim}
\noindent
produces the coercion represented by
$$\hbox{\verb|REF INT|\co\verb|[]REF INT|}$$
Likewise, a name of mode \verb|REF[]INT| can be rowed to a name with
mode \verb|REF[,]INT| or a non-name with mode \verb|[]REF[]INT|,
depending on the mode required. Although \verb|INT| has been taken as
an example, any mode could have been used.
\end{enumerate}

\Subsection{Voiding}{gram-voiding}
\hx{In}{coercion!voiding} a \hx{strong context}{context!strong}, a
value can be thrown away, either because the mode \ixtt{VOID} is
explicitly stated, as in a procedure yielding \verb|VOID|, or because
the context demands it, as in the case of a
\hx{semicolon}{;@\texttt{;}} (the go-on symbol).  In this case, there
are two exceptions to the rule that the value yielded depends only on
the context. Casts and assignments are voided after the elaboration
of the unit, but all other units are subjected to the usual coercions
in a strong context.  The following program illustrates this:
\begin{verbatim}
   PROGRAM tproc CONTEXT VOID
   USE standard
   BEGIN
      PROC INT p;
      PROC pp = INT:
      (
         INT i=random int(6);
         print(i);
         i
      );
      p:=pp;
      print((" p:=pp",newline));
      pp;
      print((" pp",newline));
      p;
      print((" p",newline));
      PROC INT(p);
      print((" PROC INT(p)",newline))
   END
   FINISH
\end{verbatim}
\noindent
The output is
\begin{verbatim}
    p:=pp
            +6 pp
            +1 p
    PROC INT(p)
\end{verbatim}
\noindent
In the \ix{assignment} \verb|p:=pp|, the mode required on the
right-hand side is \verb|PROC INT| so \verb|pp| is not
\hx{deprocedured}{coercion!deproceduring}, and \verb|p| is neither
\hx{dereferenced}{coercion!dereferencing} nor deprocedured after the
assignment has been elaborated. The cast \verb|PROC INT(p)| is
elaborated (that is, \verb|p| is dereferenced) and then voided
without the procedure \verb|p| (or \verb|pp|) being
\hx{called}{elaboration}.

\Subsection{Legal coercions}{gram-legal}
In any context, you have a unit which has, or yields, a value of some
mode; and in that context you have a mode which you need. If the
value of the mode you have can be coerced to a value of the mode you
need (assuming that the two modes differ), then the coercion is legal.

For example, suppose you have a value of mode \verb|PROC| \verb|REF|
\verb|INT| in a strong context and the mode you want is
\verb|[]COMPL|.  The required mode can be got via
\begin{itemize}
\item deproceduring to mode \verb|REF INT|
\item dereferencing to mode \verb|INT|
\item widening to mode \verb|REAL|
\item widening to mode \verb|COMPL|
\item rowing to mode \verb|[]COMPL|
\end{itemize}
In practice, coercions are not usually as complicated as this.

Notice that deproceduring can take place before or after
dereferencing, that widening must occur before rowing and that
voiding can only take place after all other coercions. For example,
you cannot coerce \verb|[]INT| to \verb|[]REAL|.

\begin{exercise}
\item Which coercions are available in a meek context?
\ans Deproceduring and dereferencing (not weakly-deref\-eren\-cing).
'
\item Which coercions are not available in a strong context?
\ans None.
'
\item For each of the following, state whether the given mode can be
coerced to the mode to the right of the arrow:
\begin{subex}
\item Weak context: \verb|REF REF BOOL|\co\verb|REF BOOL| \subans Yes.
'
\item Firm context:\newline
\verb|PROC INT|\co\verb|UNION(REAL,COMPL)|
\subans No (cannot widen).
'
\item Soft context: \verb|REF PROC CHAR|\co\verb|CHAR| \subans No
(cannot dereference).
'
\item Meek context: \verb|PROC REF REAL|\co\verb|[]REAL| \subans No
(cannot row).
'
\item Weak context: \verb|PROC REF BOOL|\co\verb|BOOL| \subans No
(cannot dereference).
'
\item Strong context:\newline
\verb|PROC INT|\co\verb|UNION([]INT,[]REAL)|
\subans No (cannot unite after rowing).
'
\end{subex}
\end{exercise}

\Section{Enclosed clauses}{gram-enc}
There are nine kinds of enclosed \hx{clause}{clause!enclosed}, most
of which we have already met.\footnote{Note that a serial clause is
\protect\emph{not} an enclosed clause.}
\begin{enumerate}
\item The simplest is the closed \hx{clause}{clause!closed} which
consists of a serial clause enclosed in parentheses (or \ixtt{BEGIN}
and \ixtt{END}).  The \ix{range} of any identifiers declared in the
closed clause is limited to the closed clause.  The
\hx{a68toc}{a68toc!identifier range} compiler limits the use of any
identifiers declared in the closed clause to the closed clause at and
after their declaration.  Here are some examples of closed clauses:
\begin{verbatim}
   (3)
   BEGIN p + 3 END
   (INT s;  read(s);  s)
   (REAL q:=i+2;  sqrt(q))
\end{verbatim}
\noindent
\item \hx{Collateral clauses}{clause!collateral} look like
row-displays: there must be at least two units. Remember that
declarations are not units.  The units are elaborated collaterally.
This means that the order is undefined and may well be in parallel.
Examples of collateral clauses are\footnote{The
\protect\hx{a68toc}{a68toc!collateral clauses}
compiler does not provide collateral clauses other than row-
and structure-displays.}
\begin{verbatim}
   (m:=3, n:=-2)
   ((INT m:=2; m),(CHAR a=REPR i; a))
\end{verbatim}
\noindent
The second collateral clause has two units each of which is a
closed clause.

A parallel \hx{clause}{clause!parallel} looks exactly like a
collateral clause preceded by \verb|PAR|. The constituent units
(there must be at least two) are executed in parallel.\footnote{The
\protect\hx{a68toc}{a68toc!parallel clauses} compiler does not
provide parallel clauses.}
\end{enumerate}
The other enclosed clauses have already been discussed: 
\begin{enumerate}\setcounter{enumi}{2}
\item \hx{row-display}{row!display} in section~\hyref{rep-rowdisp}
\item loop \hx{clause}{clause!loop} in section \hyref{rep-rep}
\item conditional \hx{clause}{clause!conditional} in section
\hyref{choice-cond}
\item case \hx{clause}{clause!case} in section \hyref{choice-case}
\item \hx{structure-display}{structure!display} in
section~\hyref{struc-denot}
\item conformity \hx{clause}{clause!conformity} in
section~\hyref{union-conf}
\end{enumerate}
It should be noted that the enquiry clause (in a conditional- or
case-clause) is in a \hx{meek context}{context!meek} whatever the
context of the whole clause. Thus, the context of the clause is
passed on only to the final phrase (it must be a unit) in the
\verb|THEN|, \verb|ELSE|, \verb|IN| or \verb|OUT| clauses.

\begin{exercise}
\item What kind of enclosed clause could each of the following be?
\begin{subex}
\item \verb|((INT p:=ENTIER-4.7; p),37.5)| \subans Row-display,
structure-display, collateral clause.
'
\item \verb|PAR BEGIN 3, 15 END| \subans Parallel clause.
'
\item \verb!(i|3,-3|4)! \subans Case clause.
'
\item \verb!(si|(INT i): i,(STRING i): i)! \subans Conformity clause.
'
\item \verb!(a < 3|4|5)! \subans Conditional clause.
'
\item \verb|(a:=2; b:=-a)| \subans Closed clause or enclosed clause.
'
\end{subex}
\end{exercise}

\Section{Primaries}{gram-prim}
Primaries are denotations, applied identifiers, casts, calls and
slices. We have met \hx{denotations}{denotation} in chapters 1,~4
and~6. Only plain \hx{values}{values!plain}, routines and a special
name (\verb|NIL|) have a denotation.  \verb|NIL| is dealt with in the
section on tertiaries and the mode \verb|BITS| is covered in
chapter~11.  Applied-identifiers means identifiers being used in a
context, rather than in their declarations.  We have met numerous
examples of these.  Routine denotations are not primaries.

A \bfix{cast} consists of a \hx{mode indicant}{mode!indicant}
followed by an enclosed clause, usually a closed clause. Here is a
formula with a cast:
\begin{verbatim}
   3.4 * REAL (i)
\end{verbatim}
\noindent
where \verb|i| has mode \verb|INT|. The cast puts the enclosed clause
in a strong context. Thus, in the above formula, the normal context
of an operand is firm (see chapter~2), but the cast causes the value
of \verb|i| to be widened to a \verb|REAL|. Casts are usually used in
formul\ae{} and \hx{identity relations}{identity!relation} (see
sections \hyref{gram-quat} and \hyref{adv-idrel}). Casts are
sometimes used to coerce a conditional or case clause where
\ix{balancing} is insufficient to provide the mode required (see
section~\hyref{gram-bal} later in this chapter). The mode indicant
can be any mode and can contain any of the mode-constructors such as
\verb|REF| or \verb|PROC| or \verb|[]| (but it should not be a
\ix{generator}, which is not a mode indicant). Care should be taken
when using a structured mode. For example, in this formula,
\begin{verbatim}
   3 * STRUCT(INT k)(4)
\end{verbatim}
\noindent
assuming that the operator has been declared for operands of modes
\verb|INT| and \verb|STRUCT(INT k)|, the cast must include the
\ix{field selector} because it is part of the mode.

\hx{Calls}{procedure!call} were discussed in sections
\hyref{rout-npprocs} and \hyref{rout-parprocs}. Here is a simple example:
\begin{verbatim}
   sqrt(0.7)
\end{verbatim}
\noindent
In this call, \verb|sqrt| is itself a primary (it is an
applied-\hx{identifier}{identifier!applied}). In
section~\hyref{gram-contexts}, it was mentioned that the primary of a
call is in a meek \hx{context}{context!meek}. This applies even if
the call itself (as a whole) is in a strong context. The primary of a
call can be an enclosed clause. For example,
\begin{verbatim}
   (a>4|sqrt|sin)(x)
\end{verbatim}
\noindent
which yields \verb|sqrt(x)| if \verb|a > 4| and \verb|sin(x)|
otherwise. In this case, the primary is
\begin{verbatim}
   (a>4|sqrt|sin)
\end{verbatim}

We discussed \hx{slices}{slice} in section \hyref{rep-slice}. They
include simple subscripting.  For example, given the declaration
\begin{verbatim}
   [,]INT r = ((1,2,3),(4,5,6))
\end{verbatim}
\noindent
the units \verb|r[1,]| and \verb|r[2,3]| are both slices. Whatever the
context of the slice, the context of the primary of the slice (\verb|r|
in these examples) is always \hx{weak}{context!weak}. This means that
only weak-dereferencing is allowed. Thus, given the
\hx{phrases}{coercion!weakly-dereferencing}
\begin{verbatim}
   [2,3]INT s:=r;  INT p:=s[2,1]
\end{verbatim}
\noindent
the slice \verb|s[2,1]| is in a strong \hx{context}{context!strong},
but the \verb|s| is in a weak context, so the name that \verb|s|
identifies, which has the mode \verb|REF[,]INT| will not be
dereferenced, though the slice, which has mode \verb|REF INT|, will
be.

\hylabel{gram-slice}%
There is another consequence of the weak context of the primary of a
slice: row-displays can only be used in a strong context. So if you
want to change the bounds of a row-display, because the slicer will
produce a weak context, the row-display must be enclosed in a cast.

The context of subscripts and bounds in trimmers is
\hx{meek}{context!meek} and they must be units.

All enclosed clauses are primaries, but not all primaries are
enclosed clauses.

\begin{exercise}
\item What are the contexts of
\begin{subex}
\item \verb|p| (mode \verb|REF[]REAL|) in \verb|[]REAL (p[3])|
\subans Weak.
'
\item \verb|q| (mode \verb|PROC(REAL)INT|) in \verb|REAL(q(0.5))|
\subans Meek.
'
\end{subex}
\item How many primaries are there in each of the following units:
\begin{subex}
\item \verb|3 * (1.4 + r)/2**6| \subans 6 (4 denotations, 1
applied-identifier, 1 closed clause).
'
\item \verb|p:=sqrt(r) - 6| \subans 5 (1 denotation, 3
applied-identifiers, 1 call).
'
\item \verb|num:=x[3,ENTIER r]| \subans 5 (1 denotation, 3
applied-identifiers, 1 slice).
'
\item \verb|i * []CHAR("e")| \subans (1 denotation, 1 closed clause,
1 cast, 1 applied-identifier).
'
\end{subex}
\end{exercise}

\Section{Secondaries}{gram-sec}
We have discussed both kinds of secondary (\hx{selections}{selection}
and \hx{generators}{generator}), but there are other points which
need mentioning.

There are two kinds of generator (see section~\hyref{name-assign}).
Occasionally, when a procedure has a name parameter, the name may not
be needed. Instead, therefore, of using an identifier of a name which
is used for another purpose, which would be confusing, or declaring a
name just for this purpose, which would be unnecessary, an
\ix{anonymous name} can be used. For example, a
possible call of the procedure \verb|char in string| could be
\begin{verbatim}
   char in string(ch,LOC INT,str)
\end{verbatim}
\noindent
if you are only interested in whether the character is in the string
and not in its position.

Another case where an anonymous name is useful is in the creation of
odd-shaped multiples.  Consider the program fragment:
\begin{verbatim}
   [10]REF[]INT ri;  INT j;
   
   FOR i TO UPB ri
   DO
      read(j);
      ri[i]:=LOC[j]INT;  read(ri[i])
   OD
\end{verbatim}
\noindent
Since there are no declarations in the loop \hx{clause}{clause!loop},
the \ix{scope} of the name created by the generator is the
enclosed clause surrounding the loop clause, which includes the
declarations for \verb|ri| and \verb|j|.  The mode of the slice
\verb|ri[i]| is \verb|REF REF[]INT|.  Thus the value of \verb|ri[i]| is
a name with two \verb|REF|s in its mode, and it is made to refer to a
name of mode \verb|REF[]INT|, which has one \verb|REF| less. 
Assignments of this type will be considered in detail in the next
chapter. Note that the context of a parameter to \verb|read| is firm
so the parameter is dereferenced once before a value is read.

\hylabel{gram-paren}%
When discussing \hx{selections}{secondary} in section
\hyref{struc-field}, you may have wondered about the peculiar rules of
placing parentheses when talking about rows in structures, rows of
structures and rows in rows of structures.  Firstly, it should be
mentioned that in the secondary
\begin{verbatim}
   im OF z
\end{verbatim}
\noindent
where z has mode \verb|COMPL| or \verb|REF COMPL|, the \verb|z|
itself is not only a secondary, it is also a primary (it is an
\hx{applied-identifier}{identifier!applied}). This means that using
the declarations
\begin{verbatim}
   MODE AM = STRUCT(INT i,CHAR c),
        BM = STRUCT(INT i,AM a);
   BM b
\end{verbatim}
\noindent
the selection
\begin{verbatim}
   c OF a OF b
\end{verbatim}
\noindent
is valid because
\begin{verbatim}
   a OF b
\end{verbatim}
\noindent
is also a secondary. We shall meet extended selections like this in
chapter~11.

Secondly, a primary is a secondary, but not necessarily the other way
round.  Consider these declarations:
\begin{verbatim}
   STRUCT(INT i,[3]REAL r)s1;
   [3]STRUCT(INT i,REAL r)s2
\end{verbatim}
\noindent
The selection \verb|r OF s1| has the mode \verb|REF[]REAL|.  If you
want to slice it, to get one of the constituent names of mode
\verb|REF REAL| say, you cannot do so directly.  The reason is that
in a \ix{slice}, as mentioned in the previous section, what is sliced
must be a \ix{primary}.  To convert the \ix{secondary} into a primary
you have to enclose it in \ix{parentheses} thus converting it into an
enclosed clause; and enclosed clauses are also primaries (in
section~\hyref{gram-phrases}, it was said that the four classes of
units are arranged in a hierarchy in which each class includes the
lower classes).  So the second name of \verb|r OF s1| is yielded by
\verb|(r OF s1)[2]|.

On the other hand, considering the name identified by \verb|s2|, the
selection
\begin{verbatim}
   r OF s2[2]
\end{verbatim}
\noindent
can be written without parentheses because \verb|s2| is not only a
secondary, it is also a primary (an applied-identifier) with mode
\verb|REF[]STRUCT...|. The phrase \verb|s2[2]| is perfectly valid, it
having mode \verb|REF STRUCT(...)|.  The selection \verb|r OF s2| has
the mode \verb|REF[]REAL| and so it too can be sliced by writing
\verb|(r OF s2)[2]|. The effect is the same for both of the cases
involving \verb|s2|. Note that the \hx{a68toc}{a68toc!selections}
compiler does not permit selection of a field from a row of
structures. Doing so will yield the following error message:-
\begin{verbatim}
OPERATORS - select: []struct not implemented
FATAL ERROR (661) Compiler error:
   ENVIRONMENT (ASSERT) - assertion failure
\end{verbatim}

To summarise, any primary can be regarded as a secondary, but not
vice-versa.

\begin{exercise}
\item Give an example of a primary which is also a secondary.
\ans The identifier of a structure or a name referring to a structure.
'
\item Give an example of a secondary which is not a primary. \ans A
selection.
'
\item In this exercise, the following declarations hold:
\begin{verbatim}
   MODE AM = STRUCT(CHAR a,b),
        BM = STRUCT(AM a,
                    STRUCT(CHAR a,AM b) c,
                    REF BM d);
   BM p
\end{verbatim}
\noindent
How many secondaries are there in each of the following units?
\begin{subex}
\item \verb|a OF p| \subans 2.
'
\item \verb|a OF a OF p| \subans 3.
'
\item \verb|a OF c OF p| \subans 3.
'
\item \verb|a OF a OF d OF p| \subans 4.
'
\end{subex}
\end{exercise}

\Section{Tertiaries}{gram-ter}
Tertiaries are formul\ae{} and \verb|NIL|.  \hx{Formul\ae{}}{formula}
were covered in chapter two.  All that needs to be said here is that a
formula can consist solely of a single \ix{secondary} or \ix{primary}
or enclosed clause although this is not usual.  If a formula,
containing at least one operator, is to be used as a primary or a
secondary, it must be enclosed in parentheses (or \verb|BEGIN| and
\verb|END|).  For example, in the formula
\verb|next OF (H declarer)|, where \verb|H = (INT)REF HAND| and
\verb|HAND = STRUCT(...,REF HAND next)|, the formula must be
surrounded by parentheses to make it into a secondary.

The only name having a denotation is \verb|NIL|. Its mode is
\verb|REF| \verb|whatever|. In other words, it can have any mode
which starts with \verb|REF|. It does not refer to any value and,
although it must only occur in a strong context, it cannot be
coerced. Its uses are described in the next chapter.

\Section{Quaternaries}{gram-quat}
Quaternaries are assignments, routine denotations,
\hx{identity relations}{identity!relation} and \verb|SKIP|. Of the
four, the \ix{assignment} is the most common.  An assignment consists
of three parts. The left-hand side must be a \ix{tertiary}. It is
usually an \ix{applied identifier} or, less commonly, an
\hx{enclosed clause}{clause!closed}. Its value must be a \ix{name}.
Its context is \hx{soft}{context!soft}, so no dereferencing is
allowed unless a \ix{cast} is used (see the next chapter), but
\hx{deproceduring}{coercion!deproceduring} is allowed.  The second
part is the \ix{assignment token}.  The right-hand side (the third
part) can be any quaternary (including, of course, another
assignment). Its context is \hx{strong}{context!strong} so any
coercion is permitted.  The mode of its value must contain one less
\verb|REF| than the mode of the left-hand side.

The right-hand side of an assignment is, most commonly, a formula
which is a tertiary (all tertiaries are quaternaries, but not
vice-versa).  The left-hand side can also be a formula provided that
the value yielded is a name (which is the case with the
\ix{assigning operators}---see section~\hyref{name-assops}). If an
assignment is to
be used as a primary, a secondary or a tertiary, then it must be
enclosed in \ix{parentheses}(or \ixtt{BEGIN} and \ixtt{END}).  The
value of an assignment is the value of the left-hand side: that is,
it is a name.  Assignments were discussed in chapter~5.

\hx{Routine denotations}{denotation!routine} were discussed in
chapter~6.

\ixtt{SKIP} yields an undefined value of any mode
and can only occur in a strong context. It is particularly useful in
the following case. Consider the procedure
\begin{verbatim}
   PROC p=(REAL a,b)REAL:
   IF b=0
   THEN print(("Division by zero",newline)):
      stop;  SKIP
   ELSE a/b
   FI
\end{verbatim}
\noindent
Since the yield has mode \verb|REAL|, and the \verb|ELSE| part of the
conditional clause yields a value of mode \verb|REAL|, by the principle
of \ix{balancing} (see below) the \verb|THEN| part also
must yield a value of mode \verb|REAL|.  Now the construct \verb|stop|
yields a value of mode \verb|VOID| which cannot be coerced to
\verb|REAL| in any context.  If the procedure is going to compile
successfully, the \verb|THEN| part must yield \verb|REAL| (or, at
least, a value which can be coerced to \verb|REAL| in the context of
the body of the procedure which is strong) even though the value
yielded will never be used (because the \verb|stop| will terminate the
program).  The \verb|SKIP| will yield an undefined value of mode
\verb|REAL|.  Although \verb|SKIP| must occur in a strong context, it
cannot be \hx{coerced}{SKIP@\texttt{SKIP}}.

Another use for \verb|SKIP| is in \hx{row-}{row!display} or
\hx{structure-displays}{structure!display} where not all the units
are known at the time of a declaration.  For example:
\begin{verbatim}
   [3]INT ii:=(4,?,5)
\end{verbatim}
\noindent
Before the multiple \verb|ii| is used, the second element should be
given a value. If no such value is assigned, and you try to print the
value of \verb|ii[2]| the \hx{a68toc}{a68toc!unassigned names}
compiler will generate code which will print whatever value was there
at the time the multiple was generated, which may well be rubbish.

The \hx{identity relation}{identity!relation} is discussed in the
next chapter, but its grammar has important consequences. The
identity relation consists of two tertiaries separated by an
\hx{identity relator}{identity!relator} (one of \ixtt{:=:} or
\ixtt{:/=:}). Since a \ix{formula} is a \ix{tertiary}, it can safely
be included in an identity relation.  For example, given the
declarations
\begin{verbatim}
   INT x:=3, y:=1;
   PROC x or y = (REAL r)REF INT: (r<0.5|x|y)
\end{verbatim}
\noindent
the identity relation
\begin{verbatim}
   x or y(random) :=: x
\end{verbatim}
\noindent
is legal. However, if you want to include an identity relation in a
formula then you must surround it with parentheses to make it into a
tertiary, as in
\begin{verbatim}
   IF (x or y(random) :=: x) AND x*y > 0
   THEN
\end{verbatim}
\noindent
Since one side of an identity relation is in a soft
\hx{context}{context!soft} while the other is in a strong
\hx{context}{context!strong}, only one side of an identity
relation can be strongly-dereferenced. The soft side can be
weakly-dereferenced which means that one \verb|REF| will always be
left on that side. \hx{Balancing}{balancing} applies to
\hx{identity relations}{coercion!weakly-dereferencing} (see the
discussion in section~\hyref{adv-idrel}).

This completes the general discussion of units.

\begin{exercise}
\item What kind of units are each of the following:
\begin{subex}
\item A cast. \subans A primary.
'
\item An applied-identifier. \subans A primary.
'
\item A selection. \subans A secondary.
'
\item A multiple. \subans A primary.
'
\item A name. \subans A primary.
'
\item A formula. \subans Tertiary.
'
\item A loop clause. \subans Enclosed clause.
'
\item An assignment. \subans A quaternary.
'
\item A declaration. \subans It is not a unit.
'
\item A procedure denotation. \subans A quaternary.
'
\end{subex}
\item Which units are to be found in each of the following:
\begin{subex}
\item \verb|3.5 * (a - 2 * x)| \subans 2 denotations $+$ 2
applied-identifiers $=$ 4 primaries. 1 closed clause. 3~formul\ae{}
$=$ 3 tertiaries.
'
\item \verb|p OR q AND a = 4| \subans 1 denotation $+$ 3
applied-identifiers $=$ 4 primaries. 3 formul\ae{} $=$ 3 tertiaries.
'
\item \verb|sin(x)| \subans 2 applied-identifiers $+$ 1 call $=$ 3
primaries.
'
\item \verb|a[3,2:4]| \subans 3 denotations $+$ 1 applied-identifier
$+$ 1 slice $=$ 5 primaries.
'
\item \verb!x:=(c<"e"|2.4|-y)! \subans 2 denotations $+$ 3
applied-identifiers $=$ 5 primaries; 1 conditional clause $=$ 1
enclosed clause, 2 formul\ae{} $=$ 2 tertiaries, 1 assignment $=$ 1
quaternary.
'
\item \verb!(i|x,y,z):=(p|2|-4)! \subans 2 denotations $+$ 5
applied-identifiers $=$ 7 primaries, 1 formula $=$ 1 tertiary, 1
assignation $=$ 1 quaternary, 1 case clause $+$ 1 conditional clause
$=$ 2 enclosed clauses.
'
\item \verb|PAR(x:=1.2,y:=3.6)| \subans 2 denotations $+$ 2
applied-identifiers $=$ 4 primaries, 2 assignments $=$ 2
quaternaries, 1 parallel clause $=$ 1 enclosed clause.
'
\end{subex}
\end{exercise}

\Section{Balancing}{gram-bal}
In section~\hyref{rout-cont}, it was pointed out that the context of a
routine \hx{denotation}{denotation!routine} is passed on to the last
unit in the denotation.  In the example given, the body of the
routine denotation was a closed \hx{clause}{clause!closed}.  The
\ix{yield} of the routine was a value of mode \verb|INT|, but the
yield of the last unit was a name with mode \verb|REF INT|.  Since
the context of the body of a routine denotation is strong, the name
is dereferenced to get an \verb|INT|. This principle is applicable to
all enclosed clauses.

Now conditional clauses, case clauses and conformity clauses can
yield one of a number of units, and so it is quite possible for the
units to yield different values of different modes.  The principle of
\ix{balancing} allows the context of all these units, except one, to
be promoted to strong whatever the context of the enclosed clause.
Balancing is also invoked for
\hx{identity relations}{identity!relation} which are dealt with in
the next chapter.

Considering, for example, the formula
\begin{verbatim}
   x * (a > 0|3.0|2)
\end{verbatim}
\noindent
the context of the conditional clause is firm which means that widening
is not allowed. Without balancing, the conditional clause could yield a
\verb|REAL| or an \verb|INT|. In this example, the principle of
balancing would promote the context of the \verb|INT| to strong and
widen it to \verb|REAL|. Balancing thus means ``making the modes the
same''.

In a balanced clause, one of the yielded units is in the context of the
whole clause and all the others are in a strong context, irrespective
of the actual context of the clause. Here is an example of a balanced
case \hx{clause}{clause!case}
\begin{verbatim}
   INT i:=3,j:=4,a:=2;
   PROC ij = REF INT: (random < 0.5|i|j);
   print(2 + (a|i,ij|random))
\end{verbatim}
\noindent
where the \verb|a| yields an \verb|INT| in a meek
\hx{context}{context!meek} (that of the enquiry clause). In this
example, the modes of the values that can be yielded by the case
clause are \verb|REF INT| (\verb|i|), \verb|PROC REF INT| (\verb|ij|)
and \verb|PROC REAL| (\verb|random|).  In a firm context, the modes
become \verb|INT|, \verb|INT| and \verb|REAL|. Thus the context of
\verb|random| is taken to be firm, and the context of \verb|i| and
\verb|ij| is promoted to strong and they are both dereferenced and
widened to \verb|REAL|.  The result is that the case clause will
yield a \verb|REAL| value even though the clause as a whole is in a
firm context (it is an operand of the operator \verb|+|).

If instead, we had
\begin{verbatim}
   PROC REAL r:=random;
   (a|i,ij|j):=ENTIER r
\end{verbatim}
\noindent
using the declaration of \verb|ij| in the previous example, then
balancing would not be needed to produce the required mode.  The
modes of the yielded units are \verb|REF INT|, \verb|PROC| \verb|REF|
\verb|INT| and \verb|REF INT| respectively.  In a soft
\hx{context}{context!soft}, these modes would yield \verb|REF INT|
(no dereferencing allowed), \verb|REF INT| (deproceduring is allowed)
and \verb|REF INT|.  Thus the case clause would yield \verb|REF INT|
on the left-hand side of the \ix{assignment}.

Here is an example of a conditional \hx{clause}{clause!conditional}
which cannot be balanced:
\begin{verbatim}
   INT i:=2, REAL a:=3.0;
   (random > 0.5|i|r):=random
\end{verbatim}
\noindent
In this case, the two parts of the conditional clause yield
\verb|REF INT| and \verb|REF REAL|. There is no coercion which will
convert a \verb|REF INT| into a \verb|REF REAL|. When you try to
compile this, the \hx{a68toc}{a68toc!balancing} compiler gives the
following error message:
\begin{verbatim}
   lhs of assignment must be a reference
\end{verbatim}
\noindent
The balancing means that one of the yields is in a strong context and
so is dereferenced which yields a value which is not a name.

The method of determining whether balancing is possible is as follows:
\begin{enumerate}
\item Determine the context of the choice clause.
\item In the context found in step~1, determine the mode yielded by
each unit in the choice clause.
\item If there is a mode such that all the modes but that one can be
strongly coerced to that mode, the clause can be balanced.
\end{enumerate}

\begin{exercise}
\item In each of the following clauses, state whether balancing is
possible, and if so, the mode yielded by the balanced clause. These
declarations are in force:
\begin{verbatim}
   INT i,j, REAL a,b:=random;
   PROC ij = REF INT: (b>0.5|i|j);
   PROC r = REAL: random * random;
   UNION(INT,REAL) ri:=(random>0.6|i|b)
\end{verbatim}
\indent\begin{subex}
\item \verb!a:=2.0*(random<0.3|i|b)! \subans The conditional clause
can yield a value of mode \verb|REF INT| or \texttt{REF REAL}. In a
firm context, these can be coerced to \verb|INT| and \verb|REAL|.
Thus the \verb|INT| is widened to \verb|REAL| and the balanced clause
yields a value of mode \verb|REAL|.
'
\item \verb!(j<2|ij|b):=r! \subans The conditional clause in a soft
context will yield \verb|REF INT| or \verb|REF REAL|. Neither can be
coerced to the other in a strong context, so the clause cannot be
balanced. The error message from the compiler arises from the
coercions applied in a strong context for the attempted balancing.
'
\item \verb!a:=((ri|(INT r):r,(REAL r):r)<1|2|3)! \subans The
conformity clause yields \verb|INT| or \verb|REAL|. In a strong
context, \verb|INT| can be widened to \verb|REAL|. Thus the balanced
clause will yield \verb|REAL|.
'
\item \verb!b:=2.0*(j>3|4|SKIP)! \subans The conditional clause
yields \verb|INT| or whatever. In a strong context, \verb|SKIP| will
yield \verb|INT|. Thus the balanced clause yields \verb|INT|.
However, the result will be undefined if the \verb|SKIP| is used in
the assignment.
'
\end{subex}
\end{exercise}

\Section{Well-formed modes}{gram-well}
In chapter 6, the mode \hx{declaration}{declaration!mode} was
presented and it was pointed out that not all possible mode
declarations are allowed.  The rules for determining whether a mode
declaration is well-\hx{formed}{mode!well-formed} are
straightforward.

There are two reasons why a mode might not be well-formed:
\begin{enumerate}
\item the \ix{elaboration} of a declaration using that
mode would need an infinite amount of memory\hylabel{gram-well1}
\item the mode can be strongly coerced to a related
mode\hylabel{gram-well2}
\end{enumerate}

Let us look at some examples of modes which are not well-formed.
Firstly, in the mode declaration
\begin{verbatim}
   MODE WRONG = STRUCT(CHAR c,WRONG w)
\end{verbatim}
\noindent
the \verb|WRONG| within the \verb|STRUCT| would expand to a further
\verb|STRUCT| and so on \emph{ad infinitum}. Even this declaration
\begin{verbatim}
   MODE WRONGAGAIN = STRUCT(WRONGAGAIN wa)
\end{verbatim}
\noindent
will not work for the same reason. However, if the mode within the
\verb|STRUCT| is \hx{shielded}{mode!shielding} by \verb|REF| or
\verb|PROC|, then the mode declaration is
\hx{legal}{mode!recursion}:
\begin{verbatim}
   MODE ALRIGHT = STRUCT(CHAR c,REF ALRIGHT a);
\end{verbatim}
\noindent
In the declaration
\begin{verbatim}
   ALRIGHT ar = ("A",LOC ALRIGHT)
\end{verbatim}
\noindent
the second field of the structure is a name which is quite different
from a structure. Likewise, the declaration
\begin{verbatim}
   MODE OKP = STRUCT(CHAR c,PROC OKP po)
\end{verbatim}
\noindent
is well-formed because in any declaration, the second field is a
procedure (or a name referring to such a procedure) which is not the
original structure and so does not require an infinite amount of
storage. It should be noted, however, that a \ixtt{UNION} does not
shield the mode sufficiently.  Thus, the mode declarations
\begin{verbatim}
   MODE MW1 = UNION(INT,MW1);
   MODE MW2 = STRUCT(UNION(CHAR,MW2) u,CHAR c)
\end{verbatim}
\noindent
are not well-formed. In fact, the mode declaration of \verb|MW1| fails
on reason~\hyref{gram-well2} above.

Secondly, a mode which could be strongly coerced to a related mode
would lead to ambiguity in coercions. Thus the mode declarations
\begin{verbatim}
   MODE WINT = PROC WINT;
   MODE WREF = REF WREF;
   MODE WROW = [5]WROW
\end{verbatim}
\noindent
are not well-formed.

All the above declarations have been recursive, but not mutually
\hx{recursive}{recursion!mutual}. Is it possible to declare
\begin{verbatim}
   MODE WA = STRUCT(WB wb,INT i),
        WB = STRUCT(WA wa,CHAR c)
\end{verbatim}
\noindent
Again, the elaboration of declarations using either mode would
require an infinite amount of storage, so the modes are not
well-formed. The following pair of mode declarations are all right:
\begin{verbatim}
   MODE RA = STRUCT(REF RB rb,INT i),
        RB = STRUCT(PROC RA pra,CHAR c)
\end{verbatim}

All non-recursive mode declarations are well-formed. It is only in
recursive and mutually-recursive modes that we have to apply a test
for well-formedness.

\subsection*{Determination of well-formedness}\hylabel{gram-detwell}
In any mutually-recursive mode declarations, or any recursive mode
declaration, to get from a particular mode on the left-hand side of
a mode declaration to the same mode \hx{indicant}{mode!indicant}
written on the right-hand side of a mode declaration, it is necessary
to traverse various mode constructors such as \verb|REF|, \verb|PROC|
or \verb|UNION|. Above each \verb|STRUCT| or set of procedure
parameters write ``\ix{yang}''. Above each \verb|REF| or \verb|PROC|
write ``\ix{yin}''. Now trace the path from the mode in question on
the left-hand side of the mode declaration until you arrive at the
same mode indicant on the right-hand side.  If you have at least one
``yin'' and at least one ``yang'', the mode is well-formed.

Let us try this method on the recursive mode declarations given in
this section. In the mode declaration for \verb|WRONG|, write
``yang'' above the \verb|STRUCT|. Thus to get from \verb|WRONG| on
the left to \verb|WRONG| on the right, a single ``yang'' is
traversed. Thus \verb|WRONG| is not well-formed. Likewise,
\verb|WRONGAGAIN| is not well-formed.  In mode \verb|ALRIGHT|, you
have to traverse a ``yang'' (\verb|STRUCT|) and a ``yin''
(\verb|REF|), so \verb|ALRIGHT| is well-formed.  Try it with the mode
\verb|OKP|.

Conversely, to get from \verb|MW1| to \verb|MW1| requires neither
``yin'' nor ``yang'', so \verb|MW1| is not well-formed. To get from
\verb|MW2| to \verb|MW2|, only a \verb|STRUCT| is traversed (the
\ixtt{UNION} does not count) so \verb|MW2| is also not well-formed.
Similar arguments hold for \verb|WINT|, \verb|WREF| and \verb|WROW|.

Now consider the mutually-recursive mode declarations of \verb|WA| and
\verb|WB|. At whichever mode we start, getting back to that mode means
traversing two ``yangs'' (both \verb|STRUCT|). Two ``yangs'' are all
right, but there should be at least one ``yin'', so the modes are not
well-formed.  On the other hand, from \verb|RA| to \verb|RA| traverses
a \verb|STRUCT| and a \verb|REF| and, via \verb|RB|, a \verb|STRUCT|
and a \verb|PROC| giving ``yang-yin-yang-yin'', so both \verb|RA| and
\verb|RB| are well-formed.

Remember that if you want to declare modes which are
mutually-recursive, the \hx{a68toc}{a68toc!mode declaration} compiler
requires that one of the modes should first be declared with a stub
\hx{declaration}{declaration!stub}.

\begin{exercise}
\item For each of the following mode declarations, determine whether
the modes are well-formed:
\begin{subex}
\item \texttt{MODE MA = INT} \subans Yes.
'
\item \texttt{MODE MB = PROC(MB)VOID} \subans Yes.
'
\item \texttt{MODE MC =[3,2]MC} \subans No.
'
\item \texttt{MODE MD = STRUCT(BOOL p,MD m)} \subans No.
'
\item \texttt{MODE ME = STRUCT(STRING s,REF ME m)} \subans Yes.
'
\item \ %
\begin{verbatim}
   MODE MF2,
       MF1 = STRUCT(REF MF2 f),
       MF2 = PROC(INT)MF1
\end{verbatim}
\indent \subans Yes.
'
\item \ %
\begin{verbatim}
   MODE MGB,
       MGA = PROC(MGB)VOID,
       MGB = STRUCT(MGA a)
\end{verbatim}
\indent \subans Yes.
'
\item \ %
\begin{verbatim}
   MODE B, C,
   MODE A = PROC(B)A,
   MODE B = STRUCT(PROC C c,
                   STRUCT(B b,INT i)d),
   MODE C = UNION(A,B)
\end{verbatim}
\indent \subans No.
'
\item \verb|C = PROC(C)C| \subans Yes! It's an example in the
``Revised Report''.
'
\end{subex}
\end{exercise}

\Section{Flexible names}{gram-flex}
Flexible names were introduced in section \hyref{name-flex}, but only
one-dimensional names. What has not been made apparent in the text
hitherto is that a multiple consists of two parts: a
\hx{\textbf{descrip\-tor}}{descriptor} and the actual elements. The
descriptor contains the lower and upper bounds of each dimension, the
``stride'' (that is, the number of bytes between two successive
elements of the dimension in question), the address in memory of the
first element of that dimension and whether the dimension is
flexible. Consider the declaration
\begin{verbatim}
   FLEX[1:0][1:3]INT flexfix
\end{verbatim}
\noindent
Because the mode of \verb|flexfix| is \verb|REF FLEX[][]INT|, when it
is subscripted, the mode of each element is \verb|REF[]INT| with
bounds of \verb|[1:3]|. Clearly, after the declaration,
\verb|flexfix| has no elements. In practice, because the first (and
only) dimension is flexible, there must be some way of referring to a
``ghost'' element whose descriptor (it is a one-dimensional multiple)
will give its properties. \verb|flexfix| is quite different from
\begin{verbatim}
   FLEX[1:0]FLEX[1:3]INT flexflex
\end{verbatim}
\noindent
each of whose elements (when it has any) have the mode \verb|REF|
\verb|FLEX[]INT| with initial bounds \verb|[1:3]|.

If the declaration of \verb|flexfix| is followed by the assignment
and slice
\begin{verbatim}
   flexfix:=LOC[1:1][1:3]INT;
   flexfix[1]:=(1,2,3)
\end{verbatim}
\noindent
then it is clear that the mode of \verb|flexfix[1]| is \verb|REF[]INT|.
Note that after
\begin{verbatim}
   flexfix:=LOC[1:4][1:3]INT
\end{verbatim}
\noindent
\verb|flexfix| refers to a multiple of which each element has the mode
\verb|[]INT|. However, the single dimension of
\begin{verbatim}
   flexfix[1]
\end{verbatim}
\noindent
is \emph{not} flexible, which is why the assignment
\begin{verbatim}
   flexfix:=LOC[1:4][1:4]REAL #this is wrong#
\end{verbatim}
\noindent
will fail\footnote{The a68toc compiler will wrongly allow this last
assignment both at compile-time and run-time.}.

\Section{Orthogonality}{gram-orth}
\hx{We}{orthogonality} have come a long way and introduced many new
ideas, yet all these ideas are based on the \ix{primitive concepts}
of \ix{value}, \ix{mode}, \ix{context}, \ix{coercion} and
\ix{phrase}.  These concepts are independent of each other, but their
combination provides Algol~68 with a \ix{flexibility} that few
programming languages possess.  For example, if a value of mode
\verb|INT| is required, such as in a trimmer or the bounds of the
declaration of a multiple, then any unit which will yield an integer
in that context will suffice.  The consequence is that Algol~68
programs can be written in a wide variety of styles.  Here is a
simple example: given the problem of printing the sum of two numbers
read from the keyboard, it could be programmed in two completely
different ways.  The conventional solution would be something like
\begin{verbatim}
   INT a,b;  read((a,b));
   print((a+b,newline))
\end{verbatim}
\noindent
but an equally valid solution is
\begin{verbatim}
   print(((INT a,b;
           read((a,b));
           a+b),newline))
\end{verbatim}
\noindent
Provided that what you write is legal Algol~68, you can adopt any
approach you please. Orthogonality refers to the independence of the
basic concepts in that you can combine them without side-effects.

Another consequence of that independence is that there are very few
exceptions to the rules of the language.  This makes the language
much easier to learn.

\Section{Summary}{gram-summ}
The grammar of Algol~68 is expressed in terms of a few primitive
concepts: value, mode, context, coercion and phrase. A phrase is
either a declaration or a unit. There are 5 contexts, 7 coercions, 22
different kinds of unit and potentially an infinite number of values
and modes.  The coercions available in each context have been
described.  Balancing is the means by which alternatives in
conditional, case and conformity clauses and the two sides of an
identity relation are coerced to a common mode, possibly making
coercions available which would not normally be so in the context of
the construct concerned.

No exercises are provided at this point.
